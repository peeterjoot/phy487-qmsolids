%
% Copyright � 2013 Peeter Joot.  All Rights Reserved.
% Licenced as described in the file LICENSE under the root directory of this GIT repository.
%
%\input{../blogpost.tex}
%\renewcommand{\basename}{condensedMatterLecture3}
%\renewcommand{\dirname}{notes/phy487/}
%\newcommand{\keywords}{Condensed matter physics, PHY487H1F}
%\input{../peeter_prologue_print2.tex}
%
%%\citep{harald2003solid} \S x.y
%%\citep{ibach2009solid} \S x.y
%
%\usepackage{mhchem}
%\usepackage{amssymb}
%
%\beginArtNoToc
%\generatetitle{PHY487H1F Condensed Matter Physics.  Lecture 3: Bonding and lattice structure.  Taught by Prof.\ Stephen Julian}
%%\chapter{Bonding and lattice structure}
%\label{chap:condensedMatterLecture3}
%
%\section{Disclaimer}
%
%Peeter's lecture notes from class.  May not be entirely coherent.
%
%\section{Bonding (cont.)}
%
%\section{Ionic bonding}
\index{ionic bonding}
\index{lattice structure}

We introduce the \underlineAndIndex{Madelung constant} \(A\) for the potential energy of the solid configuration
%
\begin{equation}\label{eqn:condensedMatterLecture3:20}
\begin{aligned}
\Phi_{tot} &= \sum_i \phi_i = \inv{2} \sum_{i \ne j} \phi_{ij}
\\ &= \inv{2}
%%\mathLabelBox
%%[
%%   labelstyle={xshift=-2cm},
%%   linestyle={out=270,in=90, latex-}
%%]
%%{N}{number of ions in solid}
N
\Bigl(
-\frac{e^2}{4 \pi \epsilon_\nought r}
\mathLabelBox
[
   labelstyle={yshift=0.3cm},
   linestyle={out=270,in=90, latex-}
]
{
\sum_{i \ne j} \frac{\lr{\pm 1}}{ p_{ij} }
}{A}
+ \frac{B}{r^n}
\sum_{i \ne j} \inv{p_{ij}^n}
\Bigr),
\end{aligned}
\end{equation}
%
where \( N \) is the number of ions in the solid,
\(r\) is the nn separation (center to center), and \(r_{ij} = p_{ij} r\), as illustrated in \cref{fig:qmSolidsL3:qmSolidsL3Fig1}.
%
\imageFigure{../figures/phy487-qmsolids/qmSolidsL3Fig1}{Madelung separation.}{fig:qmSolidsL3:qmSolidsL3Fig1}{0.1}

As an additional illustration, we have the \ce{NaCl} configuration in \cref{fig:qmSolidsL3:qmSolidsL3Fig2}.
%
\imageFigure{../figures/phy487-qmsolids/qmSolidsL3Fig2}{\ce{NaCl} lattice separation.}{fig:qmSolidsL3:qmSolidsL3Fig2}{0.2}

\examhint{Eminently examinable material (since it can be calculated).}

Examples
\begin{itemize}
\item \ce{NaCl} structure \(A = 1.748\)
\item \ce{CsCl} structure \(A = 1.763\)
\end{itemize}

Ionic bonds are weaker than covalent, non-directional.  One indicator of this is the melting points
%
\begin{subequations}
\begin{equation}\label{eqn:condensedMatterLecture3:40}
T_m = 1074 K \quad \mbox{\ce{NaCl}}
\end{equation}
\begin{equation}\label{eqn:condensedMatterLecture3:60}
T_m = 918 K \quad \mbox{\ce{CsCl}}.
\end{equation}
\end{subequations}
%
\section{Metallic bonding.}
\index{metallic bonding}
We now focus on the regions of the periodic table illustrated in \cref{fig:qmSolidsL3:qmSolidsL3Fig3}.
%
\imageFigure{../figures/phy487-qmsolids/qmSolidsL3Fig3}{Metallic bonding regions in the periodic table.}{fig:qmSolidsL3:qmSolidsL3Fig3}{0.2}

Curiously, the name is somewhat misleading.  Just because something is a metal doesn't mean it is metallic bonded.
\begin{itemize}
\item s-orbitals from \(2s\) to \(5s\),
\item p-orbitals from \(n = 4, 5, 6, 7\).
\end{itemize}
These are big orbitals that extend beyond the nn, as illustrated in \cref{fig:qmSolidsL3:qmSolidsL3Fig4}.
%
\imageFigure{../figures/phy487-qmsolids/qmSolidsL3Fig4}{Extensive wave function.}{fig:qmSolidsL3:qmSolidsL3Fig4}{0.2}

Weakly bound electrons overlap many nearby potential wells.  This lowers the Coulomb energy.  This is like a non-directional covalent bond.  This non-directionality results in malleability.

Pure metallic bonds are weak.  Melting points are correspondingly low, where for column 1 elements \(T_m\) ranges from room temperature to \(200^\circ\) C.
%
\section{Transition metals.}
\index{transition metals}

Here we have both metallically bonded s-orbitals and covalent d-orbitals \cref{fig:qmSolidsL3:qmSolidsL3Fig6}.
%
\imageFigure{../figures/phy487-qmsolids/qmSolidsL3Fig6}{Overlapping \(s\) and \(p\) orbitals.}{fig:qmSolidsL3:qmSolidsL3Fig6}{0.2}

%\cref{fig:qmSolidsL3:qmSolidsL3Fig5}.
%\imageFigure{../figures/phy487-qmsolids/qmSolidsL3Fig5}{Overlapping orbitals}{fig:qmSolidsL3:qmSolidsL3Fig5}{0.2}
%
\begin{equation}\label{eqn:condensedMatterLecture3:80}
T_m \gtrsim 2000^\circ C.
\end{equation}
%
\paragraph{On sign conventions.}  The \(+, -\)'s assume real representation of wave functions.  Here bonding is matching signs (constructive interference), and antibonding is when the signs are in opposition (destructive interference).
%
\reading \S 1.5, 1.6 \citep{ibach2009solid}.

