%
% Copyright � 2013 Peeter Joot.  All Rights Reserved.
% Licenced as described in the file LICENSE under the root directory of this GIT repository.
%
%\input{../blogpost.tex}
%\renewcommand{\basename}{condensedMatterLecture13}
%\renewcommand{\dirname}{notes/phy487/}
%\newcommand{\keywords}{Condensed matter physics, PHY487H1F}
%\input{../peeter_prologue_print2.tex}
%
%%\citep{harald2003solid} \S x.y
%%\citep{ibach2009solid} \S x.y
%
%%\usepackage{mhchem}
%\usepackage[version=3]{mhchem}
%\newcommand{\nought}[0]{\circ}
%\newcommand{\EF}[0]{\epsilon_{\txtF}}
%\newcommand{\kF}[0]{k_{\txtF}}
%
%\beginArtNoToc
%\generatetitle{PHY487H1F Condensed Matter Physics.  Lecture 13: Free electron model of metals.  Taught by Prof.\ Stephen Julian}
%%\chapter{Free electron model of metals}
\label{chap:condensedMatterLecture13}
%
%\section{Disclaimer}
%
%Peeter's lecture notes from class.  May not be entirely coherent.
%
\paragraph{Heat capacity of free electrons (cont.)}
\index{specific heat}

Last time we found the \textAndIndex{density of states} for Fermions in a period potential
%
\begin{dmath}\label{eqn:condensedMatterLecture13:20}
D(E) = \inv{2 \pi^2} \lr{\frac{2m}{\Hbar^2}}^{3/2} \sqrt{E}.
\end{dmath}
%
Using the \textAndIndex{Fermi-Dirac distribution} \cref{fig:fermiDiracLevelCurves:fermiDiracLevelCurvesFig1}
%
\mathImageFigure{../figures/phy487-qmsolids/fermiDiracLevelCurvesFig1}{Fermi-Dirac distribution}{fig:fermiDiracLevelCurves:fermiDiracLevelCurvesFig1}{0.3}{fermiDiracPlot.nb}
%
\begin{dmath}\label{eqn:condensedMatterLecture13:40}
f(E, T) = \inv{ e^{(E - \mu)/\kB T} + 1},
\end{dmath}
%
we calculated an approximate value for the specific heat
%
\begin{dmath}\label{eqn:condensedMatterLecture13:60}
C(T) \sim 2 \kB D(\EF) T.
\end{dmath}
%
We will now move on and calculate a more exact expression for the specific heat, defined by
%
\begin{subequations}
\begin{dmath}\label{eqn:condensedMatterLecture13:80}
U(T) = \int dE D(E) E f(E, T)
\end{dmath}
\begin{dmath}\label{eqn:condensedMatterLecture13:100}
C(T) = \PD{T}{U},
\end{dmath}
\end{subequations}
%
or, in terms of the density of states
%
\begin{dmath}\label{eqn:condensedMatterLecture13:120}
C(T) = \int dE D(E) E \PD{T}{f(E, T)}.
\end{dmath}
%
In \cref{fig:fermiDiracLevelCurves:fermiDiracLevelCurvesFig2}, are plots of the Fermi-Dirac distribution functions at \(T_2 > T_1\) and their difference.  Observe that this difference is zero most everywhere
%
\mathImageFigure{../figures/phy487-qmsolids/fermiDiracLevelCurvesFig2}{Fermi-Dirac curves and their difference}{fig:fermiDiracLevelCurves:fermiDiracLevelCurvesFig2}{0.3}{fermiDiracPlot.nb}

Calculating that derivative explicitly, we have
%
\begin{dmath}\label{eqn:condensedMatterLecture13:140}
\PD{T}{f(E, T)}
=
-
\frac{
e^{(E - \mu)/\kB T}
}
{
(e^{(E - \mu)/\kB T} + 1)^2
}
\frac{E - \mu}{\kB} \frac{-1}{T^2}
=
\frac{
e^{(E - \mu)/\kB T}
}
{
(e^{(E - \mu)/\kB T} + 1)^2
}
\frac{E - \mu}{\kB T^2}.
\end{dmath}
%
This is plotted in \cref{fig:fermiDiracLevelCurvesAndDerivatives:fermiDiracLevelCurvesAndDerivativesFig3}.
%
\mathImageFigure{../figures/phy487-qmsolids/fermiDiracLevelCurvesAndDerivativesFig3}{Fermi-Dirac distribution and derivatives}{fig:fermiDiracLevelCurvesAndDerivatives:fermiDiracLevelCurvesAndDerivativesFig3}{0.3}{fermiDiracPlot.nb}

The expected value of the number density is
%
\begin{dmath}\label{eqn:condensedMatterLecture13:160}
n = \int dE D(E) f(E, T).
\end{dmath}
%
It's derivative with respect to temperature is
%
\begin{dmath}\label{eqn:condensedMatterLecture13:180}
\PD{T}{n} = \int dE D(E) \PD{T}{f}.
\end{dmath}
%
With a constant constraint on the number of states, this derivative is zero, allowing us to write
%
\begin{dmath}\label{eqn:condensedMatterLecture13:200}
C(T)
=
C(T) - \EF \PD{T}{n}
=
\EF \int dE D(E) (E - \EF) \PD{T}{f}
= \int_0^\infty dE D(E)
( E - \EF)
\mathLabelBox
{
\frac{ E - \EF}{\kB T^2}
\frac{
e^{(E - \mu)/\kB T}
}
{
(e^{(E - \mu)/\kB T} + 1)^2
}
}
{
zero except within a few \(\kB T\) of \(\EF\)
}
\approx
D(\EF)
\int_0^\infty dE
( E - \EF)
\frac{ E - \EF}{\kB T^2}
\frac{
e^{(E - \EF)/\kB T}
}
{
(e^{(E - \EF)/\kB T} + 1)^2
}.
\end{dmath}
%
Because we have a zero in the integrand, except close to \(\EF\), we have made the approximation \(D(E) \rightarrow D(\EF)\), with the density function with its value at \(\EF\), perhaps similar to the point sampling sketched in \cref{fig:qmSolidsL13:qmSolidsL13Fig3}.
%
\imageFigure{../figures/phy487-qmsolids/qmSolidsL13Fig3}{Density of states point approximation}{fig:qmSolidsL13:qmSolidsL13Fig3}{0.2}

This mostly zero in the integrand also allows us to extend the integration range
%
\begin{dmath}\label{eqn:condensedMatterLecture13:220}
\int_0^\infty \rightarrow \int_{-\infty}^\infty,
\end{dmath}
%
To proceed with the integration, let
%
\begin{dmath}\label{eqn:condensedMatterLecture13:240}
x = \frac{E - \EF}{\kB T}.
\end{dmath}
%
\begin{dmath}\label{eqn:condensedMatterLecture13:260}
C(T) \approx
D(\EF)
\int_{-\infty}^\infty
\mathLabelBox
[
   labelstyle={xshift=2cm, yshift=0.5cm},
   linestyle={out=270,in=90, latex-}
]
{
(dx \kB T)
}
{
\(dE\)
}
\frac{x^2 \cancel{\kB^2} \cancel{T^2}}{\cancel{\kB} \cancel{T^2}}
\frac{e^x}{e^x + 1)^2}
=
D(\EF)
\kB^2 T
\mathLabelBox
[
%   labelstyle={below of=m\themathLableNode, below of=m\themathLableNode, xshift=-1cm, yshift=-1cm},
   labelstyle={xshift=3cm, yshift=0.5cm},
   linestyle={out=270,in=90, latex-}
]
{
\int_{-\infty}^\infty
dx \frac{x^2 e^x}{(e^x + 1)^2}
}
{
\(\pi^2/3\)
}.
\end{dmath}
%
We have finally
\boxedEquation{eqn:condensedMatterLecture13:280}{
C(T) \approx \frac{\pi^2}{3} \kB^2 D(\EF) T.
}

The linear T specific heat is a signature of the Fermi surface (sharp boundary in k-space between occupied and unoccupied states).  This doesn't depend on the details form of \(D(E)\), but only on \(D(\EF)\).  This therefore works for \textunderline{all} metals.

If you see a \(C \sim T^3\) (cubic) relationship you can realize that we are dealing with a Bosonic system with a linear frequency relationship.

%\cref{fig:qmSolidsL13:qmSolidsL13Fig4a}.
\imageFigure{../figures/phy487-qmsolids/qmSolidsL13Fig4a}{Thermally excited region}{fig:qmSolidsL13:qmSolidsL13Fig4a}{0.2}
%\cref{fig:qmSolidsL13:qmSolidsL13Fig4b}.
\imageFigure{../figures/phy487-qmsolids/qmSolidsL13Fig4b}{Momentum space for linear frequency temperature region}{fig:qmSolidsL13:qmSolidsL13Fig4b}{0.2}
%
\paragraph{Electrons}
%
Each thermally excited electron has thermal energy \(\kB T\), so
%
\begin{dmath}\label{eqn:condensedMatterLecture13:320}
\begin{aligned}
U(T) &\sim T^2 \\
\implies \\
C(T) &\propto T
\end{aligned}
\end{dmath}
%
If you see a \(C \sim T\) (linear) relationship you can realize that we are dealing with a Fermionic system.

At \(T > 0\), thin shell of width \(\kB T\) thermally excited.  Volume is
%
\begin{dmath}\label{eqn:condensedMatterLecture13:300}
4 \pi \kF^2 \delta k \propto 4 \pi \kF^2 \delta T
\end{dmath}
%
%\cref{fig:qmSolidsL13:qmSolidsL13Fig5a}.
\imageFigure{../figures/phy487-qmsolids/qmSolidsL13Fig5a}{Free particle energy levels}{fig:qmSolidsL13:qmSolidsL13Fig5a}{0.2}
%\cref{fig:qmSolidsL13:qmSolidsL13Fig5b}.
\imageFigure{../figures/phy487-qmsolids/qmSolidsL13Fig5b}{Phase space region for Fermi momentum}{fig:qmSolidsL13:qmSolidsL13Fig5b}{0.2}
%
\sectionAndIndex{Thomas-Fermi screening}
\index{Thomas-Fermi screening}

\reading \citep{ashcroft1976solid} \textchapref{17}, \citep{ibach2009solid} \S 6.5.

Recall that \(\BE = 0\) inside a metal in equilibrium as illustrated in \cref{fig:qmSolidsL13:qmSolidsL13Fig6}.
%
\imageFigure{../figures/phy487-qmsolids/qmSolidsL13Fig6}{Field in conducting metal}{fig:qmSolidsL13:qmSolidsL13Fig6}{0.2}

%\cref{fig:qmSolidsL13:qmSolidsL13Fig7}.
\imageFigure{../figures/phy487-qmsolids/qmSolidsL13Fig7}{Point charge in a box}{fig:qmSolidsL13:qmSolidsL13Fig7}{0.2}

Put a fixed point charge \(Q\) at \(\Br = 0\), and an electric potential \(\phi(\Br)\).  Our Maxwell equation is
%
\begin{dmath}\label{eqn:condensedMatterLecture13:340}
\spacegrad^2 \phi(\Br) = - \frac{\rho(\Br)}{\epsilon_\nought}.
\end{dmath}
%
Split the charge density as
%
\begin{dmath}\label{eqn:condensedMatterLecture13:360}
\rho(\Br) =
%\cancel{
   \mathLabelBox
   [
      labelstyle={xshift=-2cm},
      linestyle={out=270,in=90, latex-}
   ]
   {
   \overbar{\rho}_{\mathrm{el}}
   }
   {average electron density}
   +
   \mathLabelBox
   [
      labelstyle={below of=m\themathLableNode, below of=m\themathLableNode}
   ]
   {
   \overbar{\rho}_{\mathrm{ion}}
   }
   {
   positive background
   }
%}
+
\mathLabelBox
[
   labelstyle={xshift=2cm},
   linestyle={out=270,in=90, latex-}
]
{
\delta \rho_{\mathrm{el}}
}
{
perturbation to Q
}
\end{dmath}
%
The first two terms cancel giving
%
\begin{dmath}\label{eqn:condensedMatterLecture13:380}
\rho_{\mathrm{el}}
= -e
\int dE D(E)
\inv{ e^{(E
- \mu - e \phi(\Br)
)/\kB T} + 1}
\approx
-e
\int dE D(E)
\lr{
\inv{ e^{(E - \mu)/\kB T} + 1}
- e \phi(\Br)
\evalbar{\PD{E}{f} }{E = E - \mu}
+ \cdots
}
\end{dmath}
%
Here \(\mu + e \phi(\Br)\) is the chemical potential shifted at \(\Br\) by \(e \phi(\Br)\).  Note that as \(T \rightarrow 0\), we have
%
\begin{dmath}\label{eqn:condensedMatterLecture13:400}
\int_
{\EF - \Delta}
^{\EF + \Delta}
dE \PD{E}{f}
= -1,
\end{dmath}
%
where the width of \(\PDi{E}{f} \rightarrow 0\).  This is very much like a delta function \cref{fig:fermiDiracLevelCurvesAndEnergyDerivatives:fermiDiracLevelCurvesAndEnergyDerivativesFig4}
%as sketched in \cref{fig:qmSolidsL13:qmSolidsL13Fig8}
, so we can write
%
\mathImageFigure{../figures/phy487-qmsolids/fermiDiracLevelCurvesAndEnergyDerivativesFig4}{Delta function like region}{fig:fermiDiracLevelCurvesAndEnergyDerivatives:fermiDiracLevelCurvesAndEnergyDerivativesFig4}{0.3}{fermiDiracPlot.nb}
%\imageFigure{../figures/phy487-qmsolids/qmSolidsL13Fig8}{Delta function like region}{fig:qmSolidsL13:qmSolidsL13Fig8}{0.2}
%
\begin{dmath}\label{eqn:condensedMatterLecture13:420}
\rho_{\mathrm{el}}
\approx
\overbar{\rho}_{\mathrm{el}}
- e^2 \phi(\Br) \int dE D(E) \delta(E - \EF)
\approx
\overbar{\rho}_{\mathrm{el}}
- e^2 \delta \phi(\Br) D(\EF)
\end{dmath}
%
Recall that the spherical form of the Laplacian of a function with only radial dependence is
%
\begin{equation}\label{eqn:condensedMatterLecture13:540}
\spacegrad^2 f = \PDSq{r}{f} + \frac{2}{r} \PD{r}{f} = \inv{r^2} \PD{r}{} \lr{ r^2 \PD{r}{f} }.
\end{equation}
%
For the potential we have
%
\begin{dmath}\label{eqn:condensedMatterLecture13:440}
\phi(\Br) = \phi_{\mathrm{avg}} + \delta \phi(\Br).
\end{dmath}
%
so that the electrostatic equation is
%
\begin{dmath}\label{eqn:condensedMatterLecture13:460}
\inv{r^2} \PD{r}{} \lr{ r^2 \PD{r}{\delta \phi(\Br)} } = \frac{e^2 D(\EF)}{\epsilon_\nought} \delta \phi(\Br).
\end{dmath}
%
This has solution (see: \cref{pr:condensedMatterLecture13:1})
%
\begin{dmath}\label{eqn:condensedMatterLecture13:480}
\delta \phi(\Br) = \frac{\alpha e^{ -r/r_{\mathrm{TF} } }}{r},
\end{dmath}
%
where \(r_{\mathrm{TF} }\) is the \textAndIndex{Thomas-Fermi screening} length
%
\begin{dmath}\label{eqn:condensedMatterLecture13:500}
r_{\mathrm{TF}}
= \sqrt{\frac{\epsilon_\nought}{e^2 D(\EF) }}.
\end{dmath}
%
The functional form of the screened and unscreened potentials are plotted in \cref{fig:thomasFermiScreening:thomasFermiScreeningFig5}.
\mathImageFigure{../figures/phy487-qmsolids/thomasFermiScreeningFig5}{Screened vs unscreened Coulomb potential}{fig:thomasFermiScreening:thomasFermiScreeningFig5}{0.2}{fermiDiracPlot.nb}
%\cref{fig:qmSolidsL13:qmSolidsL13Fig9}.
%\imageFigure{../figures/phy487-qmsolids/qmSolidsL13Fig9}{Screened vs unscreened Coulomb potential}{fig:qmSolidsL13:qmSolidsL13Fig9}{0.2}
\makeexample{Copper}{example:condensedMatterLecture13:1}{
\begin{dmath}\label{eqn:condensedMatterLecture13:520}
r_{\mathrm{TF} }
\sim
0.5
\angstrom
\end{dmath}
}

Reading: \S 6.5, especially Mott transition.
%\EndArticle
