%
% Copyright � 2013 Peeter Joot.  All Rights Reserved.
% Licenced as described in the file LICENSE under the root directory of this GIT repository.
%
\makeoproblem{Geometrical construction of the reciprocal lattice}{condensedMatter:problemSet3:1}{2013 ps3 p1}{
\Cref{fig:reciprocal_lattice:reciprocal_latticeFig1}.
shows a two-dimensional lattice together with two
  vectors.
\imageFigure{../figures/phy487-qmsolids/reciprocal_latticeFig1}{A sample lattice.}{fig:reciprocal_lattice:reciprocal_latticeFig1}{0.3}
\makesubproblem{}{condensedMatter:problemSet3:1a}
Demonstrate that \(\Ba_1\) and \(\Ba_2\)
  are basis vectors by showing that you can reach the sites numbered
  1 through 5 by a combination of \(\Ba_1\) and \(\Ba_2\).
(e.g. to reach site 1 use \(0*\Ba_1 + 1*\Ba_2\), etc.. )
\makesubproblem{}{condensedMatter:problemSet3:1b}
On the page and using a ruler and/or protractor and/or any other
  drafting tools you may require, draw the two basis vectors \(\Bg_1\) and
  \(\Bg_2\) of the corresponding reciprocal lattice.
Draw \(\Bg_1\) and \(\Bg_2\) to scale (so they have the correct
  length relative to each other).
\makesubproblem{}{condensedMatter:problemSet3:1c}
Consider three waves, with wave-vectors \(\Bk_a = \Bg_1\),
  \(\Bk_b = \Bg_2\) and \(\Bk_c = \Bg_1 + 2\Bg_2\),
  draw lines on the diagram to indicate the positions of the wave-crests,
  assuming that one of the crests passes through the point ``0".
  If these lines correspond to ``planes" in the two-dimensional crystal,
  give the Miller indices of the planes.
} % makeproblem

\makeanswer{condensedMatter:problemSet3:1}{
\makeSubAnswer{}{condensedMatter:problemSet3:1a}
Our linear combinations are
\begin{enumerate}
\item \(\Ba_2\)
\item \(\Ba_2 - \Ba_1\)
\item \(2 \Ba_2\)
\item \(\Ba_1\)
\item \(\Ba_1 - \Ba_2\)
\end{enumerate}
\makeSubAnswer{}{condensedMatter:problemSet3:1b}
With the x-axis measured along \(\Ba_1\) in cm, I measure
%
\begin{dmath}\label{eqn:condensedMatterProblemSet3Problem1:20}
\begin{aligned}
\Ba_1 &= (2, 0) \pm 0.05 \\
\Ba_2 &= \lr{(1.75, 2.6) \pm 0.05}
\end{aligned}
\end{dmath}
%
Computing the reciprocal frame by inversion, we have
%
\begin{dmath}\label{eqn:condensedMatterProblemSet3Problem1:40}
\begin{bmatrix}
\Bg_1 & \Bg_2
\end{bmatrix}
=
2 \pi
\begin{bmatrix}
\Ba_1^\T \\
\Ba_2^\T
\end{bmatrix}
=
\begin{bmatrix}
 3.14159 & 0. \\
 -2.11453 & 4.83322 \\
\end{bmatrix}
\end{dmath}
%
These are drawn out on the \cref{fig:qmSolidsPs3:qmSolidsPs3Fig3}.
%
%\imageFigure{../figures/phy487-qmsolids/qmSolidsPs3Fig2}{Drawings of reciprocal vectors}{fig:qmSolidsPs3:qmSolidsPs3Fig2}{0.5}
\imageFigure{../figures/phy487-qmsolids/qmSolidsPs3Fig3}{reciprocal vectors and sample wave trains.}{fig:qmSolidsPs3:qmSolidsPs3Fig3}{0.5}
\makeSubAnswer{}{condensedMatter:problemSet3:1c}
Referring again to \cref{fig:qmSolidsPs3:qmSolidsPs3Fig3}.
\begin{itemize}
\item The wave crests for \(\Bk_a = \Bg_1\) are shown in blue.  For this wave we have \(\Bk_a \cdot \Ba_1 = g_1 a_1 \cos\theta = 2 \pi\), or a wave length of \(2 \pi/\Abs{\Bg_1}\).  The Miller indices for this wave are \(1,0,0\).

\item The wave crests for \(\Bk_b = \Bg_2\) are shown in red.  For this wave we have \(\Bk_b \cdot \Ba_2 = g_2 a_2 \cos\theta = 2 \pi\), or a wave length of \(2 \pi/\Abs{\Bg_2}\).  The Miller indices for this wave are \(0, 1, 0\).

\item The wave crests for \(\Bk_c = \Bg_1 + 2 \Bg_2\) are shown in pink and green.

For the pink wave we have \(\Bk_c \cdot \Ba_1 = k_c a_1 \cos \theta = g_1 a_1 = 2 \pi\), or a wave length of \(2 \pi/\Abs{\Bk_c}\).  The angle between \(\Bk_c\) and \(\Ba_1\) is \(\cos\theta = 1/\sqrt{5}\), or \(\theta \sim 63^\circ\).

For the green wave we have \(\Bk_c \cdot \Ba_2 = k_c a_2 \cos \theta = 2 g_2 a_2 = 4 \pi\), or a wave length of \(4 \pi/\Abs{\Bk_c}\).  The angle between \(\Bk_c\) and \(\Ba_2\) is \(\cos\theta = 2/\sqrt{5}\), or \(\theta \sim 26^\circ\).

The Miller indices for this wave are \(1, 2, 0\).

\end{itemize}
}

