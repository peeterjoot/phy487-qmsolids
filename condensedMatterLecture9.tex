%
% Copyright � 2013 Peeter Joot.  All Rights Reserved.
% Licenced as described in the file LICENSE under the root directory of this GIT repository.
%
%\input{../blogpost.tex}
%\renewcommand{\basename}{condensedMatterLecture9}
%\renewcommand{\dirname}{notes/phy487/}
%\newcommand{\keywords}{Condensed matter physics, PHY487H1F}
%\input{../peeter_prologue_print2.tex}
%
%%\citep{harald2003solid} \S x.y
%
%%\usepackage{mhchem}
%\usepackage[version=3]{mhchem}
%
%\beginArtNoToc
%\generatetitle{PHY487H1F Condensed Matter Physics.  Lecture 9: Thermal properties.  Taught by Prof.\ Stephen Julian}
%\chapter{Thermal properties}
\label{chap:condensedMatterLecture9}
%
%\section{Disclaimer}
%
%Peeter's lecture notes from class.  May not be entirely coherent.
%
\section{Thermal properties.}
\section{lattice energy.}
%
We'll want to calculate the total energy stored in the lattice.  There are a number of steps to this problem

\begin{enumerate}
\item Quantization.

As an example, the classic SHO problem
%
\begin{dmath}\label{eqn:condensedMatterLecture9:20}
m \ddot{u} = -k u,
\end{dmath}
%
results in a single frequency
%
\begin{dmath}\label{eqn:condensedMatterLecture9:40}
\omega = \sqrt{\frac{k}{m}}.
\end{dmath}
%
With quantization we find only discrete frequencies, as in \cref{fig:qmSolidsL9:qmSolidsL9Fig1}.
%
\imageFigure{../figures/phy487-qmsolids/qmSolidsL9Fig1}{Quantized SHO energy levels.}{fig:qmSolidsL9:qmSolidsL9Fig1}{0.1}
%
\begin{dmath}\label{eqn:condensedMatterLecture9:60}
\epsilon_n = \lr{ n + \inv{2} } \Hbar \omega.
\end{dmath}
%
For lattices we've been seeking solutions of the force equations
%
\begin{dmath}\label{eqn:condensedMatterLecture9:80}
M_\alpha \ddot{u}_{n \alpha i} = - \sum_{m \beta j}
\Phi_{n \alpha i}^{m \beta j} u_{m \beta j}.
\end{dmath}
%
This provided us lattice frequencies \(\omega_q\), which quantize as
%
\begin{dmath}\label{eqn:condensedMatterLecture9:100}
\epsilon_{q, n} = \lr{ n_q + \inv{2} } \Hbar \omega_q.
\end{dmath}
%
Prof hoping that we are willing to this without proof since the proof is hard, and would take a couple days (see: \citep{ashcroft1976solid} appendix L.)

We've apparently seen such a derivation indirectly in blackbody calculations.

\item

If we accept \eqnref{eqn:condensedMatterLecture9:100}, then the thermal energy is
%
\begin{dmath}\label{eqn:condensedMatterLecture9:120}
U(T) = \mathLabelBox
[
   labelstyle={below of=m\themathLableNode, below of=m\themathLableNode}
]
{\sum_q}{sum over modes}
\Bigl(
\mathLabelBox
{
n_q
}
{average thermal occupancy}
+ \inv{2}
\Bigr) \Hbar \omega_q.
\end{dmath}
%
\item figuring out how to evaluate such a sum.

\end{enumerate}
%
\section{Density of states.}
\index{density of states}

\reading \citep{ibach2009solid} \S 5.1

The density of states is the number of Phonon modes per unit energy.

\begin{itemize}
\item
\(n_q\) depends only on energy (see below), so that we can group states by energy.
\item
\(q\) is quasi-continuous
\end{itemize}

%\cref{fig:qmSolidsL9:qmSolidsL9Fig2}.
\imageFigure{../figures/phy487-qmsolids/qmSolidsL9Fig2}{Period boundary conditions.}{fig:qmSolidsL9:qmSolidsL9Fig2}{0.15}

due to periodic boundary
%
\begin{dmath}\label{eqn:condensedMatterLecture9:140}
e^{i \Bq \cdot \lr { \Br_n + \BL_x } } =
e^{i \Bq \cdot \Br_n },
\end{dmath}
%
or
%
\begin{dmath}\label{eqn:condensedMatterLecture9:160}
q_x L_x = 2 \pi l,
%\Bq \cdot \Br_n = 2 \pi l,.
\end{dmath}
%
for integer \(l\).

Similarly for \(q_y\), \(q_z\) we have
%
\begin{dmath}\label{eqn:condensedMatterLecture9:180}
\Bq = l \frac{ 2 \pi }{L_x} \xcap
+ m \frac{ 2 \pi }{L_y} \ycap
+ o \frac{ 2 \pi }{L_z} \zcap.
\end{dmath}
%
The volume per \(\Bq\) point is
%
\begin{dmath}\label{eqn:condensedMatterLecture9:200}
\frac{2 \pi}{L_x}
\frac{2 \pi}{L_y}
\frac{2 \pi}{L_z}
= \frac{\lr{ 2 \pi }^3 }{V}.
\end{dmath}
%
%\cref{fig:qmSolidsL9:qmSolidsL9Fig3}.
\imageFigure{../figures/phy487-qmsolids/qmSolidsL9Fig3}{constant energy surface.}{fig:qmSolidsL9:qmSolidsL9Fig3}{0.15}
%
\begin{dmath}\label{eqn:condensedMatterLecture9:220}
\sum_\Bn
=
\sum_{n_x}
\sum_{n_y}
\sum_{n_z}
\sim \int dn_x dn_y dn_z
=
\int
\frac{L_x}{2 \pi}
dq_x
\frac{L_y}{2 \pi}
dq_y
\frac{L_z}{2 \pi}
dq_z
=
\frac{V}{(2 \pi)^3} \int d^3\Bq.
\end{dmath}
%
We group states of same \(\Hbar \omega_q\).  In \(d^3\Bq\) group same energy states

%\cref{fig:qmSolidsL9:qmSolidsL9Fig4}.
\imageFigure{../figures/phy487-qmsolids/qmSolidsL9Fig4}{area element.}{fig:qmSolidsL9:qmSolidsL9Fig4}{0.15}

\(d f_\omega\) are constant energy area elements, so that the volume element is
%
\begin{dmath}\label{eqn:condensedMatterLecture9:240}
df_\omega dq_\perp.
\end{dmath}
%
We can write
%
\begin{dmath}\label{eqn:condensedMatterLecture9:260}
dq_\perp = \frac{d\omega}{\Abs{\spacegrad_\Bq \omega(\Bq) }},
\end{dmath}
%
or
%
\begin{dmath}\label{eqn:condensedMatterLecture9:280}
\frac{V}{(2\pi)^3 } \int d^3\Bq
\rightarrow
\frac{V}{(2\pi)^3 } \int d f_\omega d q_\perp
\rightarrow
\frac{V}{(2\pi)^3 } \int
%d f_\omega d q_\perp
\frac{df_\omega}{\Abs{\spacegrad_\Bq \omega(\Bq) }} d \omega
=
\int Z(\omega) d\omega.
\end{dmath}
%
where \(Z(\omega)\) is the \textAndIndex{density of states}, the number of modes per unit energy.

\examhint{He'll expect us to look at a phonon diagram and see where the density of states is high or low.}

\makeexample{1D diatomic chain}{example:condensedMatterLecture9:1}{
%\cref{fig:qmSolidsL9:qmSolidsL9Fig5}.
\imageFigure{../figures/phy487-qmsolids/qmSolidsL9Fig5}{frequency distribution and density of states for diatomic chain.}{fig:qmSolidsL9:qmSolidsL9Fig5}{0.15}

%\cref{fig:qmSolidsL9:qmSolidsL9Fig6}.
\imageFigure{../figures/phy487-qmsolids/qmSolidsL9Fig6}{density of states.}{fig:qmSolidsL9:qmSolidsL9Fig6}{0.15}
}

\makeexample{3D solid}{example:condensedMatterLecture9:2}{
%\cref{fig:qmSolidsL9:qmSolidsL9Fig7}.
\imageFigure{../figures/phy487-qmsolids/qmSolidsL9Fig7}{3D solid frequency distribution and density of states.}{fig:qmSolidsL9:qmSolidsL9Fig7}{0.15}
%\cref{fig:qmSolidsL9:qmSolidsL9Fig8}.
\imageFigure{../figures/phy487-qmsolids/qmSolidsL9Fig8}{3D solid density of states.}{fig:qmSolidsL9:qmSolidsL9Fig8}{0.15}
}
%
\section{Isotropic model (Debye).}
\index{isotropic model}
\index{Debye model}

With a 1 atom basis (for now), we have only acoustic modes
%
\begin{dmath}\label{eqn:condensedMatterLecture9:300}
\omega(\Bq) \rightarrow \omega(q).
\end{dmath}
%
same in all directions, so that the constant energy surfaces are spheres, as in \cref{fig:qmSolidsL9:qmSolidsL9Fig9}.
%
\imageFigure{../figures/phy487-qmsolids/qmSolidsL9Fig9}{Debye surface.}{fig:qmSolidsL9:qmSolidsL9Fig9}{0.15}
%
\begin{dmath}\label{eqn:condensedMatterLecture9:320}
\spacegrad_\Bq \omega(q) = \frac{d\omega}{dq} \qcap,
\end{dmath}
%
at small \(q\), the frequency for this 1 atom basis is
%
\begin{dmath}\label{eqn:condensedMatterLecture9:340}
\omega =
\left\{
\begin{array}{l l}
C_{\txtL} q & \quad \mbox{longitudinal acoustic} \\
C_{\txtT} q & \quad \mbox{transverse acoustic (two of these)}
\end{array}
\right.
\end{dmath}
%
This gives
%
\begin{dmath}\label{eqn:condensedMatterLecture9:360}
\int Z(\omega) d\omega
=
\sum_{LA, TA} \frac{V}{(2\pi)^3} \int \frac{d f_\omega }{\Abs{ \spacegrad_\Bq \omega(\Bq)}} d\omega
=
\frac{V}{(2\pi)^3}
\int
\sum_{LA, TA}
\frac{
\mathLabelBox
[
   labelstyle={xshift=2cm},
   linestyle={out=270,in=90, latex-}
]
{
d f_\omega
}
{\(q\) space surface area element}
}{\frac{d\omega}{dq}}
d\omega
=
\int
\frac{V}{(2\pi)^3}
\lr{
\inv{C_{\txtL}} + \frac{2}{C_{\txtT}}
}
\mathLabelBox
[
   labelstyle={xshift=2cm},
   linestyle={out=270,in=90, latex-}
]
{
4 \pi q^2
}
{\(= \int df_\omega\)}
d\omega
=
\int
\frac{V}{2\pi^2}
\lr{
\frac{q^2}{C_{\txtL}} + \frac{2 q^2}{C_{\txtT}}
}
d \omega
=
\int
\frac{V}{2\pi^2}
\lr{
\frac{1}{C_{\txtL}^3} + \frac{2}{C_{\txtT}^3}
}
\omega^2 d \omega,
\end{dmath}
%
Here we sum over the two transverse and the single longitudinal state, and refer back to \eqnref{eqn:condensedMatterLecture9:340} introduce the \(C_{\txtL}\) and \(C_{\txtT}\) factors.

Because of \(4 \pi q^2\) phase space factor, we have
%
\begin{dmath}\label{eqn:condensedMatterLecture9:420}
Z(\omega) \propto \omega^2.
\end{dmath}
%
\paragraph{Debye approximation}
\index{Debye model}

Define the Debye frequency \(\omega_{\txtD}\) \index{Debye frequency} by
%
\begin{dmath}\label{eqn:condensedMatterLecture9:380}
\int_0^{\omega_{\txtD}} Z(\omega) d\omega = 3 r N.
\end{dmath}
%
%\cref{fig:qmSolidsL9:qmSolidsL9Fig10}.
\imageFigure{../figures/phy487-qmsolids/qmSolidsL9Fig10}{Debye approximation.}{fig:qmSolidsL9:qmSolidsL9Fig10}{0.15}

%\cref{fig:qmSolidsL9:qmSolidsL9Fig11}.
\imageFigure{../figures/phy487-qmsolids/qmSolidsL9Fig11}{Linear Debye approximation.}{fig:qmSolidsL9:qmSolidsL9Fig11}{0.15}

and pretend we can cut off in a way that applies to all \(q\).  We get
%
\begin{dmath}\label{eqn:condensedMatterLecture9:400}
3 r N
=
\frac{V}{ 2 \pi^2} \lr{ \inv{C_{\txtL}^3} + \frac{ 2}{C_{\txtT}^3} } \int_0^{\omega_{\txtD}} \omega^2 d\omega
=
\frac{V}{ 2 \pi^2} \lr{ \inv{C_{\txtL}^3} + \frac{ 2}{C_{\txtT}^3} } \inv{3} \omega_{\txtD}^3,
\end{dmath}
%
or
\boxedEquation{eqn:condensedMatterLecture9:440}{
\frac{V}{ 2 \pi^2} \lr{ \inv{C_{\txtL}^3} + \frac{ 2}{C_{\txtT}^3} } \omega_{\txtD}^3 = 9 r N.
}
%\EndArticle
