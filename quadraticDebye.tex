%
% Copyright � 2013 Peeter Joot.  All Rights Reserved.
% Licenced as described in the file LICENSE under the root directory of this GIT repository.
%
%\input{../blogpost.tex}
%\renewcommand{\basename}{quadraticDebye}
%\renewcommand{\dirname}{notes/phy487/}
%\newcommand{\keywords}{Condensed matter physics, PHY487H1F, Debye temperature, Debye frequency, density of states, phonon}
%\input{../peeter_prologue_print2.tex}
%
%%\citep{harald2003solid} \S x.y
%%\citep{ibach2009solid} \S x.y
%%\reading \citep{ashcroft1976solid} \chaptext N.
%
%\usepackage{mhchem}
%\usepackage[version=3]{mhchem}
%\usepackage{units}
%\newcommand{\nought}[0]{\circ}
%%\newcommand{\EF}[0]{\epsilon_{\txtF}}
%\newcommand{\EF}[0]{E_{\txtF}}
%\newcommand{\kF}[0]{k_{\txtF}}
%\newcommand{\CV}[0]{C_{\txtV}}
%
%\beginArtNoToc
%\generatetitle{Quadratic Debye}
%\chapter{Quadratic Deybe}
\label{chap:quadraticDebye}
\makeoproblem{Quadratic Debye phonons.}{pr:quadraticDebye:1}{2013 midterm pr B2}{
\index{Deybe model}
Assume a quadratic dispersion relation for the longitudinal and transverse modes
%
\begin{equation}\label{eqn:quadraticDebye:10}
\omega =
\left\{
\begin{array}{l}
b_{\txtL} q^2 \\
b_{\txtT} q^2
\end{array}
\right.
.
\end{equation}
%
\makesubproblem{}{pr:quadraticDebye:1:a}
Find the \textAndIndex{density of states}.
\makesubproblem{}{pr:quadraticDebye:1:b}
Find the \textAndIndex{Debye frequency}.
\makesubproblem{}{pr:quadraticDebye:1:c}
In terms of \(\kB \Theta = \Hbar \omega_{\txtD}\), and
%
\begin{equation}\label{eqn:quadraticDebye:11}
\calI = \int_0^\infty \frac{y^{5/2} e^{y} dy}{
\lr{ e^y - 1}^2
 },
\end{equation}
%
find the \textAndIndex{specific heat} for \(\kB T \ll \Hbar \omega_{\txtD}\).
\makesubproblem{}{pr:quadraticDebye:1:d}
Find the specific heat for \(\kB T \gg \Hbar \omega_{\txtD}\).
} % makeoproblem
\makeanswer{pr:quadraticDebye:1}{
\makeSubAnswer{}{pr:quadraticDebye:1:a}
Working straight from the definition
%
\begin{equation}\label{eqn:quadraticDebye:30}
\begin{aligned}
Z(\omega)
&= \frac{V}{(2 \pi)^3 } \sum_{L, T} \int \frac{df_\omega}{ \Abs{ \spacegrad_\Bq \omega } }
\\ &= \frac{V}{(2 \pi)^3 }
\lr{ \evalbar{\frac{4 \pi q^2}{2 b_{\txtL} q} }{\txtL} + \evalbar{\frac{2 \times 4 \pi q^2}{2 b_{\txtT} q} }{\txtT} }
\\ &= \frac{V}{4 \pi^2 }
\lr{ \frac{q_{\txtL}}{b_{\txtL}} + \frac{2 q_{\txtT}}{b_{\txtT}} }.
\end{aligned}
\end{equation}
%
With \(q_{\txtL} = \sqrt{\omega/b_{\txtL}}\) and \(q_{\txtT} = \sqrt{\omega/b_{\txtT}}\), this is
%
\begin{equation}\label{eqn:quadraticDebye:50}
Z(\omega)
= \frac{V}{4 \pi^2 }
\lr{ \frac{1}{b_{\txtL}^{3/2}} + \frac{2}{b_{\txtT}^{3/2}} }
\sqrt{\omega}.
\end{equation}
%
\makeSubAnswer{}{pr:quadraticDebye:1:b}
The Debye frequency was given implicitly by
%
\begin{equation}\label{eqn:quadraticDebye:70}
\int_0^{\omega_{\txtD}} Z(\omega) d\omega = 3 r N,
\end{equation}
%
which gives
%
\begin{equation}\label{eqn:quadraticDebye:90}
\begin{aligned}
3 r N
&=
\frac{2}{3} \frac{V}{4 \pi^2 }
\lr{ \frac{1}{b_{\txtL}^{3/2}} + \frac{2}{b_{\txtT}^{3/2}} }
\omega_{\txtD}^{3/2}
\\ &=
\frac{V}{6 \pi^2 }
\lr{ \frac{1}{b_{\txtL}^{3/2}} + \frac{2}{b_{\txtT}^{3/2}} }
\omega_{\txtD}^{3/2}.
\end{aligned}
\end{equation}
%
\makeSubAnswer{}{pr:quadraticDebye:1:c}
Assuming a Bose distribution and ignoring the zero point energy, which has no temperature dependence, the specific heat, the temperature derivative of the energy density, is
%
\begin{equation}\label{eqn:quadraticDebye:110}
\begin{aligned}
\CV
&= \frac{d}{d T} \inv{V} \int Z(\omega) \frac{\Hbar \omega}{ e^{\Hbar \omega/ \kB T } - 1} d\omega
\\ &= \inv{V} \frac{d}{d T} \int Z(\omega) \frac{\Hbar \omega}{ \Hbar \omega/ \kB T + \inv{2}( \Hbar \omega/\kB T)^2 + \cdots } d\omega \\
&\approx \inv{V} \frac{d}{d T} \int Z(\omega) \kB T d\omega
\\ &= \inv{V} \kB 3 r N.
\end{aligned}
\end{equation}
%
\makeSubAnswer{}{pr:quadraticDebye:1:d}
First note that the density of states can be written
%
\begin{equation}\label{eqn:quadraticDebye:130}
Z(\omega) =
\frac{9 r N}{ 2 \omega_{\txtD}^{3/2} } \omega^{1/2},
\end{equation}
%
for a specific heat of
%
\begin{equation}\label{eqn:quadraticDebye:150}
\begin{aligned}
\CV
&= \frac{d}{d T} \inv{V} \int_0^\infty \frac{9 r N}{ 2 \omega_{\txtD}^{3/2} } \omega^{1/2} \frac{\Hbar \omega}{ e^{\Hbar \omega/ \kB T } - 1} d\omega \\
&=
\frac{9 r N}{ 2 V \omega_{\txtD}^{3/2} }
\int_0^\infty
d\omega \omega^{1/2}
\frac{d}{d T}
\frac{\Hbar \omega}{ e^{\Hbar \omega/ \kB T } - 1} \\
&=
\frac{9 r N}{ 2 V \omega_{\txtD}^{3/2} }
\int_0^\infty
d\omega \omega^{1/2}
\frac{-\Hbar \omega}{
\lr{e^{\Hbar \omega/ \kB T } - 1}^2
}  e^{\Hbar \omega/\kB T} \Hbar \omega/\kB
\lr{-\inv{T^2}} \\
&=
\frac{9 r N \kB }{ 2 V \omega_{\txtD}^{3/2} }
\lr{ \frac{ \kB T}{\Hbar} }
^{3/2} \\
&\qquad
\int_0^\infty
d \frac{\Hbar \omega}{\kB T}
\lr{\frac{\Hbar \omega}{\kB T}}
^{1/2}
\frac{1}{
\lr{e^{\Hbar \omega/ \kB T } - 1}^2
}  e^{\Hbar \omega/\kB T}
\lr{ \frac{\Hbar \omega}{\kB T} }^2 \\
&=
\frac{9 r N \kB }{ 2 V \omega_{\txtD}^{3/2} }
\lr{ \frac{ \kB T}{\Hbar} }
^{3/2}
\int_0^\infty dy \frac{y^{5/2} e^y }{
\lr{e^y - 1}^2
 } \\
&=
\frac{9 r N \kB }{ 2 V }
\lr{ \frac{ T}{\Theta} }
^{3/2} \calI.
\end{aligned}
\end{equation}
} % makeanswer

%\EndNoBibArticle
