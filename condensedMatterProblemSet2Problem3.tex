%
% Copyright � 2013 Peeter Joot.  All Rights Reserved.
% Licenced as described in the file LICENSE under the root directory of this GIT repository.
%
\makeoproblem{Lattice vectors, primitive vs. conventional unit cell}{condensedMatter:problemSet2:3}{2013 ps2 p3}{
A crystal has a basis of one atom per lattice point and
a set of primitive translation vectors is
%
\begin{eqnarray*}
\Ba = 3\xcap \qquad \Bb = 3\ycap \qquad
\Bc = 1.5(\xcap + \ycap + \zcap),
\end{eqnarray*}
%
where \(\xcap\), \(\ycap\) and \(\zcap\) are the
unit vectors in the x,y and z directions.

Assume that the dimensions are Angstroms, \AA.

What is the lattice type of this crystal, what is the
volume of the primitive unit cell, and what is the volume of
the conventional unit cell?
} % makeproblem
\makeanswer{condensedMatter:problemSet2:3}{
First observe that the integer multiple span of vectors \(\Ba\) and \(\Bb\) cover all the grid points separated by 3 units (\AA).  Also observe that the combination
%
\begin{dmath}\label{eqn:condensedMatterProblemSet2Problem3:20}
2 \Bc - \Ba - \Bb = 3 \zcap,
\end{dmath}
%
means that we have coverage of all cubic grid points separated by 3 units, and will have the same grid coverage from any plane that we can reach.  Finally, note that moving \(\Bc\) from the origin takes us to the center of the width three cubic lattice in the first quadrant.  This means that we have cubic coverage with all centers filled.

This is a bcc lattice, and is plotted in \cref{fig:problemSet2Problem3Visualization:problemSet2Problem3VisualizationFig1}.
% \nbref{problemSet2Problem3Visualization.nb}
%
\imageFigure{../figures/phy487-qmsolids/problemSet2Problem3VisualizationFig1}{Bcc lattice}{fig:problemSet2Problem3Visualization:problemSet2Problem3VisualizationFig1}{0.2}

The primitive unit cell volume is that of the parallelepiped formed by the vectors \(\Ba, \Bb, \Bc\), which is
%
\begin{dmath}\label{eqn:condensedMatterProblemSet2Problem3:40}
V_{\mathrm{primitive}}
= 3 \times 3 \times \frac{3}{2}
\begin{vmatrix}
1 & 0 & 0 \\
0 & 1 & 0 \\
1 & 1 & 1
\end{vmatrix}
\angstrom^3
= \frac{27}{2} \angstrom^3
\end{dmath}
%
The conventional unit cell is depicted in \cref{fig:bccPacking:bccPackingFig1}, and has volume
%
\begin{dmath}\label{eqn:condensedMatterProblemSet2Problem3:60}
V_{\mathrm{conventional}}
=
27 \angstrom^3,
\end{dmath}
%
which is twice the primitive cell volume in this case.
}
