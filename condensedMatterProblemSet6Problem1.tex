%
% Copyright � 2013 Peeter Joot.  All Rights Reserved.
% Licenced as described in the file LICENSE under the root directory of this GIT repository.
%
\makeoproblem{Fermi wave-vector, energy, Fermi temperature, etc.}{condensedMatter:problemSet6:1}{2013 ps6 p1}{

\makesubproblem{}{condensedMatter:problemSet6:1a}
Treating potassium
as a free electron metal with one conduction electron per potassium,
calculate the theoretical value of the Fermi wave-vector \(\kF\),
the Fermi energy \(\EF\) and the Fermi temperature \(\TF\).

\makesubproblem{}{condensedMatter:problemSet6:1b}
Using \cref{fig:potassium_CofT:potassium_CofTFig1} below, determine the experimental value of the
Fermi temperature \(\TF\) of potassium,
and compare it with the theoretical free-electron Fermi temperature.
(Ignore the break in the \(y\)-axis, that is, only use the numbers at the
bottom of the \(y\)-axis.)

\imageFigure{../../figures/phy487/potassium_CofTFig1}{Potassium specific heat temperature dependence}{fig:potassium_CofT:potassium_CofTFig1}{0.3}
%\vspace{-0.5cm}
%
%\begin{center}
%\includegraphics[width=8cm]{./figures/potassium_CofT.pdf}
%\end{center}
%
%\vspace{-0.5cm}

\paragraph{Potentially useful facts:}  The density of potassium is 0.862 g/cm\(^3\); the atomic mass of potassium is 39.10; the atomic mass unit is \(u = 1.66\times 10^{-27}\) kg.)

} % makeproblem

\makeanswer{condensedMatter:problemSet6:1}{

\makeSubAnswer{}{condensedMatter:problemSet6:1a}

Using the following values

\begin{dmath}\label{eqn:condensedMatterProblemSet6Problem1:20}
\begin{aligned}
\text{1 amu} &= 1.66 \times 10^{-24} \Unit{g} \\
\rho_K &= 0.86 \Unitfrac{g}{{cm}^3} \\
\text{Atomic mass of K} &= 39.0983 \\
m_e &= 9.109 \times 10^{-28} \Unit{g} \\
\Hbar &= 1.055 \times 10^{-27} \Unit{ergs} \Unit{s} \\
\kB &= 1.381 \times 10^{-16} \Unitfrac{ergs}{K},
\end{aligned}
\end{dmath}

we find

\begin{dmath}\label{eqn:condensedMatterProblemSet6Problem1:60}
m_K = \text{the mass of 1 Potassium atom} = \text{atomic mass of K \(\times\) 1 amu} = 6.49032 \times 10^{-23} \Unit{g}
\end{dmath}
\begin{dmath}\label{eqn:condensedMatterProblemSet6Problem1:80}
n_K = \rho_K / m_K = n_e (\mbox{with presumption of one free electron per atom}) = 1.32505 \times 10^{22} \Unitinv{{cm}^3}.
\end{dmath}

This gives

\begin{dmath}\label{eqn:condensedMatterProblemSet6Problem1:40}
\begin{array}{l l l}
\kF &= \lr{ 3 \pi^2 n_e }^{1/3} &= 7.32068 \times 10^7 \Unitinv{cm} \\
\EF &= \Hbar^2 \kF^2/(2 m_e) &= 3.27421 \times 10^{-12} \Unit{ergs} \\
\TF &= \EF/\kB &= 23709 \Unit{K}
\end{array}
\end{dmath}

\paragraph{Grading remark:} The grader didn't like my use of ergs as a unit, and said I should use either \(\Unit{eV}\) or \(\Unit{J}\).  Asking Prof Julian, if SI now dominates, he confirms that this was a pretty old school thing to do ``Yes, it's a long time since I have seen an erg.''  We both recall the horror of taking introductory electromagnetism with the CGS Berkley physics series while our Professors used SI.

\makeSubAnswer{}{condensedMatter:problemSet6:1b}

Looking at the graph, we see that the units are different from what we have been using.  The dimensions of specific heat \([ \CV ] = \frac{[ U ]}{ \Theta }\) are those of energy density per unit temperature, or energy per unit volume per unit temperature.  The graph provides the specific heat per mole, not per unit volume, so we have to convert by multiplying by the volume per mole \(\NA/n\) where \(\NA\) is Avagadro's number and \(n\) is the number density.

\begin{dmath}\label{eqn:condensedMatterProblemSet6Problem1:100}
C
= \CV \frac{\NA}{n}
= \frac{\pi^2}{2} \frac{ \kB n T}{\TF} \frac{\NA}{n}
= \frac{\pi^2}{2} \frac{ \kB \NA }{\TF} T
\end{dmath}

Our graphed relation is

\begin{dmath}\label{eqn:condensedMatterProblemSet6Problem1:120}
\frac{C}{T}
=
\gamma + \beta T^2
\approx
\lr{
2.08 + \frac{0.85}{0.3} T^2}
 \Unitfrac{mJ}{mole\, deg^2}
\end{dmath}

With \(\NA = 6.022 \times 10^{23}\), plugging in the numbers yields

\begin{dmath}\label{eqn:condensedMatterProblemSet6Problem1:140}
{\TF}_{\text{experimental}}
=
\frac{\pi^2 \kB \NA}{2 \times 2.08 \Unitfrac{mJ}{K^2} }
=
\frac{\pi^2
1.381 \times 10^{-23} \Unitfrac{J}{K}
\times 6.022 \times 10^{23}
\times 10^3}{2 \times 2.08 \Unitfrac{J}{K^2} }
= 19730.6 \Unit{K}.
\end{dmath}

The ratio of theoretical to experimental is

\begin{dmath}\label{eqn:condensedMatterProblemSet6Problem1:160}
\frac{\TF}{
{\TF}_{\text{experimental} }
}
=
1.20163.
\end{dmath}

The numerical calculations for this problem can also be found in \nbref{problemSet6Problem1.nb}

\paragraph{Grading remark:} ``What do you think is the reason for the discrepency?''.  Lost half a mark for that.
}
