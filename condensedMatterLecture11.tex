%
% Copyright � 2013 Peeter Joot.  All Rights Reserved.
% Licenced as described in the file LICENSE under the root directory of this GIT repository.
%
%\input{../blogpost.tex}
%\renewcommand{\basename}{condensedMatterLecture11}
%\renewcommand{\dirname}{notes/phy487/}
%\newcommand{\keywords}{Condensed matter physics, PHY487H1F}
%\input{../peeter_prologue_print2.tex}
%
%%\citep{harald2003solid} \S x.y
%
%%\usepackage{mhchem}
%\usepackage[version=3]{mhchem}
%\newcommand{\nought}[0]{\circ}
%\newcommand{\kF}[0]{k_{\txtF}}
%
%\beginArtNoToc
%\generatetitle{PHY487H1F Condensed Matter Physics.  Lecture 11: Free electron model of metals.  Taught by Prof.\ Young-June Kim}
%%\chapter{Free electron model of metals}
%\label{chap:condensedMatterLecture11}
%
%\section{Disclaimer}
%
%Peeter's lecture notes from class.  May not be entirely coherent.
%
\section{Free electron model of metals.}
\index{free electron}

\reading \citep{ibach2009solid} \textchapref{6}.

We will treat electrons as a gas of free Fermions, a statistical mechanical approach.

Some characteristics of metals that we intuitively understand (i.e. should you ask a 3rd grader)

\begin{itemize}
\item Shiny.
\item Cold to the touch in cold, or hot if warmed.
\item Can bend and hammer it (malleability).
\item Can use as a wire (conduct electricity)
\end{itemize}

Metallic bonding, the sharing of valence electrons with many neighbours, is the fundamental property that we rely on to make our free electron gas model.  We can justify this by reflecting on the overlap of the \(d\) orbital electrons in a metal.  These often have a radial wave function magnitude as illustrated in \cref{fig:qmSolidsL11:qmSolidsL11Fig1}.
\imageFigure{../figures/phy487-qmsolids/qmSolidsL11Fig1}{Rough sketch of d orbital radial magnitude.}{fig:qmSolidsL11:qmSolidsL11Fig1}{0.15}

We can justify our treatment of electrons as a gas by noting that we have significant overlap of these orbitals over many sites in the metallic lattice as in \cref{fig:qmSolidsL11:qmSolidsL11Fig2}.
\imageFigure{../figures/phy487-qmsolids/qmSolidsL11Fig2}{Overlapping d orbital wavefunctions.}{fig:qmSolidsL11:qmSolidsL11Fig2}{0.15}

\begin{itemize}
\item electrons must be treated quantum mechanically
\item periodic lattice complicates things.  We'll attempt to ignore that.
\end{itemize}

We try the simple \underlineAndIndex{jellium model}, where we ``smear'' out the positive ions to a uniform background as in the typical 2nd year particle in an infinite potential box problem, as illustrated in \cref{fig:qmSolidsL11:qmSolidsL11Fig3}.
\imageFigure{../figures/phy487-qmsolids/qmSolidsL11Fig3}{One dimensional particle in a box.}{fig:qmSolidsL11:qmSolidsL11Fig3}{0.15}

where we had solutions like those of \cref{fig:qmSolidsL11:qmSolidsL11Fig4}.
\imageFigure{../figures/phy487-qmsolids/qmSolidsL11Fig4}{First few solutions for particle in a 1D box.}{fig:qmSolidsL11:qmSolidsL11Fig4}{0.15}

The difference is that we will consider the 3D generalization of this problem, where the potentials are infinite outside of a cubic space as in \cref{fig:qmSolidsL11:qmSolidsL11Fig5}.
\imageFigure{../figures/phy487-qmsolids/qmSolidsL11Fig5}{3D particle in a box.}{fig:qmSolidsL11:qmSolidsL11Fig5}{0.15}

The free electron in an infinite 3D

solve for 1 electron
%
\begin{dmath}\label{eqn:condensedMatterLecture11:20}
-\frac{\Hbar^2}{2 m} \spacegrad^2 \Psi(\Br) + V(\Br) \Psi(\Br) = E \Psi(\Br),
\end{dmath}
%
where
%
\begin{dmath}\label{eqn:condensedMatterLecture11:40}
V(\Br) =
\left\{
\begin{array}{l l}
V_\nought & \quad \mbox{inside cube} \\
\infty & \quad \mbox{outside}
\end{array}
\right.
\end{dmath}
%
This implies a boundary condition \(\Psi(\Br) = 0\) outside of the box.  Using separation of variables
%
\begin{dmath}\label{eqn:condensedMatterLecture11:60}
-\frac{\Hbar^2}{2 m} \lr{
\PDSq{x}{}
+\PDSq{y}{}
+\PDSq{z}{}
}
\Psi_x(x)
\Psi_y(y)
\Psi_z(z)
=
\mathLabelBox{E}{\(E = -E' - V_\nought\)}
\Psi_x(x)
\Psi_y(y)
\Psi_z(z),
\end{dmath}
%
i.e. \(\Psi(\Br)\) is separable.

This is a set of equations of the form
%
\begin{dmath}\label{eqn:condensedMatterLecture11:80}
-\frac{\Hbar^2}{2 m}
\PDSq{x}{\Psi_x(x)}
=
E_x
\Psi_x(x),
\end{dmath}
%
where
%
\begin{dmath}\label{eqn:condensedMatterLecture11:100}
\Psi_x(0) = \Psi_x(L) = 0.
\end{dmath}
%
Our solution is
%
\begin{dmath}\label{eqn:condensedMatterLecture11:120}
\Psi(x) = N \sin( k_x x ),
\end{dmath}
%
where
%
\begin{dmath}\label{eqn:condensedMatterLecture11:140}
k_x = \frac{ n_x \pi}{L}.
\end{dmath}
%
In 3D this is
%
\begin{dmath}\label{eqn:condensedMatterLecture11:160}
\Psi(\Br) =
\lr{\frac{2}{L}}^{3/2}
\sin( k_x x )
\sin( k_y y )
\sin( k_z z ),
\end{dmath}
%
where
\begin{dmath}\label{eqn:condensedMatterLecture11:180}
E(\Bk)
= \frac{\Hbar}{2m} \lr{
k_x^2
+k_y^2
+k_z^2 }
= \frac{\Hbar \Bk^2}{2m}.
\end{dmath}
%
This is a parabolic dispersion, sketched in \cref{fig:qmSolidsL11:qmSolidsL11Fig7}.
\imageFigure{../figures/phy487-qmsolids/qmSolidsL11Fig7}{Parabolic energy dispersion.}{fig:qmSolidsL11:qmSolidsL11Fig7}{0.15}

This is
\begin{equation}\label{eqn:condensedMatterLecture11:200}
\begin{aligned}
k_x &= \frac{n_x \pi}{L} \\
k_y &= \frac{n_y \pi}{L} \\
k_z &= \frac{n_z \pi}{L}.
\end{aligned}
\end{equation}
%
We'll one to consider the volume per k point, as roughly sketched in \cref{fig:qmSolidsL11:qmSolidsL11Fig6}.
\imageFigure{../figures/phy487-qmsolids/qmSolidsL11Fig6}{Volume per k point.}{fig:qmSolidsL11:qmSolidsL11Fig6}{0.15}
%
\begin{dmath}\label{eqn:condensedMatterLecture11:220}
\lr{\frac{\pi}{L}}^3 = \frac{\pi^3}{V}.
\end{dmath}
%
here we switch notations and use \(V\) for volume, and not \(V\) as potential.
%
\paragraph{Density of states}
\index{density of states}
As with the phonon, properties are calculated via k space sums using the density of states.
%
\begin{dmath}\label{eqn:condensedMatterLecture11:240}
\sum_{k_x, k_y, k_z, \mathrm{spin}}
\rightarrow
\sum_{\mathrm{spin}}
\int \frac{d^3 \Bk}{ \lr{\pi/L}^3 }
=
\mathLabelBox
[
   labelstyle={below of=m\themathLableNode, below of=m\themathLableNode}
]
{2}{spin, 2 states per k point}
\frac{V}{\pi^3}
\int
\mathLabelBox{\inv{8}}{1 octant}
4 \pi k^2 dk
=
\frac{V}{\pi^2} \int k^2 dk,
\end{dmath}
where \( \rightarrow \) represents a quasi continuous approximation.
%
Now turn this into \(\int dE\) as with phonons, using
%
\begin{dmath}\label{eqn:condensedMatterLecture11:260}
E = \frac{\Hbar^2 k^2}{2m},
\end{dmath}
%
so that
\begin{dmath}\label{eqn:condensedMatterLecture11:280}
dE = \frac{\Hbar^2 k}{m} dk.
\end{dmath}
%
This gives
%
\begin{dmath}\label{eqn:condensedMatterLecture11:300}
k^2 dk =
\sqrt{\frac{ 2 m E}{\Hbar^2}} \frac{m}{\Hbar^2} dE
=
\inv{2} \lr{ \frac{2m}{\Hbar^2} }^{3/2} \sqrt{E} dE,
\end{dmath}
%
so that
\begin{dmath}\label{eqn:condensedMatterLecture11:320}
\sum_{k_x, k_y, k_z, \mathrm{spin}}
\rightarrow
\frac{V}{2 \pi^2} \lr{ \frac{2m}{\Hbar^2} }^{3/2} \int \sqrt{E} dE.
\equiv V \int D(E) dE.
\end{dmath}
%
The \textAndIndex{density of states} per volume
%
\begin{dmath}\label{eqn:condensedMatterLecture11:340}
D(E) =
\frac{1}{2 \pi^2} \lr{ \frac{2m}{\Hbar^2} }^{3/2} \sqrt{E}.
\end{dmath}
%
This is sketched in \cref{fig:qmSolidsL11:qmSolidsL11Fig8}.
\imageFigure{../figures/phy487-qmsolids/qmSolidsL11Fig8}{3D density of states.}{fig:qmSolidsL11:qmSolidsL11Fig8}{0.15}

%This is in fact the same type of expression that we get for the density of a gas.

Reading: \S 6.2, Fermi gas at \(T = 0 K\).

This assumes that electrons \textunderline{do not} interact.  This is the operational definition of an ideal gas.

This may seem more drastic than it actually is.  Electrons in plane wave states have constant density.  This shifts potential, but is still uniform.

Note that this does break down for some interesting problems.  One notable such problem is that of superconductivity, where we must consider the electron electron interactions.

Fermi energy states can only be occupied by one particle with each spin, as illustrated in \cref{fig:qmSolidsL11:qmSolidsL11Fig9}.
\imageFigure{../figures/phy487-qmsolids/qmSolidsL11Fig9}{Spin packing of energy levels for 1D particle in a box.}{fig:qmSolidsL11:qmSolidsL11Fig9}{0.15}

Because we can only pack states two per point in k space, we'll consider a Fermi surface in k space as in \cref{fig:qmSolidsL11:qmSolidsL11Fig10},
\imageFigure{../figures/phy487-qmsolids/qmSolidsL11Fig10}{Fermi surface in momentum space.}{fig:qmSolidsL11:qmSolidsL11Fig10}{0.15}
and determine the energy level \(\EF\) \index{Fermi energy} and momentum \(\kF\) \index{Fermi wavevector} associated with that surface, for which no more states can be occupied.  This is roughly illustrated in \cref{fig:qmSolidsL11:qmSolidsL11Fig11}.
\imageFigure{../figures/phy487-qmsolids/qmSolidsL11Fig11}{Fermi energy and momentum.}{fig:qmSolidsL11:qmSolidsL11Fig11}{0.15}
In quasi-continuum, approximately \(1/8\) of the occupied state.

\(\kF\) is the boundary between occupied and unoccupied states (Fermi wave vector).

The maximum number of these occupied states is then
%
\begin{equation}\label{eqn:condensedMatterLecture11:360}
N
= 2 \frac{ \frac{4}{3} \pi \kF^3 }{ 8 \lr{\frac{\pi}{L}}^3 }
= \frac{V}{3 \pi^2} \kF^3.
\end{equation}
%
so that
%
\begin{dmath}\label{eqn:condensedMatterLecture11:380}
\kF = \lr{ 3 \pi^2 n}^{1/3},
\end{dmath}
%
where
%
\begin{dmath}\label{eqn:condensedMatterLecture11:400}
n \equiv \frac{N}{V}.
\end{dmath}
%
This is typically of the order \(10^{10} m^{-1} = \angstrom^{-1}\).

For reference, we get for \(\EF\)
%
\begin{dmath}\label{eqn:condensedMatterLecture11:380b}
\EF
= \frac{ \Hbar^2 \kF^2 }{2m}
= \frac{ \Hbar^2}{2m} \lr{ 3 \pi^2 n}^{2/3}.
\end{dmath}
%
%\EndArticle
