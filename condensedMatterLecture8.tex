%
% Copyright � 2013 Peeter Joot.  All Rights Reserved.
% Licenced as described in the file LICENSE under the root directory of this GIT repository.
%
%\input{../blogpost.tex}
%\renewcommand{\basename}{condensedMatterLecture8}
%\renewcommand{\dirname}{notes/phy487/}
%\newcommand{\keywords}{Condensed matter physics, PHY487H1F}
%\input{../peeter_prologue_print2.tex}
%
%%\citep{harald2003solid} \S x.y
%
%% shorthand used when taking notes, replaced after with ad-hoc vim regexes
%%\nai === {n \alpha i}
%%\mbj === {m \beta j}
%%\ai === {\alpha i}
%%\bj === {\beta j}
%%\n1 === {n, 1}
%%\n-12 === {n-1, 2}
%%\n+12 === {n+1, 2}
%
%%\usepackage{mhchem}
%\usepackage[version=3]{mhchem}
%
%\beginArtNoToc
%\generatetitle{PHY487H1F Condensed Matter Physics.  Lecture 8: Phonons (cont.).  Taught by Prof.\ Stephen Julian}
%\chapter{Phonons (cont.)}
\label{chap:condensedMatterLecture8}

%\section{Disclaimer}
%
%Peeter's lecture notes from class.  May not be entirely coherent.
%
%\section{Phonons (cont.)}
\paragraph{normal modes}
\index{normal modes}
\index{phonon}

Last time, considering a 1D \textAndIndex{linear harmonic chain}
%
\begin{subequations}
\begin{dmath}\label{eqn:condensedMatterLecture8:20}
w_q = \sqrt{ \frac{ 4 k }{m } } \Abs{ \sin \frac{ q a }{ 2 } }.
\end{dmath}
\begin{dmath}\label{eqn:condensedMatterLecture8:40}
q = \frac{2 \pi n}{L}
\end{dmath}
\end{subequations}
%
These were described as wave like solutions, but these are in fact the normal modes \index{normal mode} of oscillations.

These are sketched in \cref{fig:qmSolidsL8:qmSolidsL8Fig1}.
%
\imageFigure{../figures/phy487-qmsolids/qmSolidsL8Fig1}{Harmonic oscillator chain normal mode frequencies}{fig:qmSolidsL8:qmSolidsL8Fig1}{0.3}

(a) At \(q = 0\), \cref{fig:qmSolidsL8:qmSolidsL8Fig1a}, we really have uniform translation of the entire chain.
%
\imageFigure{../figures/phy487-qmsolids/qmSolidsL8Fig1a}{Uniform motion}{fig:qmSolidsL8:qmSolidsL8Fig1a}{0.1}

(b) At \(q = a\), \cref{fig:qmSolidsL8:qmSolidsL8Fig1b}, we have displaced, but also uniform translation of the entire chain.
%
\imageFigure{../figures/phy487-qmsolids/qmSolidsL8Fig1b}{Displaced uniform motion}{fig:qmSolidsL8:qmSolidsL8Fig1b}{0.1}

(c) At \(q = a/2\), \cref{fig:qmSolidsL8:qmSolidsL8Fig1c}, we have maximum oscillation.
%
\imageFigure{../figures/phy487-qmsolids/qmSolidsL8Fig1c}{Maximum oscillation}{fig:qmSolidsL8:qmSolidsL8Fig1c}{0.1}

%\section{Real solids, and potential energy.}
\section{3D potentials for real solids}

\reading \citep{ibach2009solid} \S 4.1

Our problems in 3D are mostly notational, where we have the problem of indexing all the particles and their directions of motion.  Our index convention is illustrated in \cref{fig:qmSolidsL8:qmSolidsL8Fig2}.
%
\imageFigure{../figures/phy487-qmsolids/qmSolidsL8Fig2}{Displacement relative to primitive cell origin}{fig:qmSolidsL8:qmSolidsL8Fig2}{0.3}
%
\begin{dmath}\label{eqn:condensedMatterLecture8:380}
u_{n \alpha i}
\end{dmath}
%
for the displacement of the \(\alpha\)th atom in the nth unit cell, in the ith (\(i \in \{x, y, z\}\) direction.

The total potential energy can be written
%
\begin{dmath}\label{eqn:condensedMatterLecture8:400}
\Phi(
\mathLabelBox{r_{n \alpha i}}{equilibrium position}
 +
\mathLabelBox
[
   labelstyle={below of=m\themathLableNode, below of=m\themathLableNode}
]
{u_{n \alpha i}}{displacement}
) =
\Phi(r_{n \alpha i}) + \inv{2} \sum_{{n \alpha i}, {m \beta j}}
\mathLabelBox{\frac{\partial^2 \Phi}{\partial r_{n \alpha i} \partial {m \beta j}} }{\(\Phi^{m \beta j}_{n \alpha i}\)}
u_{n \alpha i} u_{m \beta j},
\end{dmath}
%
or
%
\begin{dmath}\label{eqn:condensedMatterLecture8:400a}
\Phi( r_{n \alpha i} + u_{n \alpha i} ) =
\Phi(r_{n \alpha i})
+ \inv{2}
\sum_{{n \alpha i}, {m \beta j}}
\Phi^{m \beta j}_{n \alpha i}
u_{n \alpha i} u_{m \beta j}.
\end{dmath}
%
\makeexample{1D chain}{example:condensedMatterLecture8:1}{
To illustrate our index convention consider \cref{fig:qmSolidsL8:qmSolidsL8Fig3} for the harmonic oscillator chain we previously treated.
%
\imageFigure{../figures/phy487-qmsolids/qmSolidsL8Fig3}{direction relative indexing example}{fig:qmSolidsL8:qmSolidsL8Fig3}{0.15}
%
\begin{dmath}\label{eqn:condensedMatterLecture8:60}
\Phi = \inv{2} k u_{ix}^2 + \inv{2} k u_{ix}^2 =
\end{dmath}
%
\begin{dmath}\label{eqn:condensedMatterLecture8:80}
I^{nx}_{nx} = 2 k.
\end{dmath}
%
}

\paragraph{Equation of motion}

From this generalized quadradic potential, we form the Lagrangian
%
\begin{dmath}\label{eqn:condensedMatterLecture8:500}
\LL = T - U = \inv{2} \sum_{n \alpha i} M_n \dot{u}_{n \alpha i}^2 -
\inv{2}
\sum_{{n \alpha i}, {m \beta j}}
\Phi^{m \beta j}_{n \alpha i}
u_{n \alpha i} u_{m \beta j},
\end{dmath}
%
The equations of motion follow from the Euler-Lagrange equations
%
\begin{dmath}\label{eqn:condensedMatterLecture8:520}
\ddt{} \PD{\dot{u}_{n \alpha i}}{\LL} = \PD{u_{n \alpha i}}{\LL},
\end{dmath}
%
the generalized equivalent to \(\BF = -\spacegrad \Phi\).  This provides the force on atom \(\alpha\) in unit cell \(n\), in direction \(i\), due to displacement of atom \(\beta\) in cell m in direction \(j\).  That is
%
\begin{dmath}\label{eqn:condensedMatterLecture8:120}
M_\alpha \ddot{u}_{n \alpha i} + \sum_{m \beta j} \Phi^{m \beta j}_{n \alpha i} u_{m \beta j} = 0.
\end{dmath}
%
For example,
%
\begin{dmath}\label{eqn:condensedMatterLecture8:140}
m \ddot{u}_j =
k( u_{j + 1} - u_j )
+ k( u_{j-1} - u_j ).
\end{dmath}
%
Using trial solution
%
\begin{dmath}\label{eqn:condensedMatterLecture8:160}
u_{n \alpha i} = \inv{\sqrt{m_\alpha}} \sum_q u_{\alpha i}(q) e^{-i (\Bq \cdot \Br_n - \omega_q t ) }
\end{dmath}
%
This yields, after operation with \(\sum_{q'} e^{i \Bq' \cdot \Br_n}\) as before, and canceling terms \index{dynamical matrix}
%
\begin{dmath}\label{eqn:condensedMatterLecture8:180}
-\omega_q^2 u_{\alpha i}(q) + \sum_{\beta j}
\mathLabelBox
[
   labelstyle={xshift=2cm},
   linestyle={out=270,in=90, latex-}
]
{
\sum_m \inv{\sqrt{ m_\alpha m_\beta} } \Phi^{m \beta j}_{n \alpha i} e^{-i \Bq \cdot (\Br_m - \Br_n) }
}{\(D^{\beta j}_{\alpha i}\), the Dynamical matrix, independent of \(n\)}
u_{\beta j}(q) = 0,
\end{dmath}
%
or
%
\begin{dmath}\label{eqn:condensedMatterLecture8:200}
-\omega_q^2 u_{\alpha i}(q) + \sum_{\beta j} D^{\beta j}_{\alpha i} u_{\beta j}(q) = 0.
\end{dmath}
%
We want to solve
%
\begin{dmath}\label{eqn:condensedMatterLecture8:220}
\det\lr{
D^{\beta j}_{\alpha i} - \omega_q^2 \BOne
} = 0.
\end{dmath}
%
\makeexample{diatomic linear chain}{example:condensedMatterLecture8:2}{
As a second example consider \cref{fig:qmSolidsL8:qmSolidsL8Fig4} for a diatomic linear chain.  This example can also be found outlined in \citep{ibach2009solid} \S 4.3.
%
\imageFigure{../figures/phy487-qmsolids/qmSolidsL8Fig4}{Diatomic linear chain}{fig:qmSolidsL8:qmSolidsL8Fig4}{0.2}

Our potentials are
%
\begin{subequations}
\begin{dmath}\label{eqn:condensedMatterLecture8:540}
\Phi_{n, 1}
=
\frac{f}{2} \lr{ u_{n, 1} - u_{n-1, 2} }^2
+ \frac{f}{2} \lr{ u_{n, 2} - u_{n, 1} }^2
\end{dmath}
\begin{dmath}\label{eqn:condensedMatterLecture8:560}
\Phi_{n, 2}
=
\frac{f}{2} \lr{ u_{n +1, 1} - u_{n, 2} }^2
+ \frac{f}{2} \lr{ u_{n, 2} - u_{n, 1} }^2,
\end{dmath}
\end{subequations}
%
with partials
%
\begin{subequations}
\begin{dmath}\label{eqn:condensedMatterLecture8:580}
\PD{u_{n, 1}}
{\Phi_{n, 1}}
=
f \lr{ 2 u_{n, 1} - u_{n-1, 2} - u_{n, 2} }
\end{dmath}
\begin{dmath}\label{eqn:condensedMatterLecture8:600}
\PD{u_{n, 2}}
{\Phi_{n, 2}}
=
f \lr{ -u_{n +1, 1} + 2 u_{n, 2} - u_{n, 1} }
\end{dmath}
\end{subequations}
%
In the general notation the force equations are
%
\begin{subequations}
\begin{dmath}\label{eqn:condensedMatterLecture8:240}
M_1 \ddot{u}_{n, 1} + \Phi^{n - 1, 2}_{n, 1} u_{n - 1, 2} + \Phi^{n, 1}_{n, 1} u_{n, 1} + \Phi^{n, 2}_{n, 1} u_{n, 2} = 0
\end{dmath}
\begin{dmath}\label{eqn:condensedMatterLecture8:260}
M_2 \ddot{u}_{n, 2} + \Phi^{n, 1}_{n, 2} u_{n - 1, 2} + \Phi^{n, 2}_{n, 2} u_{n, 2} + \Phi^{n + 1, 1}_{n, 2} u_{n + 1, 1} = 0,
\end{dmath}
\end{subequations}
%
and from the partials and the Euler-Lagrange equations this is
%
\begin{subequations}
\label{eqn:condensedMatterLecture8:280a}
\begin{dmath}\label{eqn:condensedMatterLecture8:280}
M_1 \ddot{u}_{n, 1} + f\lr{ 2 u_{n, 1} - u_{n - 1, 2} - u_{n, 2}} = 0
\end{dmath}
\begin{dmath}\label{eqn:condensedMatterLecture8:300}
M_2 \ddot{u}_{n, 2} + f\lr{ 2 u_{n, 2} - u_{n, 1} - u_{n + 1, 1}} = 0.
\end{dmath}
\end{subequations}
%
We can read off the potential derivatives
%
\begin{subequations}
\begin{equation}\label{eqn:condensedMatterLecture8:320}
\Phi^{n, 1}_{n, 1} = \Phi^{n, 2}_{n, 2} = 2 f
\end{equation}
\begin{equation}\label{eqn:condensedMatterLecture8:340}
\Phi^{n, 2}_{n, 1} = -f.
\end{equation}
\end{subequations}
%
The trial substitution to use (the text calls this an ansatz) is:
%
\begin{dmath}\label{eqn:condensedMatterLecture8:360}
u_{n, \alpha} = \inv{M_\alpha} u_\alpha(q) e^{i (
\mathLabelBox{a n}{\(x_n\)}
- \omega_q t)
}
\end{dmath}
%
Substitution into \eqnref{eqn:condensedMatterLecture8:280a} gives
%
\begin{subequations}
\begin{dmath}\label{eqn:condensedMatterLecture8:420}
\lr{ \frac{2 f}{M_1} - \omega_q^2 } u_1(q) - \frac{f}{\sqrt{M_1 M_2}} \lr{ 1 + e^{-i q a} } u_2(q) = 0
\end{dmath}
\begin{dmath}\label{eqn:condensedMatterLecture8:440}
- \frac{f}{\sqrt{M_1 M_2}} \lr{ 1 + e^{i q a} } u_1(q)
+\lr{ \frac{2 f}{M_2} - \omega_q^2 } u_2(q)
= 0
\end{dmath}
\end{subequations}
%
We thus want to solve
%
\begin{dmath}\label{eqn:condensedMatterLecture8:460}
\begin{vmatrix}
\lr{ \frac{2 f}{M_1} - \omega_q^2 } &- \frac{f}{\sqrt{M_1 M_2}} \lr{ 1 + e^{-i q a} } \\
- \frac{f}{\sqrt{M_1 M_2}} \lr{ 1 + e^{i q a} }
&\lr{ \frac{2 f}{M_2} - \omega_q^2 }
\end{vmatrix}
= 0,
\end{dmath}
%
We find in \nbref{IbachAndLuth4_15_verify.nb} that this has solution
%
\begin{dmath}\label{eqn:condensedMatterLecture8:480}
\omega_q^2 = f\lr{ \inv{M_1} + \inv{M_2} } \pm f \sqrt{
\lr{ \inv{M_1} + \inv{M_2} }^2 - \frac{4}{M_1 M_2} \sin^2 \frac{q a}{2}
}.
\end{dmath}
%
Plotting in \cref{fig:qmSolidsL8:qmSolidsL8Fig5}.
%
\imageFigure{../figures/phy487-qmsolids/qmSolidsL8Fig5}{Optic and acoustic modes}{fig:qmSolidsL8:qmSolidsL8Fig5}{0.3}

There are two solutions for \(q = 0\)

\(\omega^2 = 0\), or \(\omega^2 = 2 f ( 1/M_1 + 1/M_2 )\).

(a)

%\cref{fig:qmSolidsL8:qmSolidsL8Fig6}.
\imageFigure{../figures/phy487-qmsolids/qmSolidsL8Fig6}{Uniform translation}{fig:qmSolidsL8:qmSolidsL8Fig6}{0.1}

(b)

%\cref{fig:qmSolidsL8:qmSolidsL8Fig7}.
\imageFigure{../figures/phy487-qmsolids/qmSolidsL8Fig7}{Pairwise oscillation}{fig:qmSolidsL8:qmSolidsL8Fig7}{0.1}

(c)

%\cref{fig:qmSolidsL8:qmSolidsL8Fig8}.
\imageFigure{../figures/phy487-qmsolids/qmSolidsL8Fig8}{Heavier atoms oscillating}{fig:qmSolidsL8:qmSolidsL8Fig8}{0.1}

(d)

%\cref{fig:qmSolidsL8:qmSolidsL8Fig9}.
\imageFigure{../figures/phy487-qmsolids/qmSolidsL8Fig9}{Lighter atoms oscillating}{fig:qmSolidsL8:qmSolidsL8Fig9}{0.1}

Reading \S 4.3
}

%\EndArticle
