%
% Copyright © 2013 Peeter Joot.  All Rights Reserved.
% Licenced as described in the file LICENSE under the root directory of this GIT repository.
%
%\section{Periodic harmonic oscillators}
\section{Phonons}
\index{Phonons}

\reading \citep{ashcroft1976solid} \textchapref{21, 22}.

%In lecture 7 we did an introductory (classical) calculation that models Phonons
%%% FIXME:TEMP
%  This is not in the text.
Consider a 1D chain of \(N\) atoms, coupled by harmonic springs \index{periodic harmonic oscillator} with periodic boundary conditions.  We suppose that we have \(N \sim 10^{23}\).  This is illustrated in \cref{fig:qmSolidsL7Phonons:qmSolidsL7Fig4}.  Vibrations of the complete structure are called phonon modes \index{phonon mode}.
%
\imageFigure{../figures/phy487-qmsolids/qmSolidsL7Fig4}{Coupled periodic oscillators}{fig:qmSolidsL7Phonons:qmSolidsL7Fig4}{0.3}

With equilibrium positions \(x_j\), and displacement distances from equilibrium of \(u_j\), as in \cref{fig:qmSolidsL7Phonons:qmSolidsL7Fig5}.
%
\imageFigure{../figures/phy487-qmsolids/qmSolidsL7Fig5}{Equilibrium and displacement positions}{fig:qmSolidsL7Phonons:qmSolidsL7Fig5}{0.15}

Our force balance is
%
\begin{dmath}\label{eqn:condensedMatterLecture7Phonons:180}
m \uddot_j = K \lr{ u_{j + 1} - u_j } + K \lr{ u_{j - 1} - u_j}
\end{dmath}
%
We have \(10^{23}\) coupled equations.

The periodicity requirement imposes a constraint on
%
\begin{dmath}\label{eqn:condensedMatterLecture7Phonons:220}
e^{i q( x_j + N a) },
\end{dmath}
%
so that
%
\begin{dmath}\label{eqn:condensedMatterLecture7Phonons:240}
q \mathLabelBox{N a}{\(L\)} = 2 \pi n,
\end{dmath}
%
or
\begin{dmath}\label{eqn:condensedMatterLecture7Phonons:260}
q = \frac{2 \pi n}{L}.
\end{dmath}
%
In class we used trial solutions of the form
%
\begin{dmath}\label{eqn:condensedMatterLecture7Phonons:200}
u_j = \inv{\sqrt{m}} \sum_q u_q e^{i \lr{ q x_j - \omega_q t} }.
\end{dmath}
%
\paragraph{Constraint on frequency}
%
%Plugging this in and working through a rough Fourier argument provides a constraint on \(\omega\), but we can get to that constraint more easily by first considering one component of that solution in isolation.  This
%%% FIXME:TEMP
In class we plugged \eqnref{eqn:condensedMatterLecture7Phonons:200} into  \eqnref{eqn:condensedMatterLecture7Phonons:180} and after some rushed arithmetic we arrived at a constraint on the frequency.

Because that was rushed I had a bit of trouble following, but thought I had the general idea.  I found a simpler treatment in \citep{kdasgupta:ph409} which used single frequency trial solution
%
\begin{dmath}\label{eqn:condensedMatterLecture7Phonons:280}
u_n = \epsilon e^{i q n a - \omega t}.
\end{dmath}
%
Our derivatives are
\begin{subequations}
\begin{dmath}\label{eqn:condensedMatterLecture7Phonons:300}
\dot{u}_n = -i \omega \epsilon e^{i q n a - \omega t}
\end{dmath}
\begin{dmath}\label{eqn:condensedMatterLecture7Phonons:320}
\ddot{u}_n = - \omega^2 \epsilon e^{i q n a - \omega t},
\end{dmath}
\end{subequations}
%
and insertion back into \eqnref{eqn:condensedMatterLecture7Phonons:180} gives
%
\begin{dmath}\label{eqn:condensedMatterLecture7Phonons:340}
0
= \epsilon e^{-i\omega t}
\lr{
\frac{m}{K} \omega^2 e^{i q n a} + e^{i q(n + 1) a} - 2 e^{i q n a} + e^{i q (n-1) a}
}
= \epsilon e^{-i\omega t}
e^{i q n a}
\lr{
\frac{m}{K} \omega^2 + e^{i q a} - 2 + e^{-i q a}
}
= \epsilon e^{-i\omega t}
e^{i q n a}
\lr{
\frac{m}{K} \omega^2 + -2 + 2 \cos q a
}
= \epsilon e^{-i\omega t}
e^{i q n a}
\lr{
\frac{m}{K} \omega^2 + - 4 \sin^2 \lr{ \frac{ q a}{2} }
}.
\end{dmath}
%
Requiring equality means that we must have
\boxedEquation{eqn:condensedMatterLecture7Phonons:360}{
\sqrt{\frac{m}{K}} \omega = 2 \sin \lr{ \frac{ q a}{2} }.
}
\paragraph{With superposition of frequency components}
Putting an index on \(\epsilon\) and \(\omega\), and expressing \(q\) explicitly we have a slightly more general trial solution
%
\begin{dmath}\label{eqn:condensedMatterLecture7Phonons:380}
u_n = \sum_s \epsilon_s
e^{i \lr{
\frac{ 2 \pi s n}{N} - \omega_s t
} },
\end{dmath}
%
where \(s\) is an integer.  Plugging this into our EOM we have
%
\begin{dmath}\label{eqn:condensedMatterLecture7Phonons:400}
0
= \sum_s \epsilon_s e^{-i \omega_s t }
\lr{
\frac{m}{K}\omega_s^2
e^{i
\frac{ 2 \pi s n}{N}
}
+
e^{i
\frac{ 2 \pi s (n+1)}{N}
}
- 2
e^{i
\frac{ 2 \pi s n}{N}
}
+
e^{i
\frac{ 2 \pi s (n-1)}{N}
}
}
= \sum_s \epsilon_s e^{-i \omega_s t }
e^{i
\frac{ 2 \pi s n}{N}
}
\lr{
\frac{m}{K}\omega_s^2
+
e^{i
\frac{ 2 \pi s}{N}
}
- 2
+
e^{-i
\frac{ 2 \pi s }{N}
}
}
= \sum_s \epsilon_s e^{-i \omega_s t }
e^{i
\frac{ 2 \pi s n}{N}
}
\lr{
\frac{m}{K}\omega_s^2
+
2 \cos
\lr{
\frac{ 2 \pi s}{N}
}
- 2
}
= \sum_s \epsilon_s e^{-i \omega_s t }
e^{i
\frac{ 2 \pi s n}{N}
}
\lr{
\frac{m}{K}\omega_s^2
- 4 \sin^2
\lr{
\frac{ \pi s}{N}
}
}.
\end{dmath}
%
From this, we see that if
%
\begin{dmath}\label{eqn:condensedMatterLecture7Phonons:420}
\frac{m}{K}\omega_s^2 =
4 \sin^2
\lr{
\frac{ \pi s}{N}
},
\end{dmath}
%
then our equation is solved.  This is just the frequency constraint of \eqnref{eqn:condensedMatterLecture7Phonons:360}.
%
\paragraph{Q:} In class, the equality above that resulted from us applying the trial solution was operated on by \(\sum_{q'} e^{-i q' n a}\) to decouple the equations.  If the frequency constraint was what was desired, why did we even apply that operator?
%
I believe that the response for this was that without doing so, it as if one assumes a-priori that the solutions are decoupled.  In retrospect, I am still not sure that this resolves my confusion, since it seems to me that this summation operator is really just a statement that the exponentials form a basis (i.e. forming a resolution of identity).

%\EndArticle
