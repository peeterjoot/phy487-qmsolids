%
% Copyright � 2013 Peeter Joot.  All Rights Reserved.
% Licenced as described in the file LICENSE under the root directory of this GIT repository.
%
%\input{../blogpost.tex}
%\renewcommand{\basename}{condensedMatterLecture2}
%\renewcommand{\dirname}{notes/phy487/}
%\newcommand{\keywords}{Condensed matter physics, PHY487H1F, Covalent bonding, Ionic bonding, orbital, promotion, hybridization, nearest neighbor, binding energy, lattice}
%\input{../peeter_prologue_print2.tex}
%
%\usepackage{mhchem}
%
%\beginArtNoToc
%\generatetitle{PHY487H1F Condensed Matter Physics.  Lecture 2: Bonding and lattice structures.  Taught by Prof.\ Stephen Julian}
%\chapter{Bonding and lattice structures}
\label{chap:condensedMatterLecture2}

%\section{Disclaimer}
%
%Peeter's lecture notes from class.  May not be entirely coherent.

%\section{Covalent bonding (cont.)}
\index{covalent bonding}
\index{lattice structure}

Covalent bonding involves electrons shared between materials, forming between partially filled orbitals on small atoms.  Example \ce{H_2}.

Only half filled orbitals (eg. \(2 p_z^1\)) form covalent bonds.  Two shared electrons in bonding orbitals.  We need small, directional orbitals.  We find this sort of bonding in the upper triangular segment of the periodic table as in \cref{fig:qmSolidsL2:qmSolidsL2Fig1}.

\imageFigure{../../figures/phy487/qmSolidsL2Fig1}{Covalent bonding region}{fig:qmSolidsL2:qmSolidsL2Fig1}{0.15}

An example of such a covalent bond is that of two \(2p_z^1\) orbitals of Fluoride \ce{F} (\(1s^2 2s^2 2 p_x^2 2 p_y^2 2 p_z\)), as in \cref{fig:qmSolidsL2:qmSolidsL2Fig2}.

\imageFigure{../../figures/phy487/qmSolidsL2Fig2}{Fluorine gas molecule}{fig:qmSolidsL2:qmSolidsL2Fig2}{0.2}

With Fluorine we cannot make a covalently bonded solid, since there are no orbitals left over for bonding with anything else.

\paragraph{Fluorine solid?}

Can we get a Fluorine solid with promotion of two \(2 p\) states to \(3s^2\), then have three orbitals left for bonding?

Probably, but the energy cost of doing so may be exorbitant, and could require high pressures.  This is more likely with Bromine since the difference between the \(4s\) and \(5p\) states is less.

\paragraph{First important cases, \ce{C}, \ce{Si}, \ce{Ge}}

%M3
Looks like two bonds form.  Normally carbon is 3 or 4 fold coordinated.  Two such mechanisms for carbon bonding are \underlineAndIndex{promotion} and \underlineAndIndex{hybridization}

See: \citep{ibach2009solid} \textchapref{1}

For carbon (\(1s^2 2s^2 2 p_x^1 2 p_y^1 2 p_z^0\)), a promotion is possible as in \cref{fig:qmSolidsL2:qmSolidsL2Fig4}.

\imageFigure{../../figures/phy487/qmSolidsL2Fig4}{\(2s\) promotion in carbon allowing for 4 way bonding}{fig:qmSolidsL2:qmSolidsL2Fig4}{0.3}

It turns out that

\begin{equation}\label{eqn:condensedMatterLecture2:20}
E_{\text{4 bonds}} + E_{\text{promotion}} < E_{\text{2 bonds}}.
\end{equation}

This allows for linear combinations of \(2s + 2p\) orbitals that have highly directional compact orbitals, perfect for covalent bonding.

\begin{enumerate}
\item \(sp_1\) hybrid.  \(2 s + 2 p_x\)

%\cref{fig:qmSolidsL2:qmSolidsL2Fig5}.
\imageFigure{../../figures/phy487/qmSolidsL2Fig5}{\(sp_1\) hybrid}{fig:qmSolidsL2:qmSolidsL2Fig5}{0.15}

\item \(sp_2\) hybrid (like graphene.)  Discussion left to problem set 1.

\item \(sp_3\) hybrid orbitals.

%\cref{fig:qmSolidsL2:qmSolidsL2Fig5b}.
\imageFigure{../../figures/phy487/qmSolidsL2Fig5b}{\(sp_3\) hybrid sign configurations}{fig:qmSolidsL2:qmSolidsL2Fig5b}{0.2}

%\cref{fig:qmSolidsL2:qmSolidsL2Fig5b1}.
\imageFigure{../../figures/phy487/qmSolidsL2Fig5b1}
{\(sp_3\) hybrid}
{fig:qmSolidsL2:qmSolidsL2Fig5b1}{0.15}

%\cref{fig:qmSolidsL2:qmSolidsL2Fig5c}.
\imageFigure{../../figures/phy487/qmSolidsL2Fig5c}
{Lobes point to the 4 vertexes of a tetrahedron.}
{fig:qmSolidsL2:qmSolidsL2Fig5c}{0.2}

Covalent bonds are some of the strongest.  In diamond, where each \ce{C} has 4 nn (nearest neighbors) we have a melting point

\begin{equation}\label{eqn:condensedMatterLecture2:40}
T_m \sim 4000 K
\end{equation}

and has the highest hardness of any material.
\end{enumerate}

\section{Ionic bonding}
\index{ionic bonding}

\reading \citep{ashcroft1976solid} \textchapref{20}.

Here we are combining different atoms, especially the left and right hand sides of the period table.

%\cref{fig:qmSolidsL2:qmSolidsL2Fig6}.
\imageFigure{../../figures/phy487/qmSolidsL2Fig6}{\(NaCl\) periodic table locations}{fig:qmSolidsL2:qmSolidsL2Fig6}{0.2}

Examples: \ce{NaCl}, \ce{KF}, \ce{CsCl}, \ce{Li_2 O}, \ce{CaO}

\paragraph{\ce{NaCl}}

\ce{Na} has 1 weakly bound 3s electron.  \ce{Cl} has one vacancy in its \(3p\) shell.

\paragraph{Energetics}

The energy transitions for ionization are illustrated in \cref{fig:qmSolidsL2:qmSolidsL2Fig7a}.

\imageFigure{../../figures/phy487/qmSolidsL2Fig7a}{Energy ionization transitions for \(NaCl\) atoms (far apart)}{fig:qmSolidsL2:qmSolidsL2Fig7a}{0.2}

The energy released moving the ions together is illustrated in \cref{fig:qmSolidsL2:qmSolidsL2Fig7b}.

\imageFigure{../../figures/phy487/qmSolidsL2Fig7b}{Energy transitions for Coulomb interaction of ionized \(NaCl\) atoms}{fig:qmSolidsL2:qmSolidsL2Fig7b}{0.2}

Relative to \cref{fig:qmSolidsL2:qmSolidsL2Fig7a} we have a \(-3.1 eV\) energy change in the transition to this final state.

%Equilibrium separation.
%\cref{fig:qmSolidsL2:qmSolidsL2Fig8}.
\imageFigure{../../figures/phy487/qmSolidsL2Fig8}{Potential well}{fig:qmSolidsL2:qmSolidsL2Fig8}{0.2}

%\cref{fig:qmSolidsL2:qmSolidsL2Fig9}.
\imageFigure{../../figures/phy487/qmSolidsL2Fig9}{Final state for pair of ions}{fig:qmSolidsL2:qmSolidsL2Fig9}{0.2}

Observe that the final state is not obviously predictable from the initial states.  The hybrid state can probably be derived from first principles, but this determination may be easier with spectroscopy.

Solid \ce{NaCl}  (See: 02_lecture.pdf, and \citep{ibach2009solid} fig 1.6)

Have 2 interpenetrating free lattices.  Each \ce{Na+ Cl-} has 6 \underlineAndIndex{nn} (nearest neighbor) \ce{Cl- Na+}.  Binding energy 7.95 eV/pair.

\paragraph{CsCl}

The \ce{CsCl} structure.  \ce{Cl} on corners.  \ce{Cs} in center of cube.  8 nn.  Better than \ce{NaCl} structure.  But, \ce{Na+} is small, so in the \ce{CsCl} (where \ce{Cs} is big compared to \ce{Na}) structure, the next nn (nnn) \ce{Cl-} would touch.  There's a strong Coulomb repulsion.

%\EndArticle
