%
% Copyright � 2013 Peeter Joot.  All Rights Reserved.
% Licenced as described in the file LICENSE under the root directory of this GIT repository.
%
%\input{../blogpost.tex}
%\renewcommand{\basename}{condensedMatterLecture17}
%\renewcommand{\dirname}{notes/phy487/}
%\newcommand{\keywords}{Condensed matter physics, PHY487H1F}
%\input{../peeter_prologue_print2.tex}
%
%%\citep{harald2003solid} \S x.y
%%\citep{ibach2009solid} \S x.y
%
%%\usepackage{mhchem}
%\usepackage[version=3]{mhchem}
%\newcommand{\nought}[0]{\circ}
%\newcommand{\EF}[0]{\epsilon_{\txtF}}
%\newcommand{\kF}[0]{k_{\txtF}}
%
%\beginArtNoToc
%\generatetitle{PHY487H1F Condensed Matter Physics.  Lecture 17: 3 dimensional band structures, Fermi surfaces of real metals.  Taught by Prof.\ Stephen Julian}
%\chapter{3 dimensional band structures, Fermi surfaces of real metals}
\label{chap:condensedMatterLecture17}

%\section{Disclaimer}
%
%Peeter's lecture notes from class.  May not be entirely coherent.

\section{Three dimensional band structures, Fermi surfaces of real metals}
\index{band structure}
\index{Fermi surface}

\reading \citep{ibach2009solid} \S 7.4

Consider a nearly free electron metal, and a hypothetical simple cubic system as in \cref{fig:qmSolidsLecture17:qmSolidsLecture17Fig1}.

\imageFigure{../../figures/phy487/qmSolidsLecture17Fig1}{Simple cubic Brillouin zone}{fig:qmSolidsLecture17:qmSolidsLecture17Fig1}{0.2}

%\cref{fig:qmSolidsLecture17:qmSolidsLecture17Fig2}.
\imageFigure{../../figures/phy487/qmSolidsLecture17Fig2}{Two frequency distributions}{fig:qmSolidsLecture17:qmSolidsLecture17Fig2}{0.3}
%\cref{fig:qmSolidsLecture17:qmSolidsLecture17Fig3}.
\imageFigure{../../figures/phy487/qmSolidsLecture17Fig3}{Center edge contour distribution}{fig:qmSolidsLecture17:qmSolidsLecture17Fig3}{0.3}

We can ask some questions

\begin{itemize}
\item
What is the occupancy?
\item
Where is \(\EF\) (the Fermi energy)?
\end{itemize}

Consider alkali metals, such as \ce{Li} \(1s^2 2 s^1\)

Is tight-binding-like \textunderline{fully occupied}.

\(2 \times 1 s\) electrons per atom.

In 1 band

\begin{dmath}\label{eqn:condensedMatterLecture17:20}
\mathLabelBox
[
   labelstyle={xshift=-2cm},
   linestyle={out=270,in=90, latex-}
]
{\frac{ \cancel{(2 \pi)^3}}{a^3} }{
volume of k space }
\times
\mathLabelBox
[
   labelstyle={below of=m\themathLableNode, below of=m\themathLableNode}
]
{2}{spin}
\mathLabelBox
[
   labelstyle={xshift=2cm},
   linestyle={out=270,in=90, latex-}
]
{\frac{ V}{ \cancel{(2 \pi)^3} }}{density of k points}
\end{dmath}

%\cref{fig:qmSolidsLecture17:qmSolidsLecture17Fig4}.
%\imageFigure{../../figures/phy487/qmSolidsLecture17Fig4}{4: CAPTION}{fig:qmSolidsLecture17:qmSolidsLecture17Fig4}{0.3}
%\cref{fig:qmSolidsLecture17:qmSolidsLecture17Fig5}.
%\imageFigure{../../figures/phy487/qmSolidsLecture17Fig5}{5: CAPTION}{fig:qmSolidsLecture17:qmSolidsLecture17Fig5}{0.3}

This gives

\begin{dmath}\label{eqn:condensedMatterLecture17:40}
2 \frac{V}{a^3} = 2 N,
\end{dmath}

where \(N\) is the number of unit cells.

The \(2 s\) orbitals stick out a long way, so this is free electron like.
We have \textunderline{half filled} orbitals.  Where \(\EF\) crosses \(E(k)\) is a Fermi surface.  It never gets close to the Brillouin zone boundary.

This is nearly spherical.  See slides for the true Brillouin zone diagram for \ce{Li}.

\paragraph{Question:}
In class when discussing \ce{Li}, \ce{K}, \ce{Na} tight binding, and it's relation to the Fermi surface, we were told that the \ce{Li} and other alkali-metal Fermi surfaces never get close to the BZ boundary, and that they were approximately spherical.

I don't understand how we arrived at the conclusion that for these s-orbital elements ``the Fermi surface never gets close to the BZ boundary'', nor why those surfaces would necessarily be approximately spherical?

\paragraph{Answer:}

Unlike the calculation in problem set 8, the band structure of the alkalai metals and alkalai earths is free-electron-like.  This means that the dispersion relation is parabolic, except near the Brillouin zone boundary.  So if the Fermi surface never gets close to a Brillouin zone boundary, \(E(\kF) \simeq \Hbar^2 \kF^2/2m\), independent of direction in k-space, so the Fermi surface is spherical.

You can tell that the Fermi surface doesn't go close to the BZ boundary by calculating the volume of the Fermi sphere, compared with the volume of the Brillouin zone.  This doesn't work for the tight-binding band structure, because the dispersion is anisotropic even far from the Brillouin zone.

A very helpful demonstration of exactly that calculation can be found in \citep{oxfordNearlyFreeBandMT}, under section 'Alkali metals'.  To understand the details of that calculation (see: \nbref{bccBasisVectors.nb}), it is helpful to note that a BCC basis is

\begin{equation}\label{eqn:condensedMatterLecture17:41}
\Ba_i \in
\left\{
\frac{a}{2}
\begin{bmatrix}
1 \\
1 \\
1
\end{bmatrix}
,
\frac{a}{2}
\begin{bmatrix}
1 \\
1 \\
-1
\end{bmatrix}
,
a
\begin{bmatrix}
1 \\
0 \\
0
\end{bmatrix}
\right\},
\end{equation}

for which the reciprocal basis is

\begin{equation}\label{eqn:condensedMatterLecture17:42}
\Bg_i \in
\left\{
\frac{2 \pi}{a}
\begin{bmatrix}
0 \\
1 \\
1
\end{bmatrix}
,
\frac{2 \pi}{a}
\begin{bmatrix}
0 \\
1 \\
-1
\end{bmatrix}
,
\frac{2 \pi}{a}
\begin{bmatrix}
1 \\
-1 \\
0
\end{bmatrix}
\right\}.
\end{equation}

This point is also discussed with typical clarity in \citep{ashcroft1976solid} \textchapref{15}, ``The Alkaki Metals''.

\paragraph{Valence 2}
\index{alkali earth metals}

For alkali earth's \ce{Mg}, \ce{Ca}, \(\cdots\), we have \(2 \times 2 s\) valence electrons.  This \textunderline{doesn't} fill the band, because of dispersion.

Define a free electron sphere.  It extends beyond the first Brillouin zone.  This is incorrectly sketched in \cref{fig:qmSolidsLecture17:qmSolidsLecture17Fig6} as a simple cubic (actual is perhaps FCC).

%F6
%F7
%F8 ...
\imageFigure{../../figures/phy487/qmSolidsLecture17Fig6}{Fermi surfaces for \ce{Cu} like simple cubic}{fig:qmSolidsLecture17:qmSolidsLecture17Fig6}{0.3}
%\cref{fig:qmSolidsLecture17:qmSolidsLecture17Fig7}.
\imageFigure{../../figures/phy487/qmSolidsLecture17Fig7}{Copper Fermi surface side view?}{fig:qmSolidsLecture17:qmSolidsLecture17Fig7}{0.3}

\examhint{For a picture like this, understand what happens at the boundary, and how it reconstructs.}

\paragraph{Valence 3}

Considering a material such as \ce{Al}, which is in valance 3.  See slide.
\examhint{Valence three won't be examinable}

\paragraph{d electron systems}

We have two kinds of valence electrons.

\begin{itemize}
\item s electrons.  Free electron like.
\item d electrons.  compact orbitals that are tight binding like.
\end{itemize}

Looking with an experienced eye, we see two types of bands.  The first are the s-bands that are free electron like and rapidly disburse.  This is roughly sketched in \cref{fig:qmSolidsLecture17:qmSolidsLecture17Fig8}.  The others are the d orbital band that disperse a bit, but not very much.  What actually happens in here where they cross is also illustrated in the magnified section.

\imageFigure{../../figures/phy487/qmSolidsLecture17Fig8}{d electron distribution}{fig:qmSolidsLecture17:qmSolidsLecture17Fig8}{0.3}

The \textunderline{occupancy} for \ce{Cu} is \(3 d^{10} 4 s^1\).  The d orbitals are \textunderline{filled}.

\(\EF\) intersects \(4 s\) bands, we have a nearly free electron Fermi surface.

The reason that copper is copper colored is because there's a high density of states, with absorption in blue, resulting in a brown look.

We have something similar for \ce{Ag} and \ce{Au}.

From \ce{Sc} to \ce{Ni}, or \ce{Y} to \ce{Pd}, or \ce{La} to \ce{Pt}, the d orbitals are partially occupied, and \(\EF\) is among the d bands.  This ends up being a complicated Fermi surface, and there is a high density of states at \(\EF\) (see table in slides).

These atoms with \(3d\) valence electrons are very prone to magnetism.
The atoms with \(4d\) valence electrons are very prone to superconductivity.

A high density of states in physics makes for interesting effects.

%\EndArticle
