%
% Copyright � 2013 Peeter Joot.  All Rights Reserved.
% Licenced as described in the file LICENSE under the root directory of this GIT repository.
%
%\input{../blogpost.tex}
%\renewcommand{\basename}{condensedMatterLecture15}
%\renewcommand{\dirname}{notes/phy487/}
%\newcommand{\keywords}{Condensed matter physics, PHY487H1F}
%\input{../peeter_prologue_print2.tex}
%
%%\citep{harald2003solid} \S x.y
%%\citep{ibach2009solid} \S x.y
%
%%\usepackage{mhchem}
%\usepackage[version=3]{mhchem}
%\newcommand{\nought}[0]{\circ}
%\newcommand{\EF}[0]{\epsilon_{\txtF}}
%\newcommand{\kF}[0]{k_{\txtF}}
%
%\beginArtNoToc
%\generatetitle{PHY487H1F Condensed Matter Physics.  Lecture 15: Nearly free electron model, periodic potential (cont.).  Taught by Prof.\ Stephen Julian}
%\chapter{Nearly free electron model, periodic potential (cont.)}
\label{chap:condensedMatterLecture15}
%
%\section{Disclaimer}
%
%Peeter's lecture notes from class.  May not be entirely coherent.
%
\paragraph{Nearly free electron model, periodic potential (cont.)}
\index{nearly free electron model}

\reading \citep{ashcroft1976solid} \textchapref{9}.

For periodic potential
%
\begin{dmath}\label{eqn:condensedMatterLecture15:20}
V(\Br) = \sum_\BG V_\BG e^{i \BG \cdot \Br},
\end{dmath}
%
and a trial solution
\begin{dmath}\label{eqn:condensedMatterLecture15:n}
\Psi(\Br) = \sum_\Bk C_\Bk e^{i \Bk \cdot \Br },
\end{dmath}
%
we found that the Schr\"{o}dinger equation takes the form
%
\begin{dmath}\label{eqn:condensedMatterLecture15:40}
\lr{ \frac{\Hbar^2 \Bk^2}{2m} - E(\Bk) }
C_\Bk
+
\sum_\BG V_\BG C_{\Bk - \BG} = 0,
\end{dmath}
%
allowing for a factorization of \(\Psi_\Bk\)
%
\begin{dmath}\label{eqn:condensedMatterLecture15:60}
\Psi_\Bk(\Br) = \lr{
\sum_\BG
C_{\Bk - \BG}
e^{i \BG \cdot \Br}
}
e^{i \Bk \cdot \Br}
= U_\Bk(\Br)
e^{i \Bk \cdot \Br}.
\end{dmath}
%
What is \(C_{\Bk - \BG}\) on branch \(\Psi_\Bk\)?
%
\begin{dmath}\label{eqn:condensedMatterLecture15:80}
\lr{ \frac{\Hbar^2 \lr{\Bk -\BG}^2}{2m} - E(\Bk -\BG) }
C_{\Bk - \BG}
+
\sum_{\BG'} V_{\BG'} C_{\Bk - \BG - \BG'} = 0.
\end{dmath}
%
Let \(\BG'' = \BG + \BG'\), and recall that \(E(\Bk - \BG) = E(\Bk)\), giving
%
\begin{dmath}\label{eqn:condensedMatterLecture15:100}
C_{\Bk - \BG}
=
\frac{
\sum_{\BG''} V_{\BG'' - \BG} C_{\Bk - \BG''}
}
{
E(\Bk) -
\frac{\Hbar^2 \lr{\Bk -\BG}^2}{2m}
}.
\end{dmath}
%
\(C_{\Bk - \BG}\) on a given branch is small unless
%
\begin{dmath}\label{eqn:condensedMatterLecture15:120}
E(\Bk) \approx
\frac{\Hbar^2 \lr{\Bk -\BG}^2}{2m}.
\end{dmath}
%
except at crossing points at \(k = \lr{0, \pm \pi/a}\).  Only one
\(C_{\Bk - \BG}\) is large.  i.e. at (1) in \cref{fig:qmSolidsL14:qmSolidsL14Fig4}
%
\begin{dmath}\label{eqn:condensedMatterLecture15:140}
E(\Bk) \approx
\frac{ \Hbar^2 \Bk^2 }{2m}.
\end{dmath}
%
%\imageFigure{../figures/phy487-qmsolids/qmSolidsL14Fig4}{First Brillouin zone}{fig:qmSolidsL14:qmSolidsL14Fig4}{0.4}

so that \(C_\Bk\) is large, and all other \(C_\Bk\)'s are small.

At (2) we have
%
\begin{dmath}\label{eqn:condensedMatterLecture15:160}
E(\Bk) \approx
\frac{ \Hbar^2 \lr{k - 2 \pi/a}^2 }{2m}.
\end{dmath}
%
so that \(C_{k - 2 \pi/a}\) is large, and all other \(C_\Bk\)'s are small.

However, at \(k = \lr{0, \pm \pi/a}\) two or more bands cross (i.e. at (4)).  Here
%
\begin{dmath}\label{eqn:condensedMatterLecture15:180}
\frac{ \Hbar^2 \lr{k}^2 }{2m}
=
\frac{ \Hbar^2 \lr{k - 2 \pi/a}^2 }{2m},
\end{dmath}
%
so that \(C_k\) and \(C_{k - 2\pi/a}\) are both large.

Generally when
%
\begin{subequations}
\begin{dmath}\label{eqn:condensedMatterLecture15:200}
C_{\Bk}
=
\frac{
\sum_{\BG''} V_{\BG'' - \BG} C_{\Bk - \BG''}
}
{
E(\Bk) -
\frac{\Hbar^2 \lr{\Bk}^2}{2m}
}
\end{dmath}
\begin{dmath}\label{eqn:condensedMatterLecture15:220}
C_{\Bk - \BG}
=
\frac{
\sum_{\BG''} V_{\BG'' - \BG} C_{\Bk - \BG''}
}
{
E(\Bk) -
\frac{\Hbar^2 \lr{\Bk -\BG}^2}{2m}
},
\end{dmath}
\end{subequations}
%
and \(\Hbar^2 \lr{\Bk}^2/2m = \Hbar^2 \lr{\Bk -\BG}^2/2m\).
\makeexample{1D}{example:condensedMatterLecture15:1}{
Keep 2 \(C_k\)'s in \eqnref{eqn:condensedMatterLecture15:40}, or
%
\begin{dmath}\label{eqn:condensedMatterLecture15:240}
\begin{aligned}
\lr{ E(k) - \frac{\Hbar^2 k^2}{2m} } C_k
- V_{2\pi/a} C_{k - 2\pi/a} &= 0 \\
\lr{ E(k) - \frac{\Hbar^2 (k - 2\pi/a)^2}{2m} } C_{k - 2\pi/a}
- V_{-2\pi/a} C_{k} &= 0,
\end{aligned}
\end{dmath}
%
With solution
%
\begin{dmath}\label{eqn:condensedMatterLecture15:260}
0 =
\begin{vmatrix}
\lr{
\frac{\Hbar^2 k^2}{2m}
-E(k)
} & V_{2\pi/a} \\
 V_{-2\pi/a} & \lr{
\frac{\Hbar^2 (k - 2\pi/a)^2}{2m}
-E(k)
}
\end{vmatrix}.
\end{dmath}
%
With \(E_k^\nought = \Hbar^2 k^2/2m\), and \(E_{k - 2 \pi/a}^\nought = \Hbar^2 (k - 2\pi/a)^2/2m\), we complete the square to find
%
\begin{dmath}\label{eqn:condensedMatterLecture15example2cKs:261}
0 =
\lr{
{E_k^\nought}
-E
}
\lr{
{E_{k - 2 \pi/a}^\nought}
-E
}
-
V_{2\pi/a}
V_{-2\pi/a}
=
E^2 - E \lr{ {E_k^\nought} + {E_{k - 2 \pi/a}^\nought} }
+ {E_k^\nought} {E_{k - 2 \pi/a}^\nought}
-
V_{2\pi/a}
V_{-2\pi/a}
=
\lr{ E - \frac{ {E_k^\nought} + {E_{k - 2 \pi/a}^\nought} }{2} }^2
- \lr{ \frac{ {E_k^\nought} + {E_{k - 2 \pi/a}^\nought} }{2} }^2
+ {E_k^\nought} {E_{k - 2 \pi/a}^\nought}
-
V_{2\pi/a}
V_{-2\pi/a}
=
\lr{ E - \frac{ {E_k^\nought} + {E_{k - 2 \pi/a}^\nought} }{2} }^2
- \inv{4}
\lr{
{E_k^\nought}^2 + {E_{k - 2 \pi/a}^\nought}^2 + 2 {E_k^\nought} {E_{k - 2 \pi/a}^\nought} - 4 {E_k^\nought} {E_{k - 2 \pi/a}^\nought}
}
-
V_{2\pi/a}
V_{-2\pi/a}
=
\lr{ E - \frac{ {E_k^\nought} + {E_{k - 2 \pi/a}^\nought} }{2} }^2
-
\lr{ \frac{{E_k^\nought} - {E_{k - 2 \pi/a}^\nought} }{2} }^2
-
V_{2\pi/a}
V_{-2\pi/a},
\end{dmath}
%
or
%
\begin{dmath}\label{eqn:condensedMatterLecture15:280}
E^{\pm}(k) =
\inv{2} \lr{ E_k^\nought + E_{k - 2\pi/a}^\nought }
\pm
\sqrt{
\inv{4} \lr{ E_k^\nought - E_{k - 2\pi/a}^\nought }^2 + \Abs{V_{2 \pi/a}}^2
}.
\end{dmath}
%
This is illustrated in \cref{fig:qmSolidsL15:qmSolidsL15Fig2}.
\imageFigure{../figures/phy487-qmsolids/qmSolidsL15Fig2}{Energy solutions for 1D}{fig:qmSolidsL15:qmSolidsL15Fig2}{0.25}
A snapshot of a exact Manipulate of this curve is plotted in \cref{fig:qmSolidsWeakBindingNearBragg:qmSolidsWeakBindingNearBraggFig1}.
\mathImageFigure{../figures/phy487-qmsolids/qmSolidsWeakBindingNearBraggFig1}{Weak binding plot with behaviour near Bragg plane}{fig:qmSolidsWeakBindingNearBragg:qmSolidsWeakBindingNearBraggFig1}{0.3}{weakBindingPotentialNearBraggPlane.nb}
At the crossing point \(k = \pi/a\), and \(k - 2 \pi/a = -\pi/a\), so that
%
\begin{dmath}\label{eqn:condensedMatterLecture15:300}
E^{\pm} = \frac{\Hbar^2}{2m} \lr{ \frac{\pi}{a} }^2 \pm \Abs{V_{2 \pi/a}},
\end{dmath}
%
or more generally
%
\begin{dmath}\label{eqn:condensedMatterLecture15:320}
E^{\pm} = \frac{\Hbar^2}{2m} \lr{ \frac{\BG}{2} }^2 \pm \Abs{V_\BG}.
\end{dmath}
%
This energy gap is sketched in \cref{fig:qmSolidsL15:qmSolidsL15Fig3}, with the periodic extension sketched in \cref{fig:qmSolidsL15:qmSolidsL15Fig4}.
%
\imageFigure{../figures/phy487-qmsolids/qmSolidsL15Fig3}{Energy gap}{fig:qmSolidsL15:qmSolidsL15Fig3}{0.2}
\imageFigure{../figures/phy487-qmsolids/qmSolidsL15Fig4}{Energy gaps and periodic structure}{fig:qmSolidsL15:qmSolidsL15Fig4}{0.2}

What does \(\Psi_k\) look like at the crossing points (4).  Here
%
\begin{dmath}\label{eqn:condensedMatterLecture15:340}
\Psi_{k = \pi/a} =
C_{k = \pi/a} e^{i \frac{\pi}{a} x}
+ C_{(k = \pi/a) - 2\pi/a } e^{i \lr{ \frac{\pi}{a} - \frac{2\pi}{a} } x}.
\end{dmath}
%
Two solutions to \(E_k^\pm\) correspond to \(C_{k = \pi/a} = \pm C_{(k = \pi/a) - 2\pi/a}\).

or
%
\begin{dmath}\label{eqn:condensedMatterLecture15:360}
\Psi_{k = \pi/a} = \inv{\sqrt{2 L}} \lr{ e^{i \pi x/a} \pm e^{-i \pi x /a} },
\end{dmath}
%
Here \(L\) is the length of the 1d system.  Note that \eqnref{eqn:condensedMatterLecture15:360} is one of
%
\begin{dmath}\label{eqn:condensedMatterLecture15:380}
\begin{aligned}
\cos\frac{\pi}{a} x \\
\sin\frac{\pi}{a} x
\end{aligned}
\end{dmath}
%
These solutions are sketched along with the potential in \cref{fig:qmSolidsL15:qmSolidsL15Fig5}.
%
\imageFigure{../figures/phy487-qmsolids/qmSolidsL15Fig5}{Real space 1D solutions and potential}{fig:qmSolidsL15:qmSolidsL15Fig5}{0.2}

The cosine is the lower energy branch because electron density is high where \(V(x)\) is low.  The sine is the higher energy branch because electron density is high when \(V(x)\) is high.
}
%\EndNoBibArticle
