%
% Copyright � 2013 Peeter Joot.  All Rights Reserved.
% Licenced as described in the file LICENSE under the root directory of this GIT repository.
%
%\input{../blogpost.tex}
%\renewcommand{\basename}{diffractionReflection}
%\renewcommand{\dirname}{notes/phy487/}
%%\newcommand{\dateintitle}{}
%\newcommand{\keywords}{Huygens diffraction, phy487}
%
%\newcommand{\nought}[0]{\circ}
%\input{../peeter_prologue_print2.tex}
%
%\beginArtNoToc
%
%\generatetitle{Huygens diffraction}
\label{chap:diffractionReflection}

We were presented with a diffraction result, that the intensity can be expressed as the Fourier transform of the aperture.  Let's review a derivation of that based on the Huygens principle.  Consider the aperture of \cref{fig:diffractionReflection:diffractionReflectionFig1}.

\imageFigure{../figures/phy487-qmsolids/diffractionReflectionFig1}{Diffraction aperture}{fig:diffractionReflection:diffractionReflectionFig1}{0.4}

The \textAndIndex{Huygens principle} expresses the amplitude of a wave \(U(\Br)\) in terms of it's amplitude \(U_\nought\) at \(\Br = 0\) as

\begin{dmath}\label{eqn:diffractionReflection:20}
U(\Br) \propto \frac{U_\nought e^{i k \Abs{\Br}}}{\Abs{\Br}}.
\end{dmath}

For multiple point diffraction, the diffracted wave is a superposition of such contributions from all points in the aperture.  For example, two exponentials are summed when considering a two slit diffraction apparatus.  For a more general aperture as above, we'd form

\begin{dmath}\label{eqn:diffractionReflection:40}
U(\BR') \propto U_\nought \int_{\txtA} d^2\Br
\frac{e^{i k \Abs{\Br - \BR}}}{\Abs{\Br - \BR}}
\frac{e^{i k \Abs{\BR' - \Br}}}{\Abs{\BR' - \Br}}.
\end{dmath}

Note that this the Huygens result is an approximation in many ways.  Fresnel later fixed up the proportionality factor and considered the angular dependence of the between the source and diffracted rays (the obliquity factor).  That corrected result is known as the \textAndIndex{Huygens-Fresnel principle}.  Kirchhoff later considered solutions to the scalar wave equations for the electromagnetic field components \(\BE\) and \(\BB\), ignoring the Maxwell coupling of these fields.  See \S 8.3.1 \citep{born1980principles}, \S A.2 \S 10.4 of \citep{hecht1998hecht}, or \S 5.2 of \citep{fowles1989introduction} for details.  See \S 9.8 \citep{jackson1975cew} for a vector diffraction treatment.

Let's proceed with using \eqnref{eqn:diffractionReflection:40} to obtain our result from class.  For simplicity, first position the origin in the aperture itself as in \cref{fig:diffractionReflection:diffractionReflectionFig2}.

\imageFigure{../figures/phy487-qmsolids/diffractionReflectionFig2}{Diffraction aperture with origin in aperture}{fig:diffractionReflection:diffractionReflectionFig2}{0.4}

Now we are set to perform a first order expansion of the vector lengths \(\Abs{\Br - \BR}\) and \(\Abs{\Br - \BR'}\).  It's sufficient to consider just one of these, expanding to first order

\begin{dmath}\label{eqn:diffractionReflection:60}
\Abs{\Br - \BR}
=
\sqrt{(\Br - \BR)^2}
=
\sqrt{
\BR^2 + \Br^2 - 2 \Br \cdot \BR
}
=
R
\sqrt{
1 + \frac{\Br^2}{\BR^2} - 2 \Br \cdot \frac{\BR}{\BR^2}
}
\approx
R
\lr{
1 + \inv{2} \frac{\Br^2}{\BR^2} - \Br \cdot \frac{\BR}{\BR^2}
}
=
R + \inv{2} \frac{\Br^2}{R} - \Br \cdot \frac{\BR}{R}.
\end{dmath}

Assume that both \(\BR\) and \(\BR'\) to be far enough from the aperture that we can approximate the \(\Abs{\BR - \Br}\) and \(\Abs{\BR' - \Br}\) terms downstairs as \(R = \Abs{\BR}\) and \(R' = \Abs{\BR'}\) respectively.  Additionally, ignore the second order term above, significant for \textAndIndex{Fresnel diffraction} where the diffraction pattern close to the aperture is examined, but not in the far field.  This gives

\begin{dmath}\label{eqn:diffractionReflection:80}
U(\BR') \sim
\frac{U_\nought }{R R'}
\int_{\txtA} d^2\Br
e^{i k \lr{R - \Br \cdot \Rcap}}
e^{i k \lr{R' - \Br \cdot \Rcap'}}
=
\frac{U_\nought }{R R'}
e^{i k (R + R')}
\int_{\txtA} d^2\Br
e^{-i k \Br \cdot \Rcap}
e^{-i k \Br \cdot \Rcap'}.
\end{dmath}

Finally write
\begin{dmath}\label{eqn:diffractionReflection:100}
\begin{aligned}
\Bk_s &= - k\Rcap \\
\Bk &= k\Rcap',
\end{aligned}
\end{dmath}

for

\begin{dmath}\label{eqn:diffractionReflection:120}
U(\BR')
\sim
\frac{U_\nought }{R R'}
e^{i k (R + R')}
\int_{\txtA} d^2\Br
e^{-i \Br \cdot (\Bk - \Bk_s)}.
\end{dmath}

Finally, write

\begin{dmath}\label{eqn:diffractionReflection:140}
\BK = \Bk - \Bk_s,
\end{dmath}

and introduce an aperture function \(\rho(\Br)\) (we called this the \textAndIndex{scattering density}).  This aperature function is unity for any regions of the aperture that light is allowed through, and zero when light is blocked, and some value in \([0,1]\) for translucent regions of the aperture.  We can now expand the integral over the surface containing the aperture

\boxedEquation{eqn:diffractionReflection:160}{
U(\BR')
\sim
\frac{U_\nought }{R R'}
e^{i k (R + R')}
\int
\rho(\Br)
e^{-i \Br \cdot \BK}.
}

%\EndArticle
