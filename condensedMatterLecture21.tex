%
% Copyright � 2013 Peeter Joot.  All Rights Reserved.
% Licenced as described in the file LICENSE under the root directory of this GIT repository.
%
%\input{../blogpost.tex}
%\renewcommand{\basename}{condensedMatterLecture21}
%\renewcommand{\dirname}{notes/phy487/}
%\newcommand{\keywords}{Condensed matter physics, PHY487H1F}
%\input{../peeter_prologue_print2.tex}
%
%%\citep{harald2003solid} \S x.y
%
%%\usepackage{mhchem}
%\usepackage[version=3]{mhchem}
%\usepackage{units}
%\usepackage{bm} % \EE
%\newcommand{\nought}[0]{\circ}
%%\newcommand{\EF}[0]{\epsilon_{\txtF}}
%\newcommand{\EF}[0]{E_{\txtF}}
%\newcommand{\kF}[0]{k_{\txtF}}
%
%\beginArtNoToc
%\generatetitle{PHY487H1F Condensed Matter Physics.  Lecture 21: Electron-phonon scattering.  Taught by Prof.\ Stephen Julian}
\label{chap:condensedMatterLecture21}
%
%\section{Disclaimer}
%
%Peeter's lecture notes from class.  May not be entirely coherent.
%
\section{Electron-phonon scattering}
\index{scattering}

\reading \citep{ibach2009solid} \textchapref{9} (pp 258-259).
%
\paragraph{Last time}
%
\begin{subequations}
\begin{dmath}\label{eqn:condensedMatterLecture21:20}
\Bj = \sigma \bcE
\end{dmath}
\begin{dmath}\label{eqn:condensedMatterLecture21:40}
\sigma = \frac{n e^2 \tau}{m^\conj}
\end{dmath}
\end{subequations}
%
Here \(\tau\) is the \dquoteAndIndex{mean scattering time}, which is the time to randomize \(\Bv\).

We now continue to discuss scattering, a phenomena due to departure from periodicity.  For phonons, this is proportional to \(\expectation{n_q}_{th}\), where \(E_q = (n_q + 1/2) \Hbar \omega_q\).

%\cref{fig:qmSolidsL21:qmSolidsL21Fig1}.
\imageFigure{../figures/phy487-qmsolids/qmSolidsL21Fig1}{k-space scattering}{fig:qmSolidsL21:qmSolidsL21Fig1}{0.2}

Small \(q\) (long \(\lambda\)) phonons are not very effective at randomizing \(\Bv\).  The effectiveness of the scattering is \(\propto q^2\).

%\cref{fig:qmSolidsL21:qmSolidsL21Fig2}.
%\imageFigure{../figures/phy487-qmsolids/qmSolidsL21Fig2}{2:CAPTION}{fig:qmSolidsL21:qmSolidsL21Fig2}{0.2}
%\cref{fig:qmSolidsL21:qmSolidsL21Fig3}.
\imageFigure{../figures/phy487-qmsolids/qmSolidsL21Fig3}{Scattering confinement to small range of k-space}{fig:qmSolidsL21:qmSolidsL21Fig3}{0.2}

We'd found
%
\begin{subequations}
\begin{dmath}\label{eqn:condensedMatterLecture21:60}
\inv{\tau_{\mathrm{ph}}(q)} \propto q^2 \expectation{n_q}
\end{dmath}
\begin{dmath}\label{eqn:condensedMatterLecture21:80}
\expectation{n_q} = \inv{e^{\Hbar \omega_q/\kB T} - 1}
\end{dmath}
\end{subequations}
%
Combining these, we have
%
\begin{dmath}\label{eqn:condensedMatterLecture21:100}
\inv{\tau_{\mathrm{ph}}} \propto
%\frac{2 V}{(2 \pi)^3}
\lr{ \cdots }
\int 4 \pi q^2 dq q^2
\inv{e^{\Hbar \omega_q/\kB T} - 1}
\end{dmath}
%
In the high temperature limit, all modes have \(\expectation{n_q} \propto T\), for
%
\begin{subequations}
\begin{dmath}\label{eqn:condensedMatterLecture21:120}
\inv{\tau_{\mathrm{ph}}} \propto  T
\end{dmath}
\begin{dmath}\label{eqn:condensedMatterLecture21:140}
\sigma \propto \inv{T}
\end{dmath}
\begin{dmath}\label{eqn:condensedMatterLecture21:160}
\rho = \inv{\sigma} \alpha T
\end{dmath}
\end{subequations}
%
In the low temperature limit.  As in the Debye theory, let
%
\begin{equation}\label{eqn:condensedMatterLecture21:180}
x = \frac{\Hbar \omega}{\kB T} =
\frac{\Hbar c q}{\kB T},
\end{equation}
%
for
%
\begin{dmath}\label{eqn:condensedMatterLecture21:200}
\inv{\tau_{\mathrm{ph}}} \propto
\lr{ \cdots } \lr{ \kB T }^5
\mathLabelBox
{
\int_0^{\Theta/T} \frac{ x^5 dx}{e^x - 1}
}
{
some number
}.
\end{dmath}
%
This gives
%
\begin{equation}\label{eqn:condensedMatterLecture21:220}
\rho = \rho_0 + A T^5.
\end{equation}
%
At high \(T\), the \(T^1\) behavior is very general.  At low \(T\), \(T^5\) is less universal.

%This is really universal for almost all materials.
The full range of resistivity is sketched in \cref{fig:qmSolidsL21:qmSolidsL21Fig4}.
%
\imageFigure{../figures/phy487-qmsolids/qmSolidsL21Fig4}{Resistivity temperature dependence}{fig:qmSolidsL21:qmSolidsL21Fig4}{0.2}
%
\section{Electron-electron scattering}
\index{scattering}

Reading: May not be in the text?

Electrons can scatter from other electrons, but they must conserve energy and (crystal) momentum.
%
\paragraph{\(T = 0\)}
%
%\cref{fig:qmSolidsL21:qmSolidsL21Fig5}.
\imageFigure{../figures/phy487-qmsolids/qmSolidsL21Fig5}{Filled Fermi sphere}{fig:qmSolidsL21:qmSolidsL21Fig5}{0.2}

Consider 1 electron outside a Fermi sphere.  The transition \(\ket{i} \rightarrow \ket{f}\) can only go to an empty state, within \(\delta E\) of \(\EF\).

%\cref{fig:qmSolidsL21:qmSolidsL21Fig6}.
\imageFigure{../figures/phy487-qmsolids/qmSolidsL21Fig6}{With temperature dependence state transitions still effectively confined to range of energies}{fig:qmSolidsL21:qmSolidsL21Fig6}{0.2}

Must scatter off an electron that starts inside \(\EF\), ends outside \(\EF\), and it's \(\Delta E \simeq \delta E\).  eg. (a) to (b).

So both scattering events are restricted by \(\delta E\), or
\boxedEquation{eqn:condensedMatterLecture21:240}{
\inv{\tau} \propto (\delta E)^2.
}
A state at \(\EF\) has infinite lifetime.

At \(T > 0\), the argument is similar.

F6

\(\ket{i}\) has a choice of states from within \(\kB T\) of \(\EF\) to scatter to.  States within \(\kB T\) of \(\EF\) to scatter from.
%
\begin{equation}\label{eqn:condensedMatterLecture21:260}
\inv{\tau} \propto T^2, \qquad \mbox{max\((T^2, (\delta E)^2)\)},
\end{equation}
%
or
\boxedEquation{eqn:condensedMatterLecture21:280}{
\rho(T) = \rho_\nought + A T^2.
}
This is \textunderline{universal}.  This is a famous result, from \dquoteAndIndex{Fermi liquid theory}, developed by Landau and Fermi.
%\EndArticle
