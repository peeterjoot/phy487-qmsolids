%
% Copyright � 2013 Peeter Joot.  All Rights Reserved.
% Licenced as described in the file LICENSE under the root directory of this GIT repository.
%
\makeoproblem{Orbitals and bonding}{condensedMatter:problemSet1:1}{2013 ps1 p1}{
({\em The reading assignments covered sections 1, 1.1, 1.5 and 1.6 of Ibach and Luth.})

\makesubproblem{Explain $5s$ $4d$ orbital filling ordering.}{condensedMatter:problemSet1:1a}

In a hydrogenic atom (nuclear charge \(Ze\), only one electron)
the \(4d\) levels have a lower energy than the
\(5s\) levels.  According to the periodic table, however, the \(5s\) levels become
occupied before the \(4d\) levels do. Explain why.

\makesubproblem{Explain power of Van der Waals potential.}{condensedMatter:problemSet1:1b}

Briefly explain why the Van der Waals potential has a \(1/r^6\) dependence.

} % makeproblem

\makeanswer{condensedMatter:problemSet1:1}{
\makeSubAnswer{}{condensedMatter:problemSet1:1a}

The \(4d\) levels has a lower energy than the \(5s\) level in a single electron system.  In a multiple electron system completely filled (and perhaps partially filled) orbitals have the effect of shielding additional electrons from a subset of the nuclear charge, reducing the effective total charge of the system with respect to that additional electron.  This shifts the \(d\), \(p\), \(f\) orbitals up in the staircase like fashion illustrated in class, and
\textunderline{puts $5s$ below $4d$} in such multiple electron systems.

\paragraph{Grading remark:} Re: underlined text above: ``Why should the screening have less of an effect on the \(5s\) orbital than the \(4d\) orbital?''  The posted solutions explain that the underlying reason for this difference is not screening, but because the \(s\) orbitals have zero angular momentum (i.e. s orbital corresponds to the angular momentum quantum number \(l = 0\)).  This is then used to argue that the electron distribution closer to the nucleus for such orbitals and lowers the energy required to fill these states.  FIXME: followup on this argument with some calcuations to remove the handwaving from this argument.

This simple shielding description breaks down in a number of cases, as there are multi-body interactions at play here too.  This results in a number of exceptions in the ordering of the \(5s\), \(4d\) filling.  For example \ce{Pd} in its ground state has the \(4d\) orbitals completely filled (\(4 d^{10}\)) with no \(5s\) state electrons.  We see the resumption of the \(5s\) orbital filling in the subsequent \ce{Ag} and \ce{Cd} atoms retaining the completely filled \(4 d^{10}\) orbitals that we see first in \ce{Pd}.  This can be loosely described by stating it is more stable (lower energy) to have a single set of completely filled \(4 d^{10}\) orbitals, than a have a pair of partially filled orbitals in both states like \(4 d^9 5s^1\).

\makeSubAnswer{}{condensedMatter:problemSet1:1b}

Our text \citep{ibach2009solid} \S 1.6 does a loose and fast description of the \(1/r^6\) \underlineAndIndex{Van der Waals} dependence, indicating that it is due to oscillation of a dipole field that has a \(1/r^3\) dependence.  When the neighboring atom has a polarizability \(\alpha\) its stated that there is a secondary dipole induced in this neighbor that also has a \(1/r^3\) dependence, but proportional to both the initiating field and the new dipole.

To translate from this descriptive rationalization of the \(1/r^6\) dependence, I found it helpful to remind myself of the specific form of this dipole dependence.  In \citep{jackson1975cew} \S 9.2 it is argued that a charge oscillation (i.e. charge and current density changes of the form \(\rho(\Bx) e^{-i \omega t}\)) lead to an electric field of the form
%
\begin{subequations}
\begin{dmath}\label{eqn:condensedMatterProblemSet1:20}
\Bp = \int \Bx' \rho(\Bx') d^3 x'
\end{dmath}
\begin{dmath}\label{eqn:condensedMatterProblemSet1:40}
\BE =
k^2 \lr{ \Bn \cross \Bp } \cross \Bn \frac{e^{ikr}}{r}
+
\lr{
3 \Bn (\Bn \cdot \Bp) - \Bp)
}
\lr{
\inv{r^3} - \frac{i k}{r^2}
}
e^{ikr}
\end{dmath}
\end{subequations}
%
In particular observe that the near field (\(r \sim 0\)) is dominated by
%
\begin{dmath}\label{eqn:condensedMatterProblemSet1:60}
\BE =
\lr{
3 \Bn (\Bn \cdot \Bp) - \Bp)
}
\inv{r^3}.
\end{dmath}
%
Referring to \citep{griffiths1999introduction} for a definition of polarizability, we find that polarizability is a dipole electric-field proportionality
%
\begin{dmath}\label{eqn:condensedMatterProblemSet1:80}
\Bp = \alpha \BE.
\end{dmath}
%
So if the second atom has polarizability \(\alpha\), the dipole field due to its internal dipole moment (say \(\Bp' = \alpha \BE\), with respect to normal \(\Bn'\)), then the electric field from this second atom (directed back towards the original oscillating atomic dipole and others) is, in the near field approximation
%
\begin{dmath}\label{eqn:condensedMatterProblemSet1:100}
\BE' =
\lr{
3 \Bn' (\Bn' \cdot \BE) - \BE
}
\frac{\alpha}{r^3}
=
\lr{
3 \Bn' (\Bn' \cdot \Bn) (\Bn \cdot \Bp)
-3 \Bn' (\Bn' \cdot \Bp)
- \Bn (\Bn \cdot \Bp) + \Bp
}
\frac{3 \alpha}{r^6}.
\end{dmath}
%
Here we see explicitly the \(1/r^6\) dependence of the field due to both the oscillation of the dipole \(\Bp\) of the atom itself, as well the polarizability \(\alpha\) of its neighbor.
}
