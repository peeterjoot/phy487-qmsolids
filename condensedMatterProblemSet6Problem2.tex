%
% Copyright � 2013 Peeter Joot.  All Rights Reserved.
% Licenced as described in the file LICENSE under the root directory of this GIT repository.
%
\makeoproblem{Specific heat, 1-2D free electron metals.}{condensedMatter:problemSet6:2}{\citep{ibach2009solid} pr 6.1}{
\makesubproblem{}{condensedMatter:problemSet6:2a}
Calculate the density of states for a two-dimensional gas of free electrons in a so-called quantum well.  The boundary conditions for the electronic wavefunction are:
%
\begin{equation}\label{eqn:condensedMatterProblemSet6Problem2:20}
\psi(x, y, z) = 0, \quad \mbox{for \(\Abs{x} > a\)},
\end{equation}
%
where \(a\) is of atomic dimensions.
\makesubproblem{}{condensedMatter:problemSet6:2b}
Calculate the density of states for a one-dimensional gas of free electrons in a so-called quantum wire, the boundary conditions:
%
\begin{equation}\label{eqn:condensedMatterProblemSet6Problem2:40}
\psi(x, y, z) = 0, \quad \mbox{for \(\Abs{x} > a\), and \(\Abs{y} > b\)},
\end{equation}
%
where \(a\) and \(b\) are of atomic dimensions.
\makesubproblem{}{condensedMatter:problemSet6:2c}
Can such electron gases be realized physically?
\paragraph{Clarification}
The question says something rather obscure about \(\psi(x,y,z)=0\)
for \(\Abs{x}>a\), where \(a\) is of atomic dimensions.  What they mean
is that a two-dimensional electron gas can be thought of as
contained in a three-dimensional box, with two of the dimensions
being macroscopic, but the third dimension (perpendicular to the
plane of the electron gas) being microscopic.  As a result, the energy
levels are quasi-continuous in two dimensions, but discrete in the
third dimension, and we assume that only the ground state of the
discrete spectrum is occupied, so there is no summation over
that third dimension. Similarly, in the one-dimensional electron gas, only
one direction in \(q\) space is quasi-continuous.)
} % makeproblem

\makeanswer{condensedMatter:problemSet6:2}{
\makeSubAnswer{}{condensedMatter:problemSet6:2a}
Consider a pizza box configuration where \(L \gg a\) as in \cref{fig:qmSolidsPs6P2:qmSolidsPs6P2Fig1}.
\imageFigure{../figures/phy487-qmsolids/qmSolidsPs6P2Fig1}{Particle in a pizza box.}{fig:qmSolidsPs6P2:qmSolidsPs6P2Fig1}{0.3}

Schr\"{o}dinger's equation for the particle inside the box is
%
\begin{equation}\label{eqn:condensedMatterProblemSet6Problem2:60}
-\frac{\Hbar^2}{2m} \spacegrad^2 \Psi + V_\nought \Psi = E \Psi.
\end{equation}
%
Solving by separation of variables with \(\Psi = X Y Z\), we have
%
\begin{dmath}\label{eqn:condensedMatterProblemSet6Problem2:80}
\frac{X''}{X}
+\frac{Y''}{Y}
+\frac{Z''}{Z}
=
- \frac{2 m}{\Hbar^2} (E - V_\nought).
\end{dmath}
%
Assuming independent solutions
%
\begin{equation}\label{eqn:condensedMatterProblemSet6Problem2:100}
\begin{aligned}
X'' &= - k_x^2 X \\
Y'' &= - k_y^2 Y \\
Z'' &= - k_z^2 Z,
\end{aligned}
\end{equation}
%
we have
%
\begin{equation}\label{eqn:condensedMatterProblemSet6Problem2:120}
k_x^2 + k_y^2 + k_z^2 = \frac{2 m}{\Hbar^2} (E - V_\nought).
\end{equation}
%
We could write \(E' = E - V_\nought\), but this is equivalent to setting the ground to zero (i.e. \(V_\nought = 0\)), so let's just do that, dispensing with any primes on the energy.

Our solution is
%
\begin{dmath}\label{eqn:condensedMatterProblemSet6Problem2:140}
\Psi = A
\sin\lr{ k_x (x - a)}
\sin\lr{ k_y y }
\sin\lr{ k_z z },
\end{dmath}
%
where
%
\begin{equation}\label{eqn:condensedMatterProblemSet6Problem2:160}
E(\Bk) = \frac{\Hbar^2 \Bk^2}{2m},
\end{equation}
%
The boundary constraints are
%
\begin{equation}\label{eqn:condensedMatterProblemSet6Problem2:180}
\begin{aligned}
\Psi(x = a) = \Psi(x = -a) &= 0 \\
\Psi(y = 2 L) = \Psi(y = 0) &= 0 \\
\Psi(z = 2 L) = \Psi(z = 0) &= 0,
\end{aligned}
\end{equation}
%
so for integers \(q, r, s\), we have
%
\begin{equation}\label{eqn:condensedMatterProblemSet6Problem2:200}
\begin{aligned}
k_x 2 a &= q \pi \\
k_y 2 L &= r \pi \\
k_z 2 L &= s \pi,
\end{aligned}
\end{equation}
%
Normalizing the wave function gives us
%
\begin{dmath}\label{eqn:condensedMatterProblemSet6Problem2:220}
\Psi = \inv{\sqrt{a} L}
\sin\lr{ \frac{q \pi (x - a)}{2 a}}
\sin\lr{ \frac{r \pi y}{2 L}}
\sin\lr{ \frac{s \pi z}{2 L}}.
\end{dmath}
%
The k-points in momentum space are illustrated in \cref{fig:qmSolidsPs6P2:qmSolidsPs6P2Fig2}.  However, since we are considering the particle to be in the ground state for the \(x\) direction, only the k-space points in the plane, separated by \(\pi/2L\) in each direction, will contribute to the density of states.
%
\imageFigure{../figures/phy487-qmsolids/qmSolidsPs6P2Fig2}{K space points for pizza box wavefunction.}{fig:qmSolidsPs6P2:qmSolidsPs6P2Fig2}{0.3}

We evaluate the density of states in the same fashion as we did in class
%
\begin{dmath}\label{eqn:condensedMatterProblemSet6Problem2:240}
\sum_{k_y, k_z, \mathrm{spin}}
\rightarrow
\sum_{\mathrm{spin}}
\int \frac{d^2 \Bk}{ \lr{\pi/2L}^2 }
=
\mathLabelBox
[
   labelstyle={below of=m\themathLableNode, below of=m\themathLableNode}
]
{2}{spin, 2 states per k point}
\frac{A}{\pi^2}
\int
\mathLabelBox{\inv{4}}{1 quadrant}
2 \pi k dk
=
\frac{A}{\pi} \int k dk.
\end{dmath}
%
To convert to an energy integral, we use
%
\begin{equation}\label{eqn:condensedMatterProblemSet6Problem2:260}
dE = \frac{\Hbar^2 k}{m} dk,
\end{equation}
%
or
\begin{equation}\label{eqn:condensedMatterProblemSet6Problem2:280}
\frac{m}{\Hbar^2} dE = k dk.
\end{equation}
%
The density of states definition is
%
\begin{equation}\label{eqn:condensedMatterProblemSet6Problem2:300}
\frac{A}{\pi} \int \frac{m}{\Hbar^2} dE
\equiv A \int D(E) dE,
\end{equation}
%
so the density of states is constant in 2D
\boxedEquation{eqn:condensedMatterProblemSet6Problem2:320}{
D(E) = \frac{m \pi}{\Hbar^2}.
}
\makeSubAnswer{}{condensedMatter:problemSet6:2b}
We now want to consider a particle in the cigar box configuration of \cref{fig:qmSolidsPs6P2:qmSolidsPs6P2Fig3}.
\imageFigure{../figures/phy487-qmsolids/qmSolidsPs6P2Fig3}{Particle in a cigar box.}{fig:qmSolidsPs6P2:qmSolidsPs6P2Fig3}{0.15}
The wave function is
%
\begin{equation}\label{eqn:condensedMatterProblemSet6Problem2:340}
\Psi = \inv{\sqrt{a b L}}
\sin\lr{ \frac{q \pi (x - a)}{2 a}}
\sin\lr{ \frac{r \pi (y - b)}{2 b}}
\sin\lr{ \frac{s \pi z}{2 L}},
\end{equation}
%
and the k space points of interest are separated by \(\pi/2 L\) as before.  The density of states calculation goes
%
\begin{dmath}\label{eqn:condensedMatterProblemSet6Problem2:360}
\sum_{k_z, \mathrm{spin}}
\rightarrow
\sum_{\mathrm{spin}}
\int \frac{dk_z }{ \pi/2L }
=
2
\frac{2 L}{\pi}
\int
dk_z
=
\frac{4 L}{\pi}
\int dk_z.
\end{dmath}
%
Converting the differential to energy space, we have
%
\begin{dmath}\label{eqn:condensedMatterProblemSet6Problem2:380}
dk_z
=
\frac{m}{\Hbar^2} \frac{dE}{k_z}
=
\frac{m}{\Hbar^2} \frac{dE}{\sqrt{ 2 m E}/\Hbar}
=
\inv{\Hbar} \sqrt{\frac{m}{2 E}} dE.
\end{dmath}
%
Inserting this gives us
%
\begin{dmath}\label{eqn:condensedMatterProblemSet6Problem2:400}
L \int D(E) dE =
\frac{4 L}{\pi}
\int
\inv{\Hbar} \sqrt{\frac{m}{2 E}} dE,
\end{dmath}
%
or
\boxedEquation{eqn:condensedMatterProblemSet6Problem2:420}{
D(E) = \frac{2}{\pi} \sqrt{\frac{2m}{\Hbar^2}} E^{-1/2}.
}
\makeSubAnswer{}{condensedMatter:problemSet6:2c}
In microelectronics the scale of charge carrying conductive pathways are reduced so much that ``wires'' can exhibit quantum mechanical effects.  When the scale of such wires are reduced enough, modeling that conduit as a 1D particle in a box as above is likely possible.  A carbon nanotube is likely another possible physical implementation of a 1D particle in a box.

Conducting and semiconducting plane layers of the same sort of microelectronics  can likely be reduced to scales that would allow for a 2D particle in a pizza box configuration for which we calculated the density of states above.  With planar configuration and electron conduction, graphene is also likely a physical implementation of this potential configuration.
}
