%
% Copyright � 2013 Peeter Joot.  All Rights Reserved.
% Licenced as described in the file LICENSE under the root directory of this GIT repository.
%
\makeoproblem{Temperature dependent carrier density of a doped semiconductor}{condensedMatter:problemSet10:2}{2013 ps10 p2}{
Germanium has an energy gap of \(E_{\txtC} - E_{\txtV} = 0.67 \Unit{e V}\), and an intrinsic carrier density \(n_i = 2.5 � 10^{19} \Unit{m^{-3}}\) at room temperature.  A sample of germanium is doped with arsenic, which has a donor level located at \(E_{\txtd} = 0.0127 \Unit{e V}\) below \(E_{\txtC}\). The concentration of arsenic atoms is \(N_{\txtD} = 1.0 \times 10^{22} \Unit{m^{-3}}\).

Show the temperature dependence of the carrier density in this sample by drawing a plot of \(\ln(n)\) vs. \(1/T\), between \(1 \Unit{K}\) and \(293 \Unit{K}\) (room temperature), and a second plot covering the temperature range from \(100 \Unit{K}\) to \(1000 \Unit{K}\). Explain the main features of the plots.
} % makeproblem

\makeanswer{condensedMatter:problemSet10:2}{

In class we didn't carry the doping discussion far enough for this problem, but  our text treats the n-type semiconductor, arriving eventually at

\begin{dmath}\label{eqn:condensedMatterProblemSet10Problem2:20}
n \approx 2 N_{\txtD}
\lr{ 1 + \sqrt{ 1 + 4 \frac{N_{\txtD}}{N_{\mathrm{eff}}^{\txtC}} e^{E_{\txtd}/\kB T}} }^{-1},
\end{dmath}

For this problem, noting that \(\kB \sim 0.086 \Unitfrac{eV}{K}\), we have

\begin{dmath}\label{eqn:condensedMatterProblemSet10Problem2:40}
\frac{E_{\txtd}}{\kB} \sim 147 \Unit{K},
\end{dmath}

so the exponential is large for small temperatures.  How about the ratio \(N_{\txtD}/N_{\mathrm{eff}}^{\txtC}\)?  For \ce{Ge}, \S 12.3 says we have \(m_n^\conj \sim 0.13 m_e\), so

\begin{dmath}\label{eqn:condensedMatterProblemSet10Problem2:60}
N_{\mathrm{eff}}^{\txtC}
\sim 2 \lr{ \frac{2 \pi \times 0.13 m_{\txte} \times \kB T}{h^2} }^{3/2}
\sim 4 \times 10^3 T^{3/2} \Unit{m^{-3}}.
\end{dmath}

\paragraph{FIXME: Grading remark:}
Review the \nbref{ps10plots.nb} numerical calculation (stated above).  Correct answer is supposed to be \(\sim 4 \times 10^{20}\).  The failure of UnitSimplify is likely related to that large difference.

This is plotted in \cref{fig:ps10p2:ps10p2Fig2}.

\mathImageFigure{../figures/phy487-qmsolids/ps10p2Fig2}{\(N_{\mathrm{eff}}^{\txtC}\) for \ce{Ge}}{fig:ps10p2:ps10p2Fig2}{0.2}{ps10plots.nb}

Combining the exponential and power dependent terms, using non-dimensionalized temperature \(t = t/(1 \Unit{K})\), we have

\begin{dmath}\label{eqn:condensedMatterProblemSet10Problem2:80}
4 \frac{N_{\txtD}}{N_{\mathrm{eff}}^{\txtC}} e^{E_{\txtd}/\kB T}
\sim \frac{9 \times 10^{18}}{t^{3/2}} e^{147/t} \Unit{m^{-3}}.
\end{dmath}

In the \(T \in [1, 293]\) interval, this ranges from \(10^{83}\) down to \(10^{15}\) as plotted in \cref{fig:ps10p2:ps10p2Fig3}.

\mathImageFigure{../figures/phy487-qmsolids/ps10p2Fig3}{Illustrating scale of term in square root}{fig:ps10p2:ps10p2Fig3}{0.2}{ps10plots.nb}

\paragraph{Grading remark:} ``Don't forget the intrinsic carriers!''

This justifies a \(4 \frac{N_{\txtD}}{N_{\mathrm{eff}}^{\txtC}} e^{E_{\txtd}/\kB T} \gg 1\) approximation of \eqnref{eqn:condensedMatterProblemSet10Problem2:20}

\begin{dmath}\label{eqn:condensedMatterProblemSet10Problem2:100}
n
\approx
2 N_{\txtD}
\lr{ 1 + \sqrt{4 \frac{N_{\txtD}}{N_{\mathrm{eff}}^{\txtC}} e^{E_{\txtd}/\kB T}} }^{-1}
\approx
\frac{
2 N_{\txtD}
}
{
2
\sqrt{
   N_{\txtD}
   N_{\mathrm{eff}}^{\txtC}
}
e^{E_{\txtd}/2 \kB T}
}
=
\sqrt{
\frac{
N_{\txtD}
}
{
N_{\mathrm{eff}}^{\txtC}
}
}
e^{-E_{\txtd}/2 \kB T}.
\end{dmath}

Taking logarithms we have

\begin{dmath}\label{eqn:condensedMatterProblemSet10Problem2:120}
\ln n
\sim \inv{2} \ln N_{\txtD} + \inv{2} \ln N_{\mathrm{eff}}^{\txtC} - \frac{E_{\txtd}}{2 \kB T}.
\sim 11 \ln 10 + \inv{2} \ln 4 + \frac{3}{2} \ln 10
- \frac{3}{4} \ln \inv{T} - 73 \inv{T}
\approx
29
- \frac{3}{4} \ln \inv{T} - 73 \inv{T}.
\end{dmath}

This is plotted for \(T^{-1} \in [1/293, 1]\) in \cref{fig:ps10p2:ps10p2Fig1}, along the nearly linear asymptote (with slope \(-E_{\txtd}/2\kB\)).  The lowest temperature range in this plot is the \dquoteAndIndex{freeze out} range where a large number of the donors still retain their electrons.

\mathImageFigure{../figures/phy487-qmsolids/ps10p2Fig1}{Concentration log vs inverse temperature}{fig:ps10p2:ps10p2Fig1}{0.2}{ps10plots.nb}

Zooming in on the high temperature domain by plotting \(T^{-1} \in [1/1000, 1/100]\) in \cref{fig:ps10p2:ps10p2Fig4}.  In this range we have \(4 \frac{N_{\txtD}}{N_{\mathrm{eff}}^{\txtC}} e^{E_{\txtd}/\kB T}\) taking values in the range \([10^{14}, 10^{16}]\), with \(n \in [10^{14}, 10^{15}] \Unit{m^{-3}}\).  That is probably close enough to flat that we can still describe this as \textunderline{approaching} the saturation range of \citep{ibach2009solid} (\texteqnref{12.28}), at which point ``the concentration of donor electrons in the condition band has reached the maximum possible value''.

\mathImageFigure{../figures/phy487-qmsolids/ps10p2Fig4}{Concentration log vs inverse temperature for higher temperatures}{fig:ps10p2:ps10p2Fig4}{0.2}{ps10plots.nb}

\paragraph{FIXME: Grading remark:} Spike for the intrinsic carriers added to my plot with comment ``intrinsic carriers appear \(\sim 0.002\)''.
}
