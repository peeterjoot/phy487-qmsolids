%
% Copyright � 2013 Peeter Joot.  All Rights Reserved.
% Licenced as described in the file LICENSE under the root directory of this GIT repository.
%
\makeoproblem{Bulk modulus}{condensedMatter:problemSet6:3}{\citep{ibach2009solid} pr 6.3}{
The bulk modulus \(\kappa\) is given by the second derivative of the total energy \(E_{\mathrm{tot}}\) with respect to the volume
%
\begin{dmath}\label{eqn:condensedMatterProblemSet6Problem3:10}
\kappa = V \PDSq{V}{E_{\mathrm{tot}}}.
\end{dmath}
%
Estimate the bulk moduli of alkali metals by assuming that the total energy is equivalent to the kinetic energy of the Fermi gas.  What has been neglected in this estimate?
} % makeoproblem
\makeanswer{condensedMatter:problemSet6:3}{
We need to start by computing the total energy.  At \(T = 0\), utilizing the step nature of the Fermi-Dirac distribution, this is approximately
%
\begin{dmath}\label{eqn:condensedMatterProblemSet6Problem3:30}
E_{\mathrm{tot}}
=
2 \frac{V}{(2 \pi)^3} \int_{\Abs{\Bk} < \kF} d^3 \Bk \frac{\Hbar^2 \Bk^2}{2m}
=
\frac{V}{4 \pi^3}
\frac{\Hbar^2 }{2m}
\int_0^\kF 4 \pi k^4 dk
=
\frac{V}{\pi^2}
\frac{\Hbar^2 }{2m}
\frac{\kF^5}{5}
=
\frac{V \Hbar^2}{10 m \pi^2}
\lr{
3 \pi^2 \frac{N}{V}
}^{5/3}
=
\frac{\Hbar^2}{10 m \pi^2}
\lr{
3 \pi^2 N
}^{5/3} V^{-2/3}.
\end{dmath}
%
This form is what we want to take derivatives with respect to \(V\).  We can however, simplify this total energy expression by expressing it in terms of the Fermi energy
%
\begin{dmath}\label{eqn:condensedMatterProblemSet6Problem2:90}
E_{\mathrm{tot}}
= \frac{V}{5 \pi^2} 3 \pi^2 \frac{N}{V} \frac{\Hbar^2}{2m} \lr{ 3 \pi^2 n }^{2/3}
=
\frac{3}{5} N \EF.
\end{dmath}
%
Taking derivatives, we have
%
\begin{dmath}\label{eqn:condensedMatterProblemSet6Problem3:70}
\kappa
=
V \PDSq{V}{ E_{\mathrm{tot}} }
=
V \frac{\Hbar^2}{10 m \pi^2}
\lr{
3 \pi^2 N
}^{5/3}
\frac{d^2}{dV^2} V^{-5/3 + 1}
=
\frac{\Hbar^2}{10 m \pi^2}
\lr{
3 \pi^2 N
}^{5/3}
\lr{ \frac{-2}{3} }
\lr{ \frac{-5}{3} }
V^{-2/3 -2 + 1}.
\end{dmath}
%
Factoring out the total energy of the free valence electrons \(E_{\mathrm{tot}}\), or equivalently, the Fermi energy, this is
\boxedEquation{eqn:condensedMatterProblemSet6Problem3:110}{
\kappa
=
\frac{10}{9} \frac{E_{\mathrm{tot}}}{V}
=
\frac{2}{3} n \EF.
}
\paragraph{What did we assume and neglect?} We have implicitly treated the Alkali metals as a free electron gas, assuming that there was no interaction between the electrons and the nuclei, and no interactions between the electrons themselves (except for Pauli exclusion interaction).  We have assumed that only the single outermost (valence) electron of each atom contributes to this energy sum.  We have also assumed that the system is big enough that surface effects are irrelevant, and that we can make a continuum approximation for the summation over all the wave-vector states.  We have also assumed that there are no differences to the density of the metal (and thus valence electron density \(n\)) at absolute zero compared to other temperatures.
}
