%
% Copyright � 2013 Peeter Joot.  All Rights Reserved.
% Licenced as described in the file LICENSE under the root directory of this GIT repository.
%
\makeoproblem{Systematic trends in the Debye temperature}{condensedMatter:problemSet5:2}{2013 ps5 p2}{
%:} (worth 5 marks)
Table 5.1 on page
120 of Ibach and Luth shows the Debye temperature for
various solids.  Discuss and explain any trends that you see in the Debye
temperature, e.g.\ as a function of location in the periodic table,
bonding type, or atomic mass.
} % makeproblem

\makeanswer{condensedMatter:problemSet5:2}{
A plot of Debye temperatures by atomic number can be found in \cref{fig:DebyeTemperaturesVsAtomicNumber:DebyeTemperaturesVsAtomicNumberFig1}.  This is based on data from \citep{knowledgedoor:debye}, and \citep{ibach2009solid}
%
\mathImageFigure{../figures/phy487-qmsolids/DebyeTemperaturesVsAtomicNumberFig1}{Debye temperature vs atomic number.}{fig:DebyeTemperaturesVsAtomicNumber:DebyeTemperaturesVsAtomicNumberFig1}{0.4}{deybeTemperatureTable.nb}
%
\paragraph{Some observed trends}
\begin{itemize}
\item There is a general trend of decreasing Debye temperature with atomic number.
\item Lowest Debye temperatures are often at points where we have completely filled or half filled orbitals: \ce{H} (\(1s^1\)), \ce{Ne} (\(1s^2 2s^2 1p^6\)), \ce{Eu}(\([Xe]4f^7\)), \ce{Yb} (\([Xe] 4 f^{14}\)), \ce{Hg} (\([Xe] 4f^{14} 5d^{10} 6s^2\)).
\item We see peak temperatures around elements that are near the middles of their respective orbital filling ranges: \ce{C}, \ce{Si} (p block elements), \ce{Cr}, \ce{Ru}, \ce{Os} (d-block elements).  Carbon in its diamond form is plotted above (its graphite form comes in much lower at \(420 K\)).
%\item There are some exceptions to the above with local minimum Debye temperatures occuring at points that are close to completely filled: \ce{Fr} (\([Rn]7s^1\))
\end{itemize}
%
\paragraph{Comments}
%
The capability of the element for making strong bonds appears to contribute significantly to high Debye temperatures.  In particular observe that the diamond form of \ce{C}, with its strong highly directional covalent bonds, has the highest Debye temperature.  \ce{Si} also in the p block with 4 available p orbital slots has a very high Debye temperature, at least compared to its period table neighbors.   The converse is also evident, since we see lack of bonding capability associated with low Debye temperatures for those elements that have completely and half filled orbitals, which have some stability in isolation.  This is similar to the previously observed low melting points (a measure of ease of lattice breakup) for elements that have half and completely filled orbitals.

Recall that the Debye frequency (proportional to the Debye temperature) had the form
%
\begin{dmath}\label{eqn:condensedMatterProblemSet5Problem2:20}
\omega_{\txtD} \propto \lr{\frac{N}{V}}^{1/3}.
\end{dmath}
%
Based on this, we expect to see small Debye temperatures when the number density is low, which should occur when the atomic radii is large.  That can be observed in \cref{fig:atomicRadiiVsAtomicNumber:atomicRadiiVsAtomicNumberFig2}, looking for example at \ce{K}, \ce{Rb}, and \ce{Cs}, that are positioned at local maximums for atomic radii, in contrast to the local minimums observed for the Debye temperature.
%
\mathImageFigure{../figures/phy487-qmsolids/atomicRadiiVsAtomicNumberFig2}{Atomic radii.}{fig:atomicRadiiVsAtomicNumber:atomicRadiiVsAtomicNumberFig2}{0.4}{deybeTemperatureTable.nb}

This plot has some confusing aspects as-is since I didn't label all the elements, just the ones that I wanted to compare to the Debye temperatures (if you look carefully the labels for Cl, Br, I are shifted to the left slightly).  It also appears that I didn't explicitly plot those elements for which I didn't have Debye temperature data, which makes it even more misleading if looking at just the radius periodicity.

In \cref{fig:atomicRadiusAndDebyeTempOverlapped:atomicRadiusAndDebyeTempOverlappedFig3} is a combined plot of both the atomic radius and the Debye temperature.  In this second plot we see that, yes, the largest radii are those with the smallest Debye temperatures.  As the radius drops from the peak, the Debye temperature increases.  However, part way towards the middle of the period, this inverse relationship starts to fail.  In fact, they both start trending downwards at these points.  Is this where the velocities of the accoustic modes, also variables in the Debye temperatures, start to factor into the mix?
%
\mathImageFigure{../figures/phy487-qmsolids/atomicRadiusAndDebyeTempOverlappedFig3}{Debye and atomic radius.}{fig:atomicRadiusAndDebyeTempOverlapped:atomicRadiusAndDebyeTempOverlappedFig3}{0.4}{deybeTemperatureTable.nb}

As a final plot, let's look at the inverse of the atomic radius and the Debye temperature together.  This is plotted in \cref{fig:figuresatomicInvRadiusAndDebyeTempOverlapped:atomicInvRadiusAndDebyeTempOverlappedFig4}.
%
\mathImageFigure{../figures/phy487-qmsolids/atomicInvRadiusAndDebyeTempOverlappedFig4}{Inverse Atomic Radius and Debye temperature.}{fig:figuresatomicInvRadiusAndDebyeTempOverlapped:atomicInvRadiusAndDebyeTempOverlappedFig4}{0.4}{deybeTemperatureTable.nb}
}
