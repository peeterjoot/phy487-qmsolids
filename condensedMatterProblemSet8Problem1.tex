%
% Copyright � 2013 Peeter Joot.  All Rights Reserved.
% Licenced as described in the file LICENSE under the root directory of this GIT repository.
%
\makeoproblem{Tight binding, square lattice.}{condensedMatter:problemSet8:1}{2013 ps8 p1}{
Consider a two-dimensional square lattice with lattice parameter \(a\),
and thus basis vectors \((a,0)\) and
\((0,a)\).  We will construct a
tight binding band from an
\(s\)-orbital
\(\phi_s\) that is a solution
of the Schrodinger equation for the isolated atom, with eigenvalue
\(E_s\):
\(\hat{\calH_A}(\Br - \Br_n) \phi_s(\Br - \Br_n) = E_s\phi_s(\Br - \Br_n)\), where
\(\Br_n\) is a lattice vector.
\makesubproblem{}{condensedMatter:problemSet8:1a}
If the tight binding integrals (defined in class) are
\begin{eqnarray*}
A &\equiv& - \int\, d\Br\, \phi_s^*(\Br - \Br_n) v(\Br - \Br_n)
                          \phi_s(\Br - \Br_n)    \ \ \ \mathrm{and} \\
B &\equiv& - \int\, d\Br\, \phi_s^*(\Br - \Br_n \pm (a,0) ) v(\Br - \Br_n)
                          \phi_s(\Br - \Br_n) = B_x  \\
  &=& - \int\, d\Br\, \phi_s^*(\Br - \Br_n \pm (0,a) ) v(\Br - \Br_n)
                          \phi_s(\Br - \Br_n) = B_y,
\end{eqnarray*}
show that
\begin{eqnarray*}
E(\Bk) \simeq E_s - A - 2B(\cos(k_x a) + \cos(k_ya)).
\end{eqnarray*}
%
\makesubproblem{}{condensedMatter:problemSet8:1b}
Plot \(E(\Bk)\) along the following lines in
\(k\)-space:
  (i) from \(\Bk = (0,0)\) to
\((2\pi/a, 0)\);
  (ii) from \(\Bk = (0,0)\) to
\((0,2\pi/a)\);
  (iii) from \(\Bk = (0,0)\) to
\((2\pi/a, 2\pi/a)\).
\makesubproblem{}{condensedMatter:problemSet8:1c}
What is the bandwidth of this tight-binding band, as a multiple of \(B\)?
\makesubproblem{}{condensedMatter:problemSet8:1d}
Plot contours of constant energy in the first Brillouin zone (i.e.\ the
\((k_x,k_y)\) plane, using only the first Brillouin zone), for the following
energies: (i)
\(E = E_s - A - 2B\); (ii)
\(E = E_s - A - B\); (iii)
\(E = E_s - A\);
(iv) \(E = E_s - A + 2B\). You may use a plotting package, or plot by hand by
calculating
\(k_x\) and
\(k_y\) along a few directions in
\(k\)-space and then
interpolating.

Use these plots to identify which constant energy contour represents the
``half-filled state" (the state where, if all of the levels up to \(E = \EF\) are filled
then there is one electron per site, or \(N\) electrons in total,
where \(N\) is the number of atoms in the lattice).
\makesubproblem{}{condensedMatter:problemSet8:1e}
By considering how the \(A\) and
\(B\) integrals would be affected, discuss in qualitative
terms how the \(E(k)\) relation changes if, instead of atomic
\(s\)-orbitals, the
basis functions for this band are \(p_x\)-orbitals.
Sketch contours of constant
energy as the `filling' of the band changes from \(\EF\) near the bottom of the
band, to \(\EF\) near the top of the band. [9 marks]
%
\paragraph{Notes and Hints:}  Note that the \(p_x\) orbitals break
the 90 degree rotational symmetry, so now \(B_y \ne B_x\).
If the lattice parameter is large,
so that there is weak overlap as is assumed in tight-binding calculations, then
you will have \(\Abs{B_y} \ll \Abs{B_x}\).  Explain why.
Note too that the \(p_x\)-orbitals have odd-parity, compared
with the even-parity of the \(s\)-orbitals.
You may find it helpful, in discussing the \(B\) orbitals, to sketch the \(p_x\) orbitals
on neighbouring atoms, to visualize how they overlap.
} % makeproblem
\makeanswer{condensedMatter:problemSet8:1}{
\makeSubAnswer{}{condensedMatter:problemSet8:1a}

In class (or \citep{ibach2009solid} (\texteqnref{7.37})) we found for the tight binding energy at the lattice point at \(\Br_n\)
%
\begin{dmath}\label{eqn:condensedMatterProblemSet8Problem1:160}
E(\Bk) \approx E_i - A - B \sum_{m = \text{nn of n}} e^{i \Bk \cdot (\Br_n - \Br_m)}.
\end{dmath}
%
The nearest neighbor differences are illustrated in \cref{fig:qmSolidsPs8a:qmSolidsPs8aFig1}, which we see are
%
\imageFigure{../figures/phy487-qmsolids/qmSolidsPs8aFig1}{Cubic nearest neighbor differences.}{fig:qmSolidsPs8a:qmSolidsPs8aFig1}{0.2}
%
\begin{dmath}\label{eqn:condensedMatterProblemSet8Problem1:180}
\Br_m - \Br \in \{(0, a), (0, -a), (a, 0), (-a, 0)\}.
\end{dmath}
%
The sum of exponentials is just
%
\begin{dmath}\label{eqn:condensedMatterProblemSet8Problem1:200}
\sum_m e^{i \Bk \cdot (\Br_n - \Br_m)}
=
e^{ i (k_x, k_y) \cdot (0, -a) }
+
e^{ i (k_x, k_y) \cdot (0, a) }
+
e^{ i (k_x, k_y) \cdot (a, 0) }
+
e^{ i (k_x, k_y) \cdot (-a, 0) }
=
e^{ -i a k_y }
+e^{ i a k_y }
+e^{ -i a k_x }
+e^{ i a k_x }
= 2 \cos a k_x + 2 \cos a k_y.
\end{dmath}
%
\Eqnref{eqn:condensedMatterProblemSet8Problem1:160} takes the form
%
\begin{dmath}\label{eqn:condensedMatterProblemSet8Problem1:220}
E(\Bk) \approx E_i - A - 2 B \lr{
\cos a k_x + \cos a k_y
},
\end{dmath}
%
as desired.
\makeSubAnswer{}{condensedMatter:problemSet8:1b}
\paragraph{(i)}  Parameterize this trajectory with
%
\begin{dmath}\label{eqn:condensedMatterProblemSet8Problem1:20}
\Bk(u) = \frac{2 \pi}{a} u (1, 0),
\end{dmath}
%
so the energy on this trajectory is
%
\begin{dmath}\label{eqn:condensedMatterProblemSet8Problem1:40}
E(\Bk(u))
=
E_s - A - 2 B \lr{
\cos 2 \pi u + 1
}
=
E_s - A - 2 B - 2B \cos 2 \pi u.
\end{dmath}
%
This is plotted in \cref{fig:qmSolidsPs8bi:qmSolidsPs8biFig1}.
%
\mathImageFigure{../figures/phy487-qmsolids/qmSolidsPs8biFig1}{\(E(k)\) on \(k \in [(0,0), 2 \pi (1,0)/a]\).}{fig:qmSolidsPs8bi:qmSolidsPs8biFig1}{0.24}{qmSolidsPs8biFig1Generated.nb}
%
\paragraph{(ii)}  Parameterize this trajectory with
%
\begin{dmath}\label{eqn:condensedMatterProblemSet8Problem1:60}
\Bk(v) = \frac{2 \pi}{a} v (0, 1),
\end{dmath}
%
so the energy on this trajectory is
%
\begin{dmath}\label{eqn:condensedMatterProblemSet8Problem1:80}
E(\Bk(v))
=
E_s - A - 2 B \lr{
1 + \cos 2 \pi v
}
=
E_s - A - 2 B - 2B \cos 2 \pi v.
\end{dmath}
%
This, identical to (i) in form, is plotted in \cref{fig:qmSolidsPs8bii:qmSolidsPs8biiFig2}.
%
\mathImageFigure{../figures/phy487-qmsolids/qmSolidsPs8biiFig2}{\(E(k)\) on \(k \in [(0,0), 2 \pi (0,1)/a]\).}{fig:qmSolidsPs8bii:qmSolidsPs8biiFig2}{0.24}{qmSolidsPs8biiFig2Generated.nb}
%
\paragraph{(iii)}  Parameterize this trajectory with
%
\begin{dmath}\label{eqn:condensedMatterProblemSet8Problem1:100}
\Bk(w) = \frac{2 \pi}{a} w (1, 1),
\end{dmath}
%
so the energy on this trajectory is
%
\begin{dmath}\label{eqn:condensedMatterProblemSet8Problem1:120}
E(\Bk(w))
=
E_s - A - 2 B \lr{
2 \cos 2 \pi w
}
=
E_s - A - 4 B \cos 2 \pi w.
\end{dmath}
%
This, identical to (i) and (ii) in form, but with different extremums, is plotted in \cref{fig:qmSolidsPs8biii:qmSolidsPs8biiiFig3}.
%
\mathImageFigure{../figures/phy487-qmsolids/qmSolidsPs8biiiFig3}{\(E(k)\) on \(k \in [(0,0), 2 \pi (1,1)/a]\).}{fig:qmSolidsPs8biii:qmSolidsPs8biiiFig3}{0.24}{qmSolidsPs8biiiFig3Generated.nb}
\makeSubAnswer{}{condensedMatter:problemSet8:1c}
\(E(\Bk)\) ranges from \(E_s - A - 2 B(1 + 1)\) to \(E_s - A - 2 B ( - 1 - 1)\).  That maximum difference is
%
\begin{dmath}\label{eqn:condensedMatterProblemSet8Problem1:140}
4 B - (-4B) = 8B.
\end{dmath}
%
\makeSubAnswer{}{condensedMatter:problemSet8:1d}
\paragraph{(i)  \(E = E_s - A - 2B\)} The first contour is that defined by
%
\begin{dmath}\label{eqn:condensedMatterProblemSet8Problem1:240}
E = E_s - A - 2 B \lr{ \cos k_x a + \cos k_y a } =
E_s - A - 2 B,
\end{dmath}
%
or
\begin{dmath}\label{eqn:condensedMatterProblemSet8Problem1:260}
\cos k_x a + \cos k_y a = 1.
\end{dmath}
%
\paragraph{(ii) \(E = E_s - A - B\)} Next we have the contour defined by
%
\begin{dmath}\label{eqn:condensedMatterProblemSet8Problem1:280}
\cos k_x a + \cos k_y a = \inv{2}.
\end{dmath}
%
\paragraph{(iii) \(E = E_s - A\)} This contour is defined by
%
\begin{dmath}\label{eqn:condensedMatterProblemSet8Problem1:300}
\cos k_x a + \cos k_y a = 0.
\end{dmath}
%
\paragraph{(iv)  \(E = E_s - A + 2B\)} And the last contour defined by
%
\begin{dmath}\label{eqn:condensedMatterProblemSet8Problem1:320}
\cos k_x a + \cos k_y a = -1.
\end{dmath}
%
These are plotted as functions of \(u = k_x a\) and
\(v = k_y a\) in \cref{fig:qmSolidsPs8d:qmSolidsPs8dFig1}.  These are level curves of the surface plotted in \cref{fig:qmSolidsPs8d:qmSolidsPs8dFig2}.

In class when discussing the tight binding characteristics of s-orbital alkali metals (\ce{Li}, \ce{K}, \ce{Na}, ...), it was noted that their Fermi surfaces never get close to the Brillouin boundary, and that they were approximately spherical.  All of the surfaces constrained to the bucket (\(E < E_s - A\)) are far from the Brillouin boundary, but at
\(E = E_s - A - 2B\) we are just starting to loose the ``spherical'' (aka. circular for this lattice) character of these surfaces.  We expect that the Fermi energy
\(\EF\) has an upper bound of
\(\EF = E_s - A - 2B\), a position in the bucket where the contours are still nearly circular.
%
\paragraph{Grading remark:} ``? half filled?''.  Two marks lost.  My attempt to BS this above clearly failed.
%
Prof Julian provided some helpful comments on this:

``Unlike the calculation in problem set 8, the band structure of the alkalai metals and alkalai earths is free-electron-like.  This means that the dispersion relation is parabolic, except near the Brillouin zone boundary.  So if the Fermi surface never gets close to a Brillouin zone boundary,
\(E(\kF) \simeq \Hbar^2 \kF^2/2m\), independent of direction in k-space, so the Fermi surface is spherical.

You can tell that the Fermi surface doesn't go close to the BZ boundary by calculating the volume of the Fermi sphere, compared with the volume of the Brillouin zone.  This doesn't work for the tight-binding band structure, because the dispersion is anisotropic even far from the Brillouin zone.''

As a followup it would be good to:

\begin{enumerate}
\item Try this calculation of \(\EF\) for the alkali metals, then compare to the BZ volume to verify.
\item Figure out how to calculate the density of states (and thus \(\EF ...\)).
\end{enumerate}
%
\mathImageFigure{../figures/phy487-qmsolids/qmSolidsPs8dFig1}{2D Contour plots of selected tight binding energy levels.}{fig:qmSolidsPs8d:qmSolidsPs8dFig1}{0.3}{qmSolidsPs8dContourPlot.nb}
%
\mathImageFigure{../figures/phy487-qmsolids/qmSolidsPs8dFig2}{Energy level curves.}{fig:qmSolidsPs8d:qmSolidsPs8dFig2}{0.2}{qmSolidsPs8dContourPlot.nb}
\makeSubAnswer{}{condensedMatter:problemSet8:1e}
The \(2 p_x\) orbital function in Cartesian coordinates are
%
\begin{dmath}\label{eqn:condensedMatterProblemSet8Problem1:340}
\phi_{2p_x} = \inv{\sqrt{\pi}} \lr{ \frac{ Z}{2 a_\nought} }^{5/2} e^{-Z r/a_\nought} x,
\end{dmath}
%
where \(r = \sqrt{x^2 + y^2 + z^2}\).  The overlap of these orbitals in the lattice could look something like \cref{fig:qmSolidsPs8e:qmSolidsPs8eFig1}.
%
\imageFigure{../figures/phy487-qmsolids/qmSolidsPs8eFig1}{\(2 p_x\) overlap.}{fig:qmSolidsPs8e:qmSolidsPs8eFig1}{0.15}
%
\paragraph{Grading remarks:} \(x \pm a\) is circled ``be careful with this.  Need to presem \textunderline{parity} of orbital'', and ``\(B_x\) is \textunderline{negative}''.
%
Putting the origin at \(\Br_n\), writing \(\alpha = Z/a_\nought\), and \(c^2 = (Z/2 a_\nought)^5/\pi\), the \(A\) and \(B\) integrals for this basis are
%
\begin{dmath}\label{eqn:condensedMatterProblemSet8Problem1:360}
\begin{aligned}
A &= - c^2 \int dx dy dz e^{-2 \alpha \sqrt{ x^2 + y^2 + z^2} } v(x, y, z) x^2 \\
B_x &= - c^2 \int dx dy dz e^{-\alpha \sqrt{ (x \pm a)^2 + y^2 + z^2}} \lr{x \pm a} v(x, y, z) e^{ -\alpha \sqrt{x^2 + y^2 + z^2}} x \\
B_y &= - c^2 \int dx dy dz e^{-\alpha \sqrt{ x^2 + (y \pm a)^2 + z^2}} x^2 v(x, y, z) e^{ -\alpha \sqrt{x^2 + y^2 + z^2}}  \\
\end{aligned}
\end{dmath}
%
We can rewrite these as
\begin{dmath}\label{eqn:condensedMatterProblemSet8Problem1:380}
\begin{aligned}
A &= - c^2 \int d^3 \Br v(\Br) e^{-2 \alpha r} x^2 \\
B_x &= - c^2 \int d^3 \Br v(\Br) e^{-\alpha \lr{ r + \sqrt{ r^2 \pm 2 x a + a^2} }} \lr{ x^2 \pm a x} \\
B_y &= - c^2 \int d^3 \Br v(\Br) e^{-\alpha \lr{ r + \sqrt{ r^2 \pm 2 y a + a^2} }} x^2
\end{aligned}
\end{dmath}
%
The energy then becomes
%
\begin{equation}\label{eqn:condensedMatterProblemSet8Problem1:400}
\begin{aligned}
E(\Bk)
&=
E_s - A
- B_{x_{+}} e^{ i (-a, 0) \cdot (k_x, k_y) }
- B_{x_{-}} e^{ i (a, 0) \cdot (k_x, k_y) } \\
&\quad - B_{y_{+}} e^{ i (0, -a) \cdot (k_x, k_y) }
- B_{y_{-}} e^{ i (0, a) \cdot (k_x, k_y) } \\
&=
E_s - A
- B_{x_{+}} e^{ -i a k_x }
- B_{x_{-}} e^{ i a k_x } \\
&\quad - B_{y_{+}} e^{ -i a k_y }
- B_{y_{-}} e^{ i a k_y } \\
&=
E_s
+ c^2 \int d^3 \Br v(\Br) e^{-2 \alpha r} x^2  \\
&+ c^2 \int d^3 \Br v(\Br) e^{-\alpha \lr{ r + \sqrt{r^2 - 2 x a + a^2} }} \lr{ x^2 - a x} e^{ i a k_x } \\
&+ c^2 \int d^3 \Br v(\Br) e^{-\alpha \lr{ r + \sqrt{r^2+ 2 x a + a^2} }} \lr{ x^2 + a x} e^{ -i a k_x } \\
&+ c^2 \int d^3 \Br v(\Br) e^{-\alpha \lr{ r + \sqrt{r^2 - 2 y a + a^2} }} x^2 e^{i a k_y} \\
&+ c^2 \int d^3 \Br v(\Br) e^{-\alpha \lr{ r + \sqrt{r^2+ 2 y a + a^2} }} x^2 e^{-i a k_y}
\end{aligned}
\end{equation}
%
\paragraph{Zeroth order approximation}
%
In the limiting case where \(v(\Br) \approx 0\) unless \(\Abs{\Br} \ll a\), this gives us
%
\begin{dmath}\label{eqn:condensedMatterProblemSet8Problem1:420}
E(\Bk) \approx E_s - \beta(2) - 2 \beta(1) e^{- Z a/a_\nought} \lr{ \cos k_x a + \cos k_y a},
\end{dmath}
%
where
%
\begin{dmath}\label{eqn:condensedMatterProblemSet8Problem1:440}
\beta(n) = - c^2 \int d^3 \Br v(\Br) e^{-n Z r/a_\nought} x^2.
\end{dmath}
%
In this limit we have the same ``bucket'' constant energy contours as found above, however for a constant surface
%
\begin{dmath}\label{eqn:condensedMatterProblemSet8Problem1:460}
E = E_s - E_s - \beta(2) - 2 \beta(1) \mu.
\end{dmath}
%
the contours are defined by
%
\begin{dmath}\label{eqn:condensedMatterProblemSet8Problem1:480}
\cos k_x a + \cos k_y a = \mu e^{ Z a/a_\nought}.
\end{dmath}
%
This flattens the surface as \(Z\) increases.
%
\paragraph{First order approximation}
%
Allowing for an additional order of \(r/a\) in the square root expansions above, so that, for instance
%
\begin{equation}\label{eqn:condensedMatterProblemSet8Problem1:500}
\begin{aligned}
e^{-\alpha \lr{ r + \sqrt{ r^2 - 2 x a + a^2 }} }
&\approx
\exp\lr{-\alpha \lr{ r + a \lr{ 1 + \inv{2} \frac{ r^2 - 2 x a}{a^2} } } } \\
&=
e^{-\alpha (r - x)}
e^{-a \alpha}
e^{-\alpha r^2/{2 a^2} }.
\end{aligned}
\end{equation}
%
Now we see the parity effects of the \(p_x\) orbital start to manifest.  The energy to this order of approximation is
%
\begin{dmath}\label{eqn:condensedMatterProblemSet8Problem1:520}
\begin{aligned}
E(\Bk) &=
E_s
+ c^2 \int d^3 \Br v(\Br) e^{-2 \alpha r} x^2  \\
&+ c^2 e^{-a \alpha} \int d^3 \Br v(\Br) e^{-\alpha \lr{ r - x } - \alpha r^2/a^2} \lr{ x^2 - a x} e^{ i a k_x } \\
&+ c^2 e^{-a \alpha} \int d^3 \Br v(\Br) e^{-\alpha \lr{ r + x } - \alpha r^2/a^2} \lr{ x^2 + a x} e^{ -i a k_x } \\
&+ c^2 e^{-a \alpha} \int d^3 \Br v(\Br) e^{-\alpha \lr{ r - y } - \alpha r^2/a^2} x^2 e^{ i a k_y } \\
&+ c^2 e^{-a \alpha} \int d^3 \Br v(\Br) e^{-\alpha \lr{ r + y } - \alpha r^2/a^2} x^2 e^{ -i a k_y }  \\
%&=
%E_s
%+ c^2 \int d^3 \Br v(\Br) e^{-2 \alpha r} x^2  \\
%&+ 2 c^2 e^{-a \alpha} \int d^3 \Br v(\Br) e^{-\alpha r - \alpha r^2/a^2}
%\lr{
%x^2 \cosh( \alpha x + i a k_x )
%+x^2 \cosh( \alpha y + i a k_y )
%- a x \sinh( \alpha x + i a k_x )
%}
\end{aligned}
\end{dmath}
%
In the \(B_x\) integrals we have integrands including factors of the form
\((x^2 - ax)e^{\alpha x}\) or
\((x^2 + a x)e^{-ax}\) both of which have
\(x^2 e^{a \Abs{x}}\) contributions for portions of the integral.  In the
\(B_y\) integrals we have
\(x^2 e^{a \Abs{y}}\) factors in the integrands, which will have less total contribution to the integral.  That justifies the
\(\Abs{B_x} \gg \Abs{B_y}\) condition for weak overlap.

Given this difference in magnitude, we can roughly expect that the energy can be written as approximately
%
\begin{dmath}\label{eqn:condensedMatterProblemSet8Problem1:540}
E(\Bk) = E_s - A
+ \Abs{B_x} \cos k_x a
- \Abs{B_y} \cos k_y a.
\end{dmath}
%
\paragraph{Grading remark:} Originally had \(-\Abs{B_x}\), which resulted in a squished plot instead of the one below that has a saddle.  The comment was ``negative \(B_x\) makes \((0, 0)\) a saddle point''.
%
Consider constant energy contours
%
\begin{dmath}\label{eqn:condensedMatterProblemSet8Problem1:560}
E(\Bk) = E_s - A - 2 \mu \sqrt{ B_x^2 + B_y^2 },
\end{dmath}
%
so that the constant surfaces are given by
%
\begin{dmath}\label{eqn:condensedMatterProblemSet8Problem1:580}
-\cos k_x a
+ \Abs{\frac{B_y}{B_x}} \cos k_y a =
2 \mu \sqrt{ 1 + \frac{B_y^2}{B_x^2} }.
\end{dmath}
%
This is plotted for \(\mu \in [-1, 1]\), and \(\Abs{B_y/B_x} = 0.41\) in \cref{fig:qmSolidsPs8e:qmSolidsPs8eFig2} and \cref{fig:qmSolidsPs8e:qmSolidsPs8eFig3}.
\mathImageFigure{../figures/phy487-qmsolids/qmSolidsPs8eFig2}{Sample energy contours for \(p_x\) orbital basis.}{fig:qmSolidsPs8e:qmSolidsPs8eFig2}{0.4}{ps8e.nb}
%\mathImageFigure{../figures/phy487-qmsolids/qmSolidsPs8eFig3f}{3D plot for \(p_x\) orbital basis}{fig:qmSolidsPs8e:qmSolidsPs8eFig3}{0.4}{ps8e.nb}
\mathImageFigure{../figures/phy487-qmsolids/qmSolidsPs8eFig3g}{3D plot for \(p_x\) orbital basis.}{fig:qmSolidsPs8e:qmSolidsPs8eFig3}{0.4}{ps8e.nb}
%Certainly for the positive sign of \(B_x\) (ignoring the proportional adjustment) we have
%
%\Abs{B_{x_{+}}} \ge \int d^3 \Br \Abs{-v(\Br)} e^{-\alpha r\lr{ 1 + \sqrt{ 1 + \frac{\pm 2 x a + a^2}{r^2}} }} \Abs{ x^2 \pm a x}
%\ge \int d^3 \Br \Abs{-v(\Br)} e^{-\alpha r\lr{ 1 + \sqrt{ 1 + \frac{\pm 2 x a + a^2}{r^2}} }} \Abs{ x^2 }
%
%For a negative potential \(v(\Br)\) this is approximately \(\Abs{B_y}\), so we have \(\Abs{B_x} > \Abs{B_y}\).
}
