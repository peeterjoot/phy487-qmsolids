%
% Copyright � 2013 Peeter Joot.  All Rights Reserved.
% Licenced as described in the file LICENSE under the root directory of this GIT repository.
%
\makeoproblem{\(sp^2\) hybrid orbitals.}{condensedMatter:problemSet1:2}{2013 ps1 p2}{
The \(2s\) and \(2p\) orbitals of a hydrogenic atom (i.e.\ one electron, nuclear
  charge \(Ze\)) are:
%
\begin{eqnarray*}
\phi_{2s}(\rho)   &=& {\cal N}\,{\rm e}^{-\rho}(1  - \rho) \\
\phi_{2p_z}(\rho) &=& {\cal N}\,{\rm e}^{-\rho}\rho\cos\theta \\
\phi_{2p_x}(\rho) &=& {\cal N}\,{\rm e}^{-\rho}\rho\sin\theta \, \cos\phi \\
\phi_{2p_y}(\rho) &=& {\cal N}\,{\rm e}^{-\rho}\rho\sin\theta \, \sin\phi
\end{eqnarray*}
%
where \(\rho = Zr/2a_\circ\),
\(a_\circ\) is the Bohr radius, \(r\) is the radial distance from
the nucleus, and \(\theta\) and \(\phi\) are the polar and azimuthal
angles. \({\cal N} = (Z/2a_\circ)^{3/2}/\sqrt{\pi}\) is the normalization constant.

Four \(sp^2\) hybrid orbitals are constructed from these orbitals as follows:
\begin{eqnarray*}
\psi_1  &=& \frac{1}{\sqrt{3}} \phi_{2s} + \sqrt{\frac{2}{3}} \phi_{2p_x} \\
\psi_2  &=& \frac{1}{\sqrt{3}} \phi_{2s} - \frac{1}{\sqrt{6}} \phi_{2p_x} + \frac{1}{\sqrt{2}} \phi_{2p_y} \\
\psi_3  &=& \frac{1}{\sqrt{3}} \phi_{2s} - \frac{1}{\sqrt{6}} \phi_{2p_x} - \frac{1}{\sqrt{2}} \phi_{2p_y} \\
\psi_4  &=& \phi_{2p_z}
\end{eqnarray*}
%
\makesubproblem{Orthonormality.}{condensedMatter:problemSet1:2a}
Assuming
  that the \(\phi_{2s}\) and \(\phi_{2p}\) orbitals are orthogonal and normalized (i.e.\
  you don't need to show this), show that the \(sp^2\) hybrid orbitals are also orthonormal.
\makesubproblem{Coordinates for maximum probability density.}{condensedMatter:problemSet1:2b}
Find the \(\phi\) and \(\theta\) values for which the
   probability density of the \(\psi_1\) hybrid
   orbital is maximized (i.e.\ find the direction in which this orbital is pointing).

% Had kdp file processing error (binary search isolated this to chapter I).  try commenting this out
% (here and below)
% 
\iftoggle{print-version}{}
{
\makesubproblem{Contour plots.}{condensedMatter:problemSet1:2c}
Using whatever plotting package you wish (e.g. gnuplot, Matlab, or using the
  `contour' or `contourf' functions in SciPy; and please see me
  if you don't know how to approach this question), make two-dimensional contour
  plots for the \(\psi_1\) and \(\psi_2\) hybrid orbital wave-functions, and their
  moduli, in the \(x-y\) plane
  (that is for \(\theta=\pi/2\)).
Hand in the code you used to generate the contour plots, as well as a printout of
  the plots (plots can be submitted by email if you want to submit a colour contour plot
  and you don't have a colour printer).

(I have put some hints for how do this using python with the Problem Set
1 questions on the blackboard site.)
\makesubproblem{Sigma bonding.}{condensedMatter:problemSet1:2d}

Bonus (for fun, will not be marked):
    Modify your program so that two adjacent atoms have \(sp^2\) hybrid
    orbitals directed towards each other to form a \(\sigma\) bond, and plot contours of the wave-function
    in the \(x-y\) plane.
    Once you know how to do this you can do quite a lot.  It is easy for example to put atoms on
    a honeycomb lattice, with the in-plane \(sp^2\) hybrid orbitals forming covalent bonds, and the
    resulting contour plot is a map of the wave function
    in the plane of the carbon atoms in graphene.  Or you can put two \(2p_z\) orbitals on adjacent atoms
    on the \(x\)-axis and map out the wave-function of a \(\pi\) orbital.
    (In addition to the in-plane \(sp^2\) hybrid orbitals graphene has \(\pi\) bonds between the
    \(2p_z\) orbitals on adjacent atoms.)  Or, you can look at the electron density in anti-bonding
    orbitals.  etc.  If you don't feel like doing this yourself, a bit of hacking around on the
    internet will turn up many nice examples, but it's also nice to see how easy it is to do
    it yourself.
}
} % makeproblem

\makeanswer{condensedMatter:problemSet1:2}{
\makeSubAnswer{}{condensedMatter:problemSet1:2a}
Verification of orthonormality of the \(2s\) and \(2p\) orbital basis functions is
\nbcite{qmSolidsPs1P2.nb}{easily computed in software.}

For the superposition orthonormality, let's rewrite the \(sp^2\) hybrid wave functions as
%
\begin{eqnarray*}
\psi_1  &=& \inv{\sqrt{6}} \lr{
\sqrt{2} \phi_{2s} + 2 \phi_{2p_x}
}
\\
\psi_2  &=&
\inv{\sqrt{6}} \lr{
\sqrt{2}
\phi_{2s} - \phi_{2p_x} + \sqrt{3} \phi_{2p_y}
} \\
\psi_3  &=&
\inv{\sqrt{6}} \lr{
\phi_{2s} - \phi_{2p_x} - \sqrt{3} \phi_{2p_y}
} \\
\psi_4  &=& \phi_{2p_z}.
\end{eqnarray*}
%
Now, let's verify the normalization of these wave functions
%
\begin{subequations}
\begin{equation}\label{eqn:condensedMatterProblemSet1Problem2:20}
\begin{aligned}
\braket{\psi_1}{\psi_1}
&=
\inv{6} \braket
{
\sqrt{2} \phi_{2s} + 2 \phi_{2p_x}
}
{
\sqrt{2} \phi_{2s} + 2 \phi_{2p_x}
} \\
&=
\inv{6} \lr{ \sqrt{2}^2 + 2^2 } \\
&=
\inv{6} \lr{ 2 + 4 } \\
&= 1
\end{aligned}
\end{equation}
%
\begin{equation}\label{eqn:condensedMatterProblemSet1Problem2:40}
\begin{aligned}
\braket{\psi_2}{\psi_2}
&=
\inv{6}
\braket
{
\sqrt{2} \phi_{2s} - \phi_{2p_x} + \sqrt{3} \phi_{2p_y}
}
{
\sqrt{2} \phi_{2s} - \phi_{2p_x} + \sqrt{3} \phi_{2p_y}
} \\
&=
\inv{6}
\lr{ 2 + 1 + 3 } \\
&= 1
\end{aligned}
\end{equation}
%
\begin{equation}\label{eqn:condensedMatterProblemSet1Problem2:60}
\begin{aligned}
\braket{\psi_3}{\psi_3}
&=
\inv{6}
\braket
{
\sqrt{2} \phi_{2s} - \phi_{2p_x} - \sqrt{3} \phi_{2p_y}
}
{
\sqrt{2} \phi_{2s} - \phi_{2p_x} - \sqrt{3} \phi_{2p_y}
} \\
&=
\inv{6}
\lr{ 2 + 1 + 3 } \\
&= 1
\end{aligned}
\end{equation}
%
\begin{equation}\label{eqn:condensedMatterProblemSet1Problem2:80}
\begin{aligned}
\braket{\psi_4}{\psi_4}
&=
\braket{\phi_{2 p_z}}{\phi_{2 p_z}} \\
&= 1.
\end{aligned}
\end{equation}
\end{subequations}
%
That verifies that these hybrid orbitals are all normalized.  Checking all the inner products pairwise zeros will complete the verification of orthonormality.  It's clear that \(\psi_4\) is normal to all others since only \(\psi_4\) has a \(\phi_{2 p_z}\) component.  For the rest we have
%
\begin{subequations}
\begin{equation}\label{eqn:condensedMatterProblemSet1Problem2:100}
\begin{aligned}
\braket{\psi_1}{\psi_2}
&=
\inv{6}
\braket
{
\sqrt{2} \phi_{2s} + 2 \phi_{2p_x}
}
{
\sqrt{2} \phi_{2s} - \phi_{2p_x} + \sqrt{3} \phi_{2p_y}
} \\
&=
\inv{6}
\lr{
2 - 2
} \\
&= 0
\end{aligned}
\end{equation}
%
\begin{equation}\label{eqn:condensedMatterProblemSet1Problem2:120}
\begin{aligned}
\braket{\psi_1}{\psi_3}
&=
\inv{6}
\braket
{
\sqrt{2} \phi_{2s} + 2 \phi_{2p_x}
}
{
\sqrt{2} \phi_{2s} - \phi_{2p_x} - \sqrt{3} \phi_{2p_y}
} \\
&=
\inv{6}
\lr{
2 - 2
} \\
&= 0
\end{aligned}
\end{equation}
%
\begin{equation}\label{eqn:condensedMatterProblemSet1Problem2:140}
\begin{aligned}
\braket{\psi_2}{\psi_3}
&=
\inv{6}
\braket
{
\sqrt{2} \phi_{2s} - \phi_{2p_x} + \sqrt{3} \phi_{2p_y}
}
{
\sqrt{2} \phi_{2s} - \phi_{2p_x} - \sqrt{3} \phi_{2p_y}
} \\
&=
\inv{6}
\lr{
2 + 1 - 3
} \\
&= 0.
\end{aligned}
\end{equation}
\end{subequations}
%
\makeSubAnswer{}{condensedMatter:problemSet1:2b}
The probability density for \(\psi_1\) has the form
%
\begin{equation}\label{eqn:condensedMatterProblemSet1Problem2:160}
\begin{aligned}
\Abs{\psi_1}^2 
&= \inv{3}
\lr{
\phi_{2s}
+ \sqrt{2} \phi_{2 p_x}
}^\conj
\lr{
\phi_{2s}
+ \sqrt{2} \phi_{2 p_x}
} \\
&=
\inv{3}
\lr{
\Abs{\phi_{2s}}^2
+ 2 \Abs{\phi_{2 p_x}}^2
+ 2 \sqrt{2} \phi_{2s} \phi_{2 p_x}
} \\
&=
\frac{{\cal N}^2}{3} e^{-2 \rho} 
\Biglr
   {
   (1 - \rho)^2 + 2 \rho^2 \sin^2 \theta \cos^2 \phi \\
&\qquad + 2 \sqrt{2} \rho (1 - \rho) \sin\theta \cos\phi
   }.
\end{aligned}
\end{equation}
%
Integrating this to find the fraction of the probability density along each radial ray, we find
%
\begin{equation}\label{eqn:condensedMatterProblemSet1Problem2:180}
\begin{aligned}
\frac{{\cal N}^2}{3}
\int_0^\infty
r^2 dr
e^{-2 \rho} \lr
{
(1 - \rho)^2 + 2 \rho^2 \sin^2 \theta \cos^2 \phi
+ 2 \sqrt{2} \rho (1 - \rho) \sin\theta \cos\phi
}
&=
\frac{{\cal N}^2}{3} \lr{ \frac{2 a_\circ}{Z} }^3
\int_0^\infty
\rho^2 d\rho
e^{-2 \rho} \lr
{
(1 - \rho)^2 + 2 \rho^2 \sin^2 \theta \cos^2 \phi
+ 2 \sqrt{2} \rho (1 - \rho) \sin\theta \cos\phi
}
\\ &=
\inv{ 3 \pi}
\int_0^\infty
\inv{8} x^2 dx
e^{-x} \lr
{
\lr{1 - \frac{x}{2}}^2 + 2 \lr{\frac{x}{2}}^2 \sin^2 \theta \cos^2 \phi
+ 2 \sqrt{2} \frac{x}{2} \lr{1 - \frac{x}{2}} \sin\theta \cos\phi
}
\\ &=
\inv{24 \pi} \lr{
\text{const} + \inv{2} 4! \sin^2 \theta \cos^2 \phi + \sqrt{2} \sin\theta \cos\phi\lr{ 3! - \frac{4!}{2} }
}.
\end{aligned}
\end{equation}
%
Setting \(\theta\) and \(\phi\) partials equal to zero respectively we have
\begin{subequations}
\begin{equation}\label{eqn:condensedMatterProblemSet1Problem2:200}
12 \times 2 \sin\theta \cos\theta \cos^2 \phi + \sqrt{ 2 } \cos\theta \cos\phi ( -6 ) = 0
\end{equation}
\begin{equation}\label{eqn:condensedMatterProblemSet1Problem2:220}
-12 \times 2 \sin^2\theta \cos\phi \sin\phi + \sqrt{ 2 } \sin\theta \sin\phi ( -6 ) = 0,
\end{equation}
\end{subequations}
%
or
\begin{subequations}
\begin{equation}\label{eqn:condensedMatterProblemSet1Problem2:240}
\lr{ 4 \sin\theta\cos \phi - \sqrt{2} } \cos\theta \cos\phi = 0
\end{equation}
\begin{equation}\label{eqn:condensedMatterProblemSet1Problem2:260}
\lr{ 4 \sin\theta\cos \phi - \sqrt{2} } \sin\theta \sin\phi = 0.
\end{equation}
\end{subequations}
%
Neglecting the first common factor, some experimentation yields the solution \(\mathLabelBox{\phi = \pi/2}{Grading remark: Marked wrong.  Review this.}, \theta \in \{0, \pi\}\).  The probability density is maximized along the \(z,y\) plane pointing towards either of the poles.
\iftoggle{print-version}{}
{
\makeSubAnswer{}{condensedMatter:problemSet1:2c}
%%\cref{fig:qmSolidsPs1P2bPsi1:qmSolidsPs1P2bPsi1Fig1}.
%\imageFigure{../figures/phy487-qmsolids/qmSolidsPs1P2bPsi1Fig1}{\(\Abs{\psi_1}^2\) in \(x-y\) plane}{fig:qmSolidsPs1P2bPsi1:qmSolidsPs1P2bPsi1Fig1}{0.3}
%%\cref{fig:qmSolidsPs1P2bPsi2:qmSolidsPs1P2bPsi2Fig2}.
%\imageFigure{../figures/phy487-qmsolids/qmSolidsPs1P2bPsi2Fig2}{\(\Abs{\psi_2}^2\) in \(x-y\) plane}{fig:qmSolidsPs1P2bPsi2:qmSolidsPs1P2bPsi2Fig2}{0.3}
%
%%\cref{fig:qmSolidsPs1P23d:qmSolidsPs1P23dFig4}.
%%\imageFigure{../figures/phy487-qmsolids/qmSolidsPs1P23dFig4}{3d \(\Abs{\psi_1}^2\)}{fig:qmSolidsPs1P23d:qmSolidsPs1P23dFig4}{0.3}

%\cref{fig:qmSolidPs1OrbitalsPsi1AbsDensity:qmSolidPs1OrbitalsPsi1AbsDensityFig3}.
%\mathImageFigure{../figures/phy487-qmsolids/qmSolidPs1OrbitalsPsi1AbsDensityFig3}{\(\Abs{\psi_1}^2\), density plot in \(x - y\) plane}{fig:qmSolidPs1OrbitalsPsi1AbsDensity:qmSolidPs1OrbitalsPsi1AbsDensityFig3}{0.2}{qmSolidsPs1P2.nb}
%\cref{fig:qmSolidPs1OrbitalsPsi1Abs:qmSolidPs1OrbitalsPsi1AbsFig2}.
\mathImageFigure{../figures/phy487-qmsolids/qmSolidPs1OrbitalsPsi1AbsFig2}{\(\Abs{\psi_1}^2\), contour plot in \(x - y\) plane.}{fig:qmSolidPs1OrbitalsPsi1Abs:qmSolidPs1OrbitalsPsi1AbsFig2}{0.2}{qmSolidsPs1P2.nb}
%\cref{fig:qmSolidPs1OrbitalsPsi1Signed:qmSolidPs1OrbitalsPsi1SignedFig1}.
\mathImageFigure{../figures/phy487-qmsolids/qmSolidPs1OrbitalsPsi1SignedFig1}{\(\psi_1\), contour plot in \(x - y\) plane.}{fig:qmSolidPs1OrbitalsPsi1Signed:qmSolidPs1OrbitalsPsi1SignedFig1}{0.2}{qmSolidsPs1P2.nb}
%\cref{fig:qmSolidPs1OrbitalsPsi2AbsDensity:qmSolidPs1OrbitalsPsi2AbsDensityFig6}.
%\mathImageFigure{../figures/phy487-qmsolids/qmSolidPs1OrbitalsPsi2AbsDensityFig6}{\(\Abs{\psi_2}^2\), density plot in \(x - y\) plane}{fig:qmSolidPs1OrbitalsPsi2AbsDensity:qmSolidPs1OrbitalsPsi2AbsDensityFig6}{0.2}{qmSolidsPs1P2.nb}
%\cref{fig:qmSolidPs1OrbitalsPsi2Abs:qmSolidPs1OrbitalsPsi2AbsFig5}.
\mathImageFigure{../figures/phy487-qmsolids/qmSolidPs1OrbitalsPsi2AbsFig5}{\(\Abs{\psi_2}^2\), contour plot in \(x - y\) plane.}{fig:qmSolidPs1OrbitalsPsi2Abs:qmSolidPs1OrbitalsPsi2AbsFig5}{0.2}{qmSolidsPs1P2.nb}
%\cref{fig:qmSolidPs1OrbitalsPsi2Signed:qmSolidPs1OrbitalsPsi2SignedFig4}.
\mathImageFigure{../figures/phy487-qmsolids/qmSolidPs1OrbitalsPsi2SignedFig4}{\(\psi_2\), contour plot in \(x - y\) plane.}{fig:qmSolidPs1OrbitalsPsi2Signed:qmSolidPs1OrbitalsPsi2SignedFig4}{0.2}{qmSolidsPs1P2.nb}
%\cref{fig:qmSolidPs1OrbitalsPsi3AbsDensity:qmSolidPs1OrbitalsPsi3AbsDensityFig9}.
%\mathImageFigure{../figures/phy487-qmsolids/qmSolidPs1OrbitalsPsi3AbsDensityFig9}{\(\Abs{\psi_3}^2\), density plot in \(x - y\) plane}{fig:qmSolidPs1OrbitalsPsi3AbsDensity:qmSolidPs1OrbitalsPsi3AbsDensityFig9}{0.2}{qmSolidsPs1P2.nb}
%\cref{fig:qmSolidPs1OrbitalsPsi3Abs:qmSolidPs1OrbitalsPsi3AbsFig8}.
\mathImageFigure{../figures/phy487-qmsolids/qmSolidPs1OrbitalsPsi3AbsFig8}{\(\Abs{\psi_3}^2\), contour plot in \(x - y\) plane.}{fig:qmSolidPs1OrbitalsPsi3Abs:qmSolidPs1OrbitalsPsi3AbsFig8}{0.2}{qmSolidsPs1P2.nb}
%\cref{fig:qmSolidPs1OrbitalsPsi3Signed:qmSolidPs1OrbitalsPsi3SignedFig7}.
\mathImageFigure{../figures/phy487-qmsolids/qmSolidPs1OrbitalsPsi3SignedFig7}{\(\psi_3\), contour plot in \(x - y\) plane.}{fig:qmSolidPs1OrbitalsPsi3Signed:qmSolidPs1OrbitalsPsi3SignedFig7}{0.2}{qmSolidsPs1P2.nb}
%\cref{fig:qmSolidPs1OrbitalsPsi4AbsDensity:qmSolidPs1OrbitalsPsi4AbsDensityFig12}.
%\mathImageFigure{../figures/phy487-qmsolids/qmSolidPs1OrbitalsPsi4AbsDensityFig12}{\(\Abs{\psi_4}^2\), density plot in \(x - y\) plane}{fig:qmSolidPs1OrbitalsPsi4AbsDensity:qmSolidPs1OrbitalsPsi4AbsDensityFig12}{0.2}{qmSolidsPs1P2.nb}
%\cref{fig:qmSolidPs1OrbitalsPsi4Abs:qmSolidPs1OrbitalsPsi4AbsFig11}.
\mathImageFigure{../figures/phy487-qmsolids/qmSolidPs1OrbitalsPsi4AbsFig11}{\(\Abs{\psi_4}^2\), contour plot in \(x - y\) plane.}{fig:qmSolidPs1OrbitalsPsi4Abs:qmSolidPs1OrbitalsPsi4AbsFig11}{0.2}{qmSolidsPs1P2.nb}
%\cref{fig:qmSolidPs1OrbitalsPsi4Signed:qmSolidPs1OrbitalsPsi4SignedFig10}.
\mathImageFigure{../figures/phy487-qmsolids/qmSolidPs1OrbitalsPsi4SignedFig10}{\(\psi_4\), contour plot in \(x - y\) plane.}{fig:qmSolidPs1OrbitalsPsi4Signed:qmSolidPs1OrbitalsPsi4SignedFig10}{0.2}{qmSolidsPs1P2.nb}
\makeSubAnswer{}{condensedMatter:problemSet1:2d}
%\cref{fig:qmSolidPs1OrbitalsPsi5AbsDensity:qmSolidPs1OrbitalsPsi5AbsDensityFig15}.
%\mathImageFigure{../figures/phy487-qmsolids/qmSolidPs1OrbitalsPsi5AbsDensityFig15}{\(\Abs{\psi_5}^2\), density plot in \(x - y\) plane}{fig:qmSolidPs1OrbitalsPsi5AbsDensity:qmSolidPs1OrbitalsPsi5AbsDensityFig15}{0.2}{qmSolidsPs1P2.nb}
%\cref{fig:qmSolidPs1OrbitalsPsi5Abs:qmSolidPs1OrbitalsPsi5AbsFig14}.
\mathImageFigure{../figures/phy487-qmsolids/qmSolidPs1OrbitalsPsi5AbsFig14}{\(\Abs{\psi_5}^2\), contour plot in \(x - y\) plane.}{fig:qmSolidPs1OrbitalsPsi5Abs:qmSolidPs1OrbitalsPsi5AbsFig14}{0.2}{qmSolidsPs1P2.nb}
%\cref{fig:qmSolidPs1OrbitalsPsi5Signed:qmSolidPs1OrbitalsPsi5SignedFig13}.
\mathImageFigure{../figures/phy487-qmsolids/qmSolidPs1OrbitalsPsi5SignedFig13}{\(\psi_5\), contour plot in \(x - y\) plane.}{fig:qmSolidPs1OrbitalsPsi5Signed:qmSolidPs1OrbitalsPsi5SignedFig13}{0.2}{qmSolidsPs1P2.nb}

%See \href{https://raw.github.com/peeterjoot/physicsplay/master/notes/phy487/mathematica/qmSolidsPs1P2.nb}{phy487/mathematica/qmSolidsPs1P2.nb} (attached).

%\cref{fig:qmSolidsPs1P2d:qmSolidsPs1P2dFig3}.
%\imageFigure{../figures/phy487-qmsolids/qmSolidsPs1P2dFig3}{Probability density in \(x-y\) plane for \(\psi_1\) superposition with shifted and reversed self}{fig:qmSolidsPs1P2d:qmSolidsPs1P2dFig3}{0.3}
}
}
