%
% Copyright � 2013 Peeter Joot.  All Rights Reserved.
% Licenced as described in the file LICENSE under the root directory of this GIT repository.
%
\input{../assignment.tex}
\renewcommand{\basename}{condensedMatterProblemSet1}
\renewcommand{\dirname}{notes/phy487/}
\newcommand{\keywords}{Condensed Matter Physics, PHY487H1F}
\newcommand{\dateintitle}{}
\input{../peeter_prologue_print2.tex}

\usepackage{mhchem}
\newcommand{\nought}[0]{\circ}
%\newcommand{\CV}[0]{C_{\txtV}}
\newcommand{\cA}[0]{c_{\txtA}}

\beginArtNoToc
\generatetitle{PHY487H1F Condensed Matter Physics.  Problem Set 5: Density of states and Debye temperature}
%\chapter{Density of states and Deybe temperature}
\label{chap:condensedMatterProblemSet1}
%
%\section{Disclaimer}
%
%This is an ungraded set of answers to the problems posed.

%\begin{center}
%{\bf PHY 487/1487 Problem Set \# 5} \\
%Due Friday Oct.\ 18 by 4:30 p.m., MP324 \\
%\end{center}
%
%Note 1: I will be away from the University on Thursday and Friday, so if you
%have questions you will have to send email to
%{\em sjulian@physics.utoronto.ca}.  I apologize for any inconvenience.
%The lecture on Friday the 18th will be given by Professor Young-June Kim,
%and assignments can be submitted to him.
%
%Note 2:  This problem set will be graded and returned by Monday,
% 21 October at 2:00 p.m., so late assignments will not be accepted beyond
% that time.  If you have a valid medical or personal excuse, and
% cannot submit the problem set by that time, I will ignore this
% problem set in calculating your final mark.
%
%Note 3:
%Collaboration is encouraged, but direct copying is not: make sure you
%understand your solutions.  (At least some exam questions will be
%based closely on problem set questions.)

%%
% Copyright � 2013 Peeter Joot.  All Rights Reserved.
% Licenced as described in the file LICENSE under the root directory of this GIT repository.
%
\makeoproblem{Density of states of a 1D chain.}{condensedMatter:problemSet5:1}{2013 ps5 p1}{
Calculate and sketch a plot of the density of states, \(Z(\omega)\), for
the vibrational modes of a 1-d monatomic chain of length \(L\),
with nearest-neighbor spring constant K, atoms of mass \(M\), and
lattice constant \(a\).
Specifically, start from \(\sum_q\) and by transforming this into
an integral over \(\omega\), obtain \(Z(\omega)\).   Then
draw a sketch of \(Z(\omega)\) vs.\ \(\omega\), labeling intercepts and
asymptotes.
} % makeproblem

\makeanswer{condensedMatter:problemSet5:1}{
For the 2D and 3D (\(d = 2,3\)) density of states we'd consider solutions for \(Z(\omega)\) of
%
\begin{dmath}\label{eqn:condensedMatterProblemSet5Problem1:20}
\int Z(\omega) d\omega = \lr{\frac{L}{2\pi}}^d \int \frac{d \Bf_\omega}{\Abs{\spacegrad_\Bq \omega(q)} } d\omega.
\end{dmath}
%
Should we wish to extend this down to \(d = 1\) we'd have to figure out how to interpret \(d\Bf_\omega\).  In 2D and 3D that was a surface area element, a factor of the differential form \(d^d \Bq = d\Bf_\omega d\omega_\perp\).  In 3D we had \(\int d\Bf_\omega = 4 \pi q^2 = d/dq( 4 \pi q^3/3)\), and for 2D \(\int d\Bf_\omega = 2 \pi q = d/dq( \pi q^2 )\).

Those 3D and 2D ``volumes'' (differentiated to obtain the ``area'' when \(q\) of the surface for \(q\) held constant) can be obtained by these respective integrals
%
\begin{subequations}
\begin{dmath}\label{eqn:condensedMatterProblemSet5Problem1:40}
\int_{x^2 + y^2 + z^2 \le q^2} dx dy dz = \frac{4}{3} \pi q^3
\end{dmath}
\begin{dmath}\label{eqn:condensedMatterProblemSet5Problem1:60}
\int_{x^2 + y^2 \le q^2} dx dy  = \pi q^2.
\end{dmath}
\end{subequations}
%
We can generalize this down to a single dimension by considering
%
\begin{dmath}\label{eqn:condensedMatterProblemSet5Problem1:80}
\int_{x^2 \le q^2} dx = 2 q.
\end{dmath}
%
for which we could conceivably consider the area of this 1D surface to be the constant \(2\).  However, does this even make sense, since writing \(dq = df_\omega dq_\perp\) would split our 1-form into the product of two 1-forms, which isn't a sensible operation?  Let's step back and consider the density of states definition from scratch.
%
\paragraph{Starting from scratch}
%
We wish to sum over all the integer values \(n\), subject to a period constraint \(2 \pi n = q L\), and employ an integral approximation to this sum.
%
\begin{dmath}\label{eqn:condensedMatterProblemSet5Problem1:100}
\sum_n
\sim \int dn
= \frac{L}{2 \pi} \int dq
= \frac{L}{2 \pi} \int \frac{dq}{d\omega} d\omega
\equiv \int Z(\omega) d\omega.
\end{dmath}
%
From this we find for one dimension
%
\begin{dmath}\label{eqn:condensedMatterProblemSet5Problem1:120}
Z(\omega) = \frac{L}{2 \pi} \frac{dq}{d\omega}.
\end{dmath}
%
Now we are ready to start.  For the 1D chain we had
%
\begin{dmath}\label{eqn:condensedMatterProblemSet5Problem1:140}
\sqrt{\frac{M}{K}} \omega(q) = 2 \sin \lr{ \frac{ q a}{2} },
\end{dmath}
%
so
%
\begin{dmath}\label{eqn:condensedMatterProblemSet5Problem1:160}
\sqrt{\frac{M}{K}} = a \cos \lr{ \frac{ q a}{2} } \frac{dq}{d\omega},
\end{dmath}
%
or
%
\begin{dmath}\label{eqn:condensedMatterProblemSet5Problem1:180}
Z(\omega)
= \frac{L}{2 \pi}
\frac{\sqrt{\frac{M}{K}}}{a \cos \lr{ \frac{ q a}{2} }}
=
\sqrt{\frac{M}{K}} \frac{L}{2 \pi a}
\inv{ \cos \lr{ \frac{ q a}{2} }}
=
\sqrt{\frac{M}{K}} \frac{L}{2 \pi a}
\inv{ \cos \sin^{-1} \lr{
\inv{2} \sqrt{\frac{M}{K}} \omega } }
=
\inv{2} \sqrt{\frac{M}{K}} \frac{L}{\pi a}
\inv{ \sqrt{1 - \inv{4} \frac{M}{K} \omega^2 } }.
\end{dmath}
%
With \(L = N a\), this is
\boxedEquation{eqn:condensedMatterProblemSet5Problem1:200}{
Z(\omega)
=
\inv{2} \sqrt{\frac{M}{K}} \frac{N}{\pi}
\inv{ \sqrt{1 - \inv{4} \frac{M}{K} \omega^2 } }.
}
This has a minimum at \(\omega = 0\), and in that neighborhood is approximately parabolic function
%
\begin{dmath}\label{eqn:condensedMatterProblemSet5Problem1:220}
Z(\omega \approx 0)
=
\inv{2} \sqrt{\frac{M}{K}} \frac{N}{\pi}
\lr{ 1 - \lr{ - \inv{2}} \inv{4} \frac{M}{K} \omega^2 }
=
\inv{2} \sqrt{\frac{M}{K}} \frac{N}{\pi}
\lr{ 1 + \inv{8} \frac{M}{K} \omega^2 }.
\end{dmath}
%
As \(\omega \rightarrow \pm \sqrt{4 K/M}\), the density of states approaches vertical asymptotes \(Z(\omega) \rightarrow \infty\).  Observe that these extremes are the edges of the Brillouin zone where \(q a/2 = \pm \pi/2\).  For \(Z(\omega)\) to be useful for probability calculations, we expect that the integral over this first Brillouin zone will be finite, despite these infinite asymptotes.  Let's verify this
%
\begin{dmath}\label{eqn:condensedMatterProblemSet5Problem1:240}
\int_{-\sqrt{4K/M}}^{\sqrt{4K/M}} Z(\omega) d\omega
=
\inv{2} \sqrt{\frac{M}{K}} \frac{N}{\pi}
\int_{-1}^1
\sqrt{ \frac{4 K}{M} }
dx
\inv{ \sqrt{1 - x^2 } }
= \frac{N}{\pi} \pi
= N.
\end{dmath}
%
Good, the area under the curve is finite as expected.  This curve is sketched in \cref{fig:1dSpringLatticeDensityOfStates:1dSpringLatticeDensityOfStatesFig1}.
%
\imageFigure{../figures/phy487-qmsolids/1dSpringLatticeDensityOfStatesFig1}{1D density of states for Harmonic chain.}{fig:1dSpringLatticeDensityOfStates:1dSpringLatticeDensityOfStatesFig1}{0.3}
}

%
% Copyright � 2013 Peeter Joot.  All Rights Reserved.
% Licenced as described in the file LICENSE under the root directory of this GIT repository.
%
\makeoproblem{Systematic trends in the Debye temperature}{condensedMatter:problemSet5:2}{2013 ps5 p2}{
%:} (worth 5 marks)
Table 5.1 on page
120 of Ibach and Luth shows the Debye temperature for
various solids.  Discuss and explain any trends that you see in the Debye
temperature, e.g.\ as a function of location in the periodic table,
bonding type, or atomic mass.

} % makeproblem

\makeanswer{condensedMatter:problemSet5:2}{

A plot of Debye temperatures by atomic number can be found in \cref{fig:DebyeTemperaturesVsAtomicNumber:DebyeTemperaturesVsAtomicNumberFig1}.  This is based on data from \citep{knowledgedoor:debye}, and \citep{ibach2009solid}

\mathImageFigure{../figures/phy487-qmsolids/DebyeTemperaturesVsAtomicNumberFig1}{Debye temperature vs atomic number}{fig:DebyeTemperaturesVsAtomicNumber:DebyeTemperaturesVsAtomicNumberFig1}{0.4}{deybeTemperatureTable.nb}

\paragraph{Some observed trends}

\begin{itemize}
\item There is a general trend of decreasing Debye temperature with atomic number.
\item Lowest Debye temperatures are often at points where we have completely filled or half filled orbitals: \ce{H} (\(1s^1\)), \ce{Ne} (\(1s^2 2s^2 1p^6\)), \ce{Eu}(\([Xe]4f^7\)), \ce{Yb} (\([Xe] 4 f^{14}\)), \ce{Hg} (\([Xe] 4f^{14} 5d^{10} 6s^2\)).
\item We see peak temperatures around elements that are near the middles of their respective orbital filling ranges: \ce{C}, \ce{Si} (p block elements), \ce{Cr}, \ce{Ru}, \ce{Os} (d-block elements).  Carbon in its diamond form is plotted above (its graphite form comes in much lower at \(420 K\)).
%\item There are some exceptions to the above with local minimum Debye temperatures occuring at points that are close to completely filled: \ce{Fr} (\([Rn]7s^1\))
\end{itemize}

\paragraph{Comments}

The capability of the element for making strong bonds appears to contribute significantly to high Debye temperatures.  In particular observe that the diamond form of \ce{C}, with its strong highly directional covalent bonds, has the highest Debye temperature.  \ce{Si} also in the p block with 4 available p orbital slots has a very high Debye temperature, at least compared to its period table neighbors.   The converse is also evident, since we see lack of bonding capability associated with low Debye temperatures for those elements that have completely and half filled orbitals, which have some stability in isolation.  This is similar to the previously observed low melting points (a measure of ease of lattice breakup) for elements that have half and completely filled orbitals.

Recall that the Debye frequency (proportional to the Debye temperature) had the form
%
\begin{dmath}\label{eqn:condensedMatterProblemSet5Problem2:20}
\omega_{\txtD} \propto \lr{\frac{N}{V}}^{1/3}.
\end{dmath}
%
Based on this, we expect to see small Debye temperatures when the number density is low, which should occur when the atomic radii is large.  That can be observed in \cref{fig:atomicRadiiVsAtomicNumber:atomicRadiiVsAtomicNumberFig2}, looking for example at \ce{K}, \ce{Rb}, and \ce{Cs}, that are positioned at local maximums for atomic radii, in contrast to the local minimums observed for the Debye temperature.

\mathImageFigure{../figures/phy487-qmsolids/atomicRadiiVsAtomicNumberFig2}{Atomic radii}{fig:atomicRadiiVsAtomicNumber:atomicRadiiVsAtomicNumberFig2}{0.4}{deybeTemperatureTable.nb}

This plot has some confusing aspects as-is since I didn't label all the elements, just the ones that I wanted to compare to the Debye temperatures (if you look carefully the labels for Cl, Br, I are shifted to the left slightly).  It also appears that I didn't explicitly plot those elements for which I didn't have Debye temperature data, which makes it even more misleading if looking at just the radius periodicity.

In \cref{fig:atomicRadiusAndDebyeTempOverlapped:atomicRadiusAndDebyeTempOverlappedFig3} is a combined plot of both the atomic radius and the Debye temperature.  In this second plot we see that, yes, the largest radii are those with the smallest Debye temperatures.  As the radius drops from the peak, the Debye temperature increases.  However, part way towards the middle of the period, this inverse relationship starts to fail.  In fact, they both start trending downwards at these points.  Is this where the velocities of the accoustic modes, also variables in the Debye temperatures, start to factor into the mix?

\mathImageFigure{../figures/phy487-qmsolids/atomicRadiusAndDebyeTempOverlappedFig3}{Debye and atomic radius}{fig:atomicRadiusAndDebyeTempOverlapped:atomicRadiusAndDebyeTempOverlappedFig3}{0.4}{deybeTemperatureTable.nb}

As a final plot, let's look at the inverse of the atomic radius and the Debye temperature together.  This is plotted in \cref{fig:figuresatomicInvRadiusAndDebyeTempOverlapped:atomicInvRadiusAndDebyeTempOverlappedFig4}.

\mathImageFigure{../figures/phy487-qmsolids/atomicInvRadiusAndDebyeTempOverlappedFig4}{Inverse Atomic Radius and Debye temperature}{fig:figuresatomicInvRadiusAndDebyeTempOverlapped:atomicInvRadiusAndDebyeTempOverlappedFig4}{0.4}{deybeTemperatureTable.nb}
}

%%
% Copyright � 2013 Peeter Joot.  All Rights Reserved.
% Licenced as described in the file LICENSE under the root directory of this GIT repository.
%
\makeoproblem{Debye calculation in two dimensions}{condensedMatter:problemSet5:3}{2013 ps5 p3}{
Repeat the Debye theory calculation that we did in class, but for a
two-dimensional lattice.  Assume (quite artificially) that the atoms
are free to move only within the plane, so that there are \(2rN\) degrees
of freedom,  and there is only one transverse acoustic
phonon mode, instead of two as in the three-dimensional calculation.

Show that the low temperature limit of the specific heat at constant
area, per unit area, is:
\begin{eqnarray*}
\cA(T) = 7.213\, \frac{4rN}{A}\,\kB\,\frac{T^2}{\Theta^2},
\end{eqnarray*}
where \(A\) is the area of the crystal, \(rN\) is the number of atoms
in the crystal,
\(\Theta\) is defined by \(\kB\Theta = \hbar\omega_{\txtD}\),
and
\begin{eqnarray*}
\int_{0}^{\infty} \frac{y^3 e^y}{(e^y-1)^2}\,dy \simeq 7.213.
\end{eqnarray*}
} % makeproblem

\makeanswer{condensedMatter:problemSet5:3}{
We first setup the 2D density of states construction as we did for 3D, also employing the periodic relations
\begin{dmath}\label{eqn:condensedMatterProblemSet5Problem3:20}
\begin{aligned}
2 \pi n_x &= L_x q_x \\
2 \pi n_y &= L_y q_y,
\end{aligned}
\end{dmath}
%
so that a sum over the quantum numbers \(\Bn\) can be approximated as
%
\begin{dmath}\label{eqn:condensedMatterProblemSet5Problem3:40}
\sum_\Bn
\approx
\int dn_x dn_y
=
\frac{A}{(2\pi)^2 } \int d^2\Bq
=
\frac{A}{(2\pi)^2 } \int d f_\omega d q_\perp
=
\frac{A}{(2\pi)^2 } \int
\frac{df_\omega}{\Abs{\spacegrad_\Bq \omega(\Bq) }} d \omega
=
\int Z(\omega) d\omega.
\end{dmath}
%
This Debye model we have
%
\begin{dmath}\label{eqn:condensedMatterProblemSet5Problem3:60}
\omega =
\left\{
\begin{array}{l l}
C_{\txtL} q & \quad \mbox{longitudinal acoustic} \\
C_{\txtT} q & \quad \mbox{transverse acoustic}
\end{array}
\right.
\end{dmath}
%
This gives
%
\begin{dmath}\label{eqn:condensedMatterProblemSet5Problem3:80}
\int Z(\omega) d\omega
=
\sum_{LA, TA} \frac{A}{(2\pi)^2} \int \frac{d f_\omega }{\Abs{ \spacegrad_\Bq \omega(\Bq)}} d\omega
=
\frac{A}{(2\pi)^2}
\int
\sum_{LA, TA}
\frac{
\mathLabelBox
[
   labelstyle={xshift=2cm},
   linestyle={out=270,in=90, latex-}
]
{
d f_\omega
}
{\(q\) space surface ``area'' element}
}{\frac{d\omega}{dq}}
d\omega
=
\int
\frac{A}{(2\pi)^2}
\lr{
\inv{C_{\txtL}} + \frac{1}{C_{\txtT}}
}
\mathLabelBox
[
   labelstyle={xshift=2cm},
   linestyle={out=270,in=90, latex-}
]
{
2 \pi q
}
{\(= \int df_\omega\)}
d\omega
=
\int
\frac{A}{2\pi}
\lr{
\frac{q}{C_{\txtL}} + \frac{q}{C_{\txtT}}
}
d \omega
=
\int
\frac{A}{2\pi}
\lr{
\frac{1}{C_{\txtL}^2} + \frac{1}{C_{\txtT}^2}
}
\omega d \omega,
\end{dmath}
%
or
\begin{dmath}\label{eqn:condensedMatterProblemSet5Problem3:160}
Z(\omega) =
\frac{A}{2\pi}
\lr{
\frac{1}{C_{\txtL}^2} + \frac{1}{C_{\txtT}^2}
}
\omega.
\end{dmath}
%
Define the Debye frequency \(\omega_{\txtD}\) by
%
\begin{dmath}\label{eqn:condensedMatterProblemSet5Problem3:100}
\int_0^{\omega_{\txtD}} Z(\omega) d\omega = 2 r N.
\end{dmath}
%
\begin{dmath}\label{eqn:condensedMatterProblemSet5Problem3:120}
2 r N
=
\frac{A}{ 2 \pi} \lr{ \inv{C_{\txtL}^2} + \frac{1}{C_{\txtT}^2} } \int_0^{\omega_{\txtD}} \omega d\omega
=
\frac{A}{ 2 \pi} \lr{ \inv{C_{\txtL}^2} + \frac{1}{C_{\txtT}^2} } \inv{2} \omega_{\txtD}^2,
\end{dmath}
%
or
%
\begin{dmath}\label{eqn:condensedMatterProblemSet5Problem3:140}
\frac{A}{ 2 \pi} \lr{ \inv{C_{\txtL}^2} + \frac{1}{C_{\txtT}^2} } \omega_{\txtD}^2 = 4 r N.
\end{dmath}
%
Inserting this Debye frequency into the density of states gives
%
\begin{dmath}\label{eqn:condensedMatterProblemSet5Problem3:180}
Z(\omega) =
\frac{4 r N \omega}{ \omega_{\txtD}^2 }.
\end{dmath}
%
We can now start the \textAndIndex{specific heat} calculation
%
\begin{dmath}\label{eqn:condensedMatterProblemSet5Problem3:200}
\cA(T)
=
\frac{dU}{dT}
=
\frac{d}{dT}
\inv{A}
 \int_0^{\omega_{\txtD}}
\calE(\omega, T)
Z(\omega)
d\omega
=
\frac{4 r N }{ A \omega_{\txtD}^2 }
 \int_0^{\omega_{\txtD}} \frac{d}{dT} \calE(\omega, T) \omega d\omega
=
\frac{4 r N }{ A \omega_{\txtD}^2 }
\frac{d}{dT}
\int_0^{\omega_{\txtD}}
\Hbar \omega
\lr{
\inv{2}
+
\inv
{
e^{\Hbar \omega/\kB T} - 1
}
}
\omega d\omega
=
\frac{4 r N }{ A \omega_{\txtD}^2 }
\int_0^{\omega_{\txtD}}
-\Hbar \omega^2
\inv
{
\lr{ e^{\Hbar \omega/\kB T} - 1 }^2
}
e^{\Hbar \omega/\kB T} \lr{ -\frac{\Hbar \omega}{\kB T^2} }
d\omega
=
\frac{4 r N }{ A \omega_{\txtD}^2 }
\int_0^{\omega_{\txtD}}
\kB^2 T \frac{\Hbar^2 \omega^3}{\kB^3 T^3}
\inv
{
\lr{ e^{\Hbar \omega/\kB T} - 1 }^2
}
e^{\Hbar \omega/\kB T}
d\omega.
\end{dmath}
%
As in class we make substitutions
\begin{subequations}
\begin{dmath}\label{eqn:condensedMatterProblemSet5Problem3:220}
y = \frac{\Hbar \omega}{\kB T}
\end{dmath}
\begin{dmath}\label{eqn:condensedMatterProblemSet5Problem3:240}
d\omega = \frac{\kB T}{\Hbar} dy
\end{dmath}
\begin{dmath}\label{eqn:condensedMatterProblemSet5Problem3:280}
y(\omega_{\txtD}) = \frac{\Hbar \omega_{\txtD}}{ \kB T} = \frac{\kB \Theta}{\kB T} = \frac{\Theta}{T}.
\end{dmath}
\end{subequations}
%
Inserting these we have
%
\begin{dmath}\label{eqn:condensedMatterProblemSet5Problem3:300}
\cA(T)
=
\frac{4 r N }{ A \omega_{\txtD}^2 }
\frac{\kB^3 T^2}{\Hbar^2}
\int_0^{\Theta/T}
\frac{
y^3
e^{y}
dy
}
{
\lr{ e^{y} - 1 }^2
}
=
\frac{4 r N \kB T^2}{ A \Theta^2 }
\int_0^{\Theta/T}
\frac{
y^3
e^{y}
dy
}
{
\lr{ e^{y} - 1 }^2
}.
\end{dmath}
%
In the \(\kB T \ll \Hbar \omega_{\txtD} = \kB \Theta\) limit where the integrand is small, the integral limit can be approximated by extension to \(\infty\).  This produces the desired result
%
\begin{dmath}\label{eqn:condensedMatterProblemSet5Problem3:320}
\cA(T) = 7.213 \frac{4 r N}{A} \kB \frac{T^2}{ \Theta^2 }.
\end{dmath}
}


\EndArticle
