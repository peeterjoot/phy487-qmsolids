%
% Copyright � 2013 Peeter Joot.  All Rights Reserved.
% Licenced as described in the file LICENSE under the root directory of this GIT repository.
%
%\input{../blogpost.tex}
%\renewcommand{\basename}{condensedMatterLecture1}
%\renewcommand{\dirname}{notes/phy487/}
%\newcommand{\keywords}{Condensed matter physics, PHY487H1F}
%\input{../peeter_prologue_print2.tex}
%
%\beginArtNoToc
%\generatetitle{PHY487H1F Condensed Matter Physics.  Lecture 1: Course overview.  Taught by Prof.\ Stephen Julian}
%\chapter{Course overview}
\label{chap:condensedMatterLecture1}
%
%\section{Disclaimer}
%
%Peeter's lecture notes from class.  Lots of pictures and qualitative stuff in today's lecture that were hard to take coherent notes of.
%

\section{Chemical bonding in solids}

\index{chemical bonding}
\reading \S 1.1, \S 1.2 \citep{ibach2009solid}, \citep{ashcroft1976solid} \textchapref{19}.

\begin{itemize}
\item Different types of chemical bonds explain many of the differences between solids.
\item Differences in solids: hard/soft.  Example: Lithium, so soft that a pure sample will flow if set on a desk ; metal vs insulating, melting points
\end{itemize}

\imageFigure{../../figures/phy487/qmSolidsL1Fig1}{Periodic table annotated with orbital filling notes}{fig:qmSolidsL1:qmSolidsL1Fig1}{0.3}

Elements in the \underlineAndIndex{periodic table} are classified according to which type of orbital is being filled.  This is roughly sketched in \cref{fig:qmSolidsL1:qmSolidsL1Fig1}, with much better figures are everywhere (such as figure 7.28 of \citep{changChemistry1991}.)

Hydrogen, or other super ionized material (example: iron with all but one electron observed in supernova spectra) \cref{fig:qmSolidsL1:qmSolidsL1Fig2}.

\imageFigure{../../figures/phy487/qmSolidsL1Fig2}{Hydrogenic atom (only one electron)}{fig:qmSolidsL1:qmSolidsL1Fig2}{0.2}

%\cref{fig:qmSolidsL1:qmSolidsL1Fig3}.
\imageFigure{../../figures/phy487/qmSolidsL1Fig3}{Two electron atom. eg: \(\text{Ti}_{22}\)}{fig:qmSolidsL1:qmSolidsL1Fig3}{0.3}


Recall that we have the following ranges for our states

\begin{itemize}
\item \(l \in n-1, n-2, \cdots 0\) (\(n\))
\item \(m \in l, l-1, \cdots, -l\) (\(2 l + 1\))
\item \(S \in \pm\) (2)
\end{itemize}

\begin{itemize}
\item upper right hand of periodic table: covalent bonding
\item lower left hand of periodic table: metallic bonding
\item mixed left hand with right hand of periodic table: ionic bonding
\end{itemize}

\section{Covalent bonding}
\index{covalent bonding}

Consider a pair of hydrogen nuclei sharing one electron.  \S 1.2 \citep{ibach2009solid} has a mathematical description (not examinable)

%\cref{fig:qmSolidsL1:qmSolidsL1Fig5}.
\imageFigure{../../figures/phy487/qmSolidsL1Fig5}{Potential and radial distribution for \(1s\) state}{fig:qmSolidsL1:qmSolidsL1Fig5}{0.3}

%\cref{fig:qmSolidsL1:qmSolidsL1Fig6}.
\imageFigure{../../figures/phy487/qmSolidsL1Fig6}{Far apart}{fig:qmSolidsL1:qmSolidsL1Fig6}{0.2}

%\cref{fig:qmSolidsL1:qmSolidsL1Fig7}.
\imageFigure{../../figures/phy487/qmSolidsL1Fig7}{Close together}{fig:qmSolidsL1:qmSolidsL1Fig7}{0.2}

%\cref{fig:qmSolidsL1:qmSolidsL1Fig8}.
\imageFigure{../../figures/phy487/qmSolidsL1Fig8}{Close together, bonding}{fig:qmSolidsL1:qmSolidsL1Fig8}{0.3}

%\cref{fig:qmSolidsL1:qmSolidsL1Fig9}.
\imageFigure{../../figures/phy487/qmSolidsL1Fig9}{Chemistry diagram}{fig:qmSolidsL1:qmSolidsL1Fig9}{0.2}

In \cref{fig:qmSolidsL1:qmSolidsL1Fig10}, observe that only partially filled orbitals can participate in covalent bonding.

\imageFigure{../../figures/phy487/qmSolidsL1Fig10}{Adding more electrons}{fig:qmSolidsL1:qmSolidsL1Fig10}{0.3}

\reading \S 10.6 \citep{changChemistry1991}.  Figures 10.23, 10.24 for example.

%\EndArticle
