%
% Copyright � 2013 Peeter Joot.  All Rights Reserved.
% Licenced as described in the file LICENSE under the root directory of this GIT repository.
%
%\input{../blogpost.tex}
%\renewcommand{\basename}{condensedMatterLecture12}
%\renewcommand{\dirname}{notes/phy487/}
%\newcommand{\keywords}{Condensed matter physics, PHY487H1F}
%\input{../peeter_prologue_print2.tex}
%
%%\citep{harald2003solid} \S x.y
%%\citep{ibach2009solid} \S x.y
%
%%\usepackage{mhchem}
%\usepackage[version=3]{mhchem}
%\newcommand{\nought}[0]{\circ}
%\newcommand{\EF}[0]{E_{\txtF}}
%\newcommand{\CV}[0]{C_{\txtV}}
%
%\beginArtNoToc
%\generatetitle{PHY487H1F Condensed Matter Physics.  Lecture 12: Free electron model (cont.).  Taught by Prof.\ Stephen Julian}
%\chapter{Free electron model (cont.)}
\index{free electron model}
\label{chap:condensedMatterLecture12}
%
%\section{Disclaimer}
%
%Peeter's lecture notes from class.  May not be entirely coherent.
%
\paragraph{Last time}
%
We want to calculate the thermal properties of the free electron gas \index{free electron gas}, characterized by energies of the form \cref{fig:qmSolidsL12:qmSolidsL12Fig1}.  At \(T = 0\), the free electron gas fills up states up to \(\EF\) \index{Fermi energy}
\imageFigure{../figures/phy487-qmsolids/qmSolidsL12Fig1}{Free electron energy distribution}{fig:qmSolidsL12:qmSolidsL12Fig1}{0.2}

%
\section{Fermi Dirac distribution for \(T > 0\)}
\index{Fermi-Dirac distribution}

Note that we are following Einstein here, not the text.

Consider a closed system and reservoir \(r\), as in \cref{fig:qmSolidsL12:qmSolidsL12Fig4}, exchanging energy and particles, with a smaller system that has energy \(\epsilon\) and \(n\) particles.  The total energy and number of particles are respectively \(U\) and \(N\).
\imageFigure{../figures/phy487-qmsolids/qmSolidsL12Fig4}{Two systems in contact}{fig:qmSolidsL12:qmSolidsL12Fig4}{0.2}

%%\cref{fig:qmSolidsL12:qmSolidsL12Fig3}.
%\imageFigure{../figures/phy487-qmsolids/qmSolidsL12Fig3}{k space region}{fig:qmSolidsL12:qmSolidsL12Fig3}{0.3}
%%\cref{fig:qmSolidsL12:qmSolidsL12Fig2}.
%\imageFigure{../figures/phy487-qmsolids/qmSolidsL12Fig2}{Fermi energy in the energy distribution range}{fig:qmSolidsL12:qmSolidsL12Fig2}{0.2}

The probability of eigenstate \(\epsilon\) with \(n\) particles
%
\begin{dmath}\label{eqn:condensedMatterLecture12:20}
P(\epsilon, n) \propto g_r(U - \epsilon, N - n),
\end{dmath}
%
where \(g_r\) is the number of ``microstates of the reservoir when it has energy \(U - \epsilon\), and \(N - n\) particles.  We have
%
\begin{dmath}\label{eqn:condensedMatterLecture12:40}
g_r(U - \epsilon, N - n)
=
e^{ \ln g_r(U - \epsilon, N - n) }
\approx
\exp\lr{
\ln g_r(U, N)
- \epsilon \PDc{U}{\ln g_r}{N}
- n \PDc{N}{\ln g_r}{U}
}.
\end{dmath}
%
Introducing the Boltzmann \textAndIndex{entropy}
%
\begin{dmath}\label{eqn:condensedMatterLecture12:60}
S_r = \kB \ln g_r,
\end{dmath}
%
we have (approximately)
%
\begin{dmath}\label{eqn:condensedMatterLecture12:80}
P(\epsilon, n) \propto
\exp\lr{
\ln g_r(U, N)
- \frac{\epsilon}{\kB} \PDc{U}{S_r}{N}
- \frac{n}{\kB} \PDc{N}{S_r}{U}
}.
\end{dmath}
%
Recall the thermodynamic relationship, which we simplify immediately, by restricting ourselves volume (and constant pressure) constraint
%
\begin{dmath}\label{eqn:condensedMatterLecture12:100}
dU_r
= T dS_r - \cancel{p dV_r} + \mu dN_r,
= T
\lr{
\PD{U_r}{S_r} dU_r
+\PD{N_r}{S_r} dN_r
}
%dS_r
+ \mu dN_r.
\end{dmath}
%
Taking wedge products \index{wedge product}, and noting that \(dx \wedge dx = 0\), we can construct a pair of 2-forms from this
%
\begin{subequations}
\begin{dmath}\label{eqn:condensedMatterLecture12:360}
\cancel{dU_r \wedge dU_r}
= T
\lr{
\PD{U_r}{S_r} \cancel{dU_r \wedge dU_r}
+\PD{N_r}{S_r} dN_r \wedge dU_r
}
+ \mu dN_r \wedge dU_r
\end{dmath}
\begin{dmath}\label{eqn:condensedMatterLecture12:380}
dU_r \wedge dN_r
= T
\lr{
\PD{U_r}{S_r} dU_r \wedge dN_r
+\PD{N_r}{S_r} \cancel{dN_r \wedge dN_r}
}
+ \mu \cancel{dN_r \wedge dN_r}.
\end{dmath}
\end{subequations}
%
Since \(dy \wedge dx = - dx \wedge dy\), we can read off the factors of the 2-forms
%
\begin{subequations}
\begin{dmath}\label{eqn:condensedMatterLecture12:120}
\PDc{U_r}{S_r}{P,V} = \inv{T}
\end{dmath}
\begin{dmath}\label{eqn:condensedMatterLecture12:140}
\PDc{N}{S_r}{P,V} = -\frac{\mu}{T}.
\end{dmath}
\end{subequations}
%
Insertion into our probability density gives us
%
\begin{dmath}\label{eqn:condensedMatterLecture12:160}
P(\epsilon, n) \propto e^{-(\epsilon - \mu n)/\kB T}.
\end{dmath}
%
This is the \underlineAndIndex{Boltzmann-Gibbs distribution}.  Normalizing we have
%
\begin{dmath}\label{eqn:condensedMatterLecture12:180}
P(\epsilon, n) =
\frac{e^{-(\epsilon - \mu n)/\kB T}}{
\sum_n \sum_{\epsilon(n)} e^{-(\epsilon - \mu n)/\kB T}
}.
\end{dmath}
%
This method is deemed old fashioned because it relies on being able to calculate the eigenstates of the system.  Imagine the impossibility of this for, say, a room full of air.

It seems that we are labeling the system as having an energy eigenstate since we imagine that it can be characterized as having a single energy level.
%
\paragraph{Application to free electrons}
%
Select one energy level \((*)\), \cref{fig:qmSolidsL12:qmSolidsL12Fig5}, as the `system', and treat all other levels as the `reservoir'.
\imageFigure{../figures/phy487-qmsolids/qmSolidsL12Fig5}{Selected energy level for system}{fig:qmSolidsL12:qmSolidsL12Fig5}{0.15}

Note that \(n_i \in \{0, 1\}\) due to the Pauli exclusion principle, and \(\epsilon \in \{0, \epsilon_i\}\).
%
\begin{subequations}
\begin{dmath}\label{eqn:condensedMatterLecture12:200}
P(\epsilon = 0, n = 0) = \inv{ 1 + e^{-(\epsilon_i - \mu)}}
\end{dmath}
\begin{dmath}\label{eqn:condensedMatterLecture12:220}
P(\epsilon_i, n = 1) = \frac{ e^{-(\epsilon_i - \mu)/\kB T }
}{ 1 + e^{-(\epsilon_i - \mu)} }
\end{dmath}
\end{subequations}
%
Average occupancy
%
\begin{dmath}\label{eqn:condensedMatterLecture12:240}
\expectation{n_i} = \frac{ 0 \times 1 + 1 \times e^{-(\epsilon_i - \mu)/ \kB T } }
{ 1 + e^{-(\epsilon_i - \mu)/\kB T} },
\end{dmath}
%
or
\boxedEquation{eqn:condensedMatterLecture12:260}{
\expectation{n_i} = \inv{ e^{(\epsilon_i - \mu)/\kB T} + 1 }.
}

This is the \underlineAndIndex{Fermi-Dirac distribution}, as sketched roughly in
\cref{fig:qmSolidsL12:qmSolidsL12Fig6}.
\imageFigure{../figures/phy487-qmsolids/qmSolidsL12Fig6}{Fermi Dirac distribution}{fig:qmSolidsL12:qmSolidsL12Fig6}{0.2}

This solved a big mystery, since the equipartition theorem says
%
\begin{dmath}\label{eqn:condensedMatterLecture12:280}
U = \frac{3}{2} \kB T \times
\mathLabelBox
{n}
{electron density},
\end{dmath}
%
so that
%
\begin{dmath}\label{eqn:condensedMatterLecture12:300}
\CV = \PD{T}{U} \sim 10^{28} \frac{\text{electrons}}{m^3} \times \frac{3}{2} \kB,
\end{dmath}
%
however the value of \(\CV\) that was measured is \(1/100\) times too small.  Because of the Pauli exclusion principle, most electrons are trapped far below \(\EF\), and can't accept thermal energy.
%
\section{Heat capacity of free electrons}
\index{specific heat}

Estimate
%
\begin{dmath}\label{eqn:condensedMatterLecture12:320}
U(T) - U(0)
\sim
\mathLabelBox
{\kB T \times D(\EF)}{from F.D. distribution, number of thermally excited electrons}
\times
\mathLabelBox
[
   labelstyle={below of=m\themathLableNode, below of=m\themathLableNode}
]
{ \kB T}
{thermal energy per thermally excited electron}
\end{dmath}
%
%F7
%\cref{fig:qmSolidsL12:qmSolidsL12Fig7}.
%\imageFigure{../figures/phy487-qmsolids/qmSolidsL12Fig7}{7: CAPTION}{fig:qmSolidsL12:qmSolidsL12Fig7}{0.2}

Plugging in the density of states from \eqnref{eqn:condensedMatterLecture11:340}
%
\begin{dmath}\label{eqn:condensedMatterLecture12:340}
C(T)
= \frac{dU}{dT}
\sim 2 \kB^2 D(\EF) T
\sim 2 \kB^2 \frac{1}{2 \pi^2} \lr{ \frac{2m}{\Hbar^2} }^{3/2} \sqrt{\EF} T
\end{dmath}
%
From \eqnref{eqn:condensedMatterLecture11:380b} we have
%
\begin{dmath}\label{eqn:condensedMatterLecture12:400}
\sqrt{\EF}
= \frac{\EF^{3/2}}{\EF}
=
\inv{\EF}
\lr{\frac{ \Hbar^2}{2m}}^{3/2} \lr{\lr{ 3 \pi^2 n}^{2/3}}^{3/2}
=
\inv{\EF}
3 \pi^2 n
\lr{\frac{ \Hbar^2}{2m}}^{3/2},
\end{dmath}
%
so that
\begin{dmath}\label{eqn:condensedMatterLecture12:420}
C(T)
\sim \cancel{2} \kB^2 \frac{1}{\cancel{2 \pi^2}} \cancel{\lr{ \frac{2m}{\Hbar^2} }^{3/2} }
\inv{\EF}
3 \cancel{\pi^2} n
\cancel{\lr{\frac{ \Hbar^2}{2m}}^{3/2}}
T
\sim 3 \kB n
\mathLabelBox
[
   labelstyle={below of=m\themathLableNode, below of=m\themathLableNode}
]
{\frac{ \kB T}{ \EF }}
{reduction from classical values}
\end{dmath}
%
%\cref{fig:qmSolidsL12:qmSolidsL12Fig8}.
\imageFigure{../figures/phy487-qmsolids/qmSolidsL12Fig8}{Specific heat}{fig:qmSolidsL12:qmSolidsL12Fig8}{0.2}

%\EndNoBibArticle
