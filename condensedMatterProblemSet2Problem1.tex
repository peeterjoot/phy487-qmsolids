%
% Copyright � 2013 Peeter Joot.  All Rights Reserved.
% Licenced as described in the file LICENSE under the root directory of this GIT repository.
%
\makeoproblem{Essay about bonding, structure and melting points}{condensedMatter:problemSet2:1}{2013 ps2 p1}{
Explain how trends in melting points in the periodic table
might be explained in terms of the relationship between the
type of bonding and the character of the valence electrons.

For the melting points see the periodic table handed out in the
first lecture, which can also be found at:

\href{http://www.sciencegeek.net/tables/lbltable.pdf}{http://www.sciencegeek.net/tables/lbltable.pdf}

Specifically discuss melting points of (a) the alkali
metals (the first column of the periodic table not
including hydrogen); (b) the noble gases (the last
column of the periodic table); (c) boron and carbon;
(d) oxygen and nitrogen; and (e) the transition metals.

You should write around 300 words (or more).

} % makeproblem

\makeanswer{condensedMatter:problemSet2:1}{
% 124
Starting with the group 1A, the alkali metals \index{alkali metal}, we see in \cref{fig:meltingPointsAlkaliMetals:meltingPointsAlkaliMetalsFig1} a clear trend of decreasing melting points \index{melting point} as the atomic number increases, starting with \ce{Li} melting at 453 K, down to \ce{Cs} melting at the hot summer temperature of 302 K.  Elemental solids for these elements are held together by metallic bonding \index{metallic bond}.  It's likely that the single ``available'' \(n s^1\) orbital electrons are most involved in this metallic bonding.  Because these orbitals increase in size with the atomic radius, we have a delocalization of the electrons involved in this metallic bonding, and it makes sense that there is an associated decrease in melting point with \(Z\) since the electrons involved in the bonding of the corresponding solids are proportionally spread spatially.
\mathImageFigure{../figures/phy487-qmsolids/meltingPointsAlkaliMetalsFig1}{alkali metal melting points}{fig:meltingPointsAlkaliMetals:meltingPointsAlkaliMetalsFig1}{0.2}{meltingPointVsZplots.nb}

Moving to group 8, the noble gases \index{noble gas}, we see in \cref{fig:meltingPointsNobleGases:meltingPointsNobleGasesFig2} that the melting points are all extremely low.  Unlike the alkali metals, the melting points increase with \(Z\).  With these elements all having completely filled orbitals (\(n s^2 n p^6\)) we expect the melting points to be low, since the bonding in the solid state is going to be dominated by \textAndIndex{Van der Waals} bonding.  We have an increase of atomic radii with \(Z\) as we go down the column.  It seems plausible that energy supplied to these elements in solid form would tend to disrupt stable dipole configurations, especially as the radii increases for larger \(Z\) elements.  That is consistent with an inhibition of Van der Waals bonding, accounting for the increase in melting points with \(Z\) that we see in the periodic table data.
%
\mathImageFigure{../figures/phy487-qmsolids/meltingPointsNobleGasesFig2}{noble gas melting points}{fig:meltingPointsNobleGases:meltingPointsNobleGasesFig2}{0.2}{meltingPointVsZplots.nb}

% https://en.wikipedia.org/wiki/Allotropes_of_boron : \ce{B12}
The melting points of \ce{B} (\(1 s^2 2 s^2 p^1\)) and \ce{C} (\(1s^2 2 s^2 p^2\)), as plotted in \cref{fig:meltingPointsBoronAndCarbon:meltingPointsBoronAndCarbonFig2} are very large compared to the alkali metals and noble gases.  Bonding in these elements is covalent, and high energies will be required to break these bonds.  Solid boron is found as \ce{B12} \citep{wiki:boronAllotropes}.  With each boron atom in such a crystal we have enough \(1 p\) orbital electrons to populate the equivalent of two covalent \(1 p^6\) bonds.  The melting points of both the tetrahedral diamond solid structure and the planar graphite structure are similar, with graphite melting 2-47 K higher than diamond.  We can have strong
% diamond
%3550 C
%Graphite
%3652 - 3697 C (sublimes)
%
\mathImageFigure{../figures/phy487-qmsolids/meltingPointsBoronAndCarbonFig2}{boron and carbon melting points}{fig:meltingPointsBoronAndCarbon:meltingPointsBoronAndCarbonFig2}{0.2}{meltingPointVsZplots.nb}

Moving along the \(1 p\) period to \ce{N} we have a remarkable drop in melting point, down to \(63\) K from \(4100\) for \ce{C}.  Observe that \ce{N} with state \(1 s^2 2 s^2 1 p^3\) has half filled orbitals and thus has an inherent stability even in isolation.  We see similar drops in melting points at other points where we have half or completely filled orbitals, such as
\ce{Mn} (\([Ar] 3d^5 4 s^2\) \cref{fig:meltingPointsTransitionMetalsPeriod4:meltingPointsTransitionMetalsPeriod4Fig5}),
\ce{Tc} (\([Kr] 4d^5 5 s^2\) \cref{fig:meltingPointsTransitionMetalsPeriod5:meltingPointsTransitionMetalsPeriod5Fig6}),
\ce{Eu} and \ce{Yb} (\([Xe] 4f^7 6 s^2\), \([Xe] 4 f^{14} 6 s^2\) \cref{fig:meltingPointsTransitionMetalsPeriod6:meltingPointsTransitionMetalsPeriod6Fig7}), and also in the noble gases with \(n p^6\) states.  We will still have covalent bonding for \ce{N} and \ce{O} but this can occur in the smaller geometric configurations \ce{N2} and \ce{O2} which is consistent with lower melting points.  The lower melting point of \ce{O} compared to \ce{N} is somewhat of an oddity and not accounted for by the drop and rise of melting point that we see in the periodic table when transitioning into and from half filled orbital states.  However, we've noted that we can have \(sp\) hybridization in \ce{O}.  It seems reasonable that a difference in stability for this hybrid bonding compared to the purely \(1p\) covalent bonding of solid \ce{N} accounts for the lower melting point of solid \ce{O} compared to solid \ce{N}.

% covalent bonding (esp: n, o)
%Diamond
%3550 C
%Graphite
%3652 - 3697 C (sublimes)
%
% higher MP for graphite?
% http://invsee.asu.edu/nmodules/carbonmod/bonding.html
% http://invsee.asu.edu/nmodules/carbonmod/point.html
% http://www.physicsforums.com/showthread.php?t=701303	
% http://www.preservearticles.com/201101022315/differences-in-physical-properties-of-diamond-and-graphite.html
% http://philosophyofscienceportal.blogspot.ca/2008/03/boron-vs-carbon.html

% http://wiki.answers.com/Q/Why_is_melting_point_of_carbon_higher_than_nitrogen
% http://wiki.answers.com/Q/Why_is_the_ionization_energy_of_nitrogen_is_higher_than_oxygn_and_carbon NOTE: nitrogen half filled orbital.  oxygen doesn't have the half filled orbital (recall: can have: hybridization 2s^2 p^4 -> 2s^1 p^5 with bonding of pairs of s^1 p unfilled.)
% https://en.wikipedia.org/wiki/Solid_oxygen (6 different types... different mp's at different pressures)
% http://www.sciencedirect.com/science/article/pii/S037015730400273X?np=y
% https://en.wikipedia.org/wiki/Oxygen
% https://en.wikipedia.org/wiki/Nitrogen

%\cref{fig:meltingPointsOxygenAndNitrogen:meltingPointsOxygenAndNitrogenFig3}.
\mathImageFigure{../figures/phy487-qmsolids/meltingPointsOxygenAndNitrogenFig3}{oxygen and nitrogen melting points}{fig:meltingPointsOxygenAndNitrogen:meltingPointsOxygenAndNitrogenFig3}{0.2}{meltingPointVsZplots.nb}
%\cref{fig:meltingPointsPeriod1:meltingPointsPeriod1Fig4}.
%\mathImageFigure{../figures/phy487-qmsolids/meltingPointsPeriod1Fig4}{period 2 melting points}{fig:meltingPointsPeriod1:meltingPointsPeriod1Fig4}{0.2}{meltingPointVsZplots.nb}

% http://www.chemguide.co.uk/atoms/questions/a-metallicbond.pdf
The transition metals \index{transition metal} are those elements characterized by incomplete filled \(d\) and \(f\) orbitals in their ground state (although some of these may also have incomplete \(s\) orbitals as the filling sequence is not necessarily uniform).  These elements, plotted in
\cref{fig:meltingPointsTransitionMetalsPeriod4:meltingPointsTransitionMetalsPeriod4Fig5},
\cref{fig:meltingPointsTransitionMetalsPeriod5:meltingPointsTransitionMetalsPeriod5Fig6}, and
\cref{fig:meltingPointsTransitionMetalsPeriod6:meltingPointsTransitionMetalsPeriod6Fig7}
form metallic bonds due to these delocalized d, and f-orbital electrons.
The melting point dips at \ce{Mn}, \ce{Tc}, as noted above, are the elements for which we have exactly half filled orbitals.
The maximum melting points are found in the neighborhood of these half filled orbital states, and decrease towards the completely filled orbital states on either side (or the column 1 elements which are fairly close to completely filled) as one progresses up or down the period.
We've seen with the noble gases with their completely filled \(p\) orbitals have the lowest melting points, but we have local minimums just past the transition metals at \ce{Zn}, \ce{Cd}, \ce{Hg} ( \([Ar]3d^10 4s^2\), \([Kr]4d^10 5s^2\), \([Xe]4 f^14 5d^10 6s^2\) ) just before the \(p\) block.  This last element mercury is very familiar as the oddball metal that is liquid at room temperature.  We see this is a consequence of the stability of its completely filled orbitals.  With the higher melting points in the central regions of the transition metal periods is appears that more delocalized (but incompletely filled) orbitals allows for stronger bonding, requiring more energy to break the metallic bonds and melt the metal.

%While the nobel gases have the absolute lowest melting points with their \(p\) orbitals are also filled, but at the extremes of each of these transition metal periods we find the low melting point elements
%Away from these completely filled orbital states,
%% and also have low melting points, the delocalized electrons in the metallic bonds of the incompletely filled \(d\)-\(f\) orbitals
%
% elements, more energy is required to break those bonds
%
%Two characteristic trends in the melting points of these elements can be observed.  We have a close to uniform (with exceptions for half filled states noted previously) downtrends in all of the ranges
%\ce{Cr}-\ce{Zn} ( \([Ar]3d^5 4s^1\) - \([Ar]3d^10 4s^2\), \cref{fig:meltingPointsTransitionMetalsPeriod4:meltingPointsTransitionMetalsPeriod4Fig5} )
%\ce{Mo}-\ce{Cd} ( \([Kr]4d^5 5s^1\) - \([Kr]4d^10 5s^2\), \cref{fig:meltingPointsTransitionMetalsPeriod5:meltingPointsTransitionMetalsPeriod5Fig6} )
%\ce{W}-\ce{Hg} ( \([Xe]4 f^14 5d^4 6s^2\) - \([Xe]4 f^14 5d^10 6s^2\), \cref{fig:meltingPointsTransitionMetalsPeriod6:meltingPointsTransitionMetalsPeriod6Fig7} ).  In particular observe that at the end of this last range \ce{Hg} where the orbitals are completely filled we have the familiar element mercury, which has the lowest melting point of the period 6 elements until we hit \ce{Rn} where the \(p\) orbitals are also completely filled.  We see the very lowest melting points for the elements that have completely filled orbitals.  At the lower end of each of the periods we have the opposite.  Consider \ce{K} and \ce{Ca} for example, the first two elements in period 4, where the melting points for the solids goes up three fold (and continues to climb as we progress along the period until the peak at \ce{Cr}).  Also associated with this progression down the period is an decrease in radii, due to shielding effects.
% due to this smaller radii.  This accounts for the increase in melting point that occurs until the \(d\) orbital filling approaches the half filled orbital states of column 7A.

%
\mathImageFigure{../figures/phy487-qmsolids/meltingPointsTransitionMetalsPeriod4Fig5}{period 4 transition metal melting points}{fig:meltingPointsTransitionMetalsPeriod4:meltingPointsTransitionMetalsPeriod4Fig5}{0.2}{meltingPointVsZplots.nb}
\mathImageFigure{../figures/phy487-qmsolids/meltingPointsTransitionMetalsPeriod5Fig6}{period 5 transition metal melting points}{fig:meltingPointsTransitionMetalsPeriod5:meltingPointsTransitionMetalsPeriod5Fig6}{0.2}{meltingPointVsZplots.nb}
\mathImageFigure{../figures/phy487-qmsolids/meltingPointsTransitionMetalsPeriod6Fig7}{period 6 transition metal melting points}{fig:meltingPointsTransitionMetalsPeriod6:meltingPointsTransitionMetalsPeriod6Fig7}{0.2}{meltingPointVsZplots.nb}
}

