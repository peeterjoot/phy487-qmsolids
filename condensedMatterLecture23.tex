%
% Copyright � 2013 Peeter Joot.  All Rights Reserved.
% Licenced as described in the file LICENSE under the root directory of this GIT repository.
%
%\input{../blogpost.tex}
%\renewcommand{\basename}{condensedMatterLecture23}
%\renewcommand{\dirname}{notes/phy487/}
%\newcommand{\keywords}{Condensed matter physics, PHY487H1F}
%\input{../peeter_prologue_print2.tex}
%
%%\citep{harald2003solid} \S x.y
%%\citep{ibach2009solid} \S x.y
%
%%\usepackage{mhchem}
%\usepackage[version=3]{mhchem}
%\usepackage{units}
%\usepackage{bm} % \bcE
%\newcommand{\nought}[0]{\circ}
%%\newcommand{\EF}[0]{\epsilon_{\txtF}}
%\newcommand{\EF}[0]{E_{\txtF}}
%\newcommand{\kF}[0]{k_{\txtF}}
%
%\beginArtNoToc
%\generatetitle{PHY487H1F Condensed Matter Physics.  Lecture 23: Superconductivity.  Taught by Prof.\ Stephen Julian}
%%\chapter{Superconductivity}
%\label{chap:condensedMatterLecture23}
%
%\section{Disclaimer}
%
%Peeter's lecture notes from class.  May not be entirely coherent.
%
\section{Superconductivity overview}
\index{superconductivity}

\reading \citep{ibach2009solid} \S 10.1

The effect is actually not as rare as one would imagine (see \citep{ibach2009solid} \textfigref{10.2} for a chart shown of many superconductive elements).  Even \ce{O} and \ce{S} are superconductive, with fairly high transition temperatures, if pressurized enough to metalize it.

Some of the non-superconductive elements are those with very spherical Fermi surfaces.

The basic phenomena, that of resistivity drop at a critical temperature, is sketched in \cref{fig:qmSolidsL23:qmSolidsL23Fig1}.  We also have a corresponding difference in the specific heat for these materials, as sketched in \cref{fig:qmSolidsL23:qmSolidsL23Fig2}.
%
\imageFigure{../figures/phy487-qmsolids/qmSolidsL23Fig1}{Electrical resistivity}{fig:qmSolidsL23:qmSolidsL23Fig1}{0.2}
\imageFigure{../figures/phy487-qmsolids/qmSolidsL23Fig2}{Specific heat of superconductor}{fig:qmSolidsL23:qmSolidsL23Fig2}{0.2}

The exponential indicates that there's a gap to the first excited state (cf exponentials from the semiconductor theory?)

The essential feature of superconductivity is not that they are ``perfect conductors'', but that they are perfect dimagnets as illustrated in \citep{ibach2009solid} \textfigref{10.4}, and more roughly in \cref{fig:qmSolidsL23:qmSolidsL23Fig3}.
%
\imageFigure{../figures/phy487-qmsolids/qmSolidsL23Fig3}{Diamagnetic phenomena}{fig:qmSolidsL23:qmSolidsL23Fig3}{0.2}

We have a circulating current that screens the field.  This costs energy.  There are two types of superconductors

\begin{itemize}
\item Type I, where the superconductivity collapses above \(H_{\txtc}\), sketched in \cref{fig:qmSolidsL23:qmSolidsL23Fig4}.
\item Type II, particle flux expulsion above \(H_{\txtc}\), sketched in \cref{fig:qmSolidsL23:qmSolidsL23Fig5}.
\end{itemize}
%
\imageFigure{../figures/phy487-qmsolids/qmSolidsL23Fig4}{Type I}{fig:qmSolidsL23:qmSolidsL23Fig4}{0.2}
\imageFigure{../figures/phy487-qmsolids/qmSolidsL23Fig5}{Type II}{fig:qmSolidsL23:qmSolidsL23Fig5}{0.2}
%
\section{London equations, and perfect conductors}
\index{London equations}
\index{perfect conductors}

\reading \citep{ibach2009solid} \S 10.2

With resistivity \(\rho = 0\), we have
%
\begin{dmath}\label{eqn:qmSL23:20}
m \dot{\Bv} = - e \bcE,
\end{dmath}
%
and current density
%
\begin{dmath}\label{eqn:qmSL23:40}
\Bj = - n_{\txts} e \Bv,
\end{dmath}
%
for
%
\begin{dmath}\label{eqn:qmSL23:60}
\PD{t}{\Bj} = \frac{n_{\txts} e^2}{m} \bcE,
\end{dmath}
%
With Maxwell's third
%
\begin{dmath}\label{eqn:qmSL23:80}
\spacegrad \cross \bcE = - \PD{t}{\BB},
\end{dmath}
%
we have
%
\begin{dmath}\label{eqn:qmSL23:100}
\PD{t}{} \lr{
\frac{m}{n_{\txts} e^2} \spacegrad \cross \Bj + \BB
}
= 0,
\end{dmath}
%
so that the equation for a \underlineAndIndex{perfect diamagnet} satisfies
\boxedEquation{eqn:qmSL23:120}{
\frac{m}{n_{\txts} e^2} \spacegrad \cross \Bj = - \BB
}
%
\begin{equation}\label{eqn:qmSL23:140}
\spacegrad \cross \Bj = - \frac{n_{\txts} e^2}{m} \BB = -\inv{\lambda_{\txtL}} \BB.
\end{equation}
%
\makeexample{Solution for constant magnetic field in \(\xcap\) direction}{example:qmSL23:n}{
Consider a unidirectional magnetic field
%
\begin{equation}\label{eqn:qmSL23:260}
\BB = (B_x, 0, 0),
\end{equation}
%
as sketched in \cref{fig:qmSolidsL23:qmSolidsL23Fig6}.
%
\imageFigure{../figures/phy487-qmsolids/qmSolidsL23Fig6}{Magnetic Field}{fig:qmSolidsL23:qmSolidsL23Fig6}{0.2}

With no time dependence of the electric field (no displacement term \(\mu_\nought \epsilon_\nought \PDi{t}{\BE}\)) Ampere's Maxwell's equations is
%
\begin{equation}\label{eqn:qmSL23:160}
\spacegrad \cross \BB = \mu_\nought \Bj.
\end{equation}
%
Taking cross products of both sides of \eqnref{eqn:qmSL23:140}, we have
%
\begin{equation}\label{eqn:qmSL23:180}
\spacegrad \cross \lr{ \spacegrad \cross \Bj }
=
\spacegrad \cross \lr{
- \frac{\BB}{\lambda_{\txtL}}
}
= - \frac{\mu_\nought}{\lambda_{\txtL}} \Bj.
\end{equation}
%
By taking dot products of \eqnref{eqn:qmSL23:160}, and noting that \(\spacegrad \cdot (\spacegrad \cross \BX) = 0\), for sufficiently continuous fields \(\BX\), we must have \(\spacegrad \cdot \Bj = 0\), and thus
%
\begin{dmath}\label{eqn:qmSL23:360}
\spacegrad \cross \lr{ \spacegrad \cross \Bj }
= -\spacegrad^2 \Bj + \spacegrad (\spacegrad \cdot \Bj)
= -\spacegrad^2 \Bj.
\end{dmath}
%
\Eqnref{eqn:qmSL23:180} is now decoupled
%
\begin{equation}\label{eqn:qmSL23:200}
-\spacegrad^2 \Bj = - \frac{\mu_\nought}{\lambda_{\txtL}} \Bj.
\end{equation}
%
This has solution
%
\begin{equation}\label{eqn:qmSL23:220}
\Bj = \Bj_\nought e^{-\sqrt{\mu_\nought/\lambda_{\txtL}} z},
\end{equation}
%
and
\begin{equation}\label{eqn:qmSL23:240}
\BB = \xcap B_x^\nought e^{-\sqrt{\mu_\nought/\lambda_{\txtL}} z}.
\end{equation}
}
\section{Cooper pairing}
\index{Cooper pairing}
\reading \citep{ibach2009solid} \S 10.2

There's an effective attraction between the electrons in a metal that is mediated by phonons, as sketched in \cref{fig:qmSolidsL23:qmSolidsL23Fig7}.
\imageFigure{../figures/phy487-qmsolids/qmSolidsL23Fig7}{Lattice distorts after the electron passes}{fig:qmSolidsL23:qmSolidsL23Fig7}{0.2}

The second electron can lower it's energy by traveling in the ``wake'' of the first.  Note that this is actually an incorrect picture, since what really goes on is that the second electron passes in the opposite direction from the first to take advantage of the wake.

It is important that the lattice reaction is retarded in time.
%
\paragraph{electron-phonon interaction}
\index{electron-phonon interaction}
The Cooper calculation considered a filled Fermi surface (such as a sphere) at \(T = 0\).  He considered what happens if you add two electrons above \(\EF\) as sketched in \cref{fig:qmSolidsL23:qmSolidsL23Fig8}.
%
\imageFigure{../figures/phy487-qmsolids/qmSolidsL23Fig8}{Filled Fermi sphere at \(T = 0\)}{fig:qmSolidsL23:qmSolidsL23Fig8}{0.2}

The electron phonon interaction looks like
%
\begin{equation}\label{eqn:qmSL23:280}
\rho_{\mathrm{el}}( x, t ) \rho_{\mathrm{ph}}(x, t )
\end{equation}
%
\begin{equation}\label{eqn:qmSL23:300}
\ket{ \Bk, \uparrow } \rightarrow \ket{ \Bk, \uparrow, \text{ 0 phonons }}
+ \sum_\Bq \alpha_q | \Bk - \Bq_i \ket{ \Bq }
\end{equation}
%
Here the state has a Bloch form
%
\begin{equation}\label{eqn:qmSL23:320}
\ket{ \Bk, \uparrow } = \inv{\sqrt{L^3}} e^{i \Bk \cdot \Br } \ket{\sigma}
\end{equation}
%
electron density
%
\begin{dmath}\label{eqn:qmSL23:340}
\begin{aligned}
\braket{\psi}{\psi} &=
\inv{L^3}
\lr{
e^{-i \Bk \cdot \Br} \bra{0_q} + \alpha_q e^{-i(\Bk - \Bq) \cdot \Br} \bra{1_q}
}
\lr{
e^{i \Bk \cdot \Br} \ket{0_q} + e^{i(\Bk - \Bq) \cdot \Br} \ket{1_q}
} \\
&=
\inv{L^3}
{
1 + \alpha_q^2 + \alpha_q \lr{
e^{i \Bq \cdot \Br }
+ e^{-i \Bq \cdot \Br }
}
} \\
&=
\inv{L^3}
{
1 + \alpha_q^2 + \alpha_q \cos\lr{ \Bq \cdot \Br }
}
\end{aligned}
\end{dmath}
%
Here the cosine is the modulated charge density.   This transition is sketched in \cref{fig:qmSolidsL23:qmSolidsL23Fig9}.
%
\imageFigure{../figures/phy487-qmsolids/qmSolidsL23Fig9}{Pair state transitions outside of Fermi sphere}{fig:qmSolidsL23:qmSolidsL23Fig9}{0.2}

%\EndArticle
