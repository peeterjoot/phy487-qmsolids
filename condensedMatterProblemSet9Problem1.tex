%
% Copyright � 2013 Peeter Joot.  All Rights Reserved.
% Licenced as described in the file LICENSE under the root directory of this GIT repository.
%
\makeoproblem{Two and three-dimensional band structure and Fermi surface}{condensedMatter:problemSet9:1}{2013 ps9 p1}{

Consider a two-dimensional square lattice with one atom per unit cell,
where each atom contributes two electrons to the conduction band.
Assume that the band structure is free-electron-like.

\makesubproblem{}{condensedMatter:problemSet9:1a}
Show that the free-electron Fermi surface extends beyond the boundary of the
first Brillouin zone, and state by how much the free-electron Fermi wave-vector,
\(\kF\), extends beyond the boundary.

\makesubproblem{}{condensedMatter:problemSet9:1b}

Sketch how the circular free-electron Fermi surface reconstructs due to
  the lattice potential, assuming that the lattice potential introduces only
  small energy gaps at the Brillouin zone boundary.
State whether the reconstructed Fermi surfaces enclose filled states
  (so-called \dquoteAndIndex{electron pockets}) or empty states (so-called \dquoteAndIndex{hole pockets}).
} % makeproblem

\makeanswer{condensedMatter:problemSet9:1}{
\makeSubAnswer{}{condensedMatter:problemSet9:1a}

Our geometry is sketched in \cref{fig:qmSolidsPs9P1a:qmSolidsPs9P1aFig1}.

\imageFigure{../figures/phy487-qmsolids/qmSolidsPs9P1aFig1}{2D cubic lattice}{fig:qmSolidsPs9P1a:qmSolidsPs9P1aFig1}{0.15}

%Our lattice and reciprocal bases are
%
%\begin{equation}\label{eqn:condensedMatterProblemSet9Problem1:20}
%\begin{aligned}
%\Ba_i &\in \{a \xcap, a \ycap \} \\
%\Bg_i &\in \{\frac{2 \pi}{a} \xcap, \frac{2 \pi}{a} \ycap \}.
%\end{aligned}
%\end{equation}
%
With just a single atom in the primitive unit cell, the number density of the electrons is just
%
\begin{equation}\label{eqn:condensedMatterProblemSet9Problem1:40}
n = \frac{N}{V} = \frac{2}{a^2}.
\end{equation}
%
The 2D Fermi wave-vector is given implicitly by this area number density, considering the unit circle of radius \(\kF\) that can contain that number of electrons
%
\begin{equation}\label{eqn:condensedMatterProblemSet9Problem1:60}
N = 2 \times \pi \kF^2 \times \frac{A}{(2\pi)^2} = \frac{A \kF^2}{2 \pi},
\end{equation}
%
or
%
\begin{equation}\label{eqn:condensedMatterProblemSet9Problem1:80}
\kF = \sqrt{ 2 \pi n }.
\end{equation}
%
For this bivalent lattice we have
%
\begin{dmath}\label{eqn:condensedMatterProblemSet9Problem1:100}
\kF = \sqrt{ 2 \pi \frac{2}{a^2} } = \frac{2 \sqrt{\pi} }{a} \approx \frac{3.54}{a}.
\end{dmath}
%
The first Brillouin zone is sketched in \cref{fig:qmSolidsPs9P1a:qmSolidsPs9P1aFig2}.

\imageFigure{../figures/phy487-qmsolids/qmSolidsPs9P1aFig2}{First BZ for cubic 2D lattice}{fig:qmSolidsPs9P1a:qmSolidsPs9P1aFig2}{0.2}

From the center of the BZ to the nearest point on the boundary we have a distance of
%
\begin{equation}\label{eqn:condensedMatterProblemSet9Problem1:120}
\frac{\pi}{a} \approx \frac{3.14}{a} < \kF
\end{equation}
%
So \(\kF\) extends beyond the BZ boundary by

\boxedEquation{eqn:condensedMatterProblemSet9Problem1:140}{
\kF - \frac{\pi}{a} = \frac{2 \sqrt{\pi} - \pi}{a} \approx \frac{0.40}{a}.
}

\makeSubAnswer{}{condensedMatter:problemSet9:1b}

The BZ and the Fermi circle has been plotted to scale, with the reconstruction surfaces sketched over top of the plot in \cref{fig:qmSolidsPs9P1b:qmSolidsPs9P1bFig1pnSurface}.

\imageFigure{../figures/phy487-qmsolids/qmSolidsPs9P1bFig1pnSurface}{Fermi surface for 2D cubic lattice}{fig:qmSolidsPs9P1b:qmSolidsPs9P1bFig1pnSurface}{0.2}

Sketching the repeated zone scheme rather roughly (no longer to scale) in \cref{fig:qmSolidsPs9P1b:qmSolidsPs9P1bFig2}, it's easier to visualize the disconnected pockets of (blue) enclosing unfilled states (hole pockets).

\paragraph{Grading remark:} my comment ``hole pockets'' was marked wrong.  This was, in truth, a guess, and I didn't understand this business of hole pocket vs. electron pocket and didn't find the text particularly helpful to explain it.  A corrected sketch was drawn (somewhat like \cref{fig:qmSolidsPs9P1b:qmSolidsPs9P1bFig2}).

Prof Julian responded about this: ``If you go back to the free electron sphere, then the ellipsoids that are along the edges of the Brillouin zone have occupied states between the Fermi surface and the Brillouin zone boundary.  So when you close the pocket on the other size of the Brillouin zone, occupied states are enclosed.

On the other hand, the corner pockets come from sections of the free electron sphere that have unoccupied states between the Fermi surface and the Brillouin zone corner.  Thus when you enclose the BZ corner by closing the Fermi surface in the repeated zone scheme, you are enclosing empty states.''

\imageFigure{../figures/phy487-qmsolids/qmSolidsPs9P1bFig2}{Repeated zone scheme for cubic lattice}{fig:qmSolidsPs9P1b:qmSolidsPs9P1bFig2}{0.3}
}
