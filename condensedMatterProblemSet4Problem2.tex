%
% Copyright © 2016 Peeter Joot.  All Rights Reserved.
% Licenced as described in the file LICENSE under the root directory of this GIT repository.
%
\makeoproblem{one-dimensional chain with two spring constants}{condensedMatter:problemSet4:2}{2013 ps4 p2}{

Consider the one-dimensional chain of \cref{fig:two_springs:two_springsFig1} in which all masses have the same value, \(m\), but the spring constant alternates between \(k\) and \(k'\), as shown below.  The equilibrium length of the springs is the same, and equal to \(a/2\), where \(a\) is the lattice spacing.

\imageFigure{../figures/phy487-qmsolids/two_springsFig1}{Two springs chain}{fig:two_springs:two_springsFig1}{0.1}
%
%\includegraphics[width=12cm]{./figures/two_springs.pdf}

\makesubproblem{}{condensedMatter:problemSet4:2a}
Find the coupled equations of motion for
\(u_{n,1}\) and \(u_{n,2}\), and then make the ansatz
\begin{eqnarray*}
%u_{n,\alpha} = u_{\alpha}(q) {\rm e}^{i(qx_n - \omega t)},
u_{n,\alpha} = u_{\alpha}(q) e^{i(qx_n - \omega t)},
\end{eqnarray*}
to obtain coupled equations for \(u_1(q)\) and \(u_2(q)\).

\makesubproblem{}{condensedMatter:problemSet4:2b}
Solve these coupled equations, putting \(k = k_\nought + \delta\)
and \(k' = k_\nought - \delta\), to show that the dispersion relation is:
\begin{eqnarray*}
\omega^2 = \frac{2k_\nought}{m}\left( 1 \pm \sqrt{ \cos^2\frac{qa}{2}
               + \frac{\delta^2}{k_\nought^2} \sin^2 \frac{qa}{2}} \,\, \right).
\end{eqnarray*}
%
\makesubproblem{}{condensedMatter:problemSet4:2c}
At \(q = 0\), plug the two solutions for \(\omega\) back into the
coupled equations for \(u_1(q)\) and  \(u_2(q)\) to find the relative
motion of masses 1 and 2 in each primitive unit cell.  Explain how
these relative motions of the two masses give rise to the two
frequencies at \(q=0\).

\makesubproblem{}{condensedMatter:problemSet4:2d}
Plot \(\omega(q)\) vs.\ \(q\) in the range \(q=0\) to \(q=2\pi/a\),
for \(\delta/k_\nought = 0.9,0.1\) and 0.  % fixed this
Indicate the boundary of the first
Brillouin zone.  (I did this by programming my equation for
\(\omega(q)\) into a plotting package; you can also do it by hand.)

\makesubproblem{}{condensedMatter:problemSet4:2e}
For \(\delta/k_\nought = 0.9\) explain why the two branches of the
phonon spectrum are so far apart.  What happens in the limit where
\(\delta = k_\nought\)?  Explain the dispersion relation (or rather, lack
thereof) of the acoustic and optical branches in this limit.

\makesubproblem{}{condensedMatter:problemSet4:2f}
For \(\delta = 0\), explain the form of your plot (hint: when
  \(\delta=0\) the two springs are identical, so think about the
  primitive unit cell and first Brillouin zone size in this case;
  also, you may recall the calculation of the structure factor
  in which we treated a bcc lattice as simple cubic with a two-atom
  basis).
} % makeproblem

\makeanswer{condensedMatter:problemSet4:2}{
\makeSubAnswer{}{condensedMatter:problemSet4:2a}

In order to avoid the mental trauma of trying to figure out all the signs for the spring constant potential coefficients, we can describe the system by the Lagrangian
%
\begin{dmath}\label{eqn:condensedMatterProblemSet4Problem2:20}
\begin{aligned}
\LL &= \sum_{n, \alpha} \frac{m}{2} \dot{u}_{n, \alpha}^2 \\
&\quad - \frac{k'}{2}\lr{ u_{n, 1} - u_{n - 1, 2}}^2
- \frac{k}{2}\lr{ u_{n, 2} - u_{n, 1}}^2
- \frac{k'}{2}\lr{ u_{n + 1, 1} - u_{n, 2}}^2
- \cdots
\end{aligned}
\end{dmath}
%
The force equations then follow directly from the Euler-Lagrange equations
%
\begin{dmath}\label{eqn:condensedMatterProblemSet4Problem2:40}
0 = \ddt{} \PD{\dot{u}_{n, \alpha}}{\LL}
- \PD{u_{n, \alpha}}{\LL}.
\end{dmath}
%
That is
%
\begin{equation}\label{eqn:condensedMatterProblemSet4Problem2:60}
\begin{aligned}
0 &= m \ddot{u}_{n, 1}
+
( k + k')
u_{n, 1}
- k
u_{n, 2}
- k'
u_{n - 1, 2} \\
0 &= m \ddot{u}_{n, 2}
+
( k + k')
u_{n, 2}
- k
u_{n, 1}
- k'
u_{n + 1, 1}
\end{aligned}
\end{equation}
%
Our trial solution functions from above are
%
\begin{equation}\label{eqn:condensedMatterProblemSet4Problem2:220}
\begin{aligned}
u_{n - 1, 2} &= u_2 e^{ i(q a (n - 1) - \omega t) } \\
u_{n, 1} &= u_1 e^{ i(q a n - \omega t) } \\
u_{n, 2} &= u_2 e^{ i(q a n - \omega t) } \\
u_{n + 1, 1} &= u_1 e^{ i(q a (n + 1) - \omega t) },
\end{aligned}
\end{equation}
%
which gives us
%
\begin{dmath}\label{eqn:condensedMatterProblemSet4Problem2:300}
\begin{aligned}
0 &= -m \omega^2
u_1 \cancel{e^{ i(q a n - \omega t) } } \\
&\quad +
( k + k')
u_1 \cancel{e^{ i(q a n - \omega t) } }
- k
u_2 \cancel{e^{ i(q a n - \omega t) } }
- k'
u_2 e^{ i(q a (\cancel{n} - 1) - \cancel{\omega t}) }
0 &= -\omega^2 m
u_2 \cancel{e^{ i(q a n - \omega t) } } \\
&\quad +
( k + k')
u_2 \cancel{e^{ i(q a n - \omega t) } }
- k
u_1 \cancel{e^{ i(q a n - \omega t) } }
- k'
u_1 e^{ i(q a (\cancel{n} + 1) - \cancel{\omega t}) },
\end{aligned}
\end{dmath}
%
or
\begin{equation}\label{eqn:condensedMatterProblemSet4Problem2:340}
\begin{aligned}
0 &= -m \omega^2
u_1
+
2 k_\nought
u_1
- (k_\nought + \delta)
u_2
- (k_\nought - \delta)
u_2 e^{ -i q a } \\
0 &= -m \omega^2
u_2
+
2 k_\nought
u_2
- (k_\nought + \delta)
u_1
- (k_\nought - \delta)
u_1 e^{ i q a }.
\end{aligned}
\end{equation}
%
\makeSubAnswer{}{condensedMatter:problemSet4:2b}

We seek \(\omega^2\) solutions to the determinant
%
\begin{dmath}\label{eqn:condensedMatterProblemSet4Problem2:380}
\begin{aligned}
0 &=
\begin{vmatrix}
-m \omega^2 + 2 k_\nought &
- (k_\nought + \delta)
- (k_\nought - \delta) e^{ -i q a }
\\
- (k_\nought + \delta)
- (k_\nought - \delta) e^{ i q a }
&
-m \omega^2 + 2 k_\nought
\end{vmatrix} \\
&=
(-m \omega^2 + 2 k_\nought )^2
-
\lr{
  (k_\nought + \delta)
+ (k_\nought - \delta) e^{ -i q a }
}
\lr{
  (k_\nought + \delta)
+ (k_\nought - \delta) e^{ i q a }
} \\
&=
(-m \omega^2 + 2 k_\nought )^2 \\
&\quad -
\lr{
  (k_\nought + \delta)
e^{ i q a/2 }
+ (k_\nought - \delta)
e^{ -i q a/2 }
}
e^{ -i q a/2 }
e^{ i q a/2 }
\lr{
  (k_\nought + \delta) e^{ -i q a/2 }
+ (k_\nought - \delta) e^{ i q a/2 }
} \\
&=
(-m \omega^2 + 2 k_\nought )^2
-
\lr{
  2 k_\nought \cos (qa/2)
+ 2i \delta \sin (qa/2)
}
\lr{
  2 k_\nought \cos (qa/2)
- 2i \delta \sin (qa/2)
} \\
&=
(-m \omega^2 + 2 k_\nought )^2
-
4 \lr{
  k_\nought^2 \cos^2 (qa/2)
+ \delta^2 \sin^2 (qa/2)
}
\end{aligned}
\end{dmath}
%
%%%Using the trial solution
%%%
%%%\begin{dmath}\label{eqn:condensedMatterProblemSet4Problem2:100}
%%%u_{n,\alpha} = u_{\alpha}(q) e^{i(qx_{n, \alpha} - \omega t)},
%%%\end{dmath}
%%%
%%%\begin{subequations}
%%%\begin{dmath}\label{eqn:condensedMatterProblemSet4Problem2:120}
%%%0 = -m \omega^2 u_1 e^{i( q a n - \omega t)}
%%%+
%%%( k + k')
%%%u_1 e^{i( q a n - \omega t)}
%%%- k
%%%u_2 e^{i( q a (n + 1/2) - \omega t)}
%%%- k'
%%%u_2 e^{i( q a (n - 1/2) - \omega t)}
%%%\end{dmath}
%%%\begin{dmath}\label{eqn:condensedMatterProblemSet4Problem2:140}
%%%0 = -m \omega^2 u_2 e^{i( q a (n + 1/2) - \omega t)}
%%%+
%%%( k + k')
%%%u_2 e^{i( q a (n + 1/2) - \omega t)}
%%%- k
%%%u_1 e^{i( q a n - \omega t)}
%%%- k'
%%%u_1 e^{i( q a (n + 1) - \omega t)}
%%%\end{dmath}
%%%\end{subequations}
%%%
%%%Cancelling terms, substitution for \(k\) and \(k'\), and a bit of rearranging yields
%%%
%%%\begin{subequations}
%%%\begin{dmath}\label{eqn:condensedMatterProblemSet4Problem2:160}
%%%0 = -m \omega^2 u_1
%%%+
%%%2 k_\nought
%%%u_1
%%%- (k_\nought + \delta)
%%%u_2 e^{i q a /2 }
%%%- (k_\nought - \delta)
%%%u_2 e^{-i q a /2 }
%%%\end{dmath}
%%%\begin{dmath}\label{eqn:condensedMatterProblemSet4Problem2:180}
%%%0 = -m \omega^2 u_2
%%%+
%%%2 k_\nought
%%%u_2
%%%- (k_\nought + \delta)
%%%u_1
%%%e^{-i q a /2 }
%%%- (k_\nought - \delta)
%%%u_1 e^{i q a/2 }.
%%%\end{dmath}
%%%\end{subequations}
%%%
%%%This has a solution when the determinant is zero
%%%\begin{dmath}\label{eqn:condensedMatterProblemSet4Problem2:200}
%%%0 =
%%%\begin{vmatrix}
%%%-m \omega^2 +
%%%2 k_\nought &
%%%- 2 k_\nought \cos \lr{ q a /2 }
%%%- 2 i \delta \sin \lr{ q a /2 }
%%%%- (k_\nought + \delta) e^{i q a /2 }
%%%%- (k_\nought - \delta) e^{-i q a /2 }
%%%\\
%%%- 2 k_\nought \cos \lr{ q a /2 }
%%%+ 2 i \delta \sin \lr{ q a /2 }
%%%%- (k_\nought + \delta) e^{-i q a /2 }
%%%%- (k_\nought - \delta) e^{i q a/2 }
%%%&
%%%-m \omega^2 + 2 k_\nought
%%%\end{vmatrix}
%%%=
%%%\lr{ -m \omega^2 + 2 k_\nought }^2
%%%+ 4 \sqrt{
%%%k_\nought^2 \cos^2 \lr{ q a /2 }
%%%+
%%%\delta^2 \sin^2 \lr{ q a /2 }
%%%},
%%%\end{dmath}
%
or
\begin{dmath}\label{eqn:condensedMatterProblemSet4Problem2:280}
\omega^2 =
\frac{2 k_\nought }{m} \lr{ 1
\pm
\sqrt{
\cos^2 \lr{ q a /2 }
+
\frac{\delta^2}{k_\nought} \sin^2 \lr{ q a /2 }
}
},
\end{dmath}
%
as desired.

\makeSubAnswer{}{condensedMatter:problemSet4:2c}

At \(q = 0\) we have
%
\begin{dmath}\label{eqn:condensedMatterProblemSet4Problem2:400}
\omega^2 = \frac{2 k_\nought }{m} \lr{ 1 \pm 1 },
\end{dmath}
%
or
\begin{dmath}\label{eqn:condensedMatterProblemSet4Problem2:420}
\omega \in \{ \pm 2 \omega_\nought, 0 \},
\end{dmath}
%
where the natural frequency for the average spring constant is
%
\begin{dmath}\label{eqn:condensedMatterProblemSet4Problem2:580}
\omega_\nought^2 = \frac{k_\nought}{m}.
\end{dmath}
%
\paragraph{Zero frequency case.}

For the \(\omega = 0, q = 0\) case, our equations of motion
\eqnref{eqn:condensedMatterProblemSet4Problem2:340}, now take the form
%
\begin{dmath}\label{eqn:condensedMatterProblemSet4Problem2:440}
2 k_\nought u_1
=
\lr{
 (k_\nought + \delta)
+ (k_\nought - \delta)
e^{ -i q a }
}
u_2
=
\lr{
 (k_\nought + \delta)
+ (k_\nought - \delta)
}
u_2
=
2 k_\nought u_2.
\end{dmath}
%
Both the ``motions'' in the primitive unit cell is described by
%
\begin{dmath}\label{eqn:condensedMatterProblemSet4Problem2:620}
\begin{aligned}
u_{n, 1} &= u_2 \\
u_{n, 2} &= u_2,
\end{aligned}
\end{dmath}
%
where \(u_2 = u_2(q)\) an arbitrary undetermined function.  This solution represents uniform translation.

%  If we consider this system subject to the period boundary conditions used in the 1D chain in class, then this distribution is subject to the constraint on \(q\) of
%
%\begin{dmath}\label{eqn:condensedMatterProblemSet4Problem2:520}
%q a N = 2 \pi m,
%\end{dmath}
%
%or
%
%\begin{dmath}\label{eqn:condensedMatterProblemSet4Problem2:540}
%q a = \frac{2 \pi m}{N},
%\end{dmath}
%
%where \(m\) is an integer.

\paragraph{Non-zero frequency case.}

For the non-constant time dependent solutions, where \(\omega \ne 0, q = 0\), we have
%
\begin{dmath}\label{eqn:condensedMatterProblemSet4Problem2:560}
\omega = \pm 2 \omega_\nought,
\end{dmath}
%
so that our equations of motion \eqnref{eqn:condensedMatterProblemSet4Problem2:340}, now take the form
%
\begin{dmath}\label{eqn:condensedMatterProblemSet4Problem2:640}
-m \omega_\nought^2 u_1
=
(k_\nought + \delta)
u_2
+ (k_\nought - \delta)
u_2
%e^{ -i q a }
=
2 k_\nought
u_2,
\end{dmath}
%
or
%
\begin{dmath}\label{eqn:condensedMatterProblemSet4Problem2:660}
u_1 = -u_2.
\end{dmath}
%
The equations for \(u_{n, 1}\) and \(u_{n, 2}\) are
%
\begin{dmath}\label{eqn:condensedMatterProblemSet4Problem2:600}
\begin{aligned}
u_{n, 1} &= -u_2 e^{ \pm 2 i \omega_\nought t } \\
u_{n, 2} &= u_2 e^{ \pm 2 i \omega_\nought t }
\end{aligned}
\end{dmath}
%
where, again, \(u_2 = u_2(0)\) is the initial time displacement of the \(x_{n, 2}\) atom.  We see that when \(q = 0\), the variation \(\delta\) of the spring constants from their average value \(k_\nought\) makes no difference to the motion of the atoms.  This motion is a pairwise oscillation of the form \cref{fig:qmSolidsPs4Qzero:qmSolidsPs4QzeroFig1}.

\mathImageFigure{../figures/phy487-qmsolids/qmSolidsPs4QzeroFig1}{Pairwise oscillation}{fig:qmSolidsPs4Qzero:qmSolidsPs4QzeroFig1}{0.3}{qmSolidsPs4P2cAnimation.nb}

\makeSubAnswer{}{condensedMatter:problemSet4:2d}

The frequencies \(\omega(q)\) are plotted in \cref{fig:qmSolidsPs4P2d:qmSolidsPs4P2dFig1}.  The Brillouin zone bisects this figure vertically along the line \(q a = \pi\) line.

\mathImageFigure{../figures/phy487-qmsolids/qmSolidsPs4P2dFig1}{Frequencies for \(\delta/k_\nought \in \{0.9, 0.1, 0\}\)}{fig:qmSolidsPs4P2d:qmSolidsPs4P2dFig1}{0.3}{qmSolidsPs4P2dGenerated.nb}

\makeSubAnswer{}{condensedMatter:problemSet4:2e}

To examine both the limiting scenario and the case \(\delta/k_\nought = 0.9\), let
%
\begin{dmath}\label{eqn:condensedMatterProblemSet4Problem2:740}
\epsilon = 1 - \lr{\frac{\delta}{k_\nought}}^2,
\end{dmath}
%
so that
%
\begin{dmath}\label{eqn:condensedMatterProblemSet4Problem2:760}
\omega^2
= 2 \omega_\nought^2 \lr{ 1 \pm \sqrt{ \cos^2\frac{qa}{2} + (1 - \epsilon) \sin^2 \frac{qa}{2}} }
= 2 \omega_\nought^2 \lr{ 1 \pm \sqrt{ 1 - \epsilon \sin^2 \frac{qa}{2}} }
\approx 2 \omega_\nought^2 \lr{ 1 \pm \lr{ 1 - \frac{\epsilon}{2} \sin^2 \frac{qa}{2}} }
=
\left\{2 \omega_\nought^2 \lr{ 2 - \frac{\epsilon}{2} \sin^2 \frac{qa}{2} },
\omega_\nought^2 \epsilon \sin^2 \frac{qa}{2}
\right\},
\end{dmath}
%
where \(\omega_\nought^2 = k_\nought/m\), as above.  Taking roots and further approximations we have
%
\begin{dmath}\label{eqn:condensedMatterProblemSet4Problem2:780}
\omega
\approx
\left\{2 \omega_\nought \sqrt{ 1 - \frac{\epsilon}{4} \sin^2 \frac{qa}{2} },
\omega_\nought \sqrt{\epsilon} \sin \frac{qa}{2}
\right\}
\approx
\left\{
2 \omega_\nought \lr{ 1 - \frac{\epsilon}{8} \sin^2 \frac{qa}{2} },
\omega_\nought \sqrt{\epsilon} \sin \frac{qa}{2}
\right\},
\end{dmath}
%
In optical \index{optical dispersion} range, we have \(\omega \approx 2 \omega_\nought\), with only small deviations from that.  As \(\delta/k_\nought\) gets small, this will become increasingly flat.  In fact, we see this close to linear even for the \(0.9\) case where we have
%
\begin{dmath}\label{eqn:condensedMatterProblemSet4Problem2:800}
\omega \approx
2 \omega_\nought \lr{ 1 - 0.024 \sin^2 \frac{qa}{2} }.
\end{dmath}
%
In the acoustic \index{acoustic dispersion} range we have a uniformly damped sinusoid.  For this \(0.9\) case that is
%
\begin{dmath}\label{eqn:condensedMatterProblemSet4Problem2:720}
\omega \approx 0.43 \omega_\nought \sin \frac{qa}{2}.
\end{dmath}
%
\paragraph{Grading remark:} ``Explain physically what is going on.''  Examining the posted solution, what was being looked for was a characterization of the motion itself (i.e. what sort of specific motions of the springs is occuring, are they in phase or out of phase, ...)

\makeSubAnswer{}{condensedMatter:problemSet4:2f}

The \(\delta = 0\) values are somewhat tricky to distinguish in \cref{fig:qmSolidsPs4P2d:qmSolidsPs4P2dFig1} above.  The optical case is re-plotted separately in \cref{fig:qmSolidsPs4OpticalDeltaZero:qmSolidsPs4OpticalDeltaZeroFig1}, and the acoustic in \cref{fig:qmSolidsPs4AccousticDeltaZero:qmSolidsPs4AccousticDeltaZeroFig1}.

\mathImageFigure{../figures/phy487-qmsolids/qmSolidsPs4OpticalDeltaZeroFig1}{Optical dispersion at \(\delta = 0\)}{fig:qmSolidsPs4OpticalDeltaZero:qmSolidsPs4OpticalDeltaZeroFig1}{0.3}{qmSolidsPs4P2d.nb}
\mathImageFigure{../figures/phy487-qmsolids/qmSolidsPs4AccousticDeltaZeroFig1}{Acoustic dispersion at \(\delta = 0\)}{fig:qmSolidsPs4AccousticDeltaZero:qmSolidsPs4AccousticDeltaZeroFig1}{0.3}{qmSolidsPs4P2d.nb}

\paragraph{Grading remark:} ``Only one mode'' was marked under \cref{fig:qmSolidsPs4AccousticDeltaZero:qmSolidsPs4AccousticDeltaZeroFig1}, with an X beside the figure.  Reviewing the posted solution, it appears that the correct response to this part of the problem is to observe that treating this primitive cell as one with two masses is no longer appropriate.  A correct analysis of this limit requires is that of a one atom basis, not two.

To examine the nature of these curves in more detail, observe that for \(\delta = 0\) we have
%
\begin{dmath}\label{eqn:condensedMatterProblemSet4Problem2:820}
\omega^2 = 2 \omega_\nought^2 \lr{ 1 \pm \cos \frac{qa}{2} },
\end{dmath}
%
or
\begin{dmath}\label{eqn:condensedMatterProblemSet4Problem2:840}
\omega = \sqrt{2} \omega_\nought \sqrt{ 1 \pm \cos \frac{qa}{2} }.
\end{dmath}
%
Noting that
\begin{equation}\label{eqn:condensedMatterProblemSet4Problem2:860}
\begin{aligned}
1 - \cos x &= 2 \sin^2 x/2 \\
1 + \cos x &= 2 \cos^2 x/2,
\end{aligned}
\end{equation}
%
provided \(qa < \pi\) we can write the optical and acoustic ranges for \(\omega\) at \(\delta = 0\) respectively as
%
\begin{dmath}\label{eqn:condensedMatterProblemSet4Problem2:880}
\omega \in \left\{
2 \omega_\nought \cos \frac{qa}{4},
2 \omega_\nought \sin \frac{qa}{4}
 \right\}.
\end{dmath}
%
These meet at \(qa = \pi\) before switching directions (the border of the Brillouin zone).
}
