%
% Copyright � 2013 Peeter Joot.  All Rights Reserved.
% Licenced as described in the file LICENSE under the root directory of this GIT repository.
%
%\input{../blogpost.tex}
%\renewcommand{\basename}{condensedMatterLecture19}
%\renewcommand{\dirname}{notes/phy487/}
%\newcommand{\keywords}{Condensed matter physics, PHY487H1F}
%\input{../peeter_prologue_print2.tex}
%
%%\citep{harald2003solid} \S x.y
%%\citep{ibach2009solid} \S x.y
%
%%\usepackage{mhchem}
%\usepackage[version=3]{mhchem}
%\usepackage{units}
%\usepackage{bm}
%\newcommand{\nought}[0]{\circ}
%%\newcommand{\EF}[0]{\epsilon_{\txtF}}
%\newcommand{\EF}[0]{E_{\txtF}}
%\newcommand{\kF}[0]{k_{\txtF}}
%
%\beginArtNoToc
%\generatetitle{PHY487H1F Condensed Matter Physics.  Lecture 19: Electrical transport (cont.).  Taught by Prof.\ Stephen Julian}
%\chapter{Electrical transport (cont.)}
\label{chap:condensedMatterLecture19}
%
%\section{Disclaimer}
%
%Peeter's lecture notes from class.  May not be entirely coherent.
%
\paragraph{Electrical transport (cont.)}
\index{electrical transport}

Last time we noted that we can't use plain Bloch waves to model this, but must introduce a wave packet centered on some \(\Bk\), such as the Gaussian of \cref{fig:qmSolidsL19:qmSolidsL19Fig1}, moving with group velocity \index{group velocity}
%
\mathImageFigure{../figures/phy487-qmsolids/qmSolidsL19Fig1}{Gaussian wave packet.}{fig:qmSolidsL19:qmSolidsL19Fig1}{0.15}{guassianPlotsL18L19.nb}
%
\begin{dmath}\label{eqn:condensedMatterLecture19:20}
\Bv = \inv{\Hbar} \spacegrad_\Bk E(\Bk).
\end{dmath}
%
For nearly free electrons where \(E(\Bk) = \Hbar^2 \Bk^2/2m\) this gives the intuitively appealing
%
\begin{equation}\label{eqn:condensedMatterLecture19:40}
\Bv = \frac{\Hbar \Bk}{m} = \frac{\Bp}{m},
\end{equation}
%
where we have velocity as momentum over mass.

For tight binding
%
\begin{dmath}\label{eqn:condensedMatterLecture19:60}
E(k) = E_i - A - 2 B \cos k a.
\end{dmath}
%
so that
%
\begin{dmath}\label{eqn:condensedMatterLecture19:80}
v = \frac{2 B a}{\Hbar} \sin k a.
\end{dmath}
%
An important quantity is the \textAndIndex{Fermi velocity}
%
\begin{equation}\label{eqn:condensedMatterLecture19:100}
\Bv_{\txtF} = \Bv(\Bk = \Bk_{\txtF}).
\end{equation}
%
Linearizing \(E(\Bk)\) around \(\Bk_{\txtF}\)
%
\begin{dmath}\label{eqn:condensedMatterLecture19:120}
E(\Bk)
= E(\Bk_{\txtF}) + \delta k_\perp \lr{ \frac{dE}{d k_\perp} }.
\equiv E(\Bk_{\txtF}) + \delta k_\perp
\mathLabelBox
{
\frac{\Hbar \Bk_{\txtF}}{m^\conj}
}
{
\(\Bp_{\txtF}\)
}.
\end{dmath}
%
Here \(m^\conj\) is the \textAndIndex{effective mass}.  Keep in mind that the Fermi surface is often not spherical as in \cref{fig:qmSolidsL19:qmSolidsL19Fig2}, and \cref{fig:qmSolidsL19:qmSolidsL19Fig3}.
%
\imageFigure{../figures/phy487-qmsolids/qmSolidsL19Fig2}{Non-spherical Fermi surface.}{fig:qmSolidsL19:qmSolidsL19Fig2}{0.15}
\imageFigure{../figures/phy487-qmsolids/qmSolidsL19Fig3}{Second non-spherical Fermi surface.}{fig:qmSolidsL19:qmSolidsL19Fig3}{0.15}
%
\begin{dmath}\label{eqn:condensedMatterLecture19:140}
\Bv_{\txtF}(\Bk) = \evalbar{\inv{\Hbar} \spacegrad_\Bk E(\Bk)}{\Bk_{\txtF}} = \frac{\Hbar \Bk_{\txtF}}{m^\conj}.
\end{dmath}
%
\paragraph{response to applied electric field \(\bcE\)}
%
Our \(\BF = m\Ba\) equivalent for a wave packet is
%
\begin{equation}\label{eqn:condensedMatterLecture19:160}
\dot{\Bp} = \Hbar \dot{\Bk} = -e \bcE.
\end{equation}
%
Wave vector of wave package advances steadily, as in \cref{fig:qmSolidsL19:qmSolidsL19Fig4}, provided we ignore scattering (for now).
%
\imageFigure{../figures/phy487-qmsolids/qmSolidsL19Fig4}{Wave packet along distribution curve.}{fig:qmSolidsL19:qmSolidsL19Fig4}{0.15}

The time rate of change of the velocity is
%
\begin{dmath}\label{eqn:condensedMatterLecture19:180}
\vdot_i
= \inv{\Hbar} \ddt{} \lr{ \spacegrad_\Bk E(\Bk)}_i
= \inv{\Hbar} \sum_j \frac{\partial^2 E}{\partial k_i \partial k_j} \dot{k}_j
= \sum_j \lr{ \inv{m^\conj}}_{i j} \lr{ -e \calE_j }.
\end{dmath}
%
This time we call
%
\begin{dmath}\label{eqn:condensedMatterLecture19:200}
\lr{ \inv{m^\conj} }_{i j}
= \inv{\Hbar^2}
\frac{\partial^2 E}{\partial k_i \partial k_j}.
\end{dmath}
%
the \textAndIndex{effective mass tensor}, which represents ``resistance'' to applied force.  For example in \cref{fig:qmSolidsL19:qmSolidsL19Fig5}.
%
\imageFigure{../figures/phy487-qmsolids/qmSolidsL19Fig5}{Effective mass tensor.}{fig:qmSolidsL19:qmSolidsL19Fig5}{0.15}

Some interesting conditions for the effective mass tensor are

\begin{itemize}
\item For \(m^\conj > 0\), then \(\dot{\Bv}\) is parallel to the force.
\item For \(\Abs{m^\conj} = \infty\), then \(\dot{\Bv} = 0\).
\item For \(m^\conj < 0\), the wave packet slows down under a force parallel to \(\Bv\).
\end{itemize}

%\cref{fig:qmSolidsL19:qmSolidsL19Fig6}.
\imageFigure{../figures/phy487-qmsolids/qmSolidsL19Fig6}{Illustration for points above.}{fig:qmSolidsL19:qmSolidsL19Fig6}{0.15}
%
\section{Electric current.}
\index{electric current}

\reading \citep{ibach2009solid} \S 9.2, \citep{ashcroft1976solid} \textchapref{12}.

We've been considering a single electron.  What about a metal?
%
\begin{dmath}\label{eqn:condensedMatterLecture19:220}
\Bj
= -\frac{e}{V} \sum_\Bk \Bv(\Bk)
= - e\frac{2}{(2 \pi)^3} \int d\Bk v(\Bk)
= - e\frac{2}{8 \pi^3} \int d\Bk
f(E(\Bk), \bcE)
\spacegrad_\Bk E(\Bk).
\end{dmath}
%
Here \(f(E(\Bk, \bcE)\) is the electron distribution in presence of field \(\bcE\).
%
\paragraph{In equilibrium \(\bcE = 0\)}
%
%\cref{fig:qmSolidsL19:qmSolidsL19Fig7}.
%\imageFigure{../figures/phy487-qmsolids/qmSolidsL19Fig7}{?}{fig:qmSolidsL19:qmSolidsL19Fig7}{0.15}

%\exists
In 1D for every occupied \(+\Bv(\Bk)\) state there exists a \(\Bv(-\Bk)\) state is occupied so that \(\Bv(-\Bk) = -\Bv(\Bk)\), so
%
\begin{dmath}\label{eqn:condensedMatterLecture19:260}
\sum \Bv(\Bk) = 0.
\end{dmath}
%
%\EndArticle
