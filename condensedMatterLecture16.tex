%
% Copyright � 2013 Peeter Joot.  All Rights Reserved.
% Licenced as described in the file LICENSE under the root directory of this GIT repository.
%
%\input{../blogpost.tex}
%\renewcommand{\basename}{condensedMatterLecture16}
%\renewcommand{\dirname}{notes/phy487/}
%\newcommand{\keywords}{Condensed matter physics, PHY487H1F}
%\input{../peeter_prologue_print2.tex}
%
%%\citep{harald2003solid} \S x.y
%%\citep{ibach2009solid} \S x.y
%
%%\usepackage{mhchem}
%\usepackage[version=3]{mhchem}
%\newcommand{\nought}[0]{\circ}
%\newcommand{\EF}[0]{\epsilon_{\txtF}}
%\newcommand{\kF}[0]{k_{\txtF}}
%
%\beginArtNoToc
%\generatetitle{PHY487H1F Condensed Matter Physics.  Lecture 16: Tight binding model.  Taught by Prof.\ Stephen Julian}
%%\chapter{Tight binding model}
\label{chap:condensedMatterLecture16}
%
%\section{Disclaimer}
%
%Peeter's lecture notes from class.  May not be entirely coherent.
%
\section{Tight binding model.}
\index{tight binding model}

\reading \citep{ibach2009solid} \S 7.3, \citep{ashcroft1976solid} \textchapref{10}.

Assume a periodic lattice with large lattice parameter \(a\), so that atomic poential \(V_a(\Br - \Br_n)\) and wave functions \(\phi(\Br- \Br_n)\) don't overlap much between neighbors, as sketched in \cref{fig:qmSolidsL16:qmSolidsL16Fig1}.  The potential \(\phi_i(\Br - \Br_n)\) satisfies
%
\imageFigure{../figures/phy487-qmsolids/qmSolidsL16Fig1}{Tight binding lattice.}{fig:qmSolidsL16:qmSolidsL16Fig1}{0.3}
%
\begin{dmath}\label{eqn:condensedMatterLecture16:20}
\hat{H}_A \phi_i(\Br - \Br_n) = E_i \phi(\Br - \Br_n).
\end{dmath}
%
where
%
\begin{dmath}\label{eqn:condensedMatterLecture16:40}
\hat{H}_A = - \frac{\Hbar}{2m} \spacegrad + V_A(\Br - \Br_n).
\end{dmath}
%
The one electron Hamiltonian
%
\begin{dmath}\label{eqn:condensedMatterLecture16:60}
\hat{H}
= -\frac{\Hbar}{2m} \spacegrad^2 + \sum_{n'} V_A(\Br - \Br_n')
= -\frac{\Hbar}{2m} \spacegrad^2 + V_A(\Br - \Br_n) + \sum_{n' \ne n} V_A(\Br - \Br_n')
= \hat{H}_A(\Br - \Br_n) + v(\Br - \Br_n).
\end{dmath}
%
To get an idea what \(v(\Br - \Br_n)\) might look like, consider \(v(x) = -1/\Abs{x}\), with \(\Br_n = 10 n \xcap\).  The potential looks like \cref{fig:tightBindingInverseRadialPotential:tightBindingInverseRadialPotentialFig1}, with the periodic extension in \cref{fig:tightBindingInverseRadialPotentialPeriodic:tightBindingInverseRadialPotentialPeriodicFig2}, and finally, \(v(\Br - \Br_0)\) in \cref{fig:tightBindingInverseRadialPotentialPeriodicOneMissing:tightBindingInverseRadialPotentialPeriodicOneMissingFig3}.  In the last figure we see the omission of the infinite negative peak at the origin, allowing the trailing contributions from neighbouring sites to add up to a value greater than the peak values at the other sites.
%
\mathImageFigure{../figures/phy487-qmsolids/tightBindingInverseRadialPotentialFig1}{Inverse radial potential.}{fig:tightBindingInverseRadialPotential:tightBindingInverseRadialPotentialFig1}{0.3}{tightBindingPotentials.nb}
\mathImageFigure{../figures/phy487-qmsolids/tightBindingInverseRadialPotentialPeriodicFig2}{Inverse radial periodically extended.}{fig:tightBindingInverseRadialPotentialPeriodic:tightBindingInverseRadialPotentialPeriodicFig2}{0.3}{tightBindingPotentials.nb}
\mathImageFigure{../figures/phy487-qmsolids/tightBindingInverseRadialPotentialPeriodicOneMissingFig3}{Inverse radial, period extension, with one omission.}{fig:tightBindingInverseRadialPotentialPeriodicOneMissing:tightBindingInverseRadialPotentialPeriodicOneMissingFig3}{0.3}{tightBindingPotentials.nb}

Look for a solution that is a Linear Combination of Atomic Orbitals (LCAO).
%
\begin{dmath}\label{eqn:condensedMatterLecture16:80}
\Psi_\Bk(\Br) \sim \Phi_\Bk(\Br)
= \sum_n a_n \phi_i(\Br - \Br_n)
= \sum_n e^{i \Bk \cdot \Br_n} \phi_i(\Br - \Br_n),
\end{dmath}
%
so by Bloch's theorem
%
\begin{dmath}\label{eqn:condensedMatterLecture16:100}
\Phi_\Bk(\Br) = U_\Bk e^{i \Bk \cdot \Br},
\end{dmath}
%
implies
%
\begin{dmath}\label{eqn:condensedMatterLecture16:120}
\Phi_\Bk(\Br +\Br_m) = \Phi_\Bk(\Br) e^{i \Bk \cdot \Br_m},
\end{dmath}
%
\begin{dmath}\label{eqn:condensedMatterLecture16:140}
\Phi_\Bk(\Br +\Br_m)
=
\sum_n
e^{i \Bk \cdot \Br_n}
\phi_i( \Br + \Br_m - \Br_n)
=
e^{i \Bk \cdot \Br_m}
\sum_n
e^{i \Bk \cdot (\Br_n - \Br_m)}
\phi_i( \Br + (\Br_n - \Br_m)).
\end{dmath}
%
\begin{dmath}\label{eqn:condensedMatterLecture16:160}
\Phi_{\Bk + \BG} = \sum_n e^{i \Bk \cdot \Br_n}
\mathLabelBox
{
e^{i \BG \cdot \Br_n}
}
{\(1\)}
\phi_i(\Br - \Br_n)
= \Phi_\Bk(\Br).
\end{dmath}
%
\paragraph{Normalization}
%
Calculate
%
\begin{dmath}\label{eqn:condensedMatterLecture16:340}
E(\Bk) =
\frac{
	\bra{
	\Phi_\Bk}
	\hat{H} \ket{\Phi_\Bk}
}
{
	\braket{\Phi_\Bk}{\Phi_\Bk}
}.
\end{dmath}
%
With
%
\begin{dmath}\label{eqn:condensedMatterLecture16:180}
\braket{\Phi_\Bk}{\Phi_\Bk}
 =
\sum_{n, m} e^{i \Bk \cdot (\Br_n - \Br_m)}
\times
\int d\Br
\phi_i^\conj( \Br - \Br_m)
\phi_i( \Br - \Br_n)
=
\left\{
\begin{array}{l l}
1 & \quad \mbox{if \(m = n\)} \\
0 & \quad \mbox{otherwise}
\end{array}
\right.
\approx N.
\end{dmath}
%
\begin{dmath}\label{eqn:condensedMatterLecture16:200}
E(\Bk) \approx \inv{N}
\sum_{n, m} e^{i \Bk \cdot (\Br_n - \Br_m)}
\int d\Br
\phi_i^\conj( \Br - \Br_m)
(
\mathLabelBox{\hat{H}_A(\Br - \Br_n) + v(\Br - \Br_n)}{exact}
)
\phi_i( \Br - \Br_n)
\approx
\sum_{n, m} e^{i \Bk \cdot (\Br_n - \Br_m)}
\int d\Br
\phi_i^\conj( \Br - \Br_m)
\lr{
E_i
+ v(\Br - \Br_n)
}
\phi_i( \Br - \Br_n).
\end{dmath}
%
In the integral we have from \(\hat{H}_A\), a value of \(E_i\) if \(m = n\)f, and zero otherwise.  For the \(v\) contribution to the integral, we have

\begin{itemize}
\item \(m = n\).  Large \(\phi_i^\conj( \Br - \Br_m) \phi_i( \Br - \Br_n)\).  We've got \(v(\Br - \Br_n)\) small near \(\Br_n\).
\item \(m = n \pm 1\).  Near \(\Br_m\), \(\phi_i\) and \(v\) are both large, and \(\phi_i(\Br - \Br_n)\) is small.
\end{itemize}

In short we have to keep both terms.  Let
%
\begin{dmath}\label{eqn:condensedMatterLecture16:220}
\begin{aligned}
- A &=
\int d\Br
\phi_i^\conj( \Br - \Br_n)
v(\Br - \Br_n)
\phi_i( \Br - \Br_n)  \\
- B &=
\int d\Br
\phi_i^\conj( \Br - \Br_m)
v(\Br - \Br_n)
\phi_i( \Br - \Br_n)
\end{aligned}
\end{dmath}
%
So
%
\begin{dmath}\label{eqn:condensedMatterLecture16:240}
E(\Bk) \approx \inv{N}
\lr{
\sum_n E_i - \sum_n A - \sum_{n, m} e^{i \Bk \cdot (\Br_n - \Br_m)} B
},
\end{dmath}
%
or
%
\begin{dmath}\label{eqn:condensedMatterLecture16:260}
E(\Bk) \approx E_i - A - B \sum_{m = \text{nn of n}} e^{i \Bk \cdot (\Br_n - \Br_m)}.
\end{dmath}
%
In one dimension
\makeexample{1d lattice}{example:condensedMatterLecture16:1}{
For 1D we have
%
\begin{dmath}\label{eqn:condensedMatterLecture16:280}
\Br_n - \Br_m = \pm a,
\end{dmath}
%
which implies the nearest neighbor sum is
%
\begin{dmath}\label{eqn:condensedMatterLecture16:300}
\sum_{nn} \lr{
e^{i k a}
+e^{-i k a}
}
= 2 \cos k a.
\end{dmath}
%
So
%
\begin{dmath}\label{eqn:condensedMatterLecture16:320}
E(k) \approx E_i - A -
\mathLabelBox
{2 B \cos k a
}
{The hopping term}.
\end{dmath}
%
%\cref{fig:qmSolidsL16:qmSolidsL16Fig2}.
\imageFigure{../figures/phy487-qmsolids/qmSolidsL16Fig2}{Tight binding 1D example.}{fig:qmSolidsL16:qmSolidsL16Fig2}{0.15}

%\cref{fig:qmSolidsL16:qmSolidsL16Fig3}.
\imageFigure{../figures/phy487-qmsolids/qmSolidsL16Fig3}{Dependence on \(a\).}{fig:qmSolidsL16:qmSolidsL16Fig3}{0.2}
}
\paragraph{Correspondence with the Nearly free electron model}
\index{nearly free electron model}
%\cref{fig:qmSolidsL16:qmSolidsL16Fig4}.
\imageFigure{../figures/phy487-qmsolids/qmSolidsL16Fig4}{NFE comparison points.}{fig:qmSolidsL16:qmSolidsL16Fig4}{0.1}
%\cref{fig:qmSolidsL16:qmSolidsL16Fig5}.
\imageFigure{../figures/phy487-qmsolids/qmSolidsL16Fig5}{(1) bonding \(s\) orbitals.}{fig:qmSolidsL16:qmSolidsL16Fig5}{0.1}
%\cref{fig:qmSolidsL16:qmSolidsL16Fig6}.
\imageFigure{../figures/phy487-qmsolids/qmSolidsL16Fig6}{(2) antibonding s orbitals.}{fig:qmSolidsL16:qmSolidsL16Fig6}{0.1}
%\cref{fig:qmSolidsL16:qmSolidsL16Fig8}.
\imageFigure{../figures/phy487-qmsolids/qmSolidsL16Fig8}{(3) bonding p orbitals.}{fig:qmSolidsL16:qmSolidsL16Fig8}{0.1}
%\cref{fig:qmSolidsL16:qmSolidsL16Fig7}.
\imageFigure{../figures/phy487-qmsolids/qmSolidsL16Fig7}{(4) antibonding p orbitals.}{fig:qmSolidsL16:qmSolidsL16Fig7}{0.1}
%\EndArticle
