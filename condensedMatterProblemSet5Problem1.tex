%
% Copyright � 2013 Peeter Joot.  All Rights Reserved.
% Licenced as described in the file LICENSE under the root directory of this GIT repository.
%
\makeoproblem{Density of states of a 1-d chain}{condensedMatter:problemSet5:1}{2013 ps5 p1}{
Calculate and sketch a plot of the density of states, \(Z(\omega)\), for
the vibrational modes of a 1-d monatomic chain of length \(L\),
with nearest-neighbour spring constant K, atoms of mass \(M\), and
lattice constant \(a\).
Specifically, start from \(\sum_q\) and by transforming this into
an integral over \(\omega\), obtain \(Z(\omega)\).   Then
draw a sketch of \(Z(\omega)\) vs.\ \(\omega\), labeling intercepts and
asymptotes.

} % makeproblem

\makeanswer{condensedMatter:problemSet5:1}{

For the 2D and 3D (\(d = 2,3\)) density of states we'd consider solutions for \(Z(\omega)\) of
%
\begin{dmath}\label{eqn:condensedMatterProblemSet5Problem1:20}
\int Z(\omega) d\omega = \lr{\frac{L}{2\pi}}^d \int \frac{d \Bf_\omega}{\Abs{\spacegrad_\Bq \omega(q)} } d\omega.
\end{dmath}
%
Should we wish to extend this down to \(d = 1\) we'd have to figure out how to interpret \(d\Bf_\omega\).  In 2D and 3D that was a surface area element, a factor of the differential form \(d^d \Bq = d\Bf_\omega d\omega_\perp\).  In 3D we had \(\int d\Bf_\omega = 4 \pi q^2 = d/dq( 4 \pi q^3/3)\), and for 2D \(\int d\Bf_\omega = 2 \pi q = d/dq( \pi q^2 )\).

Those 3D and 2D ``volumes'' (differentiated to obtain the ``area'' when \(q\) of the surface for \(q\) held constant) can be obtained by these respective integrals
%
\begin{subequations}
\begin{dmath}\label{eqn:condensedMatterProblemSet5Problem1:40}
\int_{x^2 + y^2 + z^2 \le q^2} dx dy dz = \frac{4}{3} \pi q^3
\end{dmath}
\begin{dmath}\label{eqn:condensedMatterProblemSet5Problem1:60}
\int_{x^2 + y^2 \le q^2} dx dy  = \pi q^2.
\end{dmath}
\end{subequations}
%
We can generalize this down to a single dimension by considering
%
\begin{dmath}\label{eqn:condensedMatterProblemSet5Problem1:80}
\int_{x^2 \le q^2} dx = 2 q
\end{dmath}
%
for which we could conceivably consider the area of this 1D surface to be the constant \(2\).  However, does this even make sense, since writing \(dq = df_\omega dq_\perp\) would split our 1-form into the product of two 1-forms, which isn't a sensible operation?  Let's step back and consider the density of states definition from scratch.

\paragraph{Starting from scratch}

We wish to sum over all the integer values \(n\), subject to a period constraint \(2 \pi n = q L\), and employ an integral approximation to this sum.
%
\begin{dmath}\label{eqn:condensedMatterProblemSet5Problem1:100}
\sum_n
\sim \int dn
= \frac{L}{2 \pi} \int dq
= \frac{L}{2 \pi} \int \frac{dq}{d\omega} d\omega
\equiv \int Z(\omega) d\omega,
\end{dmath}
%
From this we find for one dimension
%
\begin{dmath}\label{eqn:condensedMatterProblemSet5Problem1:120}
Z(\omega) = \frac{L}{2 \pi} \frac{dq}{d\omega}.
\end{dmath}
%
Now we are ready to start.  For the 1D chain we had
%
\begin{dmath}\label{eqn:condensedMatterProblemSet5Problem1:140}
\sqrt{\frac{M}{K}} \omega(q) = 2 \sin \lr{ \frac{ q a}{2} },
\end{dmath}
%
so
%
\begin{dmath}\label{eqn:condensedMatterProblemSet5Problem1:160}
\sqrt{\frac{M}{K}} = a \cos \lr{ \frac{ q a}{2} } \frac{dq}{d\omega},
\end{dmath}
%
or
%
\begin{dmath}\label{eqn:condensedMatterProblemSet5Problem1:180}
Z(\omega)
= \frac{L}{2 \pi}
\frac{\sqrt{\frac{M}{K}}}{a \cos \lr{ \frac{ q a}{2} }}
=
\sqrt{\frac{M}{K}} \frac{L}{2 \pi a}
\inv{ \cos \lr{ \frac{ q a}{2} }}
=
\sqrt{\frac{M}{K}} \frac{L}{2 \pi a}
\inv{ \cos \sin^{-1} \lr{
\inv{2} \sqrt{\frac{M}{K}} \omega } }
=
\inv{2} \sqrt{\frac{M}{K}} \frac{L}{\pi a}
\inv{ \sqrt{1 - \inv{4} \frac{M}{K} \omega^2 } }.
\end{dmath}
%
With \(L = N a\), this is

\boxedEquation{eqn:condensedMatterProblemSet5Problem1:200}{
Z(\omega)
=
\inv{2} \sqrt{\frac{M}{K}} \frac{N}{\pi}
\inv{ \sqrt{1 - \inv{4} \frac{M}{K} \omega^2 } }.
}

This has a minimum at \(\omega = 0\), and in that neighborhood is approximately parabolic function
%
\begin{dmath}\label{eqn:condensedMatterProblemSet5Problem1:220}
Z(\omega \approx 0)
=
\inv{2} \sqrt{\frac{M}{K}} \frac{N}{\pi}
\lr{ 1 - \lr{ - \inv{2}} \inv{4} \frac{M}{K} \omega^2 }
=
\inv{2} \sqrt{\frac{M}{K}} \frac{N}{\pi}
\lr{ 1 + \inv{8} \frac{M}{K} \omega^2 }.
\end{dmath}
%
As \(\omega \rightarrow \pm \sqrt{4 K/M}\), the density of states approaches vertical asymptotes \(Z(\omega) \rightarrow \infty\).  Observe that these extremes are the edges of the Brillouin zone where \(q a/2 = \pm \pi/2\).  For \(Z(\omega)\) to be useful for probability calculations, we expect that the integral over this first Brillouin zone will be finite, despite these infinite asymptotes.  Let's verify this
%
\begin{dmath}\label{eqn:condensedMatterProblemSet5Problem1:240}
\int_{-\sqrt{4K/M}}^{\sqrt{4K/M}} Z(\omega) d\omega
=
\inv{2} \sqrt{\frac{M}{K}} \frac{N}{\pi}
\int_{-1}^1
\sqrt{ \frac{4 K}{M} }
dx
\inv{ \sqrt{1 - x^2 } }
= \frac{N}{\pi} \pi
= N.
\end{dmath}
%
Good, the area under the curve is finite as expected.  This curve is sketched in \cref{fig:1dSpringLatticeDensityOfStates:1dSpringLatticeDensityOfStatesFig1}.

\imageFigure{../figures/phy487-qmsolids/1dSpringLatticeDensityOfStatesFig1}{1D density of states for Harmonic chain}{fig:1dSpringLatticeDensityOfStates:1dSpringLatticeDensityOfStatesFig1}{0.4}

}
