%
% Copyright � 2013 Peeter Joot.  All Rights Reserved.
% Licenced as described in the file LICENSE under the root directory of this GIT repository.
%
\makeoproblem{nearly free electron model}{condensedMatter:problemSet7:2}{2013 ps7 p2}{
\Cref{fig:condensedMatterProblemSet7:condensedMatterProblemSet7Fig1} shows
the free electron dispersion relation for a one-dimensional metal, in the reduced zone scheme.
\imageFigure{../figures/phy487-qmsolids/condensedMatterProblemSet7Fig1}{dispersion relation for a one-dimensional metal.}{fig:condensedMatterProblemSet7:condensedMatterProblemSet7Fig1}{0.3}
\makesubproblem{}{condensedMatter:problemSet7:2a}
For each of the branches (1), (2), (3) and (4),
state which of the coefficients \(C_{k -G}\) are non-zero, and give the corresponding \(u_k(x)\) in the electron wave-function
%
\begin{dmath}\label{eqn:condensedMatterProblemSet7Problem2:20}
\psi_k(x) = \sum_G C_{k-G} e^{-i G x} e^{i k x} = u_k(x) e^{i k x}.
\end{dmath}
%
\makesubproblem{}{condensedMatter:problemSet7:2b}
For the circled regions there are two nearly-degenerate energy solutions. If a nearly
free electron potential
%
\begin{dmath}\label{eqn:condensedMatterProblemSet7Problem2:40}
V(x) = V_1 cos(2\pi x/a) + V_2 cos(4 \pi x/a).
\end{dmath}
is introduced, what is the magnitude of the energy gap that opens up at the band
crossings in the circled regions `a', `b' and `c'?
\makesubproblem{}{condensedMatter:problemSet7:2c}
What are the gaps at `a', `b' and `c' if
instead the lattice potential is described by
a periodic array of delta functions:
%
\begin{dmath}\label{eqn:condensedMatterProblemSet7Problem2:60}
V(x) =
\sum_{-\infty}^\infty
V_\nought \delta(x - n a).
\end{dmath}
%
\makesubproblem{}{condensedMatter:problemSet7:2d}
Write the form of the wave-function for the two solutions at the level crossing at `b'
on the diagram (i.e. where \(k = 0\)). Sketch in real space the charge density associated
with these solutions relative to the atom positions on the one-dimensional lattice. Also,
write these wave functions in the Bloch form, \(\psi_k(x) = e^{i k x} u_k(x)\), and thus identify \(u_k(x)\)
for the two solutions at `b'.
\makesubproblem{}{condensedMatter:problemSet7:2e}
Near point `a', investigate how the wave-function of the lower-energy branch (branch
\(G = 0\)) evolves as you move away from the Brillouin zone boundary.  To do this, calculate
\(\Abs{\psi_k(x)}^2\), and from this sketch the charge density in real space, explaining how it changes
as \(k\) moves away from \(\pi/a\). If it helps, you may start far enough away from the Brillouin
zone boundary that \(E_{k - 2\pi/a}^\nought -E_{k}^\nought \gg \Abs{V_{2 \pi/a}}\).
} % makeproblem
\makeanswer{condensedMatter:problemSet7:2}{
\makeSubAnswer{}{condensedMatter:problemSet7:2a}
Considering each branch in turn:
\begin{enumerate}
\item On this branch, the wave function is
\begin{dmath}\label{eqn:condensedMatterProblemSet7Problem2:80}
\psi_k(x) =
\inv{\sqrt{a}}
e^{i k x},
\end{dmath}
%
so
%
\begin{equation}\label{eqn:condensedMatterProblemSet7Problem2:160}
\begin{aligned}
u_k(x) &=
\inv{\sqrt{a}} \\
C_k &=
\inv{\sqrt{a}} \\
C_{k - 2 \pi m/a} &= 0, \qquad m \in \{\pm 1, \pm 2, \cdots \}.
\end{aligned}
\end{equation}
%
\item On this branch, the wave function is
\begin{dmath}\label{eqn:condensedMatterProblemSet7Problem2:100}
\psi_k(x)
= \inv{\sqrt{a}}
e^{i \lr{k - 2 \pi/a} x}
=
\lr{
\inv{\sqrt{a}} e^{-i \frac{2 \pi}{a} x}
}
e^{i k x},
\end{dmath}
%
so
%
\begin{equation}\label{eqn:condensedMatterProblemSet7Problem2:180}
\begin{aligned}
u_k(x) &=
\inv{\sqrt{a}} e^{-i \frac{2 \pi}{a} x} \\
C_{k - 2\pi/a} &=
\inv{\sqrt{a}} \\
C_{k - 2 \pi m/a} &= 0, \qquad m \in \{0, -1, \pm 2, \pm 3, \cdots \}.
\end{aligned}
\end{equation}
%
\item On this branch, the wave function is
\begin{dmath}\label{eqn:condensedMatterProblemSet7Problem2:400}
\psi_k(x)
= \inv{\sqrt{a}}
e^{i \lr{k + 2 \pi/a} x}
=
\lr{
\inv{\sqrt{a}} e^{-i \frac{-2 \pi}{a} x}
}
e^{i k x},
\end{dmath}
%
so
%
\begin{equation}\label{eqn:condensedMatterProblemSet7Problem2:420}
\begin{aligned}
u_k(x) &=
\inv{\sqrt{a}} e^{i \frac{2 \pi}{a} x} \\
C_{k + 2\pi/a} &=
\inv{\sqrt{a}} \\
C_{k - 2 \pi m/a} &= 0, \qquad m \in \{0, 1, \pm 2, \pm 3, \cdots \}.
\end{aligned}
\end{equation}
%
\item On this branch, the wave function is
\begin{dmath}\label{eqn:condensedMatterProblemSet7Problem2:200}
\psi_k(x)
= \inv{\sqrt{a}}
e^{i \lr{k - 4 \pi/a} x}
=
\lr{
\inv{\sqrt{a}} e^{-i \frac{4 \pi}{a} x}
}
e^{i k x},
\end{dmath}
%
so
%
\begin{equation}\label{eqn:condensedMatterProblemSet7Problem2:220}
\begin{aligned}
u_k(x) &=
\inv{\sqrt{a}} e^{-i \frac{4 \pi}{a} x} \\
C_{k - 4 \pi/a} &=
\inv{\sqrt{a}} \\
C_{k - 2 \pi m/a} &= 0, \qquad m \in \{0, \pm 1, -2, \pm 3, \pm 4, \cdots \}.
\end{aligned}
\end{equation}
%
\end{enumerate}
\makeSubAnswer{}{condensedMatter:problemSet7:2b}
With \(G = 2 \pi/a\), we can write the potential of \eqnref{eqn:condensedMatterProblemSet7Problem2:40} as
%
\begin{dmath}\label{eqn:condensedMatterProblemSet7Problem2:240}
V(x) =
\frac{V_1}{2} \lr{
e^{ i G x }
+ e^{ -i G x }
}
+
\frac{V_2}{2} \lr{
e^{ 2 i G x }
+ e^{ -2 i G x }
}.
\end{dmath}
%
That is
\begin{equation}\label{eqn:condensedMatterProblemSet7Problem2:260}
\begin{aligned}
V_{1 G} &= V_{-1 G} = \frac{V_1}{2} \\
V_{2 G} &= V_{-2 G} = \frac{V_2}{2}.
\end{aligned}
\end{equation}
%
The coefficients \(C_k\) would follow from a solution of
%
\begin{dmath}\label{eqn:condensedMatterProblemSet7Problem2:280}
0 =
\lr{ \frac{\Hbar^2}{2m} \lr{k - n G}^2 - E } C_{k - n G}
+ V_G \lr{ C_{ k - (1 + n) G } + C_{ k + (1 - n) G } }
+ V_{2G} \lr{ C_{ k - (2 + n) G } + C_{ k + (2 - n)G } }.
\end{dmath}
%
Written out in full this includes
%
\begin{equation}\label{eqn:condensedMatterProblemSet7Problem2:300}
\begin{aligned}
0 &=
\lr{ \frac{\Hbar^2}{2m} (k + 2 G)^2 - E } C_{k + 2 G}
+ V_G \lr{ C_{ k + G } + C_{ k + 3 G } }
+ V_{2G} \lr{ C_{ k } + C_{ k + 4 G } } \\
0 &=
\lr{ \frac{\Hbar^2}{2m} (k + G)^2 - E } C_{k + G}
+ V_G \lr{ C_{ k } + C_{ k + 2 G } }
+ V_{2G} \lr{ C_{ k - G } + C_{ k + 3 G } } \\
0 &=
\lr{ \frac{\Hbar^2}{2m} k^2 - E } C_{k}
+ V_G \lr{ C_{ k - G } + C_{ k + G } }
+ V_{2G} \lr{ C_{ k - 2 G } + C_{ k + 2G } } \\
0 &=
\lr{ \frac{\Hbar^2}{2m} (k - G)^2 - E } C_{k - G}
+ V_G \lr{ C_{ k - 2 G } + C_{ k } }
+ V_{2G} \lr{ C_{ k - 3 G } + C_{ k + G } } \\
0 &=
\lr{ \frac{\Hbar^2}{2m} (k - 2 G)^2 - E } C_{k - 2 G}
+ V_G \lr{ C_{ k - 3 G } + C_{ k - G } }
+ V_{2G} \lr{ C_{ k - 4 G } + C_{ k } }.
\end{aligned}
\end{equation}
%
Dropping \(C_{k - 4 G}, C_{k - 3 G}, C_{k + 3 G}, C_{k + 4G}\), this is
%
%\begin{dmath}\label{eqn:condensedMatterProblemSet7Problem2:320}
%0 =
%\begin{vmatrix}
%\vdots \\
%V_{2G} & V_{G}  & {E_{k -2 G}^\nought} - E & V_G         & V_{2G}   & 0                & 0      & 0           & 0      \\
%0      & V_{2G} & V_{G}        & {E_{k - G}^\nought} - E & V_G      & V_{2G}           & 0      & 0           & 0      \\
%0      & 0      & V_{2G}       & V_{G}       & {E_{k }^\nought} - E & V_G              & V_{2G} & 0           & 0      \\
%0      & 0      & 0            & V_{2G}      & V_{G}       & {E_{k + G }^\nought} - E & V_G    & V_{2G}      & 0      \\
%0      & 0      & 0            & 0           & V_{2G}      & V_{G}       & {E_{k + 2 G }^\nought} - E & V_G    & V_{2G} \\
%\vdots \\
%\end{vmatrix}.
%\end{dmath}
%
\begin{dmath}\label{eqn:condensedMatterProblemSet7Problem2:440}
0 =
\begin{bmatrix}
                  {E_{k -2 G}^\nought} - E & V_G         & V_{2G}   & 0                & 0            \\
                  V_{G}        & {E_{k - G}^\nought} - E & V_G      & V_{2G}           & 0            \\
                  V_{2G}       & V_{G}       & {E_{k }^\nought} - E & V_G              & V_{2G}       \\
                  0            & V_{2G}      & V_{G}       & {E_{k + G }^\nought} - E & V_G           \\
                  0            & 0           & V_{2G}      & V_{G}       & {E_{k + 2 G }^\nought} - E \\
\end{bmatrix}
\begin{bmatrix}
C_{k - 2 G} \\
C_{k - G} \\
C_{k} \\
C_{k + G} \\
C_{k + 2 G}
\end{bmatrix},
\end{dmath}
%
or
\begin{dmath}\label{eqn:condensedMatterProblemSet7Problem2:320}
0 =
\begin{vmatrix}
                  {E_{k -2 G}^\nought} - E & V_G         & V_{2G}   & 0                & 0            \\
                  V_{G}        & {E_{k - G}^\nought} - E & V_G      & V_{2G}           & 0            \\
                  V_{2G}       & V_{G}       & {E_{k }^\nought} - E & V_G              & V_{2G}       \\
                  0            & V_{2G}      & V_{G}       & {E_{k + G }^\nought} - E & V_G           \\
                  0            & 0           & V_{2G}      & V_{G}       & {E_{k + 2 G }^\nought} - E \\
\end{vmatrix}
\end{dmath}
%
\paragraph{At point (a)} we
can get a rough idea of the separation by further dropping \(C_{k - G}, C_{k + G}, C_{k + 2G}\) terms, so that the system to solve is
%
\begin{dmath}\label{eqn:condensedMatterProblemSet7Problem2:340a}
0 =
\begin{bmatrix}
                                 {E_{k -G}^\nought} - E & V_G      \\
                                 V_{G}       & {E_{k }^\nought} - E
\end{bmatrix}
\begin{bmatrix}
C_{k - G} \\
C_{k }
\end{bmatrix}.
\end{dmath}
%
At the boundary where \({E_{k }^\nought} = {E_{k -G}^\nought}\), this is
%
\begin{dmath}\label{eqn:condensedMatterProblemSet7Problem2:340}
0 =
\begin{vmatrix}
                                 {E_{k}^\nought} - E & V_G      \\
                                 V_{G}       & {E_{k }^\nought} - E
\end{vmatrix},
\end{dmath}
%
or
%
\begin{dmath}\label{eqn:condensedMatterProblemSet7Problem2:360}
E_{\pm} = {E_{k }^\nought} \pm V_G = E \pm \frac{V_1}{2}.
\end{dmath}
%
The (approximate) separation between the energy curves at that point is
\boxedEquation{eqn:condensedMatterProblemSet7Problem2:380}{
\Delta E = E_{+} - E_{-}
%= \frac{V_1}{2} - - \frac{V_1}{2}
= V_1.
}
\paragraph{At point (b)} dropping all but the \(C_{k - G}, C_{k + G}\) terms gives us
%
\begin{dmath}\label{eqn:condensedMatterProblemSet7Problem2:460}
0 =
\begin{bmatrix}
                                 {E_{k -G}^\nought} - E & V_{2G}      \\
                                 V_{2G}       & {E_{k +G}^\nought} - E
\end{bmatrix}
\begin{bmatrix}
C_{k - G} \\
C_{k + G}
\end{bmatrix}
.
\end{dmath}
%
Taking the determinant and noting that at the intersection \({E_{k - G}^\nought} = {E_{k +G}^\nought}\), this has solution
%
\begin{dmath}\label{eqn:condensedMatterProblemSet7Problem2:480}
E_{\pm} = {E_{k -G}^\nought} \pm V_{2G} = E \pm \frac{V_2}{2}.
\end{dmath}
%
This time the approximate separation between the energy curves at that point is
\boxedEquation{eqn:condensedMatterProblemSet7Problem2:1280}{
\Delta E = E_{+} - E_{-} = V_2.
}
\paragraph{Finally, at point (c)} we have \({E_{k - 2 G}^\nought} = {E_{k + G}^\nought}\).  Our system \eqnref{eqn:condensedMatterProblemSet7Problem2:440} can be approximated by
%
\begin{dmath}\label{eqn:condensedMatterProblemSet7Problem2:1200}
0 =
\begin{bmatrix}
                  {E_{k -2 G}^\nought} - E & 0 \\
                  0            & {E_{k + G }^\nought} - E \\
\end{bmatrix}
\begin{bmatrix}
C_{k - 2 G} \\
C_{k + G}
\end{bmatrix},
\end{dmath}
%
with approximate solution
%
\begin{dmath}\label{eqn:condensedMatterProblemSet7Problem2:1220}
E_{\pm} = {E_{k + G}^\nought}.
\end{dmath}
%
We find that the approximate separation at this point is zero.  Should we require a better estimate, we must retain of more of the coefficients \(C_{k - h G}\).
\makeSubAnswer{}{condensedMatter:problemSet7:2c}
Let's start by assuming the periodic delta function potential has a Fourier representation, and computing the associated Fourier components
%
\begin{equation}\label{eqn:condensedMatterProblemSet7Problem2:500}
V(x)
= V_\nought \sum_n \delta( x - n a )
= \sum_h C_{G_h} e^{-i G_h x },
\end{equation}
%
where
\begin{equation}\label{eqn:condensedMatterProblemSet7Problem2:520}
G_h = G h = \frac{ 2 \pi }{a} h,
\end{equation}
%
and \(h\) is an integer.  The Fourier coefficient follows directly from the Fourier integral
%
\begin{equation}\label{eqn:condensedMatterProblemSet7Problem2:540}
\begin{aligned}
\int_{-1/2}^{1/2} du V( u a ) e^{i h' G a u}
&= \sum_h C_{G_h}
\int_{-1/2}^{1/2} du e^{i h' G a u}
e^{-i G h a u } \\
&= \sum_h C_{G_h}
\int_{-1/2}^{1/2} du e^{2 \pi i (h' - h) u} \\
&= \sum_h C_{G_h}
\delta_{h, h'} \\
&= C_{G_{h'}}.
\end{aligned}
\end{equation}
%
Integrating the LHS gives us
%
\begin{dmath}\label{eqn:condensedMatterProblemSet7Problem2:560}
C_{G_{h}}
=
\int_{-1/2}^{1/2} du V( u a ) e^{i h G a u}
=
V_\nought
\int_{-1/2}^{1/2} du
\sum_n
\delta( (u - n) a ) e^{2 \pi i h u}.
\end{dmath}
%
Only one of the delta functions in the \(\sum_n\) falls in \([-1/2, 1/2]\), so this integral is just \(V_\nought/a\), leaving
%
\begin{dmath}\label{eqn:condensedMatterProblemSet7Problem2:580}
V(x) = \frac{V_\nought}{a} \sum_h e^{i G_h x}.
\end{dmath}
%
Schr\"{o}dinger's equation for this potential is
%
\begin{dmath}\label{eqn:condensedMatterProblemSet7Problem2:600}
0 = \sum_{k'} \lr{ \hkSq{\lr{k'}} - E } C_{k'} e^{i k' x} + \frac{V_\nought}{a} \sum_G e^{i G x} C_{k'} e^{i k' x},
\end{dmath}
%
Operating with \(\int dx e^{-i k x}\) gives
% -k + k' + G = 0
% k' = k - G
%
\begin{dmath}\label{eqn:condensedMatterProblemSet7Problem2:620}
0 = \lr{ \hkSq{k} - E } C_{k} + \frac{V_\nought}{a} \sum_G C_{k - G}.
\end{dmath}
%
Noting that
%
\begin{dmath}\label{eqn:condensedMatterProblemSet7Problem2:640}
0 = \lr{ \frac{V_\nought}{a} + \hkSq{k} - E } C_{k} + \frac{V_\nought}{a} \sum_{G \ne k} C_{k - G},
\end{dmath}
%
we could put this in matrix form, but that's not too helpful since the matrix is infinite dimensional.  It's possible to find an implicit relation for \(E(k)\) by summing \(C_k\), since
%
\begin{dmath}\label{eqn:condensedMatterProblemSet7Problem2:660}
C_k
=
\frac{V_\nought}{a} \sum_G C_{k - G} \inv{ E - \hkSq{k} },
\end{dmath}
%
and
%
\begin{dmath}\label{eqn:condensedMatterProblemSet7Problem2:680}
\cancel{\sum_k C_k }
=
\frac{V_\nought}{a} \cancel{\sum_G C_{k - G}} \sum_k \inv{ E - \hkSq{k} },
\end{dmath}
%
or
%
\begin{dmath}\label{eqn:condensedMatterProblemSet7Problem2:700}
\frac{a}{V_\nought} = \sum_k \inv{ E - \hkSq{k} }.
\end{dmath}
%
While I'd guess that this can be summed using the Euler-MacLaren theorem, Mathematica says this is
%
\begin{dmath}\label{eqn:condensedMatterProblemSet7Problem2:720}
\sqrt{E} =
\frac{V_\nought}{a}
\frac{2 \pi  m}{ \Hbar^2 }
\cot \lr{2 \pi m \frac{\sqrt{E}}{\Hbar^2} }.
\end{dmath}
%
This implicit function has two discrete solutions, as illustrated by representative plots in \cref{fig:ps7p2c:ps7p2cFig1}.  This could clearly be solved numerically.
\mathImageFigure{../figures/phy487-qmsolids/ps7p2cFig1}{Implicit function curves for delta-function-potential energy.}{fig:ps7p2c:ps7p2cFig1}{0.3}{problemSet7problem2c.nb}
None of this helps with an estimate of the gap at the crossing points.  Considering point (a) as representative, let's make a crude approximation of \eqnref{eqn:condensedMatterProblemSet7Problem2:640} as
%
\begin{dmath}\label{eqn:condensedMatterProblemSet7Problem2:800}
0 \approx
\begin{bmatrix}
\frac{V_\nought}{a} + \hkSq{\lr{k - G}} - E & \frac{V_\nought}{a} \\
\frac{V_\nought}{a} & \frac{V_\nought}{a} + \hkSq{k} - E
\end{bmatrix}
\begin{bmatrix}
C_{k - G} \\
C_{k}
\end{bmatrix}.
\end{dmath}
%
At the crossing (a) where \(E_{k}^\nought = E_{k - G}^\nought\) we have
%
\begin{dmath}\label{eqn:condensedMatterProblemSet7Problem2:740}
0 \approx
\begin{vmatrix}
\frac{V_\nought}{a} + {E_{k}^\nought} - E & \frac{V_\nought}{a} \\
\frac{V_\nought}{a} & \frac{V_\nought}{a} + {E_{k}^\nought} - E
\end{vmatrix},
\end{dmath}
%
or
\begin{dmath}\label{eqn:condensedMatterProblemSet7Problem2:760}
E_\pm \approx
{E_{k}^\nought} - \frac{V_\nought}{a}
\pm \Abs{ \frac{V_\nought}{a} }.
\end{dmath}
%
The gap distance at this point (or the others) is thus approximately
\boxedEquation{eqn:condensedMatterProblemSet7Problem2:780}{
\Delta E \approx 2 \frac{V_\nought}{a}.
}
\makeSubAnswer{}{condensedMatter:problemSet7:2d}
From \eqnref{eqn:condensedMatterProblemSet7Problem2:460} we see that the coefficients are related by
%
\begin{dmath}\label{eqn:condensedMatterProblemSet7Problem2:820}
C_{k - G}
= - C_{k + G} \frac{V_{2G}}{ E_{k -G}^\nought - E },
\end{dmath}
%
so our wave functions \(\psi = \sum_k C_k e^{i k x}\) are
%
\begin{dmath}\label{eqn:condensedMatterProblemSet7Problem2:840}
\psi_{\pm} \propto
\lr{ E_{k -G}^\nought - E_{\pm} } e^{i (k + G) x} - V_{2G} e^{i (k - G) x}.
\end{dmath}
%
However, \(V_{2G} = V_2/2\), and \(E_{k - G}^\nought - E_{\pm} = \pm V_2/2\), so we have
%
\begin{dmath}\label{eqn:condensedMatterProblemSet7Problem2:860}
\psi_{\pm} \propto e^{i (k + G) x} \mp e^{i (k - G) x}.
\end{dmath}
%
Normalizing and putting in Bloch form, these are
%
\begin{equation}\label{eqn:condensedMatterProblemSet7Problem2:880}
\begin{aligned}
\psi_{+} &= \sqrt{\frac{2}{a}} e^{i k x} \sin( G x ) \\
\psi_{+} &= \sqrt{\frac{2}{a}} e^{i k x} \cos( G x ).
\end{aligned}
\end{equation}
%
or
%
\begin{equation}\label{eqn:condensedMatterProblemSet7Problem2:900}
\begin{aligned}
u_{k+}(x) &= \sqrt{\frac{2}{a}} \sin( G x ) \\
u_{k-}(x) &= \sqrt{\frac{2}{a}} \cos( G x ).
\end{aligned}
\end{equation}
%
These have respective charge densities
%
\begin{equation}\label{eqn:condensedMatterProblemSet7Problem2:920}
\begin{aligned}
\rho_{+} &= \frac{2 e}{a} \sin^2( G x) \\
\rho_{-} &= \frac{2 e}{a} \cos^2( G x).
\end{aligned}
\end{equation}
%
These are sketched in \cref{fig:ps7p2dDensityPlots:ps7p2dDensityPlotsFig2}.
\imageFigure{../figures/phy487-qmsolids/ps7p2dDensityPlotsFig2}{Density plots.}{fig:ps7p2dDensityPlots:ps7p2dDensityPlotsFig2}{0.3}
\makeSubAnswer{}{condensedMatter:problemSet7:2e}
We start with \eqnref{eqn:condensedMatterProblemSet7Problem2:340a}, for which we find
%
\begin{dmath}\label{eqn:condensedMatterProblemSet7Problem2:940}
E_{\pm} =
\inv{2}
\lr{
{E_{k -G}^\nought} + {E_{k }^\nought} \pm \sqrt{ \lr{ {E_{k }^\nought} - {E_{k -G}^\nought} }^2 + V_1^2 }
},
\end{dmath}
%
so that the wave function coefficients are given by
%
\begin{dmath}\label{eqn:condensedMatterProblemSet7Problem2:960}
C_{k - G}
\inv{2}
\lr{
{E_{k -G}^\nought} - {E_{k }^\nought} \mp \sqrt{ \lr{ {E_{k }^\nought} - {E_{k -G}^\nought} }^2 + V_1^2 }
}
=
- C_{k } \frac{V_1}{2}.
\end{dmath}
%
This gives
%
\begin{dmath}\label{eqn:condensedMatterProblemSet7Problem2:980}
\psi_{k,\pm}(x)
\propto -V_1 e^{i (k - G) x} +
\lr{
{E_{k -G}^\nought} - {E_{k }^\nought} \mp \sqrt{ \lr{ {E_{k }^\nought} - {E_{k -G}^\nought} }^2 + V_1^2 }
} e^{i k x}.
\end{dmath}
%
The kinetic energy difference is
%
\begin{dmath}\label{eqn:condensedMatterProblemSet7Problem2:1000}
E_{k -G}^\nought - E_{k }^\nought
= \hkSq{\lr{k- G}}
- \hkSq{k}
=
\frac{\Hbar^2 \lr{ G^2 - 2 k G} }{2m},
\end{dmath}
%
so we have
%
\begin{equation}\label{eqn:condensedMatterProblemSet7Problem2:1020}
\begin{aligned}
\psi_{k,\pm}(x)
&\propto -V_1 e^{i (k - G) x} +
\Bigl(
\frac{\Hbar^2 \lr{ G^2 - 2 k G} }{2m} \\
&\mp \sqrt{
\lr{ \frac{\Hbar^2 \lr{ G^2 - 2 k G}}{2m} }^2
+ V_1^2 }
\Bigr)
e^{i k x}
\end{aligned}
\end{equation}
%
To make this less cumbersome, let's write
%
\begin{dmath}\label{eqn:condensedMatterProblemSet7Problem2:1040}
\epsilon_\pm(k) =
\lr{
\frac{\Hbar^2 G \lr{ G - 2 k} }{2m}
\mp \sqrt{
\lr{ \frac{\Hbar^2 G \lr{ G - 2 k} }{2m} }^2
+ V_1^2 }
}.
\end{dmath}
%
Writing out \(G = 2 \pi/a\) explicitly, the (still unnormalized) wave functions are
%
\begin{dmath}\label{eqn:condensedMatterProblemSet7Problem2:1060}
\psi_{k,\pm}(x)
=
\lr{
-V_1
e^{-i \pi x/a} +
\epsilon_\pm(k)
e^{i \pi x/a}
}
e^{-i \pi x/a}
e^{i k x},
\end{dmath}
%
and the densities are proportional to
%
\begin{dmath}\label{eqn:condensedMatterProblemSet7Problem2:1080}
\rho_{k,\pm}(x)
=
\Abs{
-V_1
e^{-i \pi x/a} +
\epsilon_\pm(k)
e^{i \pi x/a}
}^2.
\end{dmath}
%
We are interested in the lower energy branch \(\rho_{k, -}\) where
%
\begin{dmath}\label{eqn:condensedMatterProblemSet7Problem2:1100}
\epsilon_{-}(k)
=
\lr{
\frac{\Hbar^2 G \lr{ G - 2 k} }{2m}
+ \sqrt{
\lr{ \frac{\Hbar^2 G \lr{ G - 2 k} }{2m} }^2
+ V_1^2 }
}
=
\lr{
\frac{h^2 \lr{ 1 - \frac{k a}{\pi}} }{2m a^2}
+ \sqrt{
\lr{ \frac{h^2 \lr{ 1 - \frac{k a}{\pi}} }{2m a^2} }^2
+ V_1^2 }
}.
\end{dmath}
%
Observe that the kinetic energy difference terms are zero at the Brillouin boundary (\(k = \pi/a\)).  At a distance approaching that boundary, say
%
\begin{dmath}\label{eqn:condensedMatterProblemSet7Problem2:1120}
k = \frac{\pi}{a}( 1 - \alpha ),
\end{dmath}
%
we have
\begin{dmath}\label{eqn:condensedMatterProblemSet7Problem2:1240}
\epsilon_{-}(k)
=
\frac{h^2 \alpha}{2m a^2}
+ \sqrt{
\lr{ \frac{h^2 \alpha}{2m a^2} }^2
+ V_1^2 }
\approx
\frac{h^2 \alpha}{2m a^2}
+ V_1
+
\inv{2} \lr{ \frac{h^2 \alpha}{2 m a^2} }^2
=
V_1 + O(\alpha).
\end{dmath}
%
Thus near the boundary we have a (without normalization) nearly sine density
%
\begin{dmath}\label{eqn:condensedMatterProblemSet7Problem2:1140}
\rho_{k,-}(x)
=
\Abs{
2 i V_1 \sin(\pi x/a)
+
O(\alpha)
e^{i \pi x/a}
}^2
\approx
V_1^2 \sin^2 (\pi x/ a).
\end{dmath}
%
This approaches zero in real space near the atomic lattice positions.

On the other extreme, far enough from the boundary that
%
\begin{dmath}\label{eqn:condensedMatterProblemSet7Problem2:1160}
\frac{h^2 \lr{ 1 - \frac{k a}{\pi}} }{2m a^2} \gg V_1^2,
\end{dmath}
%
we have
%
\begin{dmath}\label{eqn:condensedMatterProblemSet7Problem2:1180}
\epsilon_{-}(k)
\approx
\frac{h^2 \lr{1 - \frac{k a}{2}}}{m a^2},
\end{dmath}
%
and our (unnormalized) density is nearly constant
%
\begin{dmath}\label{eqn:condensedMatterProblemSet7Problem2:1260}
\rho_{k,\pm}(x)
\approx
\Abs{
-V_1
e^{-i \pi x/a} +
\frac{h^2 \lr{1 - \frac{k a}{2}}}{m a^2}
e^{i \pi x/a}
}^2
\approx
\frac{h^2 \lr{1 - \frac{k a}{2}}}{m a^2}.
\end{dmath}
%
Some of these curves are plotted in \cref{fig:ps7p2e:ps7p2eFig1} for various relative values of \({E_{k -2\pi/2}^\nought} - {E_{k }^\nought}, V_1\), where the most peaked is near the \(k = \pi/a\) boundary, and the flattest, near the origin.
%
\mathImageFigure{../figures/phy487-qmsolids/ps7p2eFig1}{Real space variation with \(k\) in the Brillouin zone.}{fig:ps7p2e:ps7p2eFig1}{0.2}{ps7p2ePlot.nb}
}
