%
% Copyright � 2013 Peeter Joot.  All Rights Reserved.
% Licenced as described in the file LICENSE under the root directory of this GIT repository.
%
\makeoproblem{The Madelung constant (Ibach and Luth, Q1, Chapter 1)}{condensedMatter:problemSet1:3}{2013 ps1 p3}{
\makesubproblem{Calculate the Madelung constant \(A\) for a linear ionic chain.}{condensedMatter:problemSet1:3a}
\makesubproblem{Approximate numerical calculations for the \ce{NaCl} lattice.}{condensedMatter:problemSet1:3b}
Make approximate numerical calculations (on a computer)
for the \ce{NaCl} lattice.  First use a cubic geometry
in which \(2ma\) is the cube side-length, with \(a\) the
separation of nearest neighbors, and second a spherical
geometry where \(ma\) is the radius of the sphere. Carry out
the calculation for \(m\) values of 97, 98 and 99, and compare
the results.  Please submit your code for
this question, as well as a discussion of why calculations in
the two geometries behave so differently.
} % makeproblem

\makeanswer{condensedMatter:problemSet1:3}{
\makeSubAnswer{}{condensedMatter:problemSet1:3a}
Consider an ionic arrangement as illustrated in \cref{fig:qmSolidsPs2P3a:qmSolidsPs2P3aFig5}.
\imageFigure{../figures/phy487-qmsolids/qmSolidsPs2P3aFig5}{Linear ionic solid configuration}{fig:qmSolidsPs2P3a:qmSolidsPs2P3aFig5}{0.2}

With \(r_{ij} = a p_{ij}\), with \(p_{ij}\) the difference in the enumeration indices, our Madelung constant for the ion at position zero is
%
\begin{dmath}\label{eqn:condensedMatterProblemSet1Problem3:20}
A_0
= \sum_{i \ne 0} \frac{z_i}{\Abs{r_{0i}/a}}
= \sum_{i \ne 0, i \,\text{odd}} \inv{i}
- \sum_{i \ne 0, i \,\text{odd}} \inv{i}
=
2 \sum_{j = 0}^\infty \inv{2 j + 1}
- 2 \sum_{j = 0}^\infty \inv{2 j + 2}
= 2 \lr{ \inv{1} - \inv{2} + \inv{3} - \inv{4} + \cdots }
= 2 \ln (1 + 1)
\approx 1.39
\end{dmath}
%
\makeSubAnswer{}{condensedMatter:problemSet1:3b}
Positioning the \ce{Cl-} ion in the center, we want to sum
%
\begin{dmath}\label{eqn:condensedMatterProblemSet1Problem3:40}
A
= \sum_{i,j,k \in [-m,m], \{i, j, k\} \ne \{0,0,0\}}
\frac{(-1)^{i + j + k + 1}}{\sqrt{i^2 + j^2 + k^2}}.
\end{dmath}
%
For the spherical geometry this sum will be limited by an additional constraint on the sum of
%
\begin{dmath}\label{eqn:condensedMatterProblemSet1Problem3:60}
i^2 + j^2 + k^2 \le m^2.
\end{dmath}
%
For the calculations see
%\href{https://raw.github.com/peeterjoot/physicsplay/master/notes/phy487/mathematica/qmSolidsPs1P3b.nb}{phy487/mathematica/qmSolidsPs1P3b.nb} (attached).
\nbref{qmSolidsPs1P3b.nb}

% http://www.physlink.com/education/askexperts/ae342.cfm
The sum over the cubic configuration appears to converge on a value in the range \((1.741, 1.753)\).  On the other hand for \(m = 97, 98, 99\) we have \(A = 15.405, -4.513, 3.570\) respectively.  This spherical summation is wildly divergent for small values of \(m\), oscillating between positive and negative values with no apparent regularity.  Given the symmetry of the cubic structure with respect to additional shells of charge it is not surprising that this sum is better behaved for low values of \(m\).  With a rough estimate of \(10^{18}\) atoms in a grain of salt, we'd still have \(m\) of the order \(10^8 \gg 99\) for a crystal structure of decent extent.  A sum with \(m = 99\) is still very small.  This is apparently small enough that we can't expect any sort of convergence for a spherical summation that has no \textunderline{inherent symmetry} as new spherical ``shells'' are added.
%
\paragraph{Grading remarks:} Lost marks for the underlined text above, with comment ``Explain specifically and in physical terms what this means.''  The posted solutions incuded a nice table that shows how the number of positive vs number of negative atoms in the lattice as \(m\) increases, and how that difference in the total charge difference oscillates wildly when we increase the spherical volume but very little when increasing the volume of the cube.
}
