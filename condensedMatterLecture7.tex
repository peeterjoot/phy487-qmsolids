%
% Copyright � 2013 Peeter Joot.  All Rights Reserved.
% Licenced as described in the file LICENSE under the root directory of this GIT repository.
%
%\input{../blogpost.tex}
%\renewcommand{\basename}{condensedMatterLecture7}
%\renewcommand{\dirname}{notes/phy487/}
%\newcommand{\keywords}{Condensed matter physics, PHY487H1F}
%\input{../peeter_prologue_print2.tex}
%
%%\citep{harald2003solid} \S x.y
%%\citep{ibach2009solid} \S x.y
%
%%\usepackage{mhchem}
%\usepackage[version=3]{mhchem}
%
%\beginArtNoToc
%\generatetitle{PHY487H1F Condensed Matter Physics.  Lecture 7: Structure factor.  Taught by Prof.\ Stephen Julian}
%\chapter{Structure factor}
\label{chap:condensedMatterLecture7}
%
%\section{Disclaimer}
%
%Peeter's lecture notes from class.  May not be entirely coherent.
%
\section{Structure factor.}
\index{structure factor}

\reading \citep{ashcroft1976solid} \textchapref{6}, \textchapref{9} (\S in monatomic lattices).

It is useful to quantify the interference between waves scattered from a unit cell.  To do so the structure factor is introduced, and the ideas are outlined here.  See \citep{ibach2009solid} \S 3.6, and \citep{ashcroft1976solid} \textchapref{6}, for more complete treatments.

Recall that the aperture function for a period lattice was defined by
%
\begin{equation}\label{eqn:condensedMatterLecture7:20}
\rho(\Br) = \sum_{h k l} \rho_{h k l} e^{i \BG_{h k l} \cdot \Br}.
\end{equation}
%
The Fourier coefficient \index{Fourier coefficient} \(\rho_{h k l}\) can be recovered by integrating over a unit cell as depicted in \cref{fig:qmSolidsL7:qmSolidsL7Fig1}
%
\begin{equation}\label{eqn:condensedMatterLecture7:40}
\rho_{h k l} = \inv{V_{\mathrm{cell}}} \int_{\mathrm{cell}} \rho(\Br) e^{-i \BG_{hkl} \cdot \Br} d\Br.
\end{equation}
%
\imageFigure{../figures/phy487-qmsolids/qmSolidsL7Fig1}{Cell relative vector positions.}{fig:qmSolidsL7:qmSolidsL7Fig1}{0.25}

Note that \(\rho(\Br)\) is large close to each nucleus.  Using the change of variables to atom centered bases, we find
%
\begin{dmath}\label{eqn:condensedMatterLecture7:60}
\rho_{h k l} \sim \inv{V_{\mathrm{cell}}} \sum_\alpha
e^{i \BG_{h k l} \cdot \Br_\alpha}
%\mathLabelBox
%[
%   labelstyle={xshift=2cm},
%   linestyle={out=270,in=90, latex-}
%]
%{
\int \rho_\alpha(\Br')
e^{-i \BG_{h k l} \cdot \Br} d\Br',
\end{dmath}
where the 
index \(\alpha\) is used to sum over all atoms in a primitive unit cell.  
The integral portion is 
the atomic scattering factor (which can be tabulated), and is designated
\begin{dmath}\label{eqn:condensedMatterLecture7:180}
f_\alpha = 
\int \rho_\alpha(\Br')
e^{-i \BG_{h k l} \cdot \Br} d\Br'.
\end{dmath}
In terms of the scattering factor, \cref{eqn:condensedMatterLecture7:60} is
%}
%
\begin{dmath}\label{eqn:condensedMatterLecture7:80}
\rho_{h k l} \sim \inv{V_{\mathrm{cell}}}
\mathLabelBox
[
   labelstyle={xshift=2cm},
   linestyle={out=270,in=90, latex-}
]
{
\sum_\alpha f_\alpha
e^{-i \BG_{h k l} \cdot \Br_\alpha}
}
{\(S_{h k l}\), the \textAndIndex{structure factor}}.
\end{dmath}
%
Recall from \eqnref{eqn:condensedMatterLecture6:40b} that the intensity at \(\BK = \BG\) is proportional to \(\Abs{\rho_\BG}^2\).  Because of this and \eqnref{eqn:condensedMatterLecture7:80}, we can utilize the structure factor as a measure of intensity at a reciprocal lattice point
%
\begin{equation}\label{eqn:condensedMatterLecture7:101}
I(\BK = \BG_{hkl}) \propto \Abs{S_{hkl}}^2.
\end{equation}
%
\makeexample{BCC lattice as simple cubic and 2 atom basis.}{example:condensedMatterLecture7:1}{
%F2
%\cref{fig:qmSolidsL7:qmSolidsL7Fig2}.
\imageFigure{../figures/phy487-qmsolids/qmSolidsL7Fig2}{Bcc.}{fig:qmSolidsL7:qmSolidsL7Fig2}{0.3}
%
\begin{subequations}
\begin{dmath}\label{eqn:condensedMatterLecture7:100}
\Br_1 =
\begin{bmatrix}
0 \\
0 \\
0
\end{bmatrix}
\end{dmath}
\begin{dmath}\label{eqn:condensedMatterLecture7:120}
\Br_2 =
\frac{a}{2}
\begin{bmatrix}
1 \\
1 \\
1
\end{bmatrix}
\end{dmath}
\end{subequations}
%
\begin{equation}\label{eqn:condensedMatterLecture7:140}
\BG_{h k l} \cdot \Br_n = 2 \pi m.
\end{equation}
%
\begin{dmath}\label{eqn:condensedMatterLecture7:160}
S_{h k l} = f (
\mathLabelBox
[
%   labelstyle={xshift=0cm},
   labelstyle={below of=m\themathLableNode, below of=m\themathLableNode},
%   linestyle={out=0,in=0, latex-}
]
{
1
}{corner}
+
e^{-i
   \mathLabelBox
   {
   \pi ( h + k + l )
   }
   {
   body center
   }
  }
).
\end{dmath}
%
A bcc lattice has same diffraction pattern as simple cubic, except all \(h + k + l = \text{odd}\) spots are missing.
}

\reading \citep{ibach2009solid} \S 3.7
%
\section{Brillouin zones.}
\index{Brillouin zones}

We can define a special primitive unit cell, by bisecting the (reciprocal) lattice vectors with a plane.  In 2D consider \cref{fig:qmSolidsL7:qmSolidsL7Fig3}
%
\imageFigure{../figures/phy487-qmsolids/qmSolidsL7Fig3}{First Brillouin zone.}{fig:qmSolidsL7:qmSolidsL7Fig3}{0.3}

All points inside the first Brillouin zone are closer to \((0, 0)\) than to any other lattice point.

Example: 07 lecture.pdf

Fcc lattice has a bcc reciprocal lattice.

Some 3D figures from \citep{wiki:BrillouinZone} were shown in slides.

\examhint{A frequent exam question will be to draw a primitive unit cell for a lattice: follow this procedure.}

