%
% Copyright � 2013 Peeter Joot.  All Rights Reserved.
% Licenced as described in the file LICENSE under the root directory of this GIT repository.
%
%\input{../blogpost.tex}
%\renewcommand{\basename}{condensedMatterLecture14}
%\renewcommand{\dirname}{notes/phy487/}
%\newcommand{\keywords}{Condensed matter physics, PHY487H1F}
%\input{../peeter_prologue_print2.tex}
%
%%\citep{harald2003solid} \S x.y
%%\citep{ibach2009solid} \S x.y
%
%%\usepackage{mhchem}
%\usepackage[version=3]{mhchem}
%\newcommand{\nought}[0]{\circ}
%\newcommand{\EF}[0]{\epsilon_{\txtF}}
%\newcommand{\kF}[0]{k_{\txtF}}
%
%\beginArtNoToc
%\generatetitle{PHY487H1F Condensed Matter Physics.  Lecture 14: Electrons in a periodic lattice.  Taught by Prof.\ Stephen Julian}
%%\chapter{Electrons in a periodic lattice}
%\label{chap:condensedMatterLecture14}
%
%%\section{Disclaimer}
%
%Peeter's lecture notes from class.  May not be entirely coherent.
%
\section{Electrons in a periodic lattice}
\index{periodic lattice}

\reading \citep{ashcroft1976solid} \textchapref{8}.

We want to look at the general properties of 1 electron in a periodic potential.  We want to solve the Schr\"{o}dinger equation
%
\begin{dmath}\label{eqn:condensedMatterLecture14:20}
\lr{
-\frac{\Hbar^2}{2m} \spacegrad^2 + V(\Br)
}
\Psi(\Br) = E \Psi(\Br).
\end{dmath}
%
Here \(V(\Br)\) is periodic \(V(\Br + \Br_n) = V(\Br)\), with
%
\begin{dmath}\label{eqn:condensedMatterLecture14:40}
\Br_n =
n_1 \Ba_1
+ n_2 \Ba_2
+ n_3 \Ba_3,
\end{dmath}
%
so that
%
\begin{dmath}\label{eqn:condensedMatterLecture14:60}
V(\Br) = \sum_\BG V_\BG e^{i \BG \cdot \Br },
\end{dmath}
%
where
%
\begin{subequations}
\begin{dmath}\label{eqn:condensedMatterLecture14:80}
\BG =
h \Bg_1
+ k \Bg_2
+ l \Bg_3
\end{dmath}
\begin{dmath}\label{eqn:condensedMatterLecture14:100}
\Bg_i \cdot \Ba_j = \delta_{ij}.
\end{dmath}
\end{subequations}
%
Example potential (roughly) sketched in \cref{fig:qmSolidsL14:qmSolidsL14Fig1}.
%
\imageFigure{../figures/phy487-qmsolids/qmSolidsL14Fig1}{Fourier decomposition of 1D periodic potential}{fig:qmSolidsL14:qmSolidsL14Fig1}{0.2}

Use a plane wave basis
%
\begin{dmath}\label{eqn:condensedMatterLecture14:120}
\Psi(\Br) = \sum_\Bk C_\Bk e^{i \Bk \cdot \Br },
\end{dmath}
%
\begin{dmath}\label{eqn:condensedMatterLecture14:140}
\sum_{\Bk'} \frac{\Hbar^2 {\Bk'}^2 }{2m} C_\Bk' e^{i \Bk' \cdot \Br}
+
\sum_{\Bk'', \BG} V_\BG C_{\Bk''}e^{i (\Bk'' + \BG) \cdot \Br}
=
E \sum_{\Bk' }
e^{i \Bk' \cdot \Br},
\end{dmath}
%
with \(\Bk' = \Bk'' + \BG\), or \(\Bk'' = \Bk' - \BG\), this is
%
\begin{dmath}\label{eqn:condensedMatterLecture14:200}
0 =
\sum_{\Bk' }
\lr{
\lr{ \frac{\Hbar^2 {\Bk'}^2 }{2m} - E} C_\Bk'
+
\sum_\BG V_\BG C_{\Bk' - \BG}
}
e^{ i \Bk' \cdot \Br }.
\end{dmath}
%
Using
%
\begin{dmath}\label{eqn:condensedMatterLecture14:220}
\int d\Br e^{ -i (\Bk' - \Bk) \cdot \Br } \propto \delta(\Bk' - \Bk),
\end{dmath}
%
and operating with
\begin{dmath}\label{eqn:condensedMatterLecture14:240}
\int d\Br e^{ -i \Bk \cdot \Br } \times \cdots,
\end{dmath}
%
we decouple the system
\boxedEquation{eqn:condensedMatterLecture14:260}{
0 =
\lr{\frac{\Hbar^2 {\Bk}^2 }{2m} - E} C_\Bk
+
\sum_\BG V_\BG C_{\Bk - \BG}.
}

Each eigenstate only involves \(C_\Bk\)'s that differ by reciprocal lattice vectors.

%\cref{fig:qmSolidsL14:qmSolidsL14Fig2}.
\imageFigure{../figures/phy487-qmsolids/qmSolidsL14Fig2}{Energy vs momentum}{fig:qmSolidsL14:qmSolidsL14Fig2}{0.2}
%
\begin{dmath}\label{eqn:condensedMatterLecture14:280}
\Psi \sim
C_k e^{i k x}
+ C_{k + 2 \pi/a} e^{i (k + 2 \pi/a) x}
+ C_{k - 2 \pi/a} e^{i (k - 2 \pi/a) x}
+ \cdots.
\end{dmath}
%
Label each eigenstate with \(\Bk\)
%
\begin{subequations}
\begin{equation}\label{eqn:condensedMatterLecture14:300}
\Psi_\Bk(\Br)
\end{equation}
\begin{equation}\label{eqn:condensedMatterLecture14:320}
E = E_\Bk = E(\Bk),
\end{equation}
\end{subequations}
%
\begin{dmath}\label{eqn:condensedMatterLecture14:340}
\Psi_\Bk(\Br)
= \sum_\BG C_{\Bk - \BG} e^{ i (\Bk - \BG) \cdot \Br}
=
\mathLabelBox
[
   labelstyle={xshift=2cm},
   linestyle={out=270,in=90, latex-}
]
{
\lr{
\sum_\BG C_{\Bk - \BG} e^{ -i \BG \cdot \Br}
}
}
{Periodic in \(\Br\) with lattice periodicity}
\mathLabelBox
[
   labelstyle={below of=m\themathLableNode, below of=m\themathLableNode}
]
{
e^{ i \Bk \cdot \Br}
}
{plane wave}.
\end{dmath}
%
\boxedEquation{eqn:condensedMatterLecture14:360}{
\Psi_\Bk(\Br) = U_\Bk(\Br) e^{i \Bk \cdot \Br},
}
where \(U_\Bk(\Br)\) is a periodic function.  This is \underlineAndIndex{Bloch's theorem}.

With \(\Psi_\Bk(\Br)\) periodic in \(\Bk\) we have
%
\begin{dmath}\label{eqn:condensedMatterLecture14:380}
\Psi_{\Bk + \BG}(\Br)
=
\sum_{\BG'} C_{\Bk + \BG - \BG'} e^{ i (\Bk + \BG - \BG') \cdot \Br}.
\end{dmath}
%
With
%
\begin{dmath}\label{eqn:condensedMatterLecture14:400}
\BG'' = \BG' - \BG,
\end{dmath}
%
this is
%
\begin{dmath}\label{eqn:condensedMatterLecture14:420}
\Psi_{\Bk + \BG}(\Br)
=
\sum_{\BG''} C_{\Bk - \BG''} e^{ i (\Bk - \BG'') \cdot \Br}
=
\Psi_\Bk(\Br).
\end{dmath}
%
\boxedEquation{eqn:condensedMatterLecture14:440}{
\begin{aligned}
\Psi_{\Bk + \BG}(\Br)
&=
\Psi_\Bk(\Br) \\
E(\Bk + \BG) &= E(\Bk).
\end{aligned}
}

We want to examine what this means.
%
\section{Nearly free electron model}
\index{nearly free electron}

Consider \(V(\Br)\) in the limit \(V(\Br) \rightarrow 0\) in 1D, but still keep periodicity.  This leads to
%
\begin{dmath}\label{eqn:condensedMatterLecture14:460}
\begin{aligned}
\Psi_{k}(x) &= \inv{\sqrt{L}} e^{i k x} \\
E(k) &= \frac{\Hbar^2 k^2}{2m}
\end{aligned}
\end{dmath}
%
%\cref{fig:qmSolidsL14:qmSolidsL14Fig3}.
%\imageFigure{../figures/phy487-qmsolids/qmSolidsL14Fig3}{Periodic energy solutions}{fig:qmSolidsL14:qmSolidsL14Fig3}{0.2}
\mathImageFigure{../figures/phy487-qmsolids/nearlyFreeFig1}{Periodic energy solutions}{fig:qmSolidsL14:qmSolidsL14Fig3}{0.3}{nearlyFreeFig1GeneratedLabelled.nb}

Solutions outside the first Brillouin zone are redundant.  This is called the \textAndIndex{reduced zone scheme}.  The periodicity folds the extended solution into the first Brillouin zone as sketched in \cref{fig:qmSolidsL14:qmSolidsL14Fig4}.
%
\imageFigure{../figures/phy487-qmsolids/qmSolidsL14Fig4}{First Brillouin zone}{fig:qmSolidsL14:qmSolidsL14Fig4}{0.4}

%The meaning of this periodicity in momen
%
\paragraph{Bloch's theorem spelled out}
\index{Bloch's theorem}

In \cref{qmSolidsL14:tab:1} this is written out explicitly for a couple values.

\captionedTable{Bloch's theorem spelled out}{qmSolidsL14:tab:1}{
    \begin{tabular}{|lllll|}
    \hline
      & \(\Psi_\Bk(x)\)                          & \(U_\Bk(x)\)                           & \(E(\Bk)\)                              & \(C_\Bk's\) \\
    \hline
    1 & \(\inv{\sqrt{L}} e^{i k x}\)             & \(\inv{\sqrt{L}}\)                     & \(\frac{\Hbar^2 k^2}{2m}\)              & \(C_k = 1 ; C_{k - \BG} = 0, \BG = 0\) \\
    2 & \(\inv{\sqrt{L}} e^{i (k - 2 \pi/a)k x}\)             & \(\frac{e^{-2 \pi x/a}}{\sqrt{L}}\)                     & \(\frac{\Hbar^2 (k - 2 \pi/a)^2}{2m}\)              & \(C_{k - 2 \pi/a} = 1 ; \text{all others zero}\) \\
    3 & \(\inv{\sqrt{L}} e^{i (k + 2 \pi/a)k x}\)             & \(\frac{e^{2 \pi x/a}}{\sqrt{L}}\)                     & \(\frac{\Hbar^2 (k + 2 \pi/a)^2}{2m}\)              & \(C_{k + 2 \pi/a} = 1 ; \text{all others zero}\) \\
    \hline
    \end{tabular}
}

\newpage

%\EndNoBibArticle
