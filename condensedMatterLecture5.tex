%
% Copyright � 2013 Peeter Joot.  All Rights Reserved.
% Licenced as described in the file LICENSE under the root directory of this GIT repository.
%
%\input{../blogpost.tex}
%\renewcommand{\basename}{condensedMatterLecture5}
%\renewcommand{\dirname}{notes/phy487/}
%\newcommand{\keywords}{Condensed matter physics, PHY487H1F}
%\input{../peeter_prologue_print2.tex}
%
%%\citep{harald2003solid} \S x.y
%%\citep{ibach2009solid} \S x.y
%
%\usepackage[version=3]{mhchem}
%
%\beginArtNoToc
%\generatetitle{PHY487H1F Condensed Matter Physics.  Lecture 5: General theory of diffraction.  Taught by Prof.\ Stephen Julian}
%%\chapter{General theory of diffraction}
\label{chap:condensedMatterLecture5}
%
%\section{Disclaimer}
%
%Peeter's lecture notes from class.  May not be entirely coherent.
%
\section{General theory of diffraction.}
\index{diffraction}

Our diffraction geometry is illustrated in \cref{fig:qmSolidsL5:qmSolidsL5Fig1}.
%
\imageFigure{../figures/phy487-qmsolids/qmSolidsL5Fig1}{Diffraction in crystal by x-rays, neutrons, ...}{fig:qmSolidsL5:qmSolidsL5Fig1}{0.2}

In the crystal we can imagine light and atom interaction as illustrated in \cref{fig:qmSolidsL5:qmSolidsL5Fig2}.
%
\imageFigure{../figures/phy487-qmsolids/qmSolidsL5Fig2}{Diffraction interaction in the crystal.}{fig:qmSolidsL5:qmSolidsL5Fig2}{0.2}

The incident beam makes electrons vibrate at frequency \(\omega_\nought\), and re-radiate at \(\omega_\nought\).

The diffraction pattern is the constructive interference of the re-radiated x-rays (or neutrons)

At \(P\), primary beam has amplitude
%
\begin{dmath}\label{eqn:condensedMatterLecture5:20}
A_p(\Br, t) =
A_\nought
\exp\lr{ i \lr{
k_\nought
\cdot
\lr{ \BR + \Br} - \omega_\nought t}  }.
\end{dmath}
%
Here \(A_\nought\) is constant everywhere and \(k_\nought = 2 \pi/\lambda_\nought\).

Scattered wave at \(P\) has amplitude
%
\begin{dmath}\label{eqn:condensedMatterLecture5:120}
\rho(\Br) A_p(\Br, t).
\end{dmath}
%
where
%
\begin{dmath}\label{eqn:condensedMatterLecture5:40}
\rho(\Br) = \text{``scattering density''}.
\end{dmath}
%
For neutrons we've got interactions with the nuclei, and things get messier, but for x-rays
%
\begin{dmath}\label{eqn:condensedMatterLecture5:60}
\rho(\Br) \sim \text{electron density}.
\end{dmath}
%
This density is largest near the nucleus, at least for large Z (heavy) atoms like \ce{La}, and \ce{Ac}.  Large Z atoms are easier to see.

Amplitude at \(B\), due to \(\Br\)
%
\begin{dmath}\label{eqn:condensedMatterLecture5:80}
A_B(\Br) = A_p(\Br) \rho(\Br)
\frac{
e^{
i \Bk \cdot
\lr{ \BR' - \Br}
}
}
{
\Abs{
\BR' - \Br
}
}
\approx
A_p(\Br, t) \rho(\Br)
\frac{
e^{
	i \Bk \cdot
	\lr{ \BR' - \Br}
  }
}
{
  \Abs{\BR'}
}.
\end{dmath}
%
time independent part of \(\rho(\Br)\) => secondary beam has frequency \(\omega_\nought\), or
%
\begin{dmath}\label{eqn:condensedMatterLecture5:100}
\Abs{\Bk} = \Abs{\Bk_\nought} = \frac{\omega_\nought}{c}.
\end{dmath}
%
This is the \underlineAndIndex{elastic scattering}.  We ignore inelastic scattering.

The vector \(\BR'\) determines the direction of \(\Bk\).

If \(R' \gg r\), \(\Bk\) is close to the same for all \(\Br\).  Then \eqnref{eqn:condensedMatterLecture5:80} is approximately
%
\begin{dmath}\label{eqn:condensedMatterLecture5:140}
A_B(\Br) \approx
\mathLabelBox
[
   labelstyle={xshift=-2cm},
   linestyle={out=270,in=90, latex-}
]
{
\frac{A_\nought}{R'} e^{ i \lr{ \Bk_\nought \cdot \BR + \Bk \cdot \BR'} }
}{independent of \(\Br\)}
\mathLabelBox
[
   labelstyle={xshift=2cm},
   linestyle={out=270,in=90, latex-}
]
{
\rho(\Br) e^{ i \lr{ \Bk_\nought - \Bk} \cdot \Br } e^{-i \omega_\nought t}
}
{dependent of \(\Br\)}.
\end{dmath}
%
Total amplitude at \(B\)
%
\begin{dmath}\label{eqn:condensedMatterLecture5:160}
A_B \propto e^{-i \omega_\nought t } \int \rho(\Br) e^{ i
\mathLabelBox
%[
%   labelstyle={below of=m\themathLableNode, below of=m\themathLableNode}
%]
{
\lr{ \Bk_\nought -\Bk}
}
{ \(\equiv - \BK\)}
\cdot \Br }
d\Br.
\end{dmath}
%
So that the intensity is
%
\begin{dmath}\label{eqn:condensedMatterLecture5:180}
I_\BB \propto \Abs{A_B}^2 \propto
\Abs{
\int \rho(\Br) e^{-i \BK \cdot \Br}
d\Br
}^2.
\end{dmath}
%
This is the Fourier transform of the scattering density.
%
\section{Reciprocal lattice.}
\index{reciprocal lattice}
%\section{3D periodic structures}
%
\reading \citep{ashcroft1976solid} \textchapref{5}.

- 1D crystal.  Illustrated in \cref{fig:qmSolidsL5:qmSolidsL5Fig3}.
%
\imageFigure{../figures/phy487-qmsolids/qmSolidsL5Fig3}{1D crystal diffraction electron density.}{fig:qmSolidsL5:qmSolidsL5Fig3}{0.24}
%
\begin{dmath}\label{eqn:condensedMatterLecture5:200}
\rho(x) = \sum_n p_n e^{ i n 2 \pi x/a }.
\end{dmath}
%
Show that
%
\begin{dmath}\label{eqn:condensedMatterLecture5:220}
\rho(x + m a) = \rho(x).
\end{dmath}
%
- 3D lattice
%
\begin{dmath}\label{eqn:condensedMatterLecture5:240}
\rho(\Br) = \sum_\BG \rho_\BG e^{ i \BG \cdot \Br }.
\end{dmath}
%
With
%
\begin{dmath}\label{eqn:condensedMatterLecture5:260}
\Br_n =
n_1 \Ba_1 +
n_2 \Ba_2 +
n_3 \Ba_3,
\end{dmath}
%
find the \(\BG\)'s, such that
%
\begin{dmath}\label{eqn:condensedMatterLecture5:280}
\rho(\Br + \Br_n) = \rho(\Br).
\end{dmath}
%
\begin{dmath}\label{eqn:condensedMatterLecture5:300}
e^{ i \BG \cdot \Br_n } = 1,
\end{dmath}
%
or
\begin{dmath}\label{eqn:condensedMatterLecture5:320}
\BG \cdot \Br_n = 2 \pi m,
\end{dmath}
%
where \(m\) is an integer.

Try
%
\begin{dmath}\label{eqn:condensedMatterLecture5:340}
\BG =
h \Bg_1 +
k \Bg_2 +
l \Bg_3,
\end{dmath}
%
where \(h, k, l\) are integers.

The \(\BG\)'s are wave vectors of waves with the periodicity of the lattice.
\makeexample{2d periodic lattice}{example:condensedMatterLecture5:1}{
An example of reciprocal projection is illustrated in \cref{fig:obliqueReciprocal:obliqueReciprocalFig1}.
%
\imageFigure{../figures/phy487-qmsolids/obliqueReciprocalFig1}{reciprocal projection.}{fig:obliqueReciprocal:obliqueReciprocalFig1}{0.4}

Here \(e^i \cdot e_j = \delta^i_j\), not using the \(2 \pi\) scaling factor that we are using in this diffraction context.

FIXME: Our prof used the following to illustrate (which I did a brute force cut and paste of from the prof's notes: 05 lecture.pdf).  Re-draw once I figure out what he was illustrating.

%\cref{fig:qmSolidsL5:qmSolidsL5Fig4}.
\imageFigure{../figures/phy487-qmsolids/qmSolidsL5Fig4}{2d periodic lattice.}{fig:qmSolidsL5:qmSolidsL5Fig4}{0.2}

Wave 1 :
%
\begin{subequations}
\begin{equation}\label{eqn:condensedMatterLecture5:360}
\Bg_1 \cdot \Ba_1 = g_1 a_1 \cos\theta_1 = 2 \pi
\end{equation}
\begin{equation}\label{eqn:condensedMatterLecture5:380}
\Bg_1 \cdot \Ba_2 = 0.
\end{equation}
\end{subequations}
%
Wave 2 : \(\Bg_2 \cdot \Ba_1 = 0, \)
%
\begin{subequations}
\begin{equation}\label{eqn:condensedMatterLecture5:400}
\Bg_2 \cdot \Ba_1 = 0
\end{equation}
\begin{equation}\label{eqn:condensedMatterLecture5:420}
\Bg_2 \cdot \Ba_2
= g_2 a_2 \cos\theta_2 = 2 \pi.
\end{equation}
\end{subequations}
%
We can read off relations between the wavelengths from these
%
\begin{equation}\label{eqn:condensedMatterLecture5:480}
\begin{aligned}
\lambda_1 &= a_1 \cos \theta_1 \\
\lambda_2 &= a_2 \cos \theta_2.
\end{aligned}
\end{equation}
}
We introduce \underlineAndIndex{reciprocal vectors} defined by
%
\begin{dmath}\label{eqn:condensedMatterLecture5:440}
\Bg_i \cdot \Ba_j = 2 \pi \delta_{ij}.
\end{dmath}
%
The general formula in 3D is
%
\begin{subequations}
\begin{dmath}\label{eqn:condensedMatterLecture5:460}
\Bg_1 = 2 \pi
\frac
{ \Ba_2 \cross \Ba_3 }
{ \Ba_1 \cdot \lr{ \Ba_2 \cross \Ba_3} }
\end{dmath}
\begin{dmath}\label{eqn:condensedMatterLecture5:460b}
\Bg_2 = 2 \pi
\frac
{ \Ba_3 \cross \Ba_2 }
{ \Ba_2 \cdot \lr{ \Ba_3 \cross \Ba_1} }
\end{dmath}
\begin{dmath}\label{eqn:condensedMatterLecture5:460c}
\Bg_3 = 2 \pi
\frac
{ \Ba_1 \cross \Ba_3 }
{ \Ba_3 \cdot \lr{ \Ba_1 \cross \Ba_2} }.
\end{dmath}
\end{subequations}
%
The numerator cross product ensures that \(\Bg_1\) is perpendicular to \(\Ba_2\) and \(\Ba_3\).  The one in the denominator ``cancels'' the \(\Ba_2 \cross \Ba_3\) in the numerator.


%\EndNoBibArticle
