%
% Copyright � 2013 Peeter Joot.  All Rights Reserved.
% Licenced as described in the file LICENSE under the root directory of this GIT repository.
%
\makeoproblem{Analytic calculations of density of states and effective mass}{condensedMatter:problemSet9:2}{2013 ps9 p2}{
The dispersion relation of
a two-dimensional square lattice with lattice parameter \(a\),
with nearest-neighbour tight-binding integrals \(A\) and \(B\), is
%
\begin{eqnarray*}
E(\Bk) \simeq E_s - A - 2B(\cos(k_xa) + \cos(k_ya)).
\end{eqnarray*}
%
\makesubproblem{}{condensedMatter:problemSet9:2a}
Expanding \(E(k)\) around the bottom and the top of the band to second
  order in \(\Bk\) (i.e.\ approximating the bands as parabolic)
  show that the density of states per unit area jumps from zero when \(E\) is
  outside the band, to a finite value as \(E\) passes the bottom or top of the band.
  What is the finite value?
\makesubproblem{}{condensedMatter:problemSet9:2b}
Consider now the analogous three-dimensional simple cubic system,
  such that
  \begin{eqnarray*}
    E(\Bk) \simeq E_s - A - 2B(\cos(k_x a) + \cos(k_y a) + \cos(k_z a)).
  \end{eqnarray*}
Show that the density of states now grows continuously from zero,
 and is proportional to
 \(\sqrt(E-E_b)\), and \(\sqrt(E_t-E)\), as the bottom (\(E_b\)) and top (\(E_t\))
 of the band are crossed.
Find the constants of proportionality.
\makesubproblem{}{condensedMatter:problemSet9:2c}
For the three-dimensional cubic system of \partref{condensedMatter:problemSet9:2b}, find the group
  velocity \(\Bv_g\) along the \(k_x\) direction, as a function of \(k_x\),
  and sketch its behaviour from \(k_x=0\) to \(\pi/a\).
\makesubproblem{}{condensedMatter:problemSet9:2d}
Using our definition of group velocity from class, and defining the
  velocity effective mass \(m^*\) by \(v_g = \Hbar k/m^*\),
  plot \(m^*\) vs \(k_x\) from 0 to \(\pi/a\).
} % makeproblem
\makeanswer{condensedMatter:problemSet9:2}{
\makeSubAnswer{}{condensedMatter:problemSet9:2a}
It's helpful to recall the geometry of the energy distribution in k-space.  This was plotted in \cref{fig:qmSolidsPs8d:qmSolidsPs8dFig2}.
%\mathImageFigure{../figures/phy487-qmsolids/qmSolidsPs8dFig2}{Energy level curves}{fig:qmSolidsPs8d:qmSolidsPs8dFig2}{0.2}{qmSolidsPs8dContourPlot.nb}
At the bottom of the distribution, to second order, we have
%
\begin{dmath}\label{eqn:condensedMatterProblemSet9Problem2:20}
E(\Bk)
\approx E_s - A - 2B \lr{
1 - \inv{2} \lr{ k_x a }^2
+ 1 - \inv{2} \lr{ k_y a }^2
}
= E_s - A - 4 B + B a^2 \lr{ k_x^2 + k_y^2 }.
\end{dmath}
%
For the gradient we have
%
\begin{dmath}\label{eqn:condensedMatterProblemSet9Problem2:40}
\spacegrad_\Bk E(\Bk)
= B a^2 \spacegrad \lr{ k_x^2 + k_y^2 }
= B a^2 \lr{ 2 k_x, 2 k_y }
= 2 B a^2 \Bk.
\end{dmath}
%
With cylindrical coordinates, we have
%
\begin{dmath}\label{eqn:condensedMatterProblemSet9Problem2:60}
d \Bk
= k d\phi_k dk
= k d\phi_k \frac{dE}{\Abs{\spacegrad_\Bk E}},
\end{dmath}
%
So that the density of states is given by
%
\begin{dmath}\label{eqn:condensedMatterProblemSet9Problem2:80}
D(E) dE
=
2 \times \inv{A} \frac{A}{(2 \pi)^2}
\int_{\phi_k = 0}^{2 \pi} \cancel{k} d\phi_k \frac{dE}{2 B a^2 \cancel{k}},
\end{dmath}
%
or
\boxedEquation{eqn:condensedMatterProblemSet9Problem2:100}{
D(E) = \frac{1}{2 \pi B a^2}.
}

For the top of the energy levels, we can also expand to second order in \(k_x, k_y\), at the points \(k_x,k_y = \pm \pi/a\).  With \(k\) for one of \(k_x\), or \(k_y\) we have at the corner \(k = \pm \pi/a\)
%
\begin{dmath}\label{eqn:condensedMatterProblemSet9Problem2:120}
\cos k a
=
\inv{0!}
\evalbar{\cos k a}{ k = \pm \pi/a }
+\inv{1!}
\evalbar{\cos' k a}{ k = \pm \pi/a }
\lr{ k \mp \pi/a }
+\inv{2!}
\evalbar{\cos' k a}{ k = \pm \pi/a } \lr{ k \mp \pi/a }^2
+ \cdots
\approx
\cos \lr{ \pm \pi }
-a \sin \lr{ \pm \pi } \lr{ k \mp \pi/a }
-a^2 \cos \lr{ \pm \pi } \lr{ k \mp \pi/a }^2
=
-1 + \inv{2} \lr{ k a \mp \pi }^2.
\end{dmath}
%
So, at the corner \(\Bk = \pm (1,1) \pi/a\) the energy is approximately
%
\begin{dmath}\label{eqn:condensedMatterProblemSet9Problem2:140}
E(\Bk) =
E_s - A + 4 B - B
\lr{
\lr{ k_x a \mp \pi}^2
+ \lr{ k_y a \mp \pi}^2
},
\end{dmath}
%
with gradient
\begin{dmath}\label{eqn:condensedMatterProblemSet9Problem2:180}
\spacegrad_\Bk E(\Bk)
=
- 2 B a
\lr{
k_x a \mp \pi,
+ k_y a \mp \pi
}
=
- 2 B a^2
\lr{
k_x \mp \pi/a,
+ k_y \mp \pi/a
}.
\end{dmath}
%
At the corners \(\Bk = \pm(1, -1) \pi/a\) we have approximately
\begin{dmath}\label{eqn:condensedMatterProblemSet9Problem2:160}
E(\Bk) =
E_s - A + 4 B - B
\lr{
\lr{ k_x a \mp \pi}^2
+ \lr{ k_y a \pm \pi}^2
},
\end{dmath}
%
which have gradients
%
\begin{dmath}\label{eqn:condensedMatterProblemSet9Problem2:200}
\spacegrad_\Bk E(\Bk)
%=
%- 2 B a
%\lr{
%k_x a \mp \pi,
%+ k_y a \pm \pi
%}
=
- 2 B a^2
\lr{
k_x \mp \pi/a,
+ k_y \pm \pi/a
}.
\end{dmath}
%
Consider the \(\Bk = -(1,1)\pi/a\) corner, and make the change of variables
%
\begin{dmath}\label{eqn:condensedMatterProblemSet9Problem2:220}
\begin{aligned}
k_x + \pi/a &= k \cos\phi_k \\
k_y + \pi/a &= k \sin\phi_k.
\end{aligned}
\end{dmath}
%
We see that we have to only consider this portion of the k-space area, quadrupling the integral, so that the density of states is
%
\begin{dmath}\label{eqn:condensedMatterProblemSet9Problem2:240}
D(E) dE = 2 \times
4
\inv{(2 \pi)^2} \int_{\phi_k = 0}^{\pi/2} k d\phi_k \frac{dE}{\Abs{-2 B a^2 k}}.
\end{dmath}
%
This is identical to \eqnref{eqn:condensedMatterProblemSet9Problem2:100}, the constant value for the density of states found for the bottom of the band, provided we expand the energy only to second order in \(k_x, k_y\).
\makeSubAnswer{}{condensedMatter:problemSet9:2b}
At the bottom of the band, again approximating the energy to second order in \(k_x, k_y, k_z\), we have
%
\begin{dmath}\label{eqn:condensedMatterProblemSet9Problem2:360}
E(\Bk) \approx
E_s - A - 2 B \lr{
3
- \inv{2} k_x^2 a^2
- \inv{2} k_y^2 a^2
- \inv{2} k_z^2 a^2
}
=
E_s - A - 6 B + B a^2 \lr{ k_x^2 + k_y^2 + k_z^2 }.
\end{dmath}
%
The energy gradient is
%
\begin{equation}\label{eqn:condensedMatterProblemSet9Problem2:380}
\spacegrad_\Bk E
= B a^2 \spacegrad_\Bk \Bk^2
= 2 B a^2 \Bk.
\end{equation}
%
Our density of states at the bottom of the band is thus
%
\begin{dmath}\label{eqn:condensedMatterProblemSet9Problem2:400}
D(E)
= 2 \times \inv{8 \pi^3 } \evalbar{\frac{ 4 \pi k^2 }{ 2 B a^2 k}}{k = k(E)}
= \inv{2 \pi^2 B a^2}
\lr{ \frac{ E - E_s + A + 6 B }{B a^2 } }^{1/2}
=
\inv{ 2 \pi^2 B^{3/2} a^3 } \sqrt{ E - E_s + A + 6 B }
\end{dmath}
%
\boxedEquation{eqn:condensedMatterProblemSet9Problem2:500}{
D(E)
=
\inv{ 2 \pi^2 B^{3/2} a^3 } \sqrt{ E - E_b },
}

where \(E_b = E_s - A - 6 B\).

For the calculation at the top of the band, we can expand the cosines around the corner coordinates.  Considering just the \(\Bk = -(1,1,1) \pi/a\) octet, and multiplying by 8 for the total density of states we have
%
\begin{dmath}\label{eqn:condensedMatterProblemSet9Problem2:420}
E(\Bk) \approx
E_s - A - 2 B \lr{
-3
+ \inv{2} \lr{ k_x a + \pi}^2
+ \inv{2} \lr{ k_y a + \pi}^2
+ \inv{2} \lr{ k_z a + \pi}^2
}
=
E_s - A + 6 B - B \lr{
\lr{ k_x a + \pi}^2
+ \lr{ k_y a + \pi}^2
+ \lr{ k_z a + \pi}^2 }.
\end{dmath}
%
The gradient at this point is
%
\begin{dmath}\label{eqn:condensedMatterProblemSet9Problem2:440}
\spacegrad_\Bk E(\Bk)
=
- B a^2 \spacegrad_\Bk \lr{
\lr{ k_x + \pi/a}^2
+ \lr{ k_y + \pi/a}^2
+ \lr{ k_z + \pi/a}^2
}
=
- 2 B a^2 \lr{
k_x + \pi/a,
k_y + \pi/a,
k_z + \pi/a
}.
\end{dmath}
%
Now introduce spherical coordinates with the origin at this point
%
\begin{equation}\label{eqn:condensedMatterProblemSet9Problem2:460}
\begin{aligned}
k_x + \pi/a &= k \sin\theta_k \cos\phi_k \\
k_y + \pi/a &= k \sin\theta_k \sin\phi_k \\
k_z + \pi/a &= k \cos\theta_k,
\end{aligned}
\end{equation}
%
So that the density of states is
%
\begin{dmath}\label{eqn:condensedMatterProblemSet9Problem2:480}
D(E) = 2 \times 8 \times \inv{ (2 \pi)^3 }
\int_0^{\pi/2} \sin \theta_k d\theta_k
\int_0^{\pi/2} d\phi_k
\evalbar
{
\frac{k^2}{ 2 B a^2 k}
}{k = k(E)}
= \inv{\pi^3 B a^2} \frac{\pi}{2} \lr{\frac{E - E_s + A - 6 B}{-B a^2}}^{1/2}
\end{dmath}
%
This is
\boxedEquation{eqn:condensedMatterProblemSet9Problem2:520}{
D(E)
= \inv{2 \pi^2 B^{3/2} a^3} \sqrt{E_t - E},
}
where \(E_t = E_s - A + 6 B\).  The constant of proportionality is the same we found for the bottom of the band in \eqnref{eqn:condensedMatterProblemSet9Problem2:500}.
\makeSubAnswer{}{condensedMatter:problemSet9:2c}
The expectation value of the velocity operator is given by
%
\begin{dmath}\label{eqn:condensedMatterProblemSet9Problem2:260}
\Bv_g
= \inv{\Hbar} \spacegrad_\Bk E
= \frac{-2 B }{\Hbar} \spacegrad_\Bk \lr{ \cos k_x a + \cos k_y a + \cos k_z a}
= \frac{2 B a}{\Hbar} \lr{ \sin k_x a, \sin k_y a, \sin k_z a}
\end{dmath}
%
Along the \(k_x\) direction we have
%
\begin{dmath}\label{eqn:condensedMatterProblemSet9Problem2:280}
\Bv_g \cdot (1, 0, 0) = \frac{2 B a}{\Hbar} \sin k_x a.
\end{dmath}
%
This is plotted for the \(k_x = [0, \pi/a]\) region in \cref{fig:qmSolidsPs9P2c:qmSolidsPs9P2cFig1}.
\mathImageFigure{../figures/phy487-qmsolids/qmSolidsPs9P2cFig1}{\(k_x\) component of \(\Bv_g\)}{fig:qmSolidsPs9P2c:qmSolidsPs9P2cFig1}{0.3}{ps9p2deFigures.nb}
\makeSubAnswer{}{condensedMatter:problemSet9:2d}
From \eqnref{eqn:condensedMatterProblemSet9Problem2:260} and our definition, we have
%
\begin{equation}\label{eqn:condensedMatterProblemSet9Problem2:300}
v_g = \frac{ 2 B a }{\hbar} \lr{ \sin^2 k_x a + \sin^2 k_y a + \sin^2 k_z a }^{1/2} = \frac{\Hbar k}{m^\conj},
\end{equation}
%
or
%
\begin{equation}\label{eqn:condensedMatterProblemSet9Problem2:320}
m^\conj = \frac{\Hbar^2}{2 B a} \sqrt{ \frac{k_x^2 + k_y^2 + k_z^2}{
\sin^2 k_x a + \sin^2 k_y a + \sin^2 k_z a }}.
\end{equation}
%
Along the \(k_x\) axis we have
%
\begin{equation}\label{eqn:condensedMatterProblemSet9Problem2:340}
m^\conj( k_x, 0, 0) = \frac{\Hbar^2}{2 B a^2} \Abs{ \frac{k_x a}{ \sin k_x a } }.
\end{equation}
%
This is plotted in \cref{fig:qmSolidsPs9P2d:qmSolidsPs9P2dFig1}.
%
\mathImageFigure{../figures/phy487-qmsolids/qmSolidsPs9P2dFig1}{Effective mass for 3D cubic system}{fig:qmSolidsPs9P2d:qmSolidsPs9P2dFig1}{0.3}{ps9p2deFigures.nb}
}
