%
% Copyright � 2013 Peeter Joot.  All Rights Reserved.
% Licenced as described in the file LICENSE under the root directory of this GIT repository.
%
%\input{../blogpost.tex}
%\renewcommand{\basename}{condensedMatterLecture20}
%\renewcommand{\dirname}{notes/phy487/}
%\newcommand{\keywords}{Condensed matter physics, PHY487H1F}
%\input{../peeter_prologue_print2.tex}
%
%%\citep{harald2003solid} \S x.y
%%\citep{ibach2009solid} \S x.y
%
%%\usepackage{mhchem}
%\usepackage{bm} % \EE
%\usepackage[version=3]{mhchem}
%\usepackage{units}
%\newcommand{\nought}[0]{\circ}
%%\newcommand{\EF}[0]{\epsilon_{\txtF}}
%\newcommand{\EF}[0]{E_{\txtF}}
%\newcommand{\kF}[0]{k_{\txtF}}
%\newcommand{\vF}[0]{v_{\txtF}}
%
%\beginArtNoToc
%\generatetitle{PHY487H1F Condensed Matter Physics.  Lecture 20: Electric current (cont.).  Taught by Prof.\ Stephen Julian}
%\chapter{Electric current (cont.)}
\label{chap:condensedMatterLecture20}

%\section{Disclaimer}
%
%Peeter's lecture notes from class.  May not be entirely coherent.

\paragraph{Electric current (cont.)}
\index{electric current}

\(\bcE \ne 0\) shifts the Fermi distribution so that so that \(+\Bv_1 - \Bv\) no longer cancel.

%\cref{fig:qmSolidsL20:qmSolidsL20Fig1}.
\imageFigure{../figures/phy487-qmsolids/qmSolidsL20Fig1}{Fermi filling for a metal}{fig:qmSolidsL20:qmSolidsL20Fig1}{0.2}

Current flows, unless the band is completely full, or empty as sketched in \cref{fig:qmSolidsL20:qmSolidsL20Fig2}.

\imageFigure{../figures/phy487-qmsolids/qmSolidsL20Fig2}{Fermi filling of an insulator}{fig:qmSolidsL20:qmSolidsL20Fig2}{0.15}

Full band has \(f(E, \BE) = F(E, 0)\), so that there is no current

An \dquoteAndIndex{insulator} has all bands either completely full or completely empty.

A \dquoteAndIndex{metal} is a solid with a Fermi surface (partly filled band(s)).

The \(\bcE\) field displaces the Fermi surface, but scattering restores equilibrium, limiting \(\Bj\).

Note that a periodic lattice does \textunderline{not} cause scattering, it causes \textunderline{band structure}.

Scattering is due to \textunderline{departures} from periodicity, and is due to impurities and/or vacancies and lattice vibrations (phonons), as sketched in \cref{fig:qmSolidsL20:qmSolidsL20Fig3}.

\imageFigure{../figures/phy487-qmsolids/qmSolidsL20Fig3}{Vacancy}{fig:qmSolidsL20:qmSolidsL20Fig3}{0.1}

A metric for this is called the resistivity, a temperature dependent effect as sketched in \cref{fig:qmSolidsL20:qmSolidsL20Fig4}.

\imageFigure{../figures/phy487-qmsolids/qmSolidsL20Fig4}{Resistivity}{fig:qmSolidsL20:qmSolidsL20Fig4}{0.15}

\paragraph{Non-equilibrium \(f(E, \bcE)\)}

With imposition of a field the Fermi filling is shifted as sketched in \cref{fig:qmSolidsL20:qmSolidsL20Fig5}.

\imageFigure{../figures/phy487-qmsolids/qmSolidsL20Fig5}{Fermi filling shift due to imposed field}{fig:qmSolidsL20:qmSolidsL20Fig5}{0.2}

Electrons in the range (1) cannot scatter due to the Pauli exclusion principle, whereas those in the energy range (2) \textunderline{can} scatter from \(+k\) to \(-k\).  The net scattering is from \(+\Bv\) to \(-\Bv\).

For 3D see \citep{ibach2009solid} \textfigref{9.5}, roughly as in \cref{fig:qmSolidsL20:qmSolidsL20Fig6}.

\imageFigure{../figures/phy487-qmsolids/qmSolidsL20Fig6}{A displaced Fermi sphere}{fig:qmSolidsL20:qmSolidsL20Fig6}{0.2}

\paragraph{Steady state} rate of scattering from \(+\Bv\) to \(-\Bv\) is the rate at which new \(+\Bv\) carriers appear, where

\begin{dmath}\label{eqn:condensedMatterLecture20:20}
\Hbar \dot{\Bk} = - e \bcE.
\end{dmath}

Introduce \(\tau\) as the mean scattering time, and let

\begin{equation}\label{eqn:condensedMatterLecture20:40}
\Delta k = \dot{k} \tau = -\frac{e \calE}{\Hbar} \tau,
\end{equation}

the amount by which the Fermi surface is displaced.

%\cref{fig:qmSolidsL20:qmSolidsL20Fig7}.
\imageFigure{../figures/phy487-qmsolids/qmSolidsL20Fig7}{Integration region, difference between displaced Fermi sphere}{fig:qmSolidsL20:qmSolidsL20Fig7}{0.2}

In the shaded region, the electrons are inert, because the \(-\Bv\) and \(\Bv\) contributions cancel.  It's only the non-shaded portions of the overlapping spheres that we have to consider.

\begin{dmath}\label{eqn:condensedMatterLecture20:60}
j_x
= -\frac{e}{V} \sum_{k, \sigma} v_x(k, \sigma)
= -\frac{e}{\cancel{V}}  \frac{2 \cancel{V} }{(2 \pi)^3}
\int
\mathLabelBox
{
\kF^2 \sin\theta d\theta d\phi
}
{
area element at \(\theta, \phi\)
}
\mathLabelBox
[
   labelstyle={below of=m\themathLableNode, below of=m\themathLableNode}
]
{
\Bigr( -\frac{e \calE}{\Hbar} \tau \cos\theta \Bigl)
}
{
\(\Delta k\) at \(\theta, \phi\)
}
\times
\mathLabelBox
{
\vF \cos\theta
}
{
\(v_x\)
}
\approx
\frac{e^2}{4 \pi^3} \tau \kF^2 \frac{\vF}{\Hbar}
\mathLabelBox
[
   labelstyle={xshift=2cm},
   linestyle={out=270,in=90, latex-}
]
{
\int_0^\pi \sin\theta \cos^2\theta d\theta
}
{$
-\int_0^\pi \cos^2\theta d(\cos\theta) = - \int_1^{-1} u^2 du = 2/3
$}
\mathLabelBox
[
   labelstyle={below of=m\themathLableNode, below of=m\themathLableNode}
]
{
2 \pi
}
{
\(\int d \phi\)
}
\calE_x
\end{dmath}

With

\begin{dmath}\label{eqn:condensedMatterLecture20:80}
\vF = \frac{\Hbar \kF}{m^\conj},
\end{dmath}

this is

\begin{dmath}\label{eqn:condensedMatterLecture20:100}
j_x \approx
\frac{e^2\tau}{m^\conj} \frac{\kF^3}{3 \pi^2} \calE_x
\end{dmath}

\begin{dmath}\label{eqn:condensedMatterLecture20:120}
j_x = \frac{n e^2 \tau}{m^\conj} \calE_x
\end{dmath}

or

\boxedEquation{eqn:condensedMatterLecture20:140}{
\sigma = \frac{n e^2 \tau}{m^\conj}
}

This is the \textAndIndex{Drude formula for conductivity}.

\paragraph{On Drude's derivation}

Note that Drude's derivation, see \citep{ashcroft1976solid} \textchapref{1}, predated quantum mechanics.  He treated the electrons classically introducing a drift velocity \cref{fig:qmSolidsL20:qmSolidsL20Fig8}.

\imageFigure{../figures/phy487-qmsolids/qmSolidsL20Fig8}{Drude drift velocity in an electric field}{fig:qmSolidsL20:qmSolidsL20Fig8}{0.15}

where

\begin{dmath}\label{eqn:condensedMatterLecture20:160}
v_{\text{drift}} = \lr{ \text{acceleration} } \tau = -\frac{e \calE}{m} \tau,
\end{dmath}

so that

\begin{dmath}\label{eqn:condensedMatterLecture20:180}
\Bj = - e n v_{\text{drift}} \bcE = \frac{ n e^2 \tau }{m} \bcE.
\end{dmath}

%\EndArticle
