%
% Copyright � 2013 Peeter Joot.  All Rights Reserved.
% Licenced as described in the file LICENSE under the root directory of this GIT repository.
%
\makeoproblem{Thomas-Fermi screening.}{condensedMatter:problemSet7:1}{2013 ps7 p1}{
\makesubproblem{}{condensedMatter:problemSet7:1a}
For \(f(E) = 1/(e^{(E-\mu)/{\kB T}} + 1)\) (i.e. the Fermi-Dirac distribution function), show
that in the limit as \(T \rightarrow 0\) K, \(-\PDi{E}{f}\) has the following properties expected of the
Dirac delta-function:

\begin{itemize}
\item It is zero everywhere, except at \(E = \mu\) where it is infinite;
\item \(- \int_{-\infty}^\infty dE \PDi{E}{f} = 1\)
\end{itemize}
%
\paragraph{Note}: you may know that the \textAndIndex{Dirac delta function} is the derivative of the so-called
\textAndIndex{Heaviside function}, so the correspondence between \(\PDi{E}{f}\) and \(\delta(E -\mu)\) is not a
surprise.
\makesubproblem{}{condensedMatter:problemSet7:1b}
Consider an externally applied periodic charge density wave of the form \(\delta \rho_\nought(x) = \delta \rho_\nought \cos( q x)\), inside a metal. In practice this could be a modulation of the ionic charge density due to a static or dynamic charge density wave.

Show, using a result we derived in class
%
\begin{equation}\label{eqn:condensedMatterProblemSet7Problem1:280}
\delta \rho_{\mathrm{el}}(\Br) = - e^2 D(\EF) U(\Br),
\end{equation}
%
that the induced electron density due to this applied periodic charge density wave is
%
\begin{equation}\label{eqn:condensedMatterProblemSet2Problem1:20}
\delta \rho_{\mathrm{el}}(x) =
-\frac{e^2 D(\EF)}{\epsilon_\nought}
\frac{\delta \rho_\nought}{q^2 + \kappa^2}
\cos(q x),
\end{equation}
%
where \(\kappa^2 = e^2 D(\EF)/\epsilon_\nought\).
} % makeproblem
\makeanswer{condensedMatter:problemSet7:1}{
\makeSubAnswer{}{condensedMatter:problemSet7:1a}
With \(\tau = \kB T\), the derivative is
%
\begin{equation}\label{eqn:condensedMatterProblemSet7Problem1:40}
\begin{aligned}
\PD{E}{f}
&=
-\inv{\tau} \frac
{
e^{(E - \mu)/\tau}
}
{
\lr{
e^{(E - \mu)/\tau}
+ 1
}^2
}
\\ &=
-\inv{\tau} \frac
{
1
}
{
\lr{ e^{(E - \mu)/\tau} + 1 }
\lr{ e^{-(E - \mu)/\tau} + 1 }
}.
\end{aligned}
\end{equation}
%
Consider this first for \(E > \mu\), where in the limit \(\tau \rightarrow 0\) we have
%
\begin{equation}\label{eqn:condensedMatterProblemSet7Problem1:60}
\begin{aligned}
\PD{E}{f}
&\approx
-\inv{\tau} \frac
{
1
}
{
\lr{ e^{(E - \mu)/\tau} + 1 }
}
\\ &=
-\inv{
2 \tau
+
(E - \mu)
+
\inv{2}
(E - \mu)^2/\tau
+
\inv{6}
(E - \mu)^3/\tau^2
+ \cdots
}.
\end{aligned}
\end{equation}
%
As \(\tau \rightarrow 0\), the denominator is dominated by the ever increasing powers of \(1/\tau\), and thus goes to zero.

Similarly for \(E < \mu\) and \(\tau \rightarrow 0\) we have
%
\begin{equation}\label{eqn:condensedMatterProblemSet7Problem1:80}
\begin{aligned}
\PD{E}{f}
&\approx
-\inv{\tau} \frac
{
1
}
{
\lr{ e^{(\mu - E)/\tau} + 1 }
}
\\ &=
-\inv{
2 \tau
+
(\mu - E)
+
\inv{2}
(\mu - E)^2/\tau
+
\inv{6}
(\mu - E)^3/\tau^2
+ \cdots
}.
\end{aligned}
\end{equation}
%
This has the same form as \eqnref{eqn:condensedMatterProblemSet7Problem1:60}, so clearly also approaches zero as \(\tau \rightarrow 0\).

For the \(E = \mu\) condition we have
%
\begin{equation}\label{eqn:condensedMatterProblemSet7Problem1:100}
\PD{E}{f} = -\inv{4 \tau}
%\xrightarrow[\tau \rightarrow 0]{}
\xrightarrow{\tau \rightarrow 0}
-\infty.
\end{equation}
%
Now consider the integral of the derivative, taking some care in the neighborhood \(\Abs{E - \mu} < \epsilon\).  We have
%
\begin{equation}\label{eqn:condensedMatterProblemSet7Problem1:120}
\begin{aligned}
&-\int_{-\infty}^\infty dE \PD{E}{f} \\
&=
\int_{-\infty}^\infty \frac{dE}{\tau}
\frac
{
e^{(E - \mu)/\tau}
}
{
\lr{
e^{(E - \mu)/\tau}
+ 1
}^2
}
\\ &=
\int_{-\infty}^{\mu - \epsilon} \frac{dE}{\tau}
\frac
{
e^{(E - \mu)/\tau}
}
{
\lr{
e^{(E - \mu)/\tau}
+ 1
}^2
}
+
\int_{\mu - \epsilon}^{\mu + \epsilon} \frac{dE}{\tau}
\frac
{
e^{(E - \mu)/\tau}
}
{
\lr{
e^{(E - \mu)/\tau}
+ 1
}^2
} \\
&\qquad +
\int_{\mu + \epsilon}^\infty \frac{dE}{\tau}
\frac
{
e^{(E - \mu)/\tau}
}
{
\lr{
e^{(E - \mu)/\tau}
+ 1
}^2
}.
\end{aligned}
\end{equation}
%
With \(x = (E - \mu)/\tau\) and \(dx = dE/\tau\), we have
%
\begin{equation}\label{eqn:condensedMatterProblemSet7Problem1:140}
\begin{aligned}
-\int_{-\infty}^\infty &dE \PD{E}{f} 
=
\int_{-\infty}^{- \epsilon/\tau} dx
\frac
{
e^x
}
{
\lr{
e^x
+ 1
}^2
}
+
\int_{\epsilon/\tau}^\infty dx
\frac
{
e^x
}
{
\lr{
e^x
+ 1
}^2
}
+
\int_{-\epsilon/\tau}^{\epsilon/\tau} dx
\frac
{
e^x
}
{
\lr{
e^x
+ 1
}^2
} \\
&=
\evalrange
{
- \frac{1}{e^x + 1}
}
{-\infty}{- \epsilon/\tau}
+
\evalrange
{
- \frac{1}{e^x + 1}
}
{-\epsilon/\tau}
{\epsilon/\tau}
+
\evalrange
{
- \frac{1}{e^x + 1}
}
{\epsilon/\tau}
{\infty} \\
&=
\lr{
\frac{1}{e^{-\infty} + 1}
-\frac{1}{e^{-\epsilon/\tau} + 1}
}
+
\lr{
\frac{1}{e^{-\epsilon/\tau} + 1}
-\frac{1}{e^{\epsilon/\tau} + 1}
} \\
&\qquad +
\lr{
\frac{1}{e^{\epsilon/\tau} + 1}
-\frac{1}{e^{\infty} + 1}
} \\
&=
\lr{1 - 1}
+\lr{1 - 0}
+\lr{0 - 0}.
\end{aligned}
\end{equation}
%
Here we (rather loosely) consider \(\epsilon\) fixed, and allow \(\tau \rightarrow 0\) for that choice of \(\epsilon\).  This leaves only the integral in the neighborhood of \(\mu\), which we find is unity as expected.
\makeSubAnswer{}{condensedMatter:problemSet7:1b}
We used the result in class in the Coulomb calculation
%
\begin{equation}\label{eqn:condensedMatterProblemSet7Problem1:160}
\spacegrad^2 U = - \frac{\rho}{\epsilon_\nought}.
\end{equation}
%
Summing the internal and the external charge densities, we wish to find
%
\begin{equation}\label{eqn:condensedMatterProblemSet7Problem1:180}
\rho = \delta \rho_\nought(x) + \delta \rho_{\mathrm{el}}
=
\delta \rho_\nought \cos(q x) - e^2 D(\EF) U,
\end{equation}
%
or
%
\begin{equation}\label{eqn:condensedMatterProblemSet7Problem1:200}
\spacegrad^2 U =
-\frac{\delta \rho_\nought}{\epsilon_\nought} \cos(q x) + \frac{e^2 D(\EF)}{\epsilon_\nought} U.
\end{equation}
%
Fourier transforming, assuming the potential has only an \(x\) dependence \(U = U(x)\), we have
%
\begin{equation}\label{eqn:condensedMatterProblemSet7Problem1:220}
\begin{aligned}
\int dx
e^{i k x}
\frac{d^2 U(x)}{dx^2}
&=
-\frac{\delta \rho_\nought}{\epsilon_\nought}
\int dx e^{i k x}
\cos(q x)
+
\kappa^2
\int dx e^{i k x} U(x)
\\ &=
-\frac{\delta \rho_\nought}{\epsilon_\nought}
\inv{2} \int dx
\lr{
e^{i (k + q) x}
+ e^{i (k - q) x}
}
+
\kappa^2 \tilde{U}(k)
\\ &=
-\frac{\pi \delta \rho_\nought}{\epsilon_\nought}
\lr{
\delta(k + q)
+\delta(k - q)
}
+
\kappa^2 \tilde{U}(k).
\end{aligned}
\end{equation}
%
Integrating the LHS twice by parts, and rearranging, this is
%
\begin{equation}\label{eqn:condensedMatterProblemSet7Problem1:240}
\tilde{U}(k) =
-\frac{\pi \delta \rho_\nought}{\epsilon_\nought
\lr{k^2 + \kappa^2}
}
\lr{
\delta(k + q)
+\delta(k - q)
}.
\end{equation}
%
A final inverse transform yields the real space potential
\begin{equation}\label{eqn:condensedMatterProblemSet7Problem1:260}
\begin{aligned}
U(x)
&=
\inv{2 \pi} \int dk e^{-i k x} \tilde{U}(k)
\\ &=
-\frac{\delta \rho_\nought}{2 \epsilon_\nought}
\lr{
\frac
{
e^{-i k q}
}
{
\lr{q^2 + \kappa^2}
}
+
\frac
{
e^{-i k q}
}
{
\lr{(-q)^2 + \kappa^2}
}
}
\\ &=
-\frac{\delta \rho_\nought \cos(q x)}{\epsilon_\nought \lr{ q^2 + \kappa^2} }.
\end{aligned}
\end{equation}
%
Referring back to \eqnref{eqn:condensedMatterProblemSet7Problem1:280}, the induced electron density is
\boxedEquation{eqn:condensedMatterProblemSet7Problem1:300}{
\delta \rho_{\mathrm{el}} = - \epsilon_\nought \kappa^2 U =
-\frac{\kappa^2 \delta \rho_\nought \cos(q x)}{q^2 + \kappa^2},
}
as desired.
}
