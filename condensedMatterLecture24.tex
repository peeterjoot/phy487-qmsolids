%
% Copyright � 2013 Peeter Joot.  All Rights Reserved.
% Licenced as described in the file LICENSE under the root directory of this GIT repository.
%
%\input{../blogpost.tex}
%\renewcommand{\basename}{condensedMatterLecture24}
%\renewcommand{\dirname}{notes/phy487/}
%\newcommand{\keywords}{Condensed matter physics, PHY487H1F}
%\input{../peeter_prologue_print2.tex}
%
%%\citep{harald2003solid} \S x.y
%%\citep{ibach2009solid} \S x.y
%
%%\usepackage{mhchem}
%\usepackage[version=3]{mhchem}
%\usepackage{units}
%\usepackage{bm} % \bcE
%\newcommand{\nought}[0]{\circ}
%%\newcommand{\EF}[0]{\epsilon_{\txtF}}
%\newcommand{\EF}[0]{E_{\txtF}}
%\newcommand{\kF}[0]{k_{\txtF}}
%
%\beginArtNoToc
%\generatetitle{PHY487H1F Condensed Matter Physics.  Lecture 24: Superconductivity (cont.).  Taught by Prof.\ Stephen Julian}
%\chapter{Superconductivity (cont.)}
%\label{chap:condensedMatterLecture24}
%
%\section{Disclaimer}
%
%Peeter's lecture notes from class.  May not be entirely coherent.
%
\paragraph{Cooper pairing (cont.)}
\index{Cooper pairing}

\reading \citep{ibach2009solid} \S 10.2

%\cref{fig:qmSolidsL24:qmSolidsL24Fig1}.
\imageFigure{../figures/phy487-qmsolids/qmSolidsL24Fig1}{Electron-phonon interaction}{fig:qmSolidsL24:qmSolidsL24Fig1}{0.2}

Electron-phonon interaction
%
\begin{dmath}\label{eqn:condensedMatterPhysicsLecture24:20}
\ket{\Bk, \uparrow} \rightarrow
\ket{\Bk, \uparrow, 0_q}
+ \alpha_q \ket{\Bk, \uparrow, 1_q}
\end{dmath}
%
%\cref{fig:qmSolidsL24:qmSolidsL24Fig2}.
%\imageFigure{../figures/phy487-qmsolids/qmSolidsL24Fig2}{No idea what this figure was about}{fig:qmSolidsL24:qmSolidsL24Fig2}{0.2}
%
\begin{dmath}\label{eqn:condensedMatterPhysicsLecture24:40}
\ket{\Bk', \downarrow} \rightarrow
\ket{\Bk' -\Bq, \downarrow, 0_q}
+ \alpha_{-q} \ket{\Bk -\Bq, \downarrow, 1_q}
\end{dmath}
%
Using perturbation theory, you set an effective interaction
%
\begin{dmath}\label{eqn:condensedMatterPhysicsLecture24:60}
V_{\mathrm{eff}} = \frac{
\Abs{V_{\mathrm{e,ph}}}^2
}
{
\lr{ \calE_{\Bk+\Bq} - \calE_\Bk }^2 - \Hbar^2 \omega_\Bq^2
}
\end{dmath}
%
%\cref{fig:qmSolidsL24:qmSolidsL24Fig4}.
%\imageFigure{../figures/phy487-qmsolids/qmSolidsL24Fig4}{What was this}{fig:qmSolidsL24:qmSolidsL24Fig4}{0.2}
%\cref{fig:qmSolidsL24:qmSolidsL24Fig5}.
\imageFigure{../figures/phy487-qmsolids/qmSolidsL24Fig5}{Limited options for paired states available}{fig:qmSolidsL24:qmSolidsL24Fig5}{0.2}
%\cref{fig:qmSolidsL24:qmSolidsL24Fig6}.
\imageFigure{../figures/phy487-qmsolids/qmSolidsL24Fig6}{Cooper filled Fermi sphere plus two electrons}{fig:qmSolidsL24:qmSolidsL24Fig6}{0.2}
%
\begin{dmath}\label{eqn:condensedMatterPhysicsLecture24:80}
\ket{\Psi} = \sum_k g(\Bk) e^{ i \Bk \cdot (\Br_1 - \Br_2 ) }
\end{dmath}
%
\begin{dmath}\label{eqn:condensedMatterPhysicsLecture24:100}
\lr{
-\frac{\Hbar^2}{2m} \lr{
\spacegrad_1^2 +
\spacegrad_2^2
}
+ V(\Br_1, \Br_2)
}
\sum_k g(\Bk) e^{ i \Bk \cdot (\Br_1 - \Br_2 ) }
=
\mathLabelBox
{
(2 \EF + \calE)
}
{
\(E \ket{\Psi}\)
}
\sum_k g(\Bk) e^{ i \Bk \cdot (\Br_1 - \Br_2 ) }
\end{dmath}
%
Operating with \(\int d\Br_1 d\Br_2 e^{-i \Bk' \cdot (\Br_1 - \Br_2) }\), we have
%
\begin{dmath}\label{eqn:condensedMatterPhysicsLecture24:120}
\int d\Br_1 d\Br_2 e^{-i \Bk' \cdot (\Br_1 - \Br_2) }
\lr{
\bcE_\Bk +
\bcE_{-\Bk}
+ V
}
\sum_k g(\Bk) e^{ i \Bk \cdot (\Br_1 - \Br_2 ) }
=
\int d\Br_1 d\Br_2 e^{-i \Bk' \cdot (\Br_1 - \Br_2) }
(2 \EF + \calE)
\sum_k g(\Bk) e^{ i \Bk \cdot (\Br_1 - \Br_2 ) }
\end{dmath}
%
This is
%
\begin{dmath}\label{eqn:condensedMatterPhysicsLecture24:140}
2 \bcE_{\Bk'} g(\Bk') + \sum_\Bk V_{\Bk'\Bk} g(\Bk)
= \lr{ 2 \EF + \calE } g(\Bk'),
\end{dmath}
%
where
\begin{dmath}\label{eqn:condensedMatterPhysicsLecture24:160}
V_{\Bk, \Bk'} =
\int d\Br_1 d\Br_2
e^{-i \Bk' \cdot (\Br_1 - \Br_2) }
V(\Br_1, \Br_2)
e^{i \Bk \cdot (\Br_1 - \Br_2) }.
\end{dmath}
%
Make the approximation
%
\begin{dmath}\label{eqn:condensedMatterPhysicsLecture24:180}
V_{\Bk, \Bk'} =
\left\{
\begin{array}{l l}
V_\nought & \quad \mbox{if \(\calE_\Bk,\calE_{\Bk'}\) between \(\EF\) and \(\EF + \Hbar \Omega_D\)} \\
0 & \quad \mbox{otherwise}
\end{array}
\right.
\end{dmath}
%
This has solution
%
\begin{dmath}\label{eqn:condensedMatterPhysicsLecture24:200}
g(\Bk') = \inv{ 2 \EF + \calE - 2 \calE_{\Bk'}} \sum_\Bk V_{\Bk', \Bk} g(\Bk)
\end{dmath}
%
Sum this over \(\Bk'\)
%
\begin{dmath}\label{eqn:condensedMatterPhysicsLecture24:220}
\cancel{\sum_{\Bk'} g(\Bk')} = \sum_{\Bk'}
\frac{V_\nought}{ 2 \EF + \calE - 2 \calE_{\Bk'}} \cancel{\sum_\Bk g(\Bk)}
\end{dmath}
%
\begin{dmath}\label{eqn:condensedMatterPhysicsLecture24:240}
1 = V_\nought \int_\EF^{\EF + \Hbar \omega_{\txtD}}
\frac{D(E) dE}{
2 \EF + \calE - 2 E
}
\approx
V_\nought D(\EF) \inv{2}
\evalrange{\ln \lr{ 2 \EF + \calE - 2 E}}{\EF}{\EF + \Hbar \omega_{\txtD}}
\approx
\frac{V_\nought D(\EF) }{2}
\ln
\frac{ 2 \EF + \calE - 2 \EF}
{ 2 \EF + \calE - 2 (\EF + \Hbar \omega_{\txtD}) }
\end{dmath}
%
Giving
%
\begin{dmath}\label{eqn:condensedMatterPhysicsLecture24:260}
e^{ 2/V_\nought D(\EF) } = \frac{\calE - 2 \Hbar \omega_{\txtD}}{\calE}
\end{dmath}
%
or
%
\begin{dmath}\label{eqn:condensedMatterPhysicsLecture24:280}
\calE = \frac{ -2 \Hbar \omega_{\txtD} }{e^{2/V_\nought D(\EF)} - 1}
\end{dmath}
%
\boxedEquation{eqn:condensedMatterPhysicsLecture24:300}{
\calE \approx -2 \Hbar \omega_{\txtD} e^{-2/V_\nought D(\EF)}
}

Bound state Fermi surface \((2 \EF + \calE) < 2 \EF\).

A filled Fermi sphere at \(T = 0\) is unstable against Fermiation of Cooper pairs.
%
\section{BCS theory}
\index{BCS theory}

\reading \citep{ibach2009solid} \S 10.4

Difficult to make a multiple Cooper pair state.

BCS guess was:
%
\begin{dmath}\label{eqn:condensedMatterPhysicsLecture24:320}
\ket{0_{\mathrm{BCS}}}
=
\Pi_\Bk \lr{
u_\Bk \ket{0_{\Bk \uparrow}, 0_{-\Bk \downarrow}}
+
v_\Bk \ket{1_{\Bk \uparrow}, 1_{-\Bk \downarrow}}
}
\end{dmath}
%
At every: states are either empty (prob \(u_\Bk^2\)) or filled (prob \(v_\Bk^2\)) in pairs.

The energy
%
\begin{dmath}\label{eqn:condensedMatterPhysicsLecture24:340}
W
= \sum_\Bk 2 \calE_\Bk v_\Bk^2
+
\mathLabelBox
{
\sum_{\Bk, \Bk'} V_{\Bk, \Bk'}
u_{\Bk'}
v_{\Bk'}
u_{\Bk}
v_{\Bk}
}
{
Need a superposition of occupied and unoccupied
}
\end{dmath}
%
%\EndArticle
