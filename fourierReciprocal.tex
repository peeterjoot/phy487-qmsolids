%
% Copyright � 2013 Peeter Joot.  All Rights Reserved.
% Licenced as described in the file LICENSE under the root directory of this GIT repository.
%
%\input{../blogpost.tex}
%\renewcommand{\basename}{fourierReciprocal}
%\renewcommand{\dirname}{notes/phy487/}
%%\newcommand{\dateintitle}{}
%\newcommand{\keywords}{PHY487, Fourier coefficient, reciprocal basis, Fourier series, unit cell}
%
%\input{../peeter_prologue_print2.tex}
%
%\beginArtNoToc
%
%\generatetitle{Fourier coefficient integral for periodic function}
\label{chap:fourierReciprocal}

We've been using the fact that a periodic function
%
\begin{equation}\label{eqn:fourierReciprocal:20}
V(\Br) = V(\Br + \Br_n),
\end{equation}
%
where
%
\begin{equation}\label{eqn:fourierReciprocal:40}
\Br_n = a_1 \Ba_1 + a_2 \Ba_2 + a_3 \Ba_3,
\end{equation}
%
has a Fourier representation
%
\begin{equation}\label{eqn:fourierReciprocal:60}
V(\Br) = \sum_\BG V_\BG e^{ i \BG \cdot \Br }.
\end{equation}
%
Here \(\BG\) is a vector in reciprocal space, say
%
\begin{equation}\label{eqn:fourierReciprocal:80}
\BG_{rst} = r \Bg_1 + s \Bg_2 + t \Bg_3,
\end{equation}
%
where
%
\begin{equation}\label{eqn:fourierReciprocal:100}
\Bg_i \cdot \Ba_j = 2 \pi \delta_{ij}.
\end{equation}
%
Now let's express the explicit form for the Fourier coefficient \(V_\BG\) so that we can compute the Fourier representation for some periodic potentials for some numerical experimentation.  In particular, let's think about what it meant to integrate over a unit cell.  Suppose we have a parameterization of the points in the unit cell
%
\begin{equation}\label{eqn:fourierReciprocal:120}
\Br = u \Ba_1 + v \Ba_2 + w \Ba_3,
\end{equation}
%
as sketched in \cref{fig:fourierReciprocalUnitCell:fourierReciprocalUnitCellFig1}.  Here \(u, v, w \in [0, 1]\).  We can compute the values of \(u, v, w\) for any vector \(\Br\) in the cell by reciprocal projection
%
\imageFigure{../figures/phy487-qmsolids/fourierReciprocalUnitCellFig1}{Unit cell.}{fig:fourierReciprocalUnitCell:fourierReciprocalUnitCellFig1}{0.3}
%
\begin{equation}\label{eqn:fourierReciprocal:140}
\Br = \inv{2 \pi} \lr{
\lr{ \Br \cdot \Bg_1} \Ba_1
+ \lr{ \Br \cdot \Bg_2} \Ba_2
+ \lr{ \Br \cdot \Bg_3} \Ba_3
},
\end{equation}
%
or
\begin{equation}\label{eqn:fourierReciprocal:160}
\begin{aligned}
u(\Br) &= \inv{2 \pi} \Br \cdot \Bg_1 \\
v(\Br) &= \inv{2 \pi} \Br \cdot \Bg_2 \\
w(\Br) &= \inv{2 \pi} \Br \cdot \Bg_3.
\end{aligned}
\end{equation}
%
Let's suppose that \(\BV(\Br)\) is period in the unit cell spanned by \(\Br = u \Ba_1 + v \Ba_2 + w \Ba_3\) with \(u, v, w \in [0, 1]\), and integrate over the unit cube for that parameterization to compute \(V_\BG\)
%
\begin{equation}\label{eqn:fourierReciprocal:180}
\int_0^1 du
\int_0^1 dv
\int_0^1 dw
V( u \Ba_1 + v \Ba_2 + w \Ba_3 )
e^{-i \BG' \cdot \Br }
=
\sum_{r s t}
V_{\BG_{r s t}}
\int_0^1 du
\int_0^1 dv
\int_0^1 dw
e^{-i \BG' \cdot \Br }
e^{i \BG \cdot \Br }.
\end{equation}
%
Let's write
\begin{equation}\label{eqn:fourierReciprocal:200}
\begin{aligned}
\BG &=
r \Bg_1
+ s \Bg_2
+ t \Bg_3 \\
\BG &=
r' \Bg_1
+ s' \Bg_2
+ t' \Bg_3,
\end{aligned}
\end{equation}
%
so that
%
\begin{equation}\label{eqn:fourierReciprocal:220}
e^{-i \BG' \cdot \Br } e^{i \BG \cdot \Br }
=
e^{ 2 \pi i (r - r') u }
e^{ 2 \pi i (s - s') u }
e^{ 2 \pi i (t - t') u }.
\end{equation}
%
Picking the \(u\) integral of this integrand as representative, we have when \(r = r'\)
%
\begin{equation}\label{eqn:fourierReciprocal:240}
\begin{aligned}
\int_0^1 du e^{ 2 \pi i (r - r') u }
&=
\int_0^1 du
\\& = 1,
\end{aligned}
\end{equation}
%
and when \(r \ne r'\)
%
\begin{equation}\label{eqn:fourierReciprocal:260}
\begin{aligned}
\int_0^1 du e^{ 2 \pi i (r - r') u }
&=
\evalrange{
   \frac{
   e^{ 2 \pi i (r - r') u }
   }
   {
   2 \pi i (r - r')
   }
}{u = 0}{1}
\\ &=
\inv{2 \pi i (r - r') } \lr{
   e^{ 2 \pi i (r - r') } - 1
}.
\end{aligned}
\end{equation}
%
This is just zero since \(r - r'\) is an integer, so we have
%
\begin{equation}\label{eqn:fourierReciprocal:280}
\int_0^1 du e^{ 2 \pi i (r - r') u } = \delta_{r, r'}.
\end{equation}
%
This gives us
%
\begin{equation}\label{eqn:fourierReciprocal:300}
\begin{aligned}
&\int_0^1 du
\int_0^1 dv
\int_0^1 dw
V( u \Ba_1 + v \Ba_2 + w \Ba_3 )
e^{ -2 \pi i r' u }
e^{ -2 \pi i s' v }
e^{ -2 \pi i t' w } \\
&=
\sum_{r s t}
V_{\BG_{r s t}}
\delta_{r s t, r' s' t'}
\\ &=
V_{\BG_{r' s' t'}}.
\end{aligned}
\end{equation}
%
This is our \textAndIndex{Fourier coefficient}.  The \textAndIndex{Fourier series} written out in gory but explicit detail is
%
\begin{equation}\label{eqn:fourierReciprocal:320}
\begin{aligned}
&V( u \Ba_1 + v \Ba_2 + w \Ba_3 ) \\
&= \sum_{r s t}
\lr{
\int_0^1 du'
\int_0^1 dv'
\int_0^1 dw'
V( u' \Ba_1 + v' \Ba_2 + w' \Ba_3 )
e^{ -2 \pi i (r u' + s v' + t w') }
}
e^{ 2 \pi i (r u + s v + t w) }.
\end{aligned}
\end{equation}
%
Also observe the unfortunate detail that we require integrability of the potential in the unit cell for the Fourier integrals to converge.  This prohibits the use of the most obvious potential for numerical experimentation, the inverse radial \(V(\Br) = -1/\Abs{\Br}\).

It was temporarily useful to expand this in terms of \(\Br = u \Ba_1 + v \Ba_2 + w \Ba_3\), but with the Fourier coefficients computed, let's put things back into vector form noting that
%
\begin{equation}\label{eqn:fourierReciprocal:319}
\begin{aligned}
2 \pi &\lr{ r u + s v + t w } 
=
2 \pi
\lr{
\inv{2 \pi} (\Br \cdot \Bg_1) r
+\inv{2 \pi} (\Br \cdot \Bg_1) s
+\inv{2 \pi} (\Br \cdot \Bg_1) t
} \\
&\qquad =
\BG_{r s t} \cdot \Br.
\end{aligned}
\end{equation}
%
So, to summarize
\boxedEquation{eqn:fourierReciprocal:320b}{
\begin{aligned}
V_{r s t}
&=
\int_0^1 du'
\int_0^1 dv'
\int_0^1 dw'
V( u' \Ba_1 + v' \Ba_2 + w' \Ba_3 )
e^{ -2 \pi i (r u' + s v' + t w') }
\\
V( \Br ) &= \sum_{r s t} V_{r s t} e^{ \BG_{r s t} \cdot \Br }.
\end{aligned}
}
%\EndNoBibArticle
