%
% Copyright � 2013 Peeter Joot.  All Rights Reserved.
% Licenced as described in the file LICENSE under the root directory of this GIT repository.
%
\makeoproblem{Fcc reciprocal lattice}{condensedMatter:problemSet3:2}{2013 ps3 p2}{
Now for a face-centered cubic lattice with conventional unit cell
  of side length \(a\):
\makesubproblem{}{condensedMatter:problemSet3:2a}
  Draw the conventional unit cell and number
  all of the corner and face-centered atoms, and demonstrate that the vectors
  \(\Ba_1 = \frac{a}{2}(1,1,0)\),
  \(\Ba_2 = \frac{a}{2}(1,0,1)\) and
  \(\Ba_3 = \frac{a}{2}(0,1,1)\), are primitive lattice vectors
  in the sense that you can get to every lattice point in the unit
  cell using these vectors.
\makesubproblem{}{condensedMatter:problemSet3:2b}
Using the formula from the lectures show that the volume of the
  primitive unit cell is 1/4 of the volume of the conventional unit cell.
\makesubproblem{}{condensedMatter:problemSet3:2c}
Using the formula from the lectures,  find the basis vectors of
  the corresponding reciprocal lattice,
  and show that these basis vectors generate a body-centered-cubic
  lattice in reciprocal space.
} % makeproblem

\makeanswer{condensedMatter:problemSet3:2}{
\makeSubAnswer{}{condensedMatter:problemSet3:2a}
The lattice and the primitive lattice vectors are sketched in \cref{fig:qmSolidsPs3P2a:qmSolidsPs3P2aFig1}.
\imageFigure{../figures/phy487-qmsolids/qmSolidsPs3P2aFig1}{Fcc lattice and primitive lattice vectors.}{fig:qmSolidsPs3P2a:qmSolidsPs3P2aFig1}{0.3}

To demonstrate that we can get to each lattice point, we first invert a matrix
of the lattice vectors
%
\begin{dmath}\label{eqn:condensedMatterProblemSet3Problem2:180}
{
\begin{bmatrix}
1 & 1 & 0 \\
1 & 0 & 1 \\
0 & 1 & 0
\end{bmatrix}
}^{-1}
=
\inv{2}
\begin{bmatrix}
1 & 1 & -1 \\
1 & -1 & 1 \\
-1 & 1 & 1
\end{bmatrix}.
\end{dmath}
%
This allows us to read off the corners of the cube in terms of the primitive lattice vectors (which perhaps could have been done by inspection)
%
\begin{subequations}
\begin{dmath}\label{eqn:condensedMatterProblemSet3Problem2:200}
\Ba_1 + \Ba_2 - \Ba_3
=
\begin{bmatrix}
a \\
0 \\
0
\end{bmatrix}
\end{dmath}
\begin{dmath}\label{eqn:condensedMatterProblemSet3Problem2:220}
\Ba_1 + \Ba_3 - \Ba_2
=
\begin{bmatrix}
0 \\
a \\
0
\end{bmatrix}
\end{dmath}
\begin{dmath}\label{eqn:condensedMatterProblemSet3Problem2:240}
\Ba_2 + \Ba_3 - \Ba_1
=
\begin{bmatrix}
0 \\
0 \\
a
\end{bmatrix}.
\end{dmath}
\end{subequations}
%
Our lattice points, in terms of the primitive vectors are
\begin{enumerate}
\item \(\Ba_1 + \Ba_2 - \Ba_3\)
\item \(2 \Ba_2\)
\item \(\Ba_1 + \Ba_2 + \Ba_3\)
\item \(2 \Ba_1\)
\item \(\Ba_1 + \Ba_2\)
\item 0
\item \(\Ba_1 + \Ba_3 + \Ba_1\)
\item \(2 \Ba_3\)
\item \(\Ba_1 + \Ba_3 - \Ba_2\)
\item \(\Ba_3\)
\item \(\Ba_2\)
\item \(\Ba_1 + \Ba_3\)
\item \(\Ba_2 + \Ba_3\)
\item \(\Ba_1\)
\end{enumerate}

\makeSubAnswer{}{condensedMatter:problemSet3:2c}
Computing the three sets of cross products we have
%
\begin{subequations}
\begin{equation}\label{eqn:condensedMatterProblemSet3Problem2:20}
\Ba_2 \cross \Ba_3 =
\lr{\frac{a}{2}}^2
\begin{vmatrix}
\xcap & \ycap & \zcap \\
1 & 0 & 1 \\
0 & 1 & 1
\end{vmatrix}
=
\lr{\frac{a}{2}}^2
\begin{bmatrix}
-1 \\
-1 \\
1
\end{bmatrix}
\end{equation}
\begin{equation}\label{eqn:condensedMatterProblemSet3Problem2:40}
\Ba_3 \cross \Ba_1 =
\lr{\frac{a}{2}}^2
\begin{vmatrix}
\xcap & \ycap & \zcap \\
0 & 1 & 1 \\
1 & 1 & 0
\end{vmatrix}
=
\lr{\frac{a}{2}}^2
\begin{bmatrix}
-1 \\
1 \\
-1
\end{bmatrix}
\end{equation}
\begin{equation}\label{eqn:condensedMatterProblemSet3Problem2:60}
\Ba_1 \cross \Ba_2 =
\lr{\frac{a}{2}}^2
\begin{vmatrix}
\xcap & \ycap & \zcap \\
1 & 1 & 0 \\
1 & 0 & 1
\end{vmatrix}
=
\lr{\frac{a}{2}}^2
\begin{bmatrix}
1 \\
-1 \\
-1
\end{bmatrix}.
\end{equation}
\end{subequations}
%
Our triplet product is
\begin{dmath}\label{eqn:condensedMatterProblemSet3Problem2:80}
\Ba_1 \cdot \lr{ \Ba_2 \cross \Ba_3 } =
\lr{\frac{a}{2}}^3 \lr{ -2 }.
\end{dmath}
%
Putting these together we have
\begin{subequations}
\label{eqn:condensedMatterProblemSet3Problem2:100a}
\begin{equation}\label{eqn:condensedMatterProblemSet3Problem2:100}
\Bg_1
= 2 \pi \frac{ \Ba_2 \cross \Ba_3 }{
\Ba_1 \cdot \lr{ \Ba_2 \cross \Ba_3 } }
=  \frac{2 \pi}{a}
\begin{bmatrix}
1 \\ 1 \\ -1
\end{bmatrix}
\end{equation}
\begin{equation}\label{eqn:condensedMatterProblemSet3Problem2:120}
\Bg_2
= 2 \pi \frac{ \Ba_3 \cross \Ba_1 }{
\Ba_1 \cdot \lr{ \Ba_2 \cross \Ba_3 } }
=  \frac{2 \pi}{a}
\begin{bmatrix}
1 \\ -1 \\ 1
\end{bmatrix}
\end{equation}
\begin{equation}\label{eqn:condensedMatterProblemSet3Problem2:140}
\Bg_3
= 2 \pi \frac{ \Ba_3 \cross \Ba_1 }{
\Ba_1 \cdot \lr{ \Ba_2 \cross \Ba_3 } }
=  \frac{2 \pi}{a}
\begin{bmatrix}
-1 \\ 1 \\ 1
\end{bmatrix}.
\end{equation}
\end{subequations}
%
Note that we can also compute all the reciprocal basis vectors more directly by inversion
%
\begin{dmath}\label{eqn:condensedMatterProblemSet3Problem2:160}
\begin{bmatrix}
\Bg_1 & \Bg_2 & \Bg_3
\end{bmatrix}
=
2 \pi
\begin{bmatrix}
\Ba_1^\T \\
\Ba_2^\T \\
\Ba_3^\T
\end{bmatrix}
=
2 \pi
\frac{2}{a}
{
\begin{bmatrix}
1&1&0 \\
1&0&1 \\
0&1&1
\end{bmatrix}
}
^{-1}
=
\frac{2 \pi}{a}
\begin{bmatrix}
1 & 1 & -1 \\
1 & -1 & 1 \\
-1 & 1 & 1
\end{bmatrix},
\end{dmath}
%
consistent with the cross product calculation of \eqnref{eqn:condensedMatterProblemSet3Problem2:100a}.
\makeSubAnswer{}{condensedMatter:problemSet3:2b}
We've already computed the (signed) volume element in \eqnref{eqn:condensedMatterProblemSet3Problem2:80}.  The absolute value of this is the volume of the primitive unit cell
%
\begin{dmath}\label{eqn:condensedMatterProblemSet3Problem2:260}
V_{\mathrm{primitive}}
=
\Abs{
\Ba_1 \cdot \lr{ \Ba_2 \cross \Ba_3 }
}
=
\inv{4} a^3,
\end{dmath}
%
which is \(1/4\) of the conventional cell volume.

FIXME: Grading remark: ``BCC?''  Ooops.  Revisit: looks like I missed doing a portion of this question.
}
