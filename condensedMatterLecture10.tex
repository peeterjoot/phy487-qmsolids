%
% Copyright � 2013 Peeter Joot.  All Rights Reserved.
% Licenced as described in the file LICENSE under the root directory of this GIT repository.
%
%\input{../blogpost.tex}
%
%\renewcommand{\basename}{condensedMatterLecture10}
%\renewcommand{\dirname}{notes/phy487/}
%\newcommand{\keywords}{Condensed matter physics, PHY487H1F}
%\input{../peeter_prologue_print2.tex}
%
%%\citep{harald2003solid} \S x.y
%%\citep{ibach2009solid} \S x.y
%
%%\usepackage{mhchem}
%\usepackage[version=3]{mhchem}
%\newcommand{\nought}[0]{\circ}
%
%\beginArtNoToc
%\generatetitle{PHY487H1F Condensed Matter Physics.  Lecture 10: Thermal properties.  Taught by Prof.\ Stephen Julian}
%\chapter{Thermal properties}
\label{chap:condensedMatterLecture10}
%
%\section{Disclaimer}
%
%Peeter's lecture notes from class.  May not be entirely coherent.
%
\section{Thermal energy of a harmonic oscillator.}
\index{thermal energy}

\reading \citep{ashcroft1976solid} \textchapref{23}.

We talk about branches of the (LA, TA, ...) of the dispersion \(\omega(\Bq)\) at discrete \(\Bq's\).

The thermal occupancy is given by the \textAndIndex{Boltzman distribution}
%
\begin{dmath}\label{eqn:condensedMatterLecture10:20}
P_n =
\frac
{ e^{-E_n/\kB T} }
{ \sum_{n = 0}^\infty e^{-E_n/\kB T} },
\end{dmath}
%
where the denominator
%
\begin{dmath}\label{eqn:condensedMatterLecture10:40}
Z = \sum_{n = 0}^\infty e^{-E_n/\kB T},
\end{dmath}
%
is a normalization sum so that
%
\begin{dmath}\label{eqn:condensedMatterLecture10:60}
\sum_{n = 0}^\infty P_n = 1.
\end{dmath}
%
For the harmonic oscillator, we have
%
\begin{dmath}\label{eqn:condensedMatterLecture10:80}
E_n = \lr{ n + \inv{2} } \Hbar \omega,
\end{dmath}
%
so that
\begin{dmath}\label{eqn:condensedMatterLecture10:100}
Z
= \sum_{n = 0}^\infty e^{-\lr{ n + \inv{2}} \Hbar \omega /\kB T},
= e^{- \Hbar \omega /2 \kB T}
\sum_{n = 0}^\infty
\lr{ e^{- \Hbar \omega /\kB T} }^n
=
\frac
{ e^{ - \inv{2} \Hbar \omega / \kB T } }
{ 1 - e^{ - \Hbar \omega / \kB T } }.
\end{dmath}
%
The average thermal energy
%
\begin{dmath}\label{eqn:condensedMatterLecture10:120}
\calE(\omega_q T)
= \expectation{ E_n }
= \sum_n E_n P_n
= \frac
{ \sum_n E_n e^{-E_n/\kB T} }
{ \sum_n e^{-E_n/\kB T} }
= \inv{Z} \lr{ - \frac{d}{d(1/\kB T)} Z }
= - \frac{d}{d(1/\kB T)} \ln Z
= - \frac{d}{d(1/\kB T)}
\lr{
\ln \lr{ e^{ - \inv{2} \Hbar \omega / \kB T } }
-
\ln \lr{ 1- e^{ - \Hbar \omega / \kB T } }
}
= \frac{d}{d(1/\kB T)}
\lr{
\inv{2} \Hbar \omega / \kB T
+
\ln \lr{ 1- e^{ - \Hbar \omega / \kB T } }
}
=
\inv{2} \Hbar \omega + \inv{
1- e^{ - \Hbar \omega / \kB T }
}
\lr{ -e^{ - \Hbar \omega / \kB T } }
\lr{ -\Hbar \omega},
\end{dmath}
%
or
\boxedEquation{eqn:condensedMatterLecture10:140}{
\calE(\omega_q T)
=
\Hbar \omega \lr{ \inv{2 } +
\inv{
e^{ \Hbar \omega / \kB T } - 1}
}.
}

Here
%
\begin{dmath}\label{eqn:condensedMatterLecture10:440}
\inv{
e^{ \Hbar \omega / \kB T } - 1} = \expectation{n}_T,
\end{dmath}
%
the \textAndIndex{Bose distribution}.  In the \(\kB T \gg \Hbar \omega\) we have
%
\begin{dmath}\label{eqn:condensedMatterLecture10:160}
\calE(\omega_q T)
\approx
\frac{\Hbar \omega}{2} + \frac{\Hbar \omega}{ \cancel{1} + \Hbar \omega/\kB T - \cancel{1} }
\approx
\mathLabelBox
[
   labelstyle={below of=m\themathLableNode, below of=m\themathLableNode}
]
{
\cancel{\frac{\Hbar \omega}{2}}
}{small}
+ \kB T.
\end{dmath}
%
Our \(\Hbar\) is ``gone''.  This is the classical limit.  This is called the equipartition where we have \(\inv{2} \kB T\) of kinetic energy and \(\inv{2} \kB T\) of potential energy.

On the other hand in the \(\kB T \ll \Hbar \omega\) we have
%
\begin{dmath}\label{eqn:condensedMatterLecture10:180}
\calE(\omega_q T)
\approx
\frac{\Hbar \omega}{2} + \Hbar \omega e^{-\Hbar \omega/\kB T}.
\end{dmath}
%
These limits are plotted in \cref{fig:qmSolidsL10:qmSolidsL10Fig1}, and \cref{fig:qmSolidsL10:qmSolidsL10Fig2}.
\imageFigure{../figures/phy487-qmsolids/qmSolidsL10Fig1}{Average energy vs temperature.}{fig:qmSolidsL10:qmSolidsL10Fig1}{0.3}
\imageFigure{../figures/phy487-qmsolids/qmSolidsL10Fig2}{Average energy vs frequency.}{fig:qmSolidsL10:qmSolidsL10Fig2}{0.3}
%
\section{Lattice specific heat capacity.}
\index{specific heat}

\reading \citep{ashcroft1976solid} \textchapref{22,23}.

Define \(\CV(T)\) or \(\CP(T)\) \index{specific heat} as the change in the energy \(U(T)\) per unit change in \(T\).

\begin{itemize}
\item
\(\CV(T)\), is a measure at constant volume (easy to calculate).
\item
\(\CP(T)\), is a measure at constant pressure (easy to measure).
\end{itemize}
%
\begin{dmath}\label{eqn:condensedMatterLecture10:200}
\CV(T) = \PDc{T}{U}{V}.
\end{dmath}
%
\begin{dmath}\label{eqn:condensedMatterLecture10:220}
U(T) = \sum_q \calE(\omega_q, T) = \inv{V} \int_0^\infty Z(\omega) \calE(\omega_q, T) d\omega.
\end{dmath}
%
In the Debye model we found in \eqnref{eqn:condensedMatterLecture9:360} and \eqnref{eqn:condensedMatterLecture9:440} that
%
\begin{subequations}
\begin{dmath}\label{eqn:condensedMatterLecture10:460}
Z(\omega) d\omega
=
\frac{V}{2\pi^2}
\lr{
\frac{1}{C_{\txtL}^3} + \frac{2}{C_{\txtT}^3}
}
\omega^2 d \omega,
\end{dmath}
\begin{dmath}\label{eqn:condensedMatterLecture10:500}
\frac{V}{ 2 \pi^2} \lr{ \inv{C_{\txtL}^3} + \frac{ 2}{C_{\txtT}^3} } \omega_{\txtD}^3 = 9 r N.
\end{dmath}
\end{subequations}
%
so the specific heat for the Debye model is
%
\begin{dmath}\label{eqn:condensedMatterLecture10:240}
\CV(T)
=
\frac{d}{dT}
 \int_0^{\omega_{\txtD}}
\calE(\omega, T)
Z(\omega)
d\omega
= \inv{2 \pi^2} \lr{
\inv{C_{\txtL}^3} + \frac{2}{C_{\txtT}^3}
}
 \int_0^{\omega_{\txtD}} \omega^2 \frac{d}{dT} \calE(\omega, T) d\omega
=
\frac{ 9 r N}{V}
\inv{\omega_{\txtD}^3}
\frac{d}{dT}
\int_0^{\omega_{\txtD}}
{\Hbar \omega^3}
\lr{
\inv{2}
+
\inv
{
e^{\Hbar \omega/\kB T} - 1
}
}
d\omega.
\end{dmath}
%
where \(r\) is the number of atoms in the unit cell.  In the limit \(\kB T \gg \Hbar \omega_{\txtD}\), we have
%
\begin{dmath}\label{eqn:condensedMatterLecture10:480}
\CV(T)
=
\frac{ 9 r N}{V}
\inv{\omega_{\txtD}^3}
\frac{d}{dT}
\int_0^{\omega_{\txtD}}
\cancel{\Hbar} \omega^3 \frac{ \kB T}{ \cancel{\Hbar} \omega} d\omega
=
\frac{ 9 r N}{V}
\inv{\omega_{\txtD}^3}
\frac{d}{dT}
\inv{3} \kB T \omega_{\txtD}^3
=
\mathLabelBox
[
   labelstyle={xshift=2cm},
   linestyle={out=270,in=90, latex-}
]
{
\frac{ 3 r N }{V}
}{number of dynamical degrees of freedom per unit volume}
\kB
=
\kB \,\mbox{per degree of freedom},
\end{dmath}
%
so that
%
\begin{dmath}\label{eqn:condensedMatterLecture10:260}
U \sim \kB T \,\mbox{per degree of freedom}.
\end{dmath}
%
This is called the \underlineAndIndex{Dulong-Petit law}.
%
\paragraph{More generally.}
%
\begin{dmath}\label{eqn:condensedMatterLecture10:280}
\CV(T) =
\lr{
   \frac{ 9 r N }{V} \inv{\omega_{\txtD}^3}
}
\frac{d}{dT}
\int_0^{\omega_{\txtD}}
\Hbar \omega^3 \lr{
\inv{2} +
\inv
{
e^{\Hbar \omega/\kB T} - 1
}
}
d\omega
=
\lr{ \frac{ 9 r N }{V} \inv{\omega_{\txtD}^3} }
\int_0^{\omega_{\txtD}}
-\Hbar \omega^3
\inv
{
\lr{e^{\Hbar \omega/\kB T} - 1}^2
}
e^{\Hbar \omega/\kB T} \lr{ -\frac{\Hbar \omega}{\kB T^2} }
d\omega.
\end{dmath}
%
Make substitutions
\begin{subequations}
\begin{dmath}\label{eqn:condensedMatterLecture10:300}
y = \frac{\Hbar \omega}{\kB T}
\end{dmath}
\begin{dmath}\label{eqn:condensedMatterLecture10:320}
d\omega = \frac{\kB T}{\Hbar} dy
\end{dmath}
\begin{dmath}\label{eqn:condensedMatterLecture10:340}
\mathLabelBox{\kB \Theta}{Debye temp} = \Hbar \omega_{\txtD}.
\end{dmath}
\end{subequations}
%
\index{Debye temperature}

The integral limit is
%
\begin{dmath}\label{eqn:condensedMatterLecture10:520}
y(\omega_{\txtD}) = \frac{\Hbar \omega_{\txtD}}{ \kB T} = \frac{\kB \Theta}{\kB T} = \frac{\Theta}{T},
\end{dmath}
%
so the specific heat is
%
\begin{dmath}\label{eqn:condensedMatterLecture10:360}
\CV(T)
=
\lr{ \frac{ 9 r N }{V} \inv{\omega_{\txtD}^3} }
\int_0^{\Theta/T}
\frac{ \lr{ \kB T y}^3 }{\Hbar^2}
\inv
{
\lr{e^{y} - 1}^2
}
e^{y}
\frac{y}{T}
\frac{\kB T}{\Hbar} dy
%=
%\lr{ \frac{ 9 r N }{V} \inv{\omega_{\txtD}^3} }
%\int_0^{\Theta/T} \frac
%{\Hbar^2 y^4 \lr{\kB T}^4}
%{
%\Hbar^4
%}
%\frac{e^y}{ (e^y - 1)^2 }
%\lr{ -\frac{\Hbar}{\kB T^2} }
%\frac{\kB T}{\Hbar} dy
=
\lr{
\frac{ 9 r N }{V}
\inv{\omega_{\txtD}^3}
}
\frac{\kB^4 T^3}{\Hbar^3}
\int_0^{\Theta/T}
\frac{y^4 e^y}{ \lr{ e^y - 1}^2 }
dy.
\end{dmath}
%
This is an exact result for the Debye model.

In the \(T \rightarrow 0\) limit where \(\Theta/T\) is large, and when \(y\) is large
%
\begin{dmath}\label{eqn:condensedMatterLecture10:380}
\frac{y^4 e^y}{\lr{e^y - 1}^2} \sim y^4 e^{-y} \rightarrow 0.
\end{dmath}
%
Modes are \textunderline{frozen out}.

In the \(T \ll \Hbar \omega_{\txtD}/\kB\) limit, we have
%
\begin{dmath}\label{eqn:condensedMatterLecture10:400}
\CV(T)
=
\frac{ 9 r N }{V}
\frac{\kB^4 T^3}{\kB^3 \Theta^3}
\int_0^{\infty} \frac{y^4 e^y}{ \lr{ e^y - 1}^2 } dy
=
\frac{ 9 r N }{V}
\frac{\kB^4 T^3}{\kB^3 \Theta^3}
\frac{4 \pi^2}{15}
=
\frac{ 3 r N }{V}
\frac{4 \pi^2 \kB }{5} \frac{T^3}{\Theta^3}.
\end{dmath}
%
The \(T^3\) dependence here is \textunderline{very important}.  This comes from

\begin{itemize}
\item
Boltzman distribution,
\item
Quantum mechanics
\item
linear dispersion relation.  This provides the
number of thermally occupied modes \(\propto\) \(T^3\)
\end{itemize}

With \(\omega \propto q\), the volume of thermally occupied \(q\) space \(\propto T^3\) and the energy of a thermally occupied mode \(= T\), we have
averaged energy of these modes propto
%
\begin{dmath}\label{eqn:condensedMatterLecture10:420}
\begin{aligned}
U(T) &\sim T^4 \\
C(T) &= T^3.
\end{aligned}
\end{dmath}
%
%\EndArticle
%\EndNoBibArticle

%\cref{fig:qmSolidsL10:qmSolidsL10Fig3}.
%\imageFigure{../figures/phy487-qmsolids/qmSolidsL10Fig3}{Occupied volume in \(q\) space}{fig:qmSolidsL10:qmSolidsL10Fig3}{0.3}
